% Colloquium Presentation - PhD Thesis Defense
% Plasma Radiation Diagnostics and Feedback Control at Wendelstein 7-X
% Works only with pdflatex

\documentclass{beamer}

% Use predefined IPP style with Eurofusion and W7-X logos
\useoutertheme[Eurofusion, w7x]{ippw7x}

% Don't show navigation symbols
\beamertemplatenavigationsymbolsempty

% Extra packages
\usepackage[english]{babel}
\usepackage{amsmath,amsfonts,amssymb,stackrel}
\usepackage{changepage}
\usepackage{tikz}
\usetikzlibrary{positioning}
\usepackage{siunitx}
\usepackage{xcolor}
\usepackage{caption}
\usepackage{subcaption}
\captionsetup{labelformat=empty,labelsep=none}

% Custom commands
% \newcommand{\diff}{\mathrm{d}}
% \newcommand{\tenpo}[1]{\cdot 10^{#1}}
% \newcommand{\ix}[1]{_{\mathrm{#1}}}
% \newcommand{\imag}{\mathbf{i}}
% \newcommand{\fett}[1]{\textbf{#1}}
% \newcommand{\tilt}[1]{\textit{#1}}
% \newcommand\inlineeqno{\stepcounter{equation}\ \quad\quad(\theequation)}
% 
% % Backup slide commands
% \newcommand{\backupend}{\setcounter{framenumber}{\value{finalframe}}}
% \newcommand{\backupbegin}{\newcounter{finalframe}%
%   \setcounter{finalframe}{\value{framenumber}}}
% 
% % Background logo
% \newcommand{\backgroundlogo}{%
%   \tikz[overlay,remember picture]{%
%   \node[at=(current page.west)] (source) {};%
%   \node[opacity = 0.04, %
%         below left= -0.4\paperheight and -0.6\paperheight of source] {%
%     \includegraphics[height=0.85\paperheight]%
%       {figures/header/minerva_gruen_ganz_ohne_hintergrund.png}%
%     }%
%   }
% }

\begin{document}
  % Title of the presentation
  % \title[Plasma Radiation Diagnostics and Feedback Control at W7-X]%
  %       {Plasma Radiation Diagnostics and Feedback Control\\at the Stellarator Wendelstein 7-X}

  % % Authors
  % \author[P. Hacker]%
  %        {P. Hacker\inst{1}\inst{2}}

  % % Institutes
  % \institute{\tiny%
  %            \inst{1}Max Planck Institute for Plasma Physics, %
  %                    Wendelsteinstr. 1, D-17491 Greifswald, Germany \and%
  %            \inst{2}Ernst-Moritz-Arndt University Greifswald, Domstr. 11, %
  %                    D-17489 Greifswald, Germany}

  % % Date
  % \date{\today}

  % % Title frame 
  % \begin{frame}
  %   \backgroundlogo%
  %   \titlepage%
  %   \acknowledgement%
  % \end{frame}

  % Contents
  % \begin{frame}{Contents}
  %   \only<1>{%
  %     \tableofcontents%
  %     \backgroundlogo%
  %   }%
  % \end{frame}%

  % \section{Introduction}
  
  % % Slide 1: Motivation
  % \begin{frame}
  %   \only<1>{%
  %     \frametitle{Motivation: Nuclear Fusion Energy}%
  %     \begin{block}{Energy Challenge}
  %       \begin{itemize}
  %         \item Depletion of fossil fuels and climate change
  %         \item Need for clean, renewable energy sources
  %         \item Nuclear fusion: virtually emission-free, abundant fuel
  %       \end{itemize}
  %     \end{block}
  %     \vspace{0.3cm}
  %     \begin{block}{Deuterium-Tritium Fusion}
  %       \begin{align}
  %         ^{2}_{1}\text{D}+\,^{3}_{1}\text{T}\longrightarrow\,^{4}_{2}\text{He}\left(\SI{3.5}{\mega\electronvolt}\right)+\,^{1}_{0}n\left(\SI{14.1}{\mega\electronvolt}\right)\nonumber
  %       \end{align}
  %       \begin{itemize}
  %         \item Requires plasma temperatures $\sim$\SI{20}{\kilo\electronvolt}
  %         \item Lawson criterion: $n\ix{e}T\tau\ix{E} \ge$ threshold
  %       \end{itemize}
  %     \end{block}
  %   }%
  % \end{frame}

  % % Slide 2: Wendelstein 7-X
  % \begin{frame}
  %   \only<1>{%
  %     \frametitle{Wendelstein 7-X Stellarator}%
  %     \begin{columns}
  %       \column{0.5\textwidth}
  %         \begin{itemize}
  %           \item World's largest stellarator
  %           \item Major radius: \SI{5.5}{\meter}
  %           \item Minor radius: \SI{0.53}{\meter}
  %           \item Volume: \SI{30}{\cubic\meter}
  %           \item Max. field: \SI{3}{\tesla}
  %           \item Heating: \SI{14}{\mega\watt}
  %           \item Goal: \SI{30}{\minute} steady-state
  %         \end{itemize}
  %       \column{0.5\textwidth}
  %         \centering
  %         \includegraphics[width=\textwidth]{figures/content/w7x_overview.pdf}
  %         \caption{\footnotesize W7-X with 5-fold symmetry, superconducting coils, island divertor}
  %     \end{columns}
  %   }%
  % \end{frame}

  % % Slide 3: Island Divertor
  % \begin{frame}
  %   \only<1>{%
  %     \frametitle{Island Divertor Concept}%
  %     \begin{block}{Challenge}
  %       Heat loads on plasma-facing components exceed material limits
  %     \end{block}
  %     \vspace{0.3cm}
  %     \begin{columns}
  %       \column{0.5\textwidth}
  %         \begin{itemize}
  %           \item Natural magnetic islands at edge
  %           \item Rotational transform $\iota \approx 1$
  %           \item 5 divertor modules per period
  %           \item Bean-shaped cross-section
  %           \item Configurations: 5/6, 5/5, 5/4
  %         \end{itemize}
  %       \column{0.5\textwidth}
  %         \centering
  %         \includegraphics[width=\textwidth]{figures/content/flux_surfaces.pdf}
  %         \caption{\footnotesize Flux surfaces and divertor location}
  %     \end{columns}
  %   }%
  % \end{frame}

  % % Slide 4: Radiation Challenge
  % \begin{frame}
  %   \only<1>{%
  %     \frametitle{Radiation and Power Exhaust}%
  %     \begin{block}{Power Balance}
  %       \begin{align}
  %         P_{\alpha}+P\ix{h}=P\ix{n}+P\ix{rad}^{\text{core}}+P\ix{SOL}\nonumber
  %       \end{align}
  %     \end{block}
  %     \vspace{0.3cm}
  %     \begin{itemize}
  %       \item Radiation distributes heat load over entire vessel
  %       \item Bremsstrahlung + line radiation from impurities
  %       \item DEMO target: $f\ix{rad}=P\ix{rad}/P\ix{heating}>0.95$
  %       \item Challenge: balance between cooling and dilution
  %       \item Impurity seeding for controlled radiation
  %     \end{itemize}
  %     \vspace{0.3cm}
  %     \begin{block}{Thesis Focus}
  %       Diagnostics and real-time control of plasma radiation at W7-X
  %     \end{block}
  %   }%
  % \end{frame}

  % % Slide 5: Transport Overview
  % \begin{frame}
  %   \only<1>{%
  %     \frametitle{Impurity Transport}%
  %     \begin{block}{Transport Mechanisms}
  %       \begin{align}
  %         \vec{\Gamma}\ix{q}=-D\ix{q}\nabla n\ix{q}+v\ix{q}n\ix{q}\nonumber
  %       \end{align}
  %     \end{block}
  %     \begin{itemize}
  %       \item \textbf{Classical:} Coulomb collisions, $D \propto 1/B^{2}$
  %       \item \textbf{Neoclassical:} Geometry effects, trapped particles, banana orbits
  %       \item \textbf{Anomalous:} Turbulence, micro-instabilities (ITG, ETG)
  %     \end{itemize}
  %     \vspace{0.3cm}
  %     \begin{block}{W7-X Optimization}
  %       Magnetic field optimized to reduce neoclassical transport
  %     \end{block}
  %   }%
  % \end{frame}

  % \section{Bolometry at W7-X}

  % % Slide 6: Bolometer Principle
  % \begin{frame}
  %   \only<1>{%
  %     \frametitle{Bolometer Diagnostic}%
  %     \begin{block}{Goals}
  %       \begin{itemize}
  %         \item Global power balance: total radiated power $P\ix{rad}$
  %         \item Local power balance: 2D radiation profiles via tomography
  %       \end{itemize}
  %     \end{block}
  %     \vspace{0.3cm}
  %     \begin{columns}
  %       \column{0.5\textwidth}
  %         \textbf{Metal Resistor Bolometer:}
  %         \begin{itemize}
  %           \item Thermal radiation detector
  %           \item \SI{5}{\micro\meter} gold foil absorber
  %           \item Carbon coating on Si$_3$N$_4$ substrate
  %           \item Wheatstone bridge readout
  %           \item Resolution: \SI{200}{\nano\watt}
  %         \end{itemize}
  %       \column{0.5\textwidth}
  %         \centering
  %         \includegraphics[width=\textwidth]{figures/content/detector_scheme.pdf}
  %         \caption{\footnotesize Detector with gold meander}
  %     \end{columns}
  %   }%
  % \end{frame}

  % % Slide 7: Camera System
  % \begin{frame}
  %   \only<1>{%
  %     \frametitle{Multicamera Bolometer System}%
  %     \begin{columns}
  %       \column{0.5\textwidth}
  %         \begin{itemize}
  %           \item 128 channels total
  %           \item HBCm: 32 channels (horizontal)
  %           \item VBCl/VBCr: 2×24 channels (vertical)
  %           \item Located at \SI{108}{\degree} toroidally
  %           \item Spatial resolution: \SI{5}{\centi\meter}
  %           \item Temporal resolution: \SI{0.8}{-}\SI{1.6}{\milli\second}
  %         \end{itemize}
  %       \column{0.5\textwidth}
  %         \centering
  %         \includegraphics[width=\textwidth]{figures/content/camera_los.pdf}
  %         \caption{\footnotesize Lines of sight through plasma}
  %     \end{columns}
  %     \vspace{0.3cm}
  %     \begin{block}{Design Features}
  %       Water-cooled, ECRH screening, steady-state capable
  %     \end{block}
  %   }%
  % \end{frame}

  % % Slide 8: Bolometer Equation
  % \begin{frame}
  %   \only<1>{%
  %     \frametitle{Calibration \& Calculation}%
  %     \begin{block}{Bolometer Equation}
  %       \begin{align}
  %         P\ix{ch}=F\ix{ch}\cdot\left(\tau\ix{ch}\frac{\diff(\Delta U\ix{ch})}{\diff t}+f\ix{ch}\cdot\Delta U\ix{ch}\right)\nonumber
  %       \end{align}
  %       \begin{itemize}
  %         \item $\Delta U$: voltage change $\propto$ temperature change
  %         \item $\tau$: cooling time constant
  %         \item $F$: calibration factor (Ohmic heating)
  %       \end{itemize}
  %     \end{block}
  %     \vspace{0.3cm}
  %     \begin{block}{Total Radiated Power}
  %       \begin{align}
  %         P\ix{rad}=\frac{V\ix{P,tor}}{V\ix{cam}}\cdot\sum\ix{ch}\frac{V\ix{ch}}{K\ix{ch}}\cdot P\ix{ch}\nonumber
  %       \end{align}
  %       \begin{itemize}
  %         \item $V\ix{ch}$: detector viewing volume
  %         \item $K\ix{ch}$: geometrical factor (etendue)
  %       \end{itemize}
  %     \end{block}
  %   }%
  % \end{frame}

  % % Slide 9: Performance Results
  % \begin{frame}
  %   \only<1>{%
  %     \frametitle{Bolometer Performance (OP1.2b)}%
  %     \centering
  %     \includegraphics[width=0.8\textwidth]{figures/content/prad_timeseries.pdf}
  %     \caption{\footnotesize Total radiated power from HBCm and VBC cameras}
  %     \vspace{0.3cm}
  %     \begin{itemize}
  %       \item Successful operation in OP1.2a and OP1.2b campaigns
  %       \item Reliable measurement of $P\ix{rad}$ up to \SI{10}{\mega\watt}
  %       \item Good agreement between camera systems
  %       \item Chord brightness profiles for tomography
  %     \end{itemize}
  %   }%
  % \end{frame}

  % \section{Real-Time Feedback Control}

  % % Slide 10: Feedback Motivation
  % \begin{frame}
  %   \only<1>{%
  %     \frametitle{Motivation for Feedback Control}%
  %     \begin{block}{Challenge}
  %       Achieve stable plasma detachment with high radiation fraction
  %     \end{block}
  %     \vspace{0.3cm}
  %     \begin{itemize}
  %       \item Target: $f\ix{rad} = P\ix{rad}/P\ix{ECRH} \ge 0.85$
  %       \item Reduce divertor heat loads by factor 2-3
  %       \item Avoid radiative collapse
  %       \item Real-time control via impurity seeding
  %     \end{itemize}
  %     \vspace{0.3cm}
  %     \begin{block}{Approach}
  %       \begin{itemize}
  %         \item Fast calculation of $P\ix{rad}$ prediction proxy
  %         \item PID controller for thermal helium beam injection
  %         \item Minimum latency: \SI{13.6}{\milli\second}
  %       \end{itemize}
  %     \end{block}
  %   }%
  % \end{frame}

  % % Slide 11: Real-Time System
  % \begin{frame}
  %   \only<1>{%
  %     \frametitle{Real-Time Radiation System}%
  %     \begin{columns}
  %       \column{0.5\textwidth}
  %         \textbf{Hardware:}
  %         \begin{itemize}
  %           \item NI 6321 analog output card
  %           \item FIFO array for noise smoothing
  %           \item Direct connection to gas valve
  %         \end{itemize}
  %         \vspace{0.3cm}
  %         \textbf{Software:}
  %         \begin{itemize}
  %           \item Two prediction proxies
  %           \item $P\ix{pred}^{(1)}$: subset of channels
  %           \item $P\ix{pred}^{(2)}$: single channel
  %           \item PID controller implementation
  %         \end{itemize}
  %       \column{0.5\textwidth}
  %         \centering
  %         \includegraphics[width=\textwidth]{figures/content/feedback_scheme.pdf}
  %         \caption{\footnotesize Real-time feedback loop}
  %     \end{columns}
  %   }%
  % \end{frame}

  % % Slide 12: PID Controller
  % \begin{frame}
  %   \only<1>{%
  %     \frametitle{PID Controller Design}%
  %     \begin{block}{Control Law}
  %       \begin{align}
  %         u(t) = K\ix{p}e(t) + K\ix{i}\int e(t)\diff t + K\ix{d}\frac{\diff e(t)}{\diff t}\nonumber
  %       \end{align}
  %       where $e(t) = f\ix{rad,target} - f\ix{rad,measured}$
  %     \end{block}
  %     \vspace{0.3cm}
  %     \begin{itemize}
  %       \item \textbf{Proportional:} Immediate response to error
  %       \item \textbf{Integral:} Eliminate steady-state error
  %       \item \textbf{Derivative:} Damping, anticipate future error
  %     \end{itemize}
  %     \vspace{0.3cm}
  %     \begin{block}{Tuning Parameters}
  %       Optimized for W7-X plasma response time and seeding efficiency
  %     \end{block}
  %   }%
  % \end{frame}

  % % Slide 13: Feedback Results 1
  % \begin{frame}
  %   \only<1>{%
  %     \frametitle{Experimental Results: XP20180920.29}%
  %     \centering
  %     \includegraphics[width=0.9\textwidth]{figures/content/feedback_20180920_029.pdf}
  %     \caption{\footnotesize Successful detachment at $f\ix{rad} \approx 90\%$}
  %     \vspace{0.3cm}
  %     \begin{itemize}
  %       \item Stable control of radiation fraction
  %       \item Smooth transition to detachment
  %       \item Factor 2 reduction in divertor heat load
  %     \end{itemize}
  %   }%
  % \end{frame}

  % % Slide 14: Feedback Results 2
  % \begin{frame}
  %   \only<1>{%
  %     \frametitle{Experimental Results: XP20181010.32}%
  %     \centering
  %     \includegraphics[width=0.9\textwidth]{figures/content/feedback_20181010_032.pdf}
  %     \caption{\footnotesize Extended detachment phase with feedback}
  %     \vspace{0.3cm}
  %     \begin{itemize}
  %       \item Maintained $f\ix{rad} \ge 85\%$ for several seconds
  %       \item Avoided radiative collapse
  %       \item Demonstrated robustness of control system
  %     \end{itemize}
  %   }%
  % \end{frame}

  % % Slide 15: Comparison with Other Methods
  % \begin{frame}
  %   \only<1>{%
  %     \frametitle{Comparison: Feedback Methods}%
  %     \begin{table}
  %       \centering
  %       \begin{tabular}{lccc}
  %         \hline
  %         Method & Response Time & Stability & $f\ix{rad}$ \\
  %         \hline
  %         Bolometer feedback & \SI{13.6}{\milli\second} & High & 85-90\% \\
  %         Density feedback & \SI{50}{\milli\second} & Medium & 75-85\% \\
  %         C-III filterscope & \SI{20}{\milli\second} & Medium & 80-85\% \\
  %         \hline
  %       \end{tabular}
  %     \end{table}
  %     \vspace{0.5cm}
  %     \begin{block}{Advantages of Bolometer Feedback}
  %       \begin{itemize}
  %         \item Direct measurement of radiated power
  %         \item Fastest response time
  %         \item Most stable control
  %         \item Highest achievable $f\ix{rad}$
  %       \end{itemize}
  %     \end{block}
  %   }%
  % \end{frame}

  % \section{Feedback Impact Analysis}

  % % Slide 16: Two-Chamber Model
  % \begin{frame}
  %   \only<1>{%
  %     \frametitle{Impurity Dynamics Model}%
  %     \begin{block}{Two-Chamber Model}
  %       \begin{align}
  %         \frac{\diff N\ix{p}}{\diff t} &= \Gamma\ix{s} + N\ix{w}\cdot\tau\ix{w,p} - N\ix{p}\cdot(\tau\ix{p,w}\cdot f + \tau\ix{p})\nonumber\\
  %         \frac{\diff N\ix{w}}{\diff t} &= N\ix{p}\cdot\tau\ix{p,w}\cdot f - N\ix{w}\cdot\tau\ix{w,p}\nonumber
  %       \end{align}
  %     \end{block}
  %     \begin{itemize}
  %       \item $N\ix{p}$: impurity population in plasma
  %       \item $N\ix{w}$: impurity population at wall
  %       \item $\Gamma\ix{s}$: seeding rate (control input)
  %       \item $\tau\ix{p,w}$, $\tau\ix{w,p}$: transport rates
  %       \item $f$: fraction leaving plasma to wall
  %     \end{itemize}
  %   }%
  % \end{frame}

  % % Slide 17: Sensitivity Analysis
  % \begin{frame}
  %   \only<1>{%
  %     \frametitle{Parameter Sensitivity Scans}%
  %     \centering
  %     \includegraphics[width=0.8\textwidth]{figures/content/sensitivity_scan.pdf}
  %     \caption{\footnotesize Impact of seeding rate and transport parameters}
  %     \vspace{0.3cm}
  %     \begin{itemize}
  %       \item Seeding rate most critical parameter
  %       \item Transport rates affect response time
  %       \item Limit values determine stability margins
  %       \item Model validated against experimental data
  %     \end{itemize}
  %   }%
  % \end{frame}

  % % Slide 18: LOS Sensitivity
  % \begin{frame}
  %   \only<1>{%
  %     \frametitle{Line-of-Sight Sensitivity Analysis}%
  %     \begin{columns}
  %       \column{0.5\textwidth}
  %         \textbf{Methodology:}
  %         \begin{itemize}
  %           \item Forward modeling with EMC3-EIRENE
  %           \item Synthetic diagnostics
  %           \item Channel-by-channel sensitivity
  %           \item Optimal channel selection
  %         \end{itemize}
  %       \column{0.5\textwidth}
  %         \centering
  %         \includegraphics[width=\textwidth]{figures/content/los_sensitivity.pdf}
  %         \caption{\footnotesize Sensitivity map}
  %     \end{columns}
  %     \vspace{0.3cm}
  %     \begin{block}{Results}
  %       Identified optimal channel subset for $P\ix{pred}^{(1)}$ proxy
  %     \end{block}
  %   }%
  % \end{frame}

  % \section{Tomographic Inversion}

  % % Slide 19: Tomography Problem
  % \begin{frame}
  %   \only<1>{%
  %     \frametitle{2D Radiation Reconstruction}%
  %     \begin{block}{Inverse Problem}
  %       \begin{align}
  %         \vec{b} = \mathbf{T}\vec{x}\nonumber
  %       \end{align}
  %       \begin{itemize}
  %         \item $\vec{b}$: measured chord brightness (128 channels)
  %         \item $\vec{x}$: 2D emissivity distribution (unknown)
  %         \item $\mathbf{T}$: geometry matrix (line integrals)
  %       \end{itemize}
  %     \end{block}
  %     \vspace{0.3cm}
  %     \begin{itemize}
  %       \item Ill-posed problem: non-unique solutions
  %       \item Requires regularization
  %       \item Challenge: limited viewing angles
  %       \item Goal: robust, physics-informed reconstruction
  %     \end{itemize}
  %   }%
  % \end{frame}

  % % Slide 20: MFR Algorithm
  % \begin{frame}
  %   \only<1>{%
  %     \frametitle{Minimum Fisher Regularization}%
  %     \begin{block}{Tikhonov Regularization}
  %       \begin{align}
  %         \vec{x}^{(n+1)} = (\mathbf{T}^T\mathbf{T} + \mu\mathbf{H}^{(n)})^{-1}\mathbf{T}^T\vec{b}\nonumber
  %       \end{align}
  %     \end{block}
  %     \vspace{0.3cm}
  %     \begin{block}{Fisher Information}
  %       \begin{align}
  %         I\ix{F} = \int\frac{1}{g(r)}\left(\frac{\partial g(r)}{\partial r}\right)^2\diff r\nonumber
  %       \end{align}
  %     \end{block}
  %     \begin{itemize}
  %       \item Minimizes Fisher information (smoothness)
  %       \item Radially dependent anisotropy (RDA) weighting
  %       \item Iterative solution with adaptive regularization
  %       \item Preserves physical features (gradients, peaks)
  %     \end{itemize}
  %   }%
  % \end{frame}

  % % Slide 21: Phantom Tests
  % \begin{frame}
  %   \only<1>{%
  %     \frametitle{Validation: Phantom Tests}%
  %     \centering
  %     \includegraphics[width=0.9\textwidth]{figures/content/phantom_comparison.pdf}
  %     \caption{\footnotesize Original phantom (left) vs. MFR reconstruction (right)}
  %     \vspace{0.3cm}
  %     \begin{itemize}
  %       \item Gaussian emission profiles
  %       \item Various anisotropy parameters tested
  %       \item Excellent reconstruction quality
  %       \item Robust against noise
  %     \end{itemize}
  %   }%
  % \end{frame}

  % % Slide 22: Experimental Tomography
  % \begin{frame}
  %   \only<1>{%
  %     \frametitle{Experimental Results: XP20181010.032}%
  %     \centering
  %     \includegraphics[width=0.9\textwidth]{figures/content/tomo_20181010_032.pdf}
  %     \caption{\footnotesize 2D radiation distribution during detachment}
  %     \vspace{0.3cm}
  %     \begin{itemize}
  %       \item Clear edge radiation during detachment
  %       \item Radiation moves outward with increasing $f\ix{rad}$
  %       \item Good agreement with chord brightness profiles
  %       \item Validates feedback control strategy
  %     \end{itemize}
  %   }%
  % \end{frame}

  % % Slide 23: Radial Profiles
  % \begin{frame}
  %   \only<1>{%
  %     \frametitle{Radial Radiation Profiles}%
  %     \centering
  %     \includegraphics[width=0.8\textwidth]{figures/content/radial_profiles.pdf}
  %     \caption{\footnotesize Emissivity vs. effective radius}
  %     \vspace{0.3cm}
  %     \begin{itemize}
  %       \item Peak radiation at $\rho\ix{eff} \approx 0.8$ during detachment
  %       \item Core radiation remains low (no impurity accumulation)
  %       \item Edge radiation increases with seeding
  %       \item Consistent with transport modeling
  %     \end{itemize}
  %   }%
  % \end{frame}

  % \section{Conclusions}

  % % Slide 24: Summary - Bolometry
  % \begin{frame}
  %   \only<1>{%
  %     \frametitle{Summary: Bolometer Diagnostic}%
  %     \begin{block}{Achievements}
  %       \begin{itemize}
  %         \item Successful commissioning of 128-channel system
  %         \item Reliable operation in OP1.2a and OP1.2b campaigns
  %         \item Spatial resolution: \SI{5}{\centi\meter}
  %         \item Temporal resolution: \SI{0.8}{-}\SI{1.6}{\milli\second}
  %         \item Power resolution: \SI{200}{\nano\watt}
  %       \end{itemize}
  %     \end{block}
  %     \vspace{0.3cm}
  %     \begin{block}{Key Results}
  %       \begin{itemize}
  %         \item Total radiated power up to \SI{10}{\mega\watt}
  %         \item Good agreement between camera systems
  %         \item Chord brightness profiles for tomography
  %         \item Foundation for real-time feedback
  %       \end{itemize}
  %     \end{block}
  %   }%
  % \end{frame}

  % % Slide 25: Summary - Feedback
  % \begin{frame}
  %   \only<1>{%
  %     \frametitle{Summary: Real-Time Feedback Control}%
  %     \begin{block}{Achievements}
  %       \begin{itemize}
  %         \item First real-time radiation feedback at W7-X
  %         \item Minimum latency: \SI{13.6}{\milli\second}
  %         \item PID controller with optimized parameters
  %         \item Stable detachment at $f\ix{rad} \ge 85\%$
  %       \end{itemize}
  %     \end{block}
  %     \vspace{0.3cm}
  %     \begin{block}{Key Results}
  %       \begin{itemize}
  %         \item Factor 2 reduction in divertor heat loads
  %         \item Avoided radiative collapse
  %         \item Superior to density and filterscope feedback
  %         \item Demonstrated in multiple experimental campaigns
  %       \end{itemize}
  %     \end{block}
  %   }%
  % \end{frame}

  % % Slide 26: Summary - Analysis
  % \begin{frame}
  %   \only<1>{%
  %     \frametitle{Summary: Impact \& Sensitivity Analysis}%
  %     \begin{block}{Two-Chamber Model}
  %       \begin{itemize}
  %         \item Describes impurity dynamics in plasma and wall
  %         \item Parameter sensitivity scans performed
  %         \item Seeding rate identified as critical parameter
  %         \item Model validated against experimental data
  %       \end{itemize}
  %     \end{block}
  %     \vspace{0.3cm}
  %     \begin{block}{LOS Sensitivity}
  %       \begin{itemize}
  %         \item Forward modeling with EMC3-EIRENE
  %         \item Optimal channel selection for feedback
  %         \item Improved prediction proxy accuracy
  %         \item Guides future diagnostic upgrades
  %       \end{itemize}
  %     \end{block}
  %   }%
  % \end{frame}

  % % Slide 27: Summary - Tomography
  % \begin{frame}
  %   \only<1>{%
  %     \frametitle{Summary: Tomographic Inversion}%
  %     \begin{block}{MFR Algorithm}
  %       \begin{itemize}
  %         \item Minimum Fisher Regularization with RDA weighting
  %         \item Robust reconstruction from limited viewing angles
  %         \item Validated with phantom tests
  %         \item Preserves physical features
  %       \end{itemize}
  %     \end{block}
  %     \vspace{0.3cm}
  %     \begin{block}{Experimental Results}
  %       \begin{itemize}
  %         \item 2D radiation distributions during detachment
  %         \item Edge radiation peak at $\rho\ix{eff} \approx 0.8$
  %         \item No core impurity accumulation
  %         \item Validates feedback control strategy
  %       \end{itemize}
  %     \end{block}
  %   }%
  % \end{frame}

  % % Slide 28: Outlook
  % \begin{frame}
  %   \only<1>{%
  %     \frametitle{Outlook \& Future Work}%
  %     \begin{block}{Near-Term Improvements}
  %       \begin{itemize}
  %         \item Reduce feedback latency to $<$\,\SI{10}{\milli\second}
  %         \item Implement adaptive PID tuning
  %         \item Extend to multiple seeding locations
  %         \item Integration with other control systems
  %       \end{itemize}
  %     \end{block}
  %     \vspace{0.3cm}
  %     \begin{block}{Long-Term Goals}
  %       \begin{itemize}
  %         \item Real-time tomographic inversion
  %         \item Multi-species impurity control
  %         \item Machine learning for predictive control
  %         \item Application to future fusion devices (ITER, DEMO)
  %       \end{itemize}
  %     \end{block}
  %   }%
  % \end{frame}

  % % Slide 29: Contributions
  % \begin{frame}
  %   \only<1>{%
  %     \frametitle{Scientific Contributions}%
  %     \begin{enumerate}
  %       \item \textbf{Diagnostic Development:} Commissioning and characterization of W7-X bolometer system
  %       \vspace{0.3cm}
  %       \item \textbf{Real-Time Control:} First radiation feedback system at W7-X with \SI{13.6}{\milli\second} latency
  %       \vspace{0.3cm}
  %       \item \textbf{Detachment Achievement:} Stable high-radiation scenarios ($f\ix{rad} \ge 85\%$) with reduced divertor loads
  %       \vspace{0.3cm}
  %       \item \textbf{Tomographic Method:} MFR algorithm with RDA weighting for robust 2D reconstruction
  %       \vspace{0.3cm}
  %       \item \textbf{Physics Understanding:} Impurity transport and radiation dynamics in stellarator geometry
  %     \end{enumerate}
  %   }%
  % \end{frame}

  % % Slide 30: Final Slide
  % \begin{frame}
  %   \only<1>{%
  %     \backgroundlogo%
  %     \frametitle{Conclusions}
  %     \begin{center}%
  %       \Large%
  %       \textbf{Plasma Radiation Diagnostics and Feedback Control\\at Wendelstein 7-X}
  %       \vspace{1cm}
  %       
  %       \normalsize
  %       \begin{itemize}
  %         \item Bolometer diagnostic successfully commissioned
  %         \item Real-time feedback enables stable detachment
  %         \item Tomographic inversion reveals radiation dynamics
  %         \item Foundation for future stellarator power plants
  %       \end{itemize}
  %       \vspace{1cm}
  %       
  %       \Large
  %       Thank you for your attention!\\
  %       \vspace{0.5cm}
  %       Questions?
  %     \end{center}
  %   }%
  % \end{frame}

  % % BACKUP SLIDES
  % \appendix
  % \backupbegin

  % % Backup 1: Detector Details
  % \begin{frame}{Backup: Detector Construction}
  %   \centering
  %   \includegraphics[width=0.7\textwidth]{figures/content/detector_details.pdf}
  %   \caption{Gold meander on Si$_3$N$_4$ substrate with carbon coating}
  % \end{frame}

  % % Backup 2: Wheatstone Bridge
  % \begin{frame}{Backup: Wheatstone Bridge Circuit}
  %   \centering
  %   \includegraphics[width=0.7\textwidth]{figures/content/wheatstone_bridge.pdf}
  %   \caption{Detector array with reference resistors in Wheatstone configuration}
  % \end{frame}

  % % Backup 3: Calibration
  % \begin{frame}{Backup: Calibration Method}
  %   \begin{itemize}
  %     \item Ohmic heating stages with known power
  %     \item Determine calibration factors $F\ix{ch}$, $\tau\ix{ch}$
  %     \item Temperature-dependent resistance
  %     \item Regular recalibration during campaigns
  %   \end{itemize}
  % \end{frame}

  % % Backup 4: ECRH Screening
  % \begin{frame}{Backup: ECRH Stray Radiation}
  %   \centering
  %   \includegraphics[width=0.7\textwidth]{figures/content/ecrh_screening.pdf}
  %   \caption{Wire mesh and ceramic coating reduce microwave background to $<$1\%}
  % \end{frame}

  % % Backup 5: Geometry Factors
  % \begin{frame}{Backup: Line-of-Sight Geometry}
  %   \begin{itemize}
  %     \item Etendue: $G = A\ix{det}\cdot A\ix{pinhole}/d^2$
  %     \item Viewing cone through pinhole aperture
  %     \item Volume calculation via ray tracing
  %     \item Geometrical factor $K\ix{ch}$ for each channel
  %   \end{itemize}
  % \end{frame}

  % % Backup 6: Feedback Latency
  % \begin{frame}{Backup: Latency Breakdown}
  %   \begin{table}
  %     \centering
  %     \begin{tabular}{lc}
  %       \hline
  %       Component & Time [\si{\milli\second}] \\
  %       \hline
  %       Data acquisition & 0.8 \\
  %       FIFO smoothing & 3.2 \\
  %       Calculation & 1.6 \\
  %       PID computation & 0.5 \\
  %       Output & 0.5 \\
  %       Gas valve response & 7.0 \\
  %       \hline
  %       Total & 13.6 \\
  %       \hline
  %     \end{tabular}
  %   \end{table}
  % \end{frame}

  % % Backup 7: PID Tuning
  % \begin{frame}{Backup: PID Parameter Tuning}
  %   \begin{itemize}
  %     \item Ziegler-Nichols method as starting point
  %     \item Manual optimization based on plasma response
  %     \item $K\ix{p}$: proportional gain
  %     \item $K\ix{i}$: integral time constant
  %     \item $K\ix{d}$: derivative time constant
  %     \item Anti-windup for integral term
  %   \end{itemize}
  % \end{frame}

  % % Backup 8: Transport Regimes
  % \begin{frame}{Backup: Neoclassical Transport Regimes}
  %   \centering
  %   \includegraphics[width=0.7\textwidth]{figures/content/transport_regimes.pdf}
  %   \caption{Banana, plateau, and Pfirsch-Schlüter regimes vs. collisionality}
  % \end{frame}

  % % Backup 9: EMC3-EIRENE
  % \begin{frame}{Backup: EMC3-EIRENE Modeling}
  %   \begin{itemize}
  %     \item 3D edge plasma transport code
  %     \item Monte Carlo neutral transport (EIRENE)
  %     \item Fluid plasma model (EMC3)
  %     \item Provides synthetic diagnostics
  %     \item Validates experimental observations
  %   \end{itemize}
  % \end{frame}

  % % Backup 10: Anisotropy Parameters
  % \begin{frame}{Backup: RDA Weighting Parameters}
  %   \begin{itemize}
  %     \item Radially dependent anisotropy
  %     \item $\kappa\ix{ani}(r)$: anisotropy factor
  %     \item Higher values: more radial smoothing
  %     \item Lower values: more poloidal smoothing
  %     \item Optimized for W7-X geometry
  %   \end{itemize}
  % \end{frame}

  % \backupend

\end{document}

% Made with Bob
