% TP: Welcome and introduce the thesis topic - bolometry diagnostics and real-time radiation feedback control at the W7-X stellarator
% TP: This work addresses critical challenges in fusion energy research: managing plasma radiation for machine protection and achieving stable detachment

% TABLE OF CONTENTS
\begin{frame}
    \frametitle{Contents}
    \backgroundlogoL%
    \tableofcontents

    % TP: Overview of presentation structure - we'll cover fusion fundamentals, W7-X stellarator, bolometer diagnostics, feedback control system, experimental results, and tomographic reconstruction
\end{frame}

\section{Motivation and Context}

\begin{frame}
    \frametitle{Nuclear Fusion: The Energy Challenge}
    \backgroundlogo%

    \begin{columns}
        \begin{column}{0.5\textwidth}
            \begin{itemize}
                \item D-T fusion reaction: $^2$H + $^3$H $\rightarrow$ $^4$He (3.5 MeV) + n (14.1 MeV)
                \item Lawson criterion for net energy gain:
                \begin{equation*}
                    n_e T \tau_E \ge \frac{12f_{tot}}{\langle\sigma_{DT}v\rangle f_H^2 E_\alpha - 4L_Z(T)} T^2
                \end{equation*}
                \item Requires: high temperature ($T \sim$ 10-20 keV), density, and confinement time
                \item Magnetic confinement: tokamaks and stellarators
            \end{itemize}
        \end{column}
        \begin{column}{0.45\textwidth}
            \only<2->{
            \begin{block}{Triple Product}
                Key figure of merit for fusion performance combining density, temperature, and energy confinement time
            \end{block}
            }
        \end{column}
    \end{columns}

    % TP: Nuclear fusion offers clean, abundant energy through D-T reactions
    % TP: The Lawson criterion defines conditions for net energy gain - need high temperature plasma confined long enough
    % TP: Magnetic confinement devices like stellarators use strong magnetic fields to contain the hot plasma
\end{frame}

\begin{frame}
    \frametitle{Wendelstein 7-X Stellarator}
    \backgroundlogo%

    \begin{columns}
        \begin{column}{0.55\textwidth}
            \begin{itemize}
                \item World's largest stellarator (R = 5.5 m, a $\sim$ 0.5 m)
                \item 50 superconducting coils, 5-fold symmetry
                \item Island divertor configuration
                \item Optimized for reduced neoclassical transport
                \item Operational phases: OP1.1, OP1.2a, OP1.2b
                \item Achieved 30 minute plasma discharges
            \end{itemize}
        \end{column}
        \begin{column}{0.4\textwidth}
            \centering
            \includegraphics[width=0.9\textwidth]{%
                ../THESIS/content/figures/chapter0/w7x.pdf}

            \only<2->{
            \begin{alertblock}{Key Challenge}
                Managing heat loads on plasma-facing components through controlled radiative cooling
            \end{alertblock}
            }
        \end{column}
    \end{columns}

    % TP: W7-X is the world's most advanced stellarator - unlike tokamaks, stellarators use 3D-shaped coils for steady-state operation
    % TP: The island divertor handles exhaust heat, but requires careful radiation control to protect components
    % TP: Goal: achieve stable detachment with radiation fractions above 90% while maintaining plasma performance
\end{frame}

\begin{frame}
    \frametitle{Plasma Radiation and Transport}
    \backgroundlogo%

    \begin{columns}
        \begin{column}{0.5\textwidth}
            \only<1->{
            \textbf{Transport Mechanisms:}
            \begin{itemize}
                \item Classical: collisional diffusion
                \item Neoclassical: trapped particles, $\propto 1/\nu$
                \item Anomalous: turbulent transport (dominant)
            \end{itemize}

            \vspace{1em}
            \textbf{Radiation Processes:}
            \begin{itemize}
                \item Bremsstrahlung: $P_{Brems} \propto n_e^2 T^{1/2}$
                \item Line radiation: atomic transitions
                \item Impurity effects on plasma performance
            \end{itemize}
            }

            \vspace{1em}
            \only<4->{
            \textbf{Radiation Fraction:}
            \begin{equation*}
                f_{rad} = \frac{P_{rad}}{P_{ECRH}}
            \end{equation*}
            }
        \end{column}

        \begin{column}{0.45\textwidth}
            \only<2->{
            \centering
            \includegraphics[width=0.9\textwidth]{%
                ../THESIS/content/figures/chapter0/oxygen_carbon_rad.pdf}
            }

            \only<3->{
            \begin{exampleblock}{Impurity Seeding}
                Deliberate injection of low-Z (He, N$_2$) or high-Z (Ne, Ar) impurities to enhance edge cooling and achieve detachment
            \end{exampleblock}
            }

        \end{column}
    \end{columns}

    % TP: Plasma transport determines how energy and particles move - anomalous transport from turbulence dominates over classical predictions
    % TP: Radiation cools the plasma through bremsstrahlung and line emission from impurities
    % TP: Controlled impurity seeding allows us to radiate power in the edge, protecting divertor targets while maintaining core performance
    % TP: Carbon and oxygen are intrinsic impurities from wall interactions - their radiation depends strongly on temperature
\end{frame}

\begin{frame}
    \frametitle{Island Divertor Configuration}
    \backgroundlogo%

    \begin{columns}
        \begin{column}{0.55\textwidth}
            \textbf{Magnetic Configuration:}
            \begin{itemize}
                \item Standard configuration: $\iota = 5/5$
                \item Five helical island chains
                \item Ten divertor modules (5 half-modules)
                \item Field lines connect upstream to downstream targets
                \item Connection lengths: $\mathcal{O}$(100-1000 m)
            \end{itemize}

            \vspace{1em}
            \textbf{Gas Injection Strategy:}
            \begin{itemize}
                \item Thermal helium beam valves
                \item Located behind divertor targets
                \item Direct seeding into island chains
                \item Fast piezo valves (ms response)
            \end{itemize}
        \end{column}
        \begin{column}{0.4\textwidth}
            \centering
            \includegraphics[width=\textwidth]{%
                ../THESIS/content/figures/chapter2/divertor_fluxtubes.pdf}

            \vspace{1em}
            \includegraphics[width=\textwidth]{%
                ../THESIS/content/figures/chapter2/divertor_effenberg_reduced.pdf}
        \end{column}
    \end{columns}

    % TP: W7-X uses an island divertor - magnetic islands at the edge intercept heat flux
    % TP: Five island chains wrap helically around the torus between upper and lower divertors
    % TP: Gas injection valves are strategically placed to seed directly into these island chains
    % TP: Field line connection between injection point and divertor targets is crucial for understanding radiation distribution
\end{frame}

\section{Bolometer Diagnostic System}

\begin{frame}
    \frametitle{Bolometry at W7-X}
    \backgroundlogo%

    \begin{columns}
        \begin{column}{0.5\textwidth}
            \only<1->{
            \textbf{Metal Resistor Bolometers:}
            \begin{itemize}
                \item Thin metal film absorbers (Pt, Au)
                \item Wheatstone bridge circuit
                \item Measures total radiation power
                \item Multicamera system: HBC, VBCl, VBCr
                \item 128 channels total
                \item Spatial resolution: $\sim$5 cm at magnetic axis
            \end{itemize}
            }
        \end{column}

        \begin{column}{0.45\textwidth}
            \centering
            \includegraphics[width=0.8\textwidth]{%
                ../THESIS/content/figures/chapter1/component_3d_bw.png}

            \only<2->{
            \textbf{Measurement Principle:}
            \begin{equation*}
                P_M = F_M \cdot \left(\frac{d(\Delta\widetilde{U}_M)}{dt} + f_M \Delta\widetilde{U}_M\right)
            \end{equation*}

            \vspace{1em}
            \textbf{Global Radiation Power:}
            \begin{equation*}
                P_{rad} = \frac{V_{P,tor}}{V_C} \sum_M \frac{P_M V_M}{K_M}
            \end{equation*}
            }
        \end{column}
    \end{columns}

    % TP: Bolometers are the primary diagnostic for measuring total plasma radiation
    % TP: Metal film absorbers heat up from incident radiation, changing resistance in a Wheatstone bridge
    % TP: Three camera arrays (horizontal and two vertical) provide comprehensive coverage of the plasma cross-section
    % TP: Each channel integrates radiation along a line of sight - we extrapolate to get total radiated power
\end{frame}

\begin{frame}
    \frametitle{Bolometer Calibration and Performance}
    \backgroundlogo%

    \begin{columns}
        \begin{column}{0.5\textwidth}
            \only<1->{
            \textbf{Calibration Methods:}
            \begin{itemize}
                \item In-situ laser calibration
                \item Electrical calibration
                \item Etendue calculations: $K_M = \int_M \widetilde{K}_M dA_M$
                \item Volume determination: $V_M$ (LOS cone volume)
            \end{itemize}

            \vspace{1em}
            \textbf{Performance Characteristics:}
            \begin{itemize}
                \item Time resolution: $\sim$1 ms
                \item Sensitivity: $\sim$10 kW/m$^3$
                \item Uncertainty: $\sim$10-15\%
            \end{itemize}
            }
        \end{column}

        \begin{column}{0.45\textwidth}
            \only<3->{
            \begin{block}{Chord Brightness}
                Line-integrated measurement along each detector's line of sight provides radial profile information
            \end{block}

            \vspace{1em}
            \textbf{Applications:}
            \begin{itemize}
                \item Power balance calculations
                \item Radiation profile monitoring
                \item Real-time feedback control
                \item Tomographic reconstruction
            \end{itemize}
            }
        \end{column}
    \end{columns}

    % TP: Accurate calibration is essential - we use laser and electrical methods to determine detector response
    % TP: Etendue describes the geometric sensitivity of each channel to radiation in different plasma regions
    % TP: Fast time resolution enables real-time monitoring and control applications
    % TP: Chord brightness profiles show where radiation is concentrated - core vs edge vs SOL
\end{frame}

\section{Real-Time Radiation Feedback Control}

\begin{frame}
    \frametitle{Feedback System Design}
    \backgroundlogo%

    \begin{columns}
        \begin{column}{0.5\textwidth}
            \only<1->{
            \textbf{System Architecture:}
            \begin{itemize}
                \item NI 6321 data acquisition hardware
                \item LabVIEW control software
                \item PID controller implementation
                \item Thermal gas valve actuation
                \item Minimum latency: 13.6 ms
            \end{itemize}

            \vspace{1em}
            \textbf{PID Controller:}
            \begin{equation*}
                u(t) = K_p e(t) + K_i \int e(t')dt' + K_d \frac{de(t)}{dt}
            \end{equation*}
            }

            \vspace{0.5em}
            \only<3->{
            \centering
            \includegraphics[width=0.9\textwidth]{%
                ../THESIS/content/figures/chapter2/oscilloscope_fix.pdf}
            }
        \end{column}

        \begin{column}{0.45\textwidth}
            \only<2->{
            \textbf{Radiation Prediction Proxies:}

            \vspace{0.5em}
            \textit{Subset-based (3-7 channels):}
            \begin{equation*}
                P_{pred}^{(1)} = \frac{V_{P,tor}}{V_S} \sum_{M}^{S} \frac{P_M V_M}{K_M}
            \end{equation*}

            \vspace{0.5em}
            \textit{Single channel (dimensionless):}
            \begin{equation*}
                P_{pred}^{(2)} \propto \Delta U_M
            \end{equation*}
            }
        \end{column}
    \end{columns}

    % TP: First real-time bolometer feedback system at W7-X, operational during OP1.2b campaign
    % TP: Uses fast data acquisition and PID control to adjust impurity gas injection rates
    % TP: Two prediction methods: subset of channels for full power estimate, or single channel for fast response
    % TP: System latency of 13.6 ms is competitive but still limits performance compared to other diagnostics
    % TP: Benchmark testing with laser diode confirmed minimum latency and system performance
\end{frame}

\begin{frame}
    \frametitle{Experimental Achievements}
    \backgroundlogo%

    \begin{columns}
        \begin{column}{0.5\textwidth}
            \only<1->{
            \textbf{XP20181010.32 - Benchmark Discharge:}
            \begin{itemize}
                \item Stable helium-seeded detachment
                \item Radiation fraction: $f_{rad} > 90\%$
                \item Peak values up to 100\%
                \item Target heat load reduction: factor of 2
                \item C$^{3+}$ detachment at $f_{rad} \sim 50\%$
                \item Validated 3-channel LOS subset
            \end{itemize}
            }

            \vspace{0.5em}
            \only<3->{
            \centering
            \includegraphics[width=0.95\textwidth]{%
                ../THESIS/content/figures/chapter2/overview20181010032.pdf}
            }
        \end{column}
        \begin{column}{0.45\textwidth}
            \only<2->{
            \begin{exampleblock}{Key Result}
                Demonstrated stable, feedback-controlled radiative cooling with $f_{rad} \ge 85\%$ without terminal plasma disruption
            \end{exampleblock}

            \vspace{1em}
            \textbf{Comparison with Other Methods:}
            \begin{itemize}
                \item Electron density feedback
                \item C-III filterscope feedback
                \item Bolometer feedback shows intrinsic connection to detachment physics
            \end{itemize}
            }
        \end{column}
    \end{columns}

    % TP: XP20181010.32 was the breakthrough experiment - achieved stable high-radiation detachment with helium seeding
    % TP: Reduced divertor heat loads by half while maintaining plasma stability
    % TP: Validated that just 3 well-chosen bolometer channels can predict total radiation with >85% accuracy
    % TP: Bolometer feedback directly controls the relevant quantity (radiation) unlike density or spectroscopy proxies
\end{frame}

\begin{frame}
    \frametitle{Feedback Performance Analysis}
    \backgroundlogo%

    \begin{itemize}
        \item \textbf{Gas Injection Dynamics:} Moderately scaled gas puffs (medium length and intensity) most effective for reliable edge cooling
        \item \textbf{System Limitations:} Computational and algorithmic constraints introduce non-negligible latencies
        \item \textbf{Optimization Challenges:} Difficult to optimize during commissioning due to limited experimental time
    \end{itemize}

    \vspace{1em}
    \only<2->{
    \begin{alertblock}{Critical Finding}
        The intrinsic connection of $P_{rad}$ and $f_{rad}$ to the detachment process makes bolometer feedback essential for future fusion reactor applications
    \end{alertblock}
    }

    % TP: Gas injection strategy matters - moderate puffs work best, avoiding plasma perturbations
    % TP: The parameter space is complex - no simple scaling law emerged from experiments
    % TP: System latency remains a challenge but is being addressed through hardware and software upgrades
    % TP: Despite limitations, this is the most direct way to control radiative cooling for reactor-relevant scenarios
\end{frame}

\section{Line of Sight Sensitivity and Modeling}

\begin{frame}
    \frametitle{Impurity Seeding Models}
    \backgroundlogo%

    \begin{columns}
        \begin{column}{0.5\textwidth}
            \only<1->{
            \textbf{Two-Chamber Model:}
            \begin{align*}
                \dot{N}\ix{w}=&\left(N\ix{w,lim}-N\ix{w}\right)\tau\ix{w,p}\\%
                \dot{N}\ix{p}=&\Gamma\ix{s}+N\ix{w}\tau\ix{w,p}-f\ix{w,p}N\ix{p}-N\ix{p}\tau\ix{p}%
            \end{align*}

            \begin{itemize}
                \item Plasma and wall compartments
                \item Particle exchange rates
                \item Pumping and recycling
            \end{itemize}
            }

            \vspace{0.5em}
            \only<3->{
            \centering
            \includegraphics[width=0.8\textwidth]{%
                ../THESIS/content/figures/chapter3/chambers/twochamber_scheme_crop_colors.pdf}
            }
        \end{column}

        \begin{column}{0.45\textwidth}
            \only<2->{
            \textbf{Three-Chamber Model:}
            \begin{align*}
                \dot{N}\ix{w}=&\left(N\ix{w,lim}-N\ix{w}\right)\tau\ix{w,s}\\%
                \dot{N}\ix{s}=&\Gamma\ix{s}-N\ix{s}\left(\tau\ix{s,p}+\tau\ix{s}\right)+g\left(N\ix{P},N\ix{S}\right)-c\ix{0}\\%
                \dot{N}\ix{p}=&N\ix{s}\tau\ix{s,p}-\frac{N\ix{p}N\ix{s}}{N\ix{p,lim}}%
            \end{align*}
            \begin{itemize}
                \item Adds SOL compartment
                \item Better represents edge physics
                \item Both models equally capable of representing feedback measurements
            \end{itemize}
            }

            \vspace{0.5em}
            \only<3->{
            \centering
            \includegraphics[width=0.8\textwidth]{%
                ../THESIS/content/figures/chapter3/chambers/threechamber_scheme_crop_colors.pdf}
            }
        \end{column}
    \end{columns}

    % TP: Developed simple models to understand gas injection dynamics - two and three chamber systems
    % TP: Models capture basic physics but don't provide detailed predictive capability
    % TP: Both models successfully reproduce experimental radiation evolution during gas injection
\end{frame}

\begin{frame}
    \frametitle{Channel Selection Sensitivity}
    \backgroundlogo%

    \begin{columns}
        \begin{column}{0.5\textwidth}
            \only<1->{
            \textbf{LOS Sensitivity Evaluation:}
            \begin{itemize}
                \item Weighted deviation metric
                \item Correlation analysis
                \item Mean deviation assessment
            \end{itemize}

            \vspace{0.5em}
            \textbf{Optimal Channel Selection:}
            \begin{itemize}
                \item No single "best" set exists
                \item Robust selection achieves $\ge$85\% prediction accuracy
                \item Agreement between HBC and VBC cameras
            \end{itemize}
            }
        \end{column}

        \begin{column}{0.45\textwidth}
            \only<2->{
            \vspace{0.5em}
            \begin{block}{Key Finding}
                Detectors viewing separatrix and SOL region most viable for real-time feedback configurations
            \end{block}

            \vspace{1em}
            \centering
            \includegraphics[width=0.95\textwidth]{%
                ../THESIS/content/figures/chapter2/ch_selection_3_13_27_g.pdf}
            }
        \end{column}
    \end{columns}

    % TP: Systematic analysis identified which bolometer channels are most sensitive to relevant radiation scenarios
    % TP: Channels viewing the separatrix and SOL are optimal - this is where detachment radiation occurs
    % TP: Multiple channel combinations can achieve similar prediction accuracy - system is robust
\end{frame}

\begin{frame}
    \frametitle{STRAHL Impurity Transport Modeling}
    \backgroundlogo%

    \begin{columns}
        \begin{column}{0.5\textwidth}
            \only<1->{
            \textbf{STRAHL Code:}
            \begin{itemize}
                \item 1D impurity transport simulation
                \item Models carbon and oxygen radiation
                \item Includes ionization, recombination
                \item Diffusion and convection transport
            \end{itemize}

            \vspace{1em}
            \textbf{Key Results:}
            \begin{itemize}
                \item Carbon dominates SOL emissivity
                \item Reduced diffusivity near separatrix
                \item Inward shift of radiation for $f_{rad} \rightarrow 1$
                \item Radiation moves from outside to inside separatrix
            \end{itemize}
            }
        \end{column}

        \begin{column}{0.45\textwidth}
            \only<2->{
            \textbf{Forward Modeling:}
            \begin{itemize}
                \item Calculate chord brightness from STRAHL profiles
                \item Compare with experimental measurements
                \item Validate transport assumptions
            \end{itemize}

            \vspace{1em}
            \begin{alertblock}{Discrepancy}
                Forward calculations show less asymmetry in chord brightness than experiments, suggesting model limitations in capturing 3D effects
            \end{alertblock}
            }
        \end{column}
    \end{columns}

    % TP: STRAHL is a 1D transport code that models how impurities move and radiate in the plasma
    % TP: Simulations show carbon is the main radiator in the SOL - intrinsic impurity from wall interactions
    % TP: As radiation fraction increases, the emission zone shifts inward toward the separatrix
    % TP: Model captures basic trends but misses some 3D asymmetries seen in experiments - stellarator complexity
\end{frame}

\section{Tomographic Reconstruction}

\begin{frame}
    \frametitle{Minimum Fisher Regularization (MFR)}
    \backgroundlogo%

    \begin{columns}
        \begin{column}{0.5\textwidth}
            \textbf{Ill-Posed Inverse Problem:}
            \begin{itemize}
                \item 128 line-integrated measurements
                \item Reconstruct 2D emissivity: $\sim$4500 pixels
                \item Requires regularization
            \end{itemize}

            \vspace{1em}
            \textbf{Fisher Information:}
            \begin{equation*}
                I_F = \int \frac{1}{g(\vec{r})} \left(\frac{\partial g(\vec{r})}{\partial \vec{r}}\right)^2 d\vec{r}
            \end{equation*}

            \vspace{0.5em}
            \textbf{Iterative Solution:}
            \begin{equation*}
                \vec{x}^{(n+1)} = \left(\mathbf{T}^T\mathbf{T} + \mu\mathbf{H}^{(n)}\right)^{-1} \mathbf{T}^T\vec{b}
            \end{equation*}
        \end{column}

        \begin{column}{0.45\textwidth}
            \textbf{Radially Dependent Anisotropy (RDA):}
            \begin{itemize}
                \item Tailored weighting: $k_{ani}(r)$
                \item $k < 1$: localized structures
                \item $k > 1$: smooth distributions
                \item Separate core and edge parameters
            \end{itemize}

            \vspace{.5em}
            \begin{block}{MFR Advantages}
                \begin{itemize}
                    \item Robust to noisy data
                    \item Smooth solutions
                    \item Incorporates a priori knowledge
                    \item Stable convergence
                \end{itemize}
            \end{block}

            % \vspace{0.5em}
            % \centering
            % \includegraphics[width=0.8\textwidth]{%
            %     ../THESIS/content/figures/chapter4/kAni_profile_comparison.pdf}
        \end{column}
    \end{columns}

    % TP: Tomography inverts line-integrated measurements to get 2D radiation distribution
    % TP: This is ill-posed - far more unknowns than measurements, needs regularization
    % TP: MFR finds the smoothest solution consistent with measurements using Fisher information theory
    % TP: Radially dependent anisotropy allows different smoothness in core vs edge - captures physics better
\end{frame}

\begin{frame}
    \frametitle{Geometry Sensitivity and Benchmarking}
    \backgroundlogo%

    \begin{columns}
        \begin{column}{0.5\textwidth}
            \only<1->{
            \textbf{Geometry Perturbation Tests:}
            \begin{itemize}
                \item Camera position variations
                \item Detector/aperture segmentation (N=2,4,8)
                \item Triangulation vs rectangular splitting
                \item Etendue calculation validation
            \end{itemize}
            }
        \end{column}

        \begin{column}{0.45\textwidth}
            \only<2->{
            \begin{block}{Key Findings:}
                \begin{itemize}
                    \item Significant robustness to geometry variations
                    \item N$\ge$8 segmentation adequate
                    \item Intrinsic bias toward upper SOL/separatrix
                    \item Camera displacement effects non-negligible
                \end{itemize}
            \end{block}
            }
        \end{column}
    \end{columns}

    % TP: Extensive testing of geometry sensitivity - how do uncertainties in camera positions affect results?
    % TP: Algorithm is robust but geometry matters - need accurate as-built measurements
\end{frame}

\begin{frame}
    \frametitle{Geometry Sensitivity and Benchmarking}
    \backgroundlogo%

    \begin{columns}
        \begin{column}{0.5\textwidth}
            \only<1->{
            \textbf{Phantom Image Benchmarks:}
            \begin{itemize}
                \item Simple and complex test profiles
                \item Quality metrics: $\chi^2$, $\rho_c$
                \item Optimal $k_{ani}$ depends on profile
                \item No universal "best" parameters
            \end{itemize}
            }
        \end{column}

        \begin{column}{0.45\textwidth}
            \only<2->{
            \begin{exampleblock}{Recommended Settings}
                $k_{ani} = \{2.0, 0.3\}$ (core, edge)\\
                Grid: 30$\times$150 pixels\\
                Domain: $1.3^2 V_P$
            \end{exampleblock}
            }
        \end{column}
    \end{columns}

    % TP: Phantom benchmarks with known input profiles test reconstruction quality
    % TP: Found optimal parameter ranges but no single "best" set - depends on radiation distribution
\end{frame}

\begin{frame}
    \frametitle{Experimental Tomography Results}
    \backgroundlogo%

    \begin{columns}
        \begin{column}{0.5\textwidth}
            \only<1->{
                \textbf{Statistical Analysis:}
                \begin{itemize}
                    \item Forward vs backward $P_{rad}$ correlation
                    \item Core radiation slightly overestimated
                    \item Total power significantly overestimated
                    \item Consistent with phantom benchmarks
                \end{itemize}
            }
        \end{column}
        \begin{column}{0.45\textwidth}
            \only<2->{
                \textbf{Power Balance Application:}
                \begin{equation*}
                    P_{bal} = P_{ECRH} - P_{rad} - P_{div}
                \end{equation*}
                \begin{itemize}
                    \item $P_{2D}^{core}$ improves balance accuracy
                    \item More stable than $P_{rad}$ alone
                    \item Alternative for quasi-steady-state analysis
                \end{itemize}
            }
        \end{column}
    \end{columns}

    % TP: Applied MFR to real experimental data from feedback-controlled discharges
    % TP: Reconstructions show radiation moving from core to edge as detachment develops
    % TP: Can resolve localized structures like X-point radiation and MARFEs
    % TP: Tomographic power integrals provide alternative to standard extrapolation for power balance
\end{frame}

\begin{frame}
    \frametitle{Tomography Quality Assessment}
    \backgroundlogo%

    \only<1->{
        \begin{itemize}
            \item \textbf{Reconstruction Accuracy:} $P_{rad}$ consistently underestimates total 2D integrated power by amount equal to SOL radiation
            \item \textbf{Quality Metrics:} $\chi^2$ and correlation coefficient $\rho_c$ not always congruent - optimal tomogram may not have $\chi^2 = 1$
            \item \textbf{Anisotropy Optimization:} Ideal $k_{ani}$ set exists for each profile but dimensionality makes finding it impractical
            \item \textbf{Experimental Validation:} Results agree well with phantom benchmarks, supporting algorithm reliability
        \end{itemize}
    }

    \only<2->{
        \vspace{1em}
        \begin{block}{Conclusion}
            MFR with RDA weighting provides robust, reliable 2D radiation reconstructions suitable for physics analysis and power balance calculations, though computational limitations prevent continuous real-time operation
        \end{block}
    }

    % TP: Quality assessment shows algorithm performs as expected from benchmarks
    % TP: Standard extrapolation underestimates total power - tomography reveals SOL contribution
    % TP: Overall, MFR is a validated, reliable tool for understanding radiation distributions at W7-X
\end{frame}

\section{Conclusions and Outlook}

\begin{frame}
    \frametitle{Key Achievements}
    \backgroundlogo%

    \begin{enumerate}
        \item \textbf{Real-Time Feedback System:} Successfully implemented, tested and operated first bolometer radiation feedback at W7-X
        \only<2->{
        \begin{itemize}
            \item Achieved stable detachment with $f_{rad} \ge 85\%$, peaks up to 100\%
            \item Reduced target heat loads by factor of 2
            \item Validated 3-channel LOS subset for fast prediction
        \end{itemize}
        }

        \vspace{0.5em}
        \item \textbf{Line of Sight Optimization:} Systematic sensitivity analysis identified optimal detector configurations
        \only<3->{
        \begin{itemize}
            \item Separatrix and SOL viewing channels most effective
            \item Robust selection achieves $\ge$85\% prediction accuracy
            \item No single "best" set - multiple viable configurations
        \end{itemize}
        }

        \vspace{0.5em}
        \item \textbf{Tomographic Reconstruction:} Developed and benchmarked tailored MFR algorithm with RDA weighting
        \only<4->{
        \begin{itemize}
            \item Robust to geometry perturbations and noise
            \item Successfully applied to experimental data
            \item Provides alternative for power balance calculations
        \end{itemize}
        }
    \end{enumerate}

    % TP: Three major achievements: feedback system, LOS optimization, and tomography algorithm
    % TP: Feedback system demonstrated stable high-radiation scenarios essential for future reactors
    % TP: Optimization work provides guidance for future diagnostic configurations
    % TP: Tomography gives detailed 2D view of radiation distribution for physics understanding
\end{frame}

\begin{frame}
    \frametitle{Scientific Insights}
    \backgroundlogo%

    \begin{columns}
        \begin{column}{0.5\textwidth}
            \only<1->{
            \textbf{Detachment Physics:}
            \begin{itemize}
                \item C$^{3+}$ detachment visible at $f_{rad} \sim 50\%$
                \item Stable detachment achievable at $f_{rad} > 90\%$
                \item Radiation shifts inward as $f_{rad} \rightarrow 1$
                \item X-point and island radiation concentration
            \end{itemize}
            }

            \vspace{1em}
            \only<2->{
            \textbf{Impurity Transport:}
            \begin{itemize}
                \item Carbon dominates SOL emissivity
                \item Reduced diffusivity near separatrix
                \item Complex 3D effects in stellarator geometry
            \end{itemize}
            }
        \end{column}

        \begin{column}{0.45\textwidth}
            \only<3->{
            \textbf{Feedback Control:}
            \begin{itemize}
                \item Moderate gas puffs most effective
                \item No simple scaling law found
                \item Intrinsic connection to detachment essential
                \item System latency limits performance
            \end{itemize}
            }

            \vspace{1em}
            \only<4->{
            \textbf{Diagnostic Performance:}
            \begin{itemize}
                \item Multicamera system provides comprehensive coverage
                \item Geometry uncertainties affect results
                \item Tomography reveals SOL radiation underestimation
            \end{itemize}
            }
        \end{column}
    \end{columns}

    % TP: Work provided new insights into detachment physics at W7-X
    % TP: Carbon from walls is main radiator - important for understanding and predicting behavior
    % TP: Feedback control works but needs optimization - latency and parameter tuning challenges
    % TP: Diagnostic capabilities well-characterized - know strengths and limitations
\end{frame}

\begin{frame}
    \frametitle{Future Outlook and Recommendations}
    \backgroundlogo%

    \begin{columns}
        \begin{column}{0.5\textwidth}
            \only<1->{
            \textbf{System Upgrades:}
            \begin{itemize}
                \item Reduce feedback latency through hardware/software optimization
                \item Implement more complex $P\ix{pred}$ models
                \item Per-experiment configurability
                \item Develop predictive scaling laws
            \end{itemize}
            }

            \vspace{1em}
            \only<2->{
            \textbf{Modeling Improvements:}
            \begin{itemize}
                \item Refine multi-chamber models with experimental data
                \item Integrate with other SOL models
                \item Better parametric optimization
                \item 3D effects in STRAHL modeling
            \end{itemize}
            }
        \end{column}

        \begin{column}{0.45\textwidth}
            \only<3->{
            \textbf{Tomography Extensions:}
            \begin{itemize}
                \item Larger phantom benchmark set
                \item Additional virtual camera arrays
                \item Real-time capable algorithms
                \item Automated parameter optimization
            \end{itemize}
            }

            \vspace{1em}
            \only<4->{
            \textbf{Reactor Relevance:}
            \begin{itemize}
                \item Essential for DEMO power plant operation
                \item Reliable radiative cooling at $f\ix{rad}\ge95\%$
                \item Machine protection and heat load management
                \item Steady-state operation capability
            \end{itemize}
            }
        \end{column}
    \end{columns}

    % TP: Clear path forward for system improvements - hardware and software upgrades
    % TP: Modeling needs refinement to capture complex physics better
    % TP: Tomography can be extended with more benchmarks and faster algorithms
    % TP: This work is directly relevant to future fusion reactors - DEMO will need reliable radiation control
\end{frame}

\begin{frame}
    \frametitle{Broader Impact}
    \backgroundlogo%

    \only<1->{
    \begin{block}{Contribution to Fusion Energy Research}
        This thesis advances the state-of-the-art in plasma diagnostics and control for stellarator fusion devices, providing essential tools and knowledge for achieving reactor-relevant operating scenarios with controlled radiative cooling and stable detachment.
    \end{block}
    }

    \vspace{1em}
    \begin{columns}
        \begin{column}{0.45\textwidth}
            \only<2->{
            \textbf{Technical Contributions:}
            \begin{itemize}
                \item First real-time bolometer feedback at W7-X
                \item Validated diagnostic optimization methods
                \item Benchmarked tomography algorithm
                \item Demonstrated stable high-$f\ix{rad}$ operation
            \end{itemize}
            }
        \end{column}

        \begin{column}{0.45\textwidth}
            \only<3->{
            \textbf{Scientific Contributions:}
            \begin{itemize}
                \item Detachment physics understanding
                \item Impurity transport insights
                \item Radiation distribution characterization
                \item Power balance methodology
            \end{itemize}
            }
        \end{column}
    \end{columns}

    % TP: This work makes significant contributions to both technical capabilities and scientific understanding
    % TP: Demonstrates that stellarators can achieve reactor-relevant radiation control
    % TP: Methods and insights applicable to other fusion devices and future reactors
    % TP: Lays groundwork for continued development toward fusion energy
\end{frame}

\begin{frame}
    \frametitle{Summary}
    \backgroundlogo%

    \begin{itemize}
        \only<1->{%
            \item Developed and operated first real-time bolometer radiation feedback system at W7-X stellarator
            \item Achieved stable, controlled detachment with radiation fractions $f_{rad} \ge 85\%$ and peaks up to 100\%
            \item Reduced divertor target heat loads by factor of 2 through controlled impurity seeding
            \item Validated 3-channel bolometer subset for fast radiation power prediction with $\ge$85\% accuracy
            \item Systematic line of sight sensitivity analysis identified optimal detector configurations for feedback
        }

        \only<2->{
            \item STRAHL modeling revealed carbon dominance in SOL radiation and inward shift at high $f_{rad}$
            \item Developed and benchmarked Minimum Fisher Regularization tomography with radially dependent anisotropy
            \item Demonstrated robust 2D radiation reconstruction from experimental data
            \item Provided alternative power balance methodology using tomographic integrals
            \item Established essential approach for future fusion reactor heat load management
        }
    \end{itemize}

    % TP: In summary, this thesis successfully developed and demonstrated real-time radiation feedback control
    % TP: Achieved reactor-relevant operating scenarios with stable high radiation fractions
    % TP: Provided comprehensive diagnostic characterization and optimization
    % TP: Advanced understanding of detachment physics and radiation distributions in stellarators
\end{frame}

\begin{frame}
    \frametitle{Collaborations}
    \backgroundlogo%

    \begin{block}{Acknowledgement}
        \acknowledgementtext%
    \end{block}

    \vspace{2em}
    \textbf{Collaborators:}
    \collaborators[1.1cm]{%
        {%
            logos/w7x.png,%
            logos/ipp.png,%
            logos/minerva.png,%
            logos/helmholtz_research_crop.png,%
            logos/eurofusion_crop.png%
        }%
    }%
\end{frame}

\begin{frame}[plain]
    \backgroundlogo%
    \begin{center}
        {\Huge Thank You}
    \end{center}

    % TP: Thank the audience and open for questions
    % TP: Be prepared to discuss: feedback system details, tomography algorithm, experimental results, future work
    % TP: Acknowledge collaborators and funding agencies
\end{frame}

\begin{frame}[allowframebreaks]{References}
    \printbibliography

    % TP: Full reference list available for detailed citations
    % TP: Key references include Anton et al. (MFR), Zhang et al. (W7-X bolometry), and experimental campaign papers
\end{frame}



% ============================================================================
% BACKUP SLIDES - APPENDIX CONTENT
% ============================================================================

\begin{frame}
    \backgroundlogo%
    \begin{center}
        \Huge\textbf{Backup Slides}

        \vspace{2em}
        \Large Additional Technical Details
    \end{center}

    % TP: Backup slides contain detailed technical information from appendices
    % TP: Available for questions on specific technical aspects
\end{frame}

% ----------------------------------------------------------------------------
% Measurement Algorithm
% ----------------------------------------------------------------------------

\begin{frame}
    \frametitle{Measurement Algorithm Overview}
    \backgroundlogo%

    \begin{columns}[T]
        \begin{column}{0.48\textwidth}
            \textbf{Algorithm Stages:}
            \begin{enumerate}
                \item \textbf{Initialization} (T1-60s)
                    \begin{itemize}
                        \item Load geometry matrices $K_M$, $V_M$
                        \item Configure channel selection $S$
                        \item Initialize analog output
                    \end{itemize}
                \item \textbf{Calibration} (4s)
                    \begin{itemize}
                        \item Offset measurement
                        \item Two-stage calibration
                        \item Determine $R$, $\tau$, $\kappa$
                    \end{itemize}
                \item \textbf{Measurement}
                    \begin{itemize}
                        \item Continuous voltage integration
                        \item Moving average filtering
                        \item Real-time prediction
                    \end{itemize}
            \end{enumerate}
        \end{column}

        \begin{column}{0.48\textwidth}
            \textbf{Real-time Feedback:}
            \begin{itemize}
                \item Parallel analog output via NI 6321
                \item Two prediction channels $P_{\text{pred}}^{(1)}$, $P_{\text{pred}}^{(2)}$
                \item FIFO buffer management
                \item Sample time $\Delta t$ configurable
            \end{itemize}

            \vspace{1em}
            \textbf{Data Storage:}
            \begin{itemize}
                \item Local HDD backup
                \item W7-X archive upload
                \item Calibration parameters
                \item Full voltage traces
            \end{itemize}
        \end{column}
    \end{columns}

    % TP: Three-stage algorithm: initialization, calibration, measurement
    % TP: Real-time capability through parallel processing and FIFO buffers
    % TP: All data archived for post-shot analysis
\end{frame}

% ----------------------------------------------------------------------------
% Fourier Transform Correlation Method
% ----------------------------------------------------------------------------

\begin{frame}
    \frametitle{Fourier Transform Correlation Metric}
    \backgroundlogo%

    \textbf{Alternative LOS Sensitivity Evaluation:}

    \begin{columns}[T]
        \begin{column}{0.55\textwidth}
            \textbf{Method:}
            \begin{itemize}
                \item Uses spectral correlation between $P_{\text{rad}}$ and $P_{\text{pred}}$
                \item Discrete Fast Fourier Transform (DFFT)
                \item Cross-spectral density analysis
                \item Coherence-based metric
            \end{itemize}

            \vspace{0.5em}
            \textbf{Mathematical Formulation:}
            \begin{align*}
                \gamma_i &= \mathcal{F}(P_{\text{rad}})(\omega) \\
                \gamma_j &= \mathcal{F}(P_{\text{pred}})(\omega) \\
                \varphi(\omega) &= \sqrt{\frac{|g_{i,j}^2|}{|\gamma_i\gamma_j|}}
            \end{align*}

            Quality: $\vartheta = \int \varphi(\omega) d\omega$
        \end{column}

        \begin{column}{0.43\textwidth}
            \textbf{Key Results:}
            \begin{itemize}
                \item Similar patterns to correlation metric
                \item Global maximum at upper inboard X-point
                \item HBC: best at $0.38r_a$
                \item VBC: best at $-0.13r_a$
                \item Spectral focus excludes absolute value contributions
            \end{itemize}

            \vspace{0.5em}
            \textbf{Advantages:}
            \begin{itemize}
                \item Frequency domain analysis
                \item Robust to noise
                \item Identifies oscillatory behavior
            \end{itemize}
        \end{column}
    \end{columns}

    % TP: Fourier transform provides frequency-domain perspective on LOS sensitivity
    % TP: Confirms X-point viewing channels have highest spectral correlation
    % TP: Complements time-domain correlation metrics
\end{frame}

\begin{frame}
    \frametitle{Plasma Parameter Sensitivity Analysis}
    \backgroundlogo%

    \begin{columns}[T]
        \begin{column}{0.48\textwidth}
            \textbf{Correlation with Plasma Parameters:}

            \includegraphics[width=\textwidth]{../THESIS/content/figures/chapter3/training/weighted_deviation_sensitivity_params_ECRH_HBCm.pdf}

            \small Heating power $P_{\text{ECRH}}$ correlation
        \end{column}

        \begin{column}{0.48\textwidth}
            \textbf{Key Findings:}
            \begin{itemize}
                \item Higher $P_{\text{ECRH}}$ ($\ge$3.5 MW) $\rightarrow$ better predictability
                \item Optimal $f_{\text{rad}} > 0.75$
                \item Core $T_e \sim$ 1.3 keV ideal
                \item Gap at $\vartheta \approx$ 0.3-0.5
                \item Both cameras in good agreement
            \end{itemize}

            \vspace{0.5em}
            \textbf{Optimal Discharge Conditions:}
            \begin{itemize}
                \item Medium-high heating power
                \item High radiation fraction
                \item Moderate core temperature
                \item Consistent across LOS combinations
            \end{itemize}
        \end{column}
    \end{columns}

    % TP: Systematic correlation between plasma parameters and prediction quality
    % TP: Identifies optimal experimental configurations for feedback
    % TP: Validates that high f_rad scenarios work well with feedback system
\end{frame}

% ----------------------------------------------------------------------------
% STRAHL Modeling Details
% ----------------------------------------------------------------------------

\begin{frame}
    \frametitle{STRAHL Parameter Variations}
    \backgroundlogo%

    \begin{columns}[T]
        \begin{column}{0.48\textwidth}
            \textbf{Decay Length Variation:}

            \includegraphics[width=\textwidth]{../THESIS/content/figures/chapter3/STRAHL/nete/compare_ne_Te_94_105_edge.pdf}

            \small SOL decay length: $\lambda$ = 2 cm vs 5 cm

            \vspace{0.5em}
            \textbf{Impact:}
            \begin{itemize}
                \item Affects SOL emissivity profiles
                \item Core radiation indirectly influenced
                \item No systematic radial shifts
            \end{itemize}
        \end{column}

        \begin{column}{0.48\textwidth}
            \textbf{Impurity Source Location:}

            \includegraphics[width=\textwidth]{../THESIS/content/figures/chapter3/STRAHL/fract_abund/compare_fract_abund_82_94_edge.pdf}

            \small Source at LCFS vs divertor ($r_a$ + 7.5 cm)

            \vspace{0.5em}
            \textbf{Impact:}
            \begin{itemize}
                \item Changes ion populations in SOL
                \item Similar core emissivity results
                \item Validates model robustness
            \end{itemize}
        \end{column}
    \end{columns}

    % TP: Parameter variations test STRAHL model sensitivity and robustness
    % TP: Decay length primarily affects SOL, source location affects ion distributions
    % TP: Core results relatively stable - important for feedback modeling
\end{frame}

\begin{frame}
    \frametitle{STRAHL: Radiation Fraction Dependence}
    \backgroundlogo%

    \begin{columns}[T]
        \begin{column}{0.48\textwidth}
            \includegraphics[width=\textwidth]{../THESIS/content/figures/chapter3/STRAHL/diag_lines/compare_strahl_rad_93_94_edge.pdf}

            \small Total radiation for $f_{\text{rad}}$ = 90\% and 100\%
        \end{column}

        \begin{column}{0.48\textwidth}
            \textbf{Critical Observations:}
            \begin{itemize}
                \item At $f_{\text{rad}}$ = 90\%: peak in SOL
                \item At $f_{\text{rad}}$ = 100\%: peak shifts to LCFS
                \item C$^{3+}$ and C$^{2+}$ most sensitive
                \item Dramatic profile changes near separatrix
            \end{itemize}

            \vspace{0.5em}
            \textbf{Implications:}
            \begin{itemize}
                \item LOS sensitivity strongly depends on $f_{\text{rad}}$
                \item Detachment scenarios predictable
                \item Validates feedback system design
                \item Critical for controlled detachment
            \end{itemize}
        \end{column}
    \end{columns}

    % TP: Radiation distribution changes dramatically with f_rad
    % TP: 100% case shows peak at separatrix - relevant for detachment control
    % TP: Lower charge states (C2+, C3+) dominate near LCFS at high f_rad
\end{frame}

% ----------------------------------------------------------------------------
% Camera Geometry Perturbations
% ----------------------------------------------------------------------------

\begin{frame}
    \frametitle{Camera Geometry Sensitivity}
    \backgroundlogo%

    \begin{columns}[T]
        \begin{column}{0.48\textwidth}
            \textbf{Unilateral Mirroring:}

            \includegraphics[width=\textwidth]{../THESIS/content/figures/chapter4/MFR/compare_fixedLoS_emissivities3D.pdf}

            \small Etendue change: $<10^{-13}$ mm$^{-3}$

            \vspace{0.5em}
            \begin{itemize}
                \item Half of HBC array mirrored
                \item Minimal impact on forward model
                \item Asymmetry reduced by 0.5-1\%
            \end{itemize}
        \end{column}

        \begin{column}{0.48\textwidth}
            \textbf{Artificial Symmetric Camera:}

            \includegraphics[width=\textwidth]{../THESIS/content/figures/chapter4/MFR/compare_symmetric_emissivities3D.pdf}

            \small Fully symmetric, equidistant array

            \vspace{0.5em}
            \begin{itemize}
                \item Perfectly symmetric geometry
                \item Similar etendue variations
                \item Validates as-built design
                \item Confirms near-ideal geometry
            \end{itemize}
        \end{column}
    \end{columns}

    % TP: Small geometry perturbations have minimal impact on measurements
    % TP: As-built HBC geometry is very close to ideal symmetric design
    % TP: Validates that manufacturing tolerances are acceptable
\end{frame}

% ----------------------------------------------------------------------------
% MFR: RDA vs RGS
% ----------------------------------------------------------------------------

\begin{frame}
    \frametitle{Tomography: RDA vs RGS Comparison}
    \backgroundlogo%

    \begin{columns}[T]
        \begin{column}{0.48\textwidth}
            \textbf{Radially Dependent Anisotropy (RDA):}
            \begin{itemize}
                \item Standard MFR with $k_{\text{ani}}$ profile
                \item Smooth reconstructions
                \item Good radial profile matching
                \item Established method
            \end{itemize}

            \vspace{0.5em}
            \textbf{Relative Gradient Smoothing (RGS):}
            \begin{itemize}
                \item Weight: $\mathbf{W}_{\text{RGS}} = (1/g_i)^2$
                \item Can be combined with RDA
                \item Stronger feature localization
                \item Higher local intensities
            \end{itemize}
        \end{column}

        \begin{column}{0.48\textwidth}
            \textbf{Comparison Results:}
            \begin{itemize}
                \item Similar integrated power values
                \item RGS: sharper features
                \item RGS: better peak localization
                \item RGS: may lose weak structures
                \item RDA: more robust for general use
            \end{itemize}

            \vspace{0.5em}
            \textbf{Conclusion:}
            \begin{itemize}
                \item RDA sufficient for most cases
                \item RGS useful for specific scenarios
                \item Combined approach possible
                \item Choice depends on physics goals
            \end{itemize}
        \end{column}
    \end{columns}

    % TP: Two regularization approaches compared systematically
    % TP: RDA chosen as primary method - good balance of accuracy and robustness
    % TP: RGS can enhance specific features but may sacrifice overall accuracy
\end{frame}

% ----------------------------------------------------------------------------
% Additional Phantom Tests
% ----------------------------------------------------------------------------

\begin{frame}
    \frametitle{Phantom Tomography: Edge Cases}
    \backgroundlogo%

    \begin{columns}[T]
        \begin{column}{0.48\textwidth}
            \textbf{Inverted Anisotropy:}

            \includegraphics[width=0.9\textwidth]{../THESIS/content/figures/chapter4/MFR/new/phantom/2D/phantom_v_tomo2D_symR0.8_fsR1.1_mx1.0e+06_aniM4_3.0_20.0_nT15_nW2_nigs1_times0.11.png}

            \small Bright SOL ring + core islands

            \begin{itemize}
                \item More challenging reconstruction
                \item SOL dominates measurement
                \item Core features partially obscured
            \end{itemize}
        \end{column}

        \begin{column}{0.48\textwidth}
            \textbf{Homogeneous Distribution:}

            \includegraphics[width=0.9\textwidth]{../THESIS/content/figures/chapter4/MFR/new/phantom/2D/phantom_v_tomo2D_blind_test_ones_mx1.0e+06_aniM3_1.0_1.0_nigs1_times0.11.png}

            \small Uniform 1 MW/m$^3$ everywhere

            \begin{itemize}
                \item Tests baseline reconstruction
                \item Validates algorithm behavior
                \item Confirms no artificial structures
            \end{itemize}
        \end{column}
    \end{columns}

    \vspace{0.5em}
    \textbf{Lessons:} Camera sensitivity in SOL can obscure core features; algorithm performs well for uniform distributions; inverted scenarios more challenging than standard core-peaked profiles.

    % TP: Edge case phantoms test algorithm limits and robustness
    % TP: Inverted case shows SOL brightness can obscure core details
    % TP: Homogeneous test confirms no artificial pattern generation
\end{frame}

% ----------------------------------------------------------------------------
% Summary of Appendix Content
% ----------------------------------------------------------------------------

\begin{frame}
    \frametitle{Appendix Summary}
    \backgroundlogo%

    \begin{columns}[T]
        \begin{column}{0.48\textwidth}
            \textbf{Measurement Algorithm:}
            \begin{itemize}
                \item Three-stage process validated
                \item Real-time capability confirmed
                \item Robust calibration procedure
                \item Full data archiving
            \end{itemize}

            \vspace{0.5em}
            \textbf{LOS Sensitivity Methods:}
            \begin{itemize}
                \item Fourier correlation complements time-domain
                \item Plasma parameter correlations identified
                \item Optimal discharge conditions determined
            \end{itemize}
        \end{column}

        \begin{column}{0.48\textwidth}
            \textbf{STRAHL Validation:}
            \begin{itemize}
                \item Parameter variations tested
                \item Model robustness confirmed
                \item Critical $f_{\text{rad}}$ dependence shown
            \end{itemize}

            \vspace{0.5em}
            \textbf{Tomography Extensions:}
            \begin{itemize}
                \item Geometry perturbations minimal
                \item RDA vs RGS compared
                \item Edge case phantoms tested
                \item Algorithm limits characterized
            \end{itemize}
        \end{column}
    \end{columns}

    \vspace{1em}
    \begin{center}
        \textbf{All appendix content validates and extends main thesis results}
    \end{center}

    % TP: Appendices provide crucial technical validation
    % TP: All methods thoroughly tested and characterized
    % TP: Results support main conclusions and demonstrate robustness
\end{frame}

\end{document}
