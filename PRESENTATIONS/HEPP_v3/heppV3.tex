\documentclass{beamer}

% use predefined IPP style
% options: St,noSt : with/without stellarator theory logo
% options: Eurofusion, noEurofusion : with/without Eurofusion logo
\useoutertheme[Eurofusion, w7x]{ippw7x}

% don't show navigation symbols
\beamertemplatenavigationsymbolsempty
% show navigation symbols
%\setbeamertemplate{navigation symbols}[only frame symbol]{}

% extra packages
\usepackage[english]{babel}
\usepackage{amsmath,amsfonts,amssymb,stackrel}
\usepackage{changepage}
\usepackage{tikz}
\usepackage{array}

\newcommand{\backgroundlogo}{%
    \tikz[overlay,remember picture]{%
    \node[at=(current page.west)] (source) {};%
    \node[opacity = 0.02] {%
    \includegraphics[height=1.\paperheight]%
        {figures/header/minerva}%
    }%
  }
}

\usepackage{units}
\usepackage{siunitx}
\usepackage{enumitem}
\usepackage{commath}
\usepackage{color}
\usepackage{media9}

\usepackage{caption}
\usepackage{subcaption}
\captionsetup{labelformat=empty,labelsep=none}

\usepackage[%
    style=authortitle,%
    % style=verbose,%
    % autocite=footnote,%
    maxbibnames=15,%
    maxcitenames=15,%
    % babel=hyphen,%
    % hyperref=true,%
    % abbreviate=false,%
    backend=bibtex%
    % mcite%
    % labelyear
    ]{biblatex}
\renewcommand*{\bibfont}{\footnotesize}

\renewcommand{\arraystretch}{1.5}%

\newcommand{\backupend}{\setcounter{framenumber}{\value{finalframe}}}
\newcommand{\backupbegin}{\newcounter{finalframe}%
  \setcounter{finalframe}{\value{framenumber}}}

\newcommand{\diff}{\text{d}}
\newcommand{\tenpo}[1]{\cdot 10^{#1}}
\newcommand{\ix}[1]{_\text{#1}}
\newcommand{\imag}{\mathbf{i}}
\newcommand{\fett}[1]{\textbf{#1}}
\newcommand{\tilt}[1]{\textit{#1}}
\newcommand\inlineeqno{\stepcounter{equation}\ \quad\quad(\theequation)}

\newenvironment{variableblock}[3]{%
    \setbeamercolor{block body}{#2}
    \setbeamercolor{block title}{#3}
    \begin{block}{#1}}{\end{block}}

\begin{document}

% title of the presentation
% short title will be shown in the footer
\title[HEPP progress talk]{%
    Real time feedback on plasma radiation at W7-X}

% authors of the presentation
% lecturer (and maybe place and date) will be shown in the footer
\author[P.Hacker]{%
    P.Hacker\inst{1, 2}, F.Reimold\inst{1}, %
    D.Zhang\inst{1}}  % , R.Burhenn\inst{1}, T.Klinger\inst{1}}

% institutes of the authors
\institute[MPI for Plasmaphysics Greifswald]{%
    \inst{1}%
    Max-Planck-Institute for Plasmaphysics, %
    Wendelsteinstr. 1, Greifswald, Germany \and%
    \inst{2}%
    University of Greifswald, Rubenowstr. 6, Greifswald, Germany}

% set date of the talk
\date{25th of Sept., 2020}


% first frame
\begin{frame}
    % show title of talk and authors
    \titlepage%
    \backgroundlogo%

    % show Logos Helmholtz, Max-Planck and EUROfusion
    \begin{minipage}[]{0.35\textwidth}
        \includegraphics[height=6ex]%
        {figures/header/2017_H_Logo_CMYK_untereinander_EN}
    \end{minipage}
    \hfill
    \begin{minipage}[]{0.2\textwidth}
        \begin{center}
            \includegraphics[height=4ex]{figures/header/minerva}
        \end{center}
    \end{minipage}
    \hfill
    \begin{minipage}[]{0.35\textwidth}
        \begin{flushright}
            \includegraphics[height=5ex]%
            {figures/header/EUROfusion-LOGO-PANTONE_REFL_BLUE}
        \end{flushright}
    \end{minipage}

    % show acknoledgement from EUROfusion
    \acknowledgement
\end{frame}

\begin{frame}{Contents}
    % display all sections
    \backgroundlogo%
    \tableofcontents%
\end{frame}%

\section{Bolometer Diagnostic}%
\begin{frame}{Bolometer}%

    \begin{block}{What is the Bolometer at W7-X?}%
        \vspace*{0.25cm}
        -- two camera system using metal film resistive %
        detector arrays to measure plasma radiation %
        based on thermal effects%
    
        %\flushleft{\color{ipp}\underline{Goals:}}%
        \begin{itemize}%
            \item[$\Rightarrow$]{%
                measure global \& local power balance, %
                investigation of radiation power loss, %
                mainly from impurities, and its distribution}%
        \end{itemize}%
    \end{block}%

    \begin{columns}%
        \column{0.4\textwidth}%
        \centering{%
            \color{ipp}%
            \textbf{\underline{HBCm}}%
        }%
        \column{0.4\textwidth}%
        \centering{%
            \color{ipp}%
            \textbf{\underline{VBCr/VBCl}}%
        }%
    \end{columns}%
    \begin{figure}%
        \centering{%
            \includegraphics[width=.8\textwidth]%
            {figures/content/linesofsight_in_vessel}%
            \caption{\tiny{%
                    (Lines of sight with individual apertures, %
                    retracted into the vacuum vessel, D.Zhang~et~al.)}%
            }
        }%
    \end{figure}%

\end{frame}%

\section{Calculations}
\begin{frame}{Calculations}

    \begin{block}{Plasma Radiation}
        \only<1-3>{%
            \begin{align}
                P_{rad}\propto\sum_{Z} %
                n_{e}\cdot n_{Z}\cdot L_{Z}\nonumber%
            \end{align}
        }%

        \only<2-3>{%
            \vspace*{-0.5cm}
            \begin{align}
                P_{rad,bolo}\,\,=\frac{\color{red}{V_{P,tor}}}%
                {\color{ipp}{V_{cam}}}\cdot%
                \sum_{ch}\frac{P_{ch}\cdot\color{ipp}{V_{ch}}}%
                {K_{ch}}\nonumber%
            \end{align}
        }%

        \only<3>{%
            \vspace*{-0.5cm}
            \flushleft{%
                \color{ippdark}\underline{%
                    Bolometer equation:}}%
            \begin{align}
                P_{ch}=F_{ch}\cdot\left(\Delta U+f_{\tau,ch}%
                \frac{\diff(\Delta U)}{\diff t}\right)\nonumber
            \end{align}
        }%

        % \only<4-5>{%
        %     \flushleft{%
        %         \color{ippdark}\underline{%
        %             Bolometer equation:}}%
        %     \begin{align}
        %         P_{ch} & =F_{ch}\cdot\left(\tau_{ch}    %
        %         \frac{\diff(\Delta U)}{\diff t}+%
        %         f_{\tau}\cdot(\Delta U)\right)\nonumber \\%
        %                & \approx                        %
        %         25.737\cdot\frac{\text{W}}{\text{V}}\left(%
        %         \diff(\Delta U_{ch})+0.014\cdot\Delta %
        %         U_{ch}\right)\nonumber%
        %     \end{align}
        % }

    \end{block}

    \only<1>{%
        \begin{itemize}
            \item[$>$]{%
                  $L_{Z}$: total radiation coefficient, %
                  incl. line \& continuum radiation %
                  of impurity $Z$\linebreak%
                  \dots $=f\left(T_{e},\,\,T_{i},\,\,T_{Z},\,\,wall%
                      \,\,material/conditions,\,\,\dots\right)$}%
        \end{itemize}
    }%

    % \only<2>{%
    %     \begin{itemize}%
    %         \item[$>$]{%
    %               $V_{ch}$: line-of-sight (\tilt{LoS}) %
    %               volume of detector $ch$}%
    %         \item[$>$]{%
    %               $V_{P, tor}$: estimated radiation volume where %
    %               emissivity is \tilt{not} negligible}%
    %         \item[$>$]{%
    %               $V_{cam}=\sum_{ch}V_{ch}$}%
    %         \item[$>$]{%
    %               $K_{ch}$: $\propto L_{ch}$ LoS length, %
    %               geometrical factor for channel $ch$}%
    %     \end{itemize}
    % }
    
    \only<2>{%
        \begin{figure}%
            \centering%
            \includegraphics[width=.6\textwidth]%
                {figures/content/hbcm_63_to_153_3d.pdf}%
            \caption{\tiny{%
                (LoS of HBCm in 90$^\circ$ subsection of magnetic %
                standard configuration fluxsurfaces; P.~Hacker~et~al)}}%
        \end{figure}%
    }

    \only<3>{%
        \begin{itemize}%
            \item[$>$]{%
                  $\Delta U\propto\Delta T$ the change in absorber %
                  temperature}%
            \item[$>$]{%
                $f_{\tau,ch}$ cooling factor of detector}
            % \item[$>$]{%
            %       $F_{ch}=sensitivity\left(\text{detector %
            %       properties~\dots}\right)$}
            % \item[$>$]{%
            %       $f_{\tau}=correction\left(\text{cable length, %
            %       impedance, heat capacity~\dots}\right)$}
            % \item[$>$]{%
            %       $\tau_{ch}=$ cooling time of absorber}
        \end{itemize}
    }

    % \only<4>{%
    %     \begin{itemize}
    %         \item[$>$]{%
    %               channel \& cabel resistances %
    %               $R_{ch}\approx\text{\SI{1}{\kilo\ohm}}$, %
    %               $R_{cab}\approx\text{\SI{40}{\ohm}}$}%
    %         \item[$>$]{%
    %               cooling/relaxation time of the gold foil %
    %               $\tau_{ch}\approx\text{\SI{110}{\milli\second}}$}%
    %         \item[$>$]{%
    %               heat capacity $\kappa_{ch}\approx%
    %                   \text{\SI{0.8}{\milli\watt/\kilo\ohm}}$}%
    %         \item[$>$]{%
    %               scaling $f_{\tau}\approx1$}%
    %         \item[$>$]{%
    %               temporal sampling in measurement %
    %               \SI{1.6}{\milli\second}}%
    %     \end{itemize}
    % }

    % \only<5>{%
    %     \vspace*{0.5cm}
    %     \centering{%
    %         \color{ipp}{\textbf{%
    %                 $\Rightarrow\,\,$%
    %                 for non-collapsing plasma scenarios: %
    %                 $\Delta U\approx\text{\SI{e-3}{\volt}}$\\%
    %                 signal derivative for sampling time %
    %                 \SI{0.8}{\milli\second}: %
    %                 $\diff(\Delta U)\approx\text{\SI{e-5}{\volt}}$}}}%
    % }%

\end{frame}%

\section{Feedback during OP1.2b}%
\begin{frame}{Feedback: Goals}%

    \begin{variableblock}{Goal}{%
            bg=example text.fg!20!bg,fg=black}{% body
            bg=example text.fg, fg=white}% title
        \centering%
        Providing feedback signal for \tilt{fast feedback gas valve} %
        diagnostic through fast (\SI{5}{\milli\second}) calculation of %
        $P_{rad}$ estimate
    \end{variableblock}%

    \begin{block}{Investigations}
        \begin{itemize}
            \item[1.-]{%
                adjust divertor heat loads, investigate radiation %
                regimes, dedicated detachment}
            \item[2.-]{find relationship between gas puff and %
                radiation power loss for both intrinsic and extrinsic %
                impurities}
        \end{itemize}
    \end{block}

\end{frame}

\begin{frame}{Feedback: Achievement}

    \begin{minipage}{0.48\textwidth}%
        \begin{block}{Prediction}%
            \begin{itemize}%
                \item[$>$]{%
                    fast radiation power proxy for subset $S$ %
                    of channel selection:}%
            \end{itemize}%

            \vspace*{-0.3cm}
            $$P_{pred}=f\cdot%
                \frac{V_{P,tor}}{V_{S}}\sum_{S}%
                \frac{V_{ch}\cdot P_{ch}}{K_{ch}}$$%
        \end{block}%
    
    \end{minipage}\hfill%
    \begin{minipage}{0.48\textwidth}%
                
        % \only<1>{%
        %     \begin{figure}%
        %         \includegraphics[width=.9\textwidth]%
        %             {figures/content/fsez_lofs2d.pdf}%
        %         \caption{\tiny{%
        %             \centering(VBCr/l LoS subset $S$; P.Hacker~et~al.)}}%
        %     \end{figure}%
        % }%
        
        \begin{figure}%
            \includegraphics[width=.9\textwidth]%
                {figures/content/overview20181010_032_prad_fb_qsq_div}%
            \caption{\tiny{\centering%
                \centering(Overview plot, $P_{rad}$, $P_{pred}$, valve %
                actuation and integrated divertor target heat load; %
                P.Hacker~et~al.)}}%
        \end{figure}%

    \end{minipage}%
\end{frame}%

\section{Sensitivity}%
\begin{frame}{Feedback: Sensitivity}%

    \begin{variableblock}{Problem}{%
            bg=alerted text.fg!20!bg, fg=black}{% body
            bg=alerted text.fg, fg=white}% title
        \centering%
        What subset of channels and/or cameras S yields the best %
        prediction and feedback performance?
    \end{variableblock}%

    \begin{block}{Response}%
        \begin{itemize}%
            % \item[$>$]{%
            %     due to time constraints during experimental campaign no %
            %     prior testing of channel selection for feedback possible}%
            \item[$>$]{%
                subset $S$ of 3 - 7 channels with LoS both tangential to %
                \tilt{scrape-off layer} (SOL) and through core}%
            \item[$>$]{%
                no more than a selection of 9 possible %
                $\Rightarrow$ calculation too slow}%
        \end{itemize}%
    \end{block}%
    
    \begin{columns}%
        \column{0.3\textwidth}%
        \centering{%
            \begin{figure}%
                \includegraphics[width=1.\textwidth]%
                    {figures/content/ch_selection_1_17_30_g}%
            \end{figure}}%
        \column{0.3\textwidth}%
        \centering{%
            \begin{figure}%
                \includegraphics[width=1.\textwidth]%
                    {figures/content/ch_selection_3_13_27_g}%
            \end{figure}}%
        \column{0.3\textwidth}%
        \centering{%
            \begin{figure}%
                \includegraphics[width=1.\textwidth]%
                    {figures/content/ch_selection_5_15_25_r}%
            \end{figure}}%
    \end{columns}%
    \centering{\tiny{%
        Examples of LoS selections $S$ from HBCm %
        (P.Hacker~et~al.)}}%

\end{frame}%

\begin{frame}{Feedback: Sensitivity}%
    \begin{block}{Prediction Quality Metric}%
            
        \only<1>{%
            \begin{itemize}%
                \item[$>$]{%
                    prediction from set $S$ using Bolometer equation %
                    $P_{pred,S}$ to define a weighted, normalised %
                    cost function:}%
            \end{itemize}%
        }%

        \centering%
        \begin{tabular}{cc}

            \only<1>{%
                \underline{deviation:} & %
                $d_{S}=\norm{P_{rad,cam}(t)-P_{pred,S}(t)}$\\%
            }%

            \only<1-2>{%
                \underline{weighted:} & %
                $\varepsilon_{S}(t)=%
                    \left\{\begin{array}{ll}%
                    1-\frac{d_{S}(t)}{P_{rad}(t)}&,%
                        \,\,d_{S}(t)<P_{rad}(t)\\%
                    0&,\text{ else}
                    \end{array}\right\}$\\%
                \underline{quality:} & %
                $\vartheta_{S}=\int_T\varepsilon_{S}(t)\diff t$%
            }%

        \end{tabular}%

        \only<1>{%
            \vspace*{0.3cm}%
            \begin{itemize}%
                \item[$>$]{%
                    e.g. $\varepsilon_{HBC}(t)=1$ and consequently %
                    $\vartheta_{HBC}=1$}%
            \end{itemize}%
        }%

    \end{block}%

    \only<2>{%
        \begin{figure}%
            \includegraphics[width=.65\textwidth]%
                {figures/content/wghtd_dev_C[_5_16_27]}%
            \caption{\tiny{%
                (XP: 20181010.032, $P_{rad}$, %
                $P_{pred}$ and cost function for one subset of %
                channels with $n=3$; P.Hacker~et~al.)}}%
        \end{figure}%
    }%

\end{frame}%

\begin{frame}{Feedback: Statistics}%

    \begin{block}{Statistics: Example}%

        \only<1-2>{%
            \begin{itemize}
                \item[$>$]{%
                    pre-selecting combinatory space for subsets, %
                    i.e. only LoS from SOL and core regions}%
                \item[$>$]{%
                    $N_{n}$ the total number of tested channel subsets %
                    $S_{n}$ with prediction consisting of $n$ individual %
                    LoS}%
            \end{itemize}%
        }%

        \only<2>{%
            \begin{itemize}
                \item[$>$]{
                    be $N^{n}_{ch}$ the amount of subsets $S_{n}$ that %
                    include channel/LoS $ch$}%
            \end{itemize}
        }%

        \only<4>{%
            \begin{itemize}
                \item[$>$]{%
                    base quality of channel combinations %
                    $\approx60$\% from educated pre-selection}%
                \item[$>$]{%
                    total sensitivity at minimum 60\% %
                    for individual contributions from LoS to predictions}%
            \end{itemize}
        }%

        \only<5>{%
            \begin{itemize}
                \item[$>$]{%
                    maximum sensitivity around SOL or %
                    \tilt{last closed fluxsurface}}%
                \item[$>$]{%
                    combination of \tilt{best} channels in one %
                    subset can achieve up to %
                    $\vartheta_{S}\approx0.95$ without %
                    any further proportionality factors}%
            \end{itemize}
        }%

        \only<2-3>{%
            \quad\quad\quad%
            \underline{total sensitivity:}%
            \begin{align}%
                \Omega_{ch}^{n}=%
                \frac{1}{N^{n}_{ch}}\sum_{S_{ch}}^{N}%
                \vartheta_{HBC,S^{ch}}\nonumber%
            \end{align}%
        }%

    \end{block}%

    \only<1>{%
        \begin{figure}%
            \includegraphics[width=.7\textwidth]%
                {figures/content/wghtd_dev_spectrum_C3HBCm}%
            \caption{\tiny{%
                (XP: 20181010.032, $P_{rad}$, %
                $P_{pred}$ and cost function for one subset of %
                channels with $n=3$; P.Hacker~et~al.)}}%
        \end{figure}%
    }%

    \only<3-5>{%
        \begin{columns}
            \column{0.33\textwidth}%
            \begin{figure}%
                \includegraphics[width=1.\textwidth]%
                    {figures/content/ch_selection_comb_23}%
                \caption{\tiny{%
                    (Selections $S$ including \# 23)}}%
            \end{figure}%

            \column{0.65\textwidth}%
            \begin{figure}%
                \includegraphics[width=1.\textwidth]%
                    {figures/content/spectrum_analysis_weighted_deviation}%
                \caption{\tiny{%
                    (XP: 20181010.032, $\Omega_{ch}^{3}$; P.Hacker~et~al.)}}%
            \end{figure}%
        \end{columns}
    }

\end{frame}%

\section{STRAHL Simulation}%
\begin{frame}{Simulation of Radiation}%

    \begin{block}{Consequences: Radiation Fraction}%
        \only<1>{%
            \begin{itemize}
                \item[$>$]{%
                    feedback controlled experiment reaches target of %
                    $f_{rad}=P_{rad}/P_{ECRH}\approx1.0$ %
                    when seeded by fast feedback gas valve}%
                \item[$>$]{%
                    increasing radiation power loss until cycling between %
                    $f_{rad}\sim0.9$ \& $1.0$}%
            \end{itemize}
        }%

        \only<2>{%
            \begin{itemize}
                \item[$>$]{%
                    line-integrated chordal profile shows majority of %
                    radiation coming from region close to separatrix or SOL}%
                \item[$>$]{%
                    increasing radiation fraction shows inward shift of %
                    brightness away from last closed fluxsurface %
                    $\Rightarrow$ \color{red}{Why?}}%
            \end{itemize}%
        }%
    \end{block}%

    \only<1>{%
        \begin{figure}%
            \includegraphics[width=.9\textwidth]%
                {figures/content/radiational_fraction_%
                 20181010_032_edge_top}%
            \caption{\tiny{%
                (XP: 20181010.032; P.Hacker~et~al.)}}%
        \end{figure}%
    }%
    
    \only<2>{%
        \begin{figure}%
            \includegraphics[width=.8\textwidth]%
                {figures/content/chordal_profile_HBCm_20181010_032}%
            \caption{\tiny{%
                (XP: 20181010.032, chordal brightness %
                (i.e.~power density) of HBCm at fix radiation %
                fractions; P.Hacker~et~al.)}}%
        \end{figure}%
    }%

\end{frame}%

\begin{frame}{Simulation of Radiation}%

    \begin{block}{Intrinsic Impurities: STRAHL Results}%

        \only<1>{%
            \begin{itemize}%
                \item[$>$]{%
                    suspecting intrinsic impurity $C$ %
                    ionisation to be very sensitive to separatrix %
                    electron temperature \& density}%
                \item[$>$]{%
                    calculating radial transport $\Gamma_{i,Z}$ %
                    and emission of impurity $i$ and ion-stage %
                    $Z$ solving continuity equation %
                    using ansatz of anomalous diffusivities %
                    $D^{*}$ and radial drift velocities $v^{*}$:}%
            \end{itemize}

            \vspace*{-0.3cm}%
            \begin{align}%
                \frac{\partial n_{i,Z}}{\partial t}=%
                    &-\nabla\,\Gamma_{i,Z}+Q_{i,Z}\nonumber\\%
                =&\frac{1}{r}\frac{\partial}{\partial r}r\left(%
                    D^{*}\frac{\partial n_{i,Z}}{\partial r}-%
                    v^{*}n_{i,Z}\right)+Q_{i,Z}\nonumber%
            \end{align}%
        }%

        \only<2>{%
            \begin{itemize}%
                \item[$>$]{%
                    radial line radiation profiles for $C^{X+}$ ions in %
                    coronal equilibrium for $f_{rad}=0.9$ and $1.0$ %
                    experience inward shift}%
            \end{itemize}
        }%

    \end{block}

    \only<2>{%
        \vspace*{.25cm}%
        \begin{minipage}{0.49\textwidth}%
            \begin{figure}%
                \includegraphics[width=1.\textwidth]%
                    {figures/content/%
                     compare_ne_Te_93_94_edge}
            \end{figure}%
        
        \end{minipage}%
        %\hfill%
        \begin{minipage}{0.49\textwidth}%

            \begin{figure}%
                \includegraphics[width=1.\textwidth]%
                    {figures/content/%
                     compare_strahl_rad_93_94_edge_bottom}
            \end{figure}%
        \end{minipage}%
        \\\centering{\tiny{%
            (Input profiles and ionization radiation for %
            $C^{X+}$ calculated by STRAHL for %
            $f_{rad}\in\left[0.9,\,1.0\right]$; P.Hacker~et~al.)}}%
    }%

\end{frame}%

\section{Tomography}%
\begin{frame}{Tomography}

    \begin{block}{Method: Mimimum Fischer Regularization}

        \only<1-2>{%
            \centering%
            \begin{tabular}{cc}
        
                \only<1-2>{%
                    \underline{general problem:} & %
                    $T\ast g=f_{m}$\\%
                    \underline{regularization:} & %
                    $\underset{g}{\text{argmin}}%
                        \left[\left(T\ast g-f_{m}\right)^{2}%
                    +\lambda\mathcal{R}\right]$\\%
                }%

                \only<2>{%
                    & \\%
                    \underline{Fisher regularization:} & %
                    $\underset{g}{\text{solve}}%
                        \left(T^{\intercal}T+\lambda%
                        \mathcal{H}^{(n)}\right)\ast g^{(n+1)}=%
                        T^{\intercal}\ast f_{m}$\\%
                    & %
                    $\mathcal{H}^{(n)}=\nabla^{\intercal}_{x}\ast%
                    W^{(n)}\ast\nabla_{x}+\nabla^{\intercal}_{y}\ast%
                    W^{(n)}\ast\nabla_{y}$%
                }%
            \end{tabular}%
        }%

    \end{block}

    \only<1>{%
        \begin{itemize}%
            \item[$>$]{%
                  $T$: LoS geometry information on %
                  inversion grid}%
            \item[$>$]{%
                  $f_{m}$: line intgrated power for each channel}%
            \item[$>$]{%
                  $g$: matrix containing radiation %
                  per channel \& pixel}%
            \item[$>$]{%
                $\lambda\in{\rm I\!R}$: regularization parameter}%
            \item[$>$]{%
                $\mathcal{R}$: regularization functional, %
                constraint to solution}
        \end{itemize}%
    }%
    
    \only<2>{%
        \begin{itemize}%
            \item[$>$]{%
                  $(\cdot)^{(n)}$: iterative process to find \tilt{argmin}}%
            \item[$>$]{%
                  $W^{(n)}$: \tilt{gradient based} weighting matrix}%
        \end{itemize}%
    }%

\end{frame}

\begin{frame}{Tomography: STRAHL Results}

    \begin{columns}%
        \column{0.6\textwidth}%

            \begin{figure}%
                \only<1>{%
                    \centering%
                    \color{ipp}{\textbf{\underline{$f_{rad}\sim 0.33$}}}\\%
                    \includegraphics[width=1.\textwidth]%
                        {figures/content/STRAHL/%
                         phantom_v_tomo2D_strahl_ID00091_%
                         aniM3_2_2_nigs1_times011}\\%
                }%
                \only<2>{%
                    \centering%
                    \color{ipp}{\textbf{\underline{$f_{rad}\sim 0.66$}}}\\%
                    \includegraphics[width=1.\textwidth]%
                        {figures/content/STRAHL/%
                         phantom_v_tomo2D_strahl_ID00092_%
                         aniM3_2_2_nigs1_times011}\\%
                }%
                \only<3>{%
                    \centering%
                    \color{ipp}{\textbf{\underline{$f_{rad}\sim 0.9$}}}\\%
                    \includegraphics[width=1.\textwidth]%
                        {figures/content/STRAHL/%
                         phantom_v_tomo2D_strahl_ID00093_%
                         aniM3_2_2_nigs1_times011}\\%
                }%
                \only<4>{%
                    \centering%
                    \color{ipp}{\textbf{\underline{$f_{rad}\sim 1.0$}}}\\%
                    \includegraphics[width=1.\textwidth]%
                        {figures/content/STRAHL/%
                         phantom_v_tomo2D_strahl_ID00094_%
                         aniM3_2_2_nigs1_times011}\\%
                }%
            \end{figure}%

        \column{0.45\textwidth}%

            \vspace*{-0.4cm}
            \begin{figure}%
                \only<1-4>{%
                    \centering%
                    \includegraphics[width=1.\textwidth]%
                        {figures/content/STRAHL/rad/new/%
                         phantom_v_tomo_profiles_strahl_ID00091_%
                         aniM3_2_2_nigs1_times011}%
                    \\%
                }%
                \only<2-4>{%
                    \centering%
                    \includegraphics[width=1.\textwidth]%
                        {figures/content/STRAHL/rad/new/%
                         phantom_v_tomo_profiles_strahl_ID00092_%
                         aniM3_2_2_nigs1_times011}%
                    \\%
                }%
                \only<3-4>{%
                    \centering%
                    \includegraphics[width=1.\textwidth]%
                        {figures/content/STRAHL/rad/new/%
                         phantom_v_tomo_profiles_strahl_ID00093_%
                         aniM3_2_2_nigs1_times011}%
                    \\%
                }%
                \only<4>{%
                    \centering%
                    \includegraphics[width=1.\textwidth]%
                        {figures/content/STRAHL/rad/new/%
                         phantom_v_tomo_profiles_strahl_ID00094_%
                         aniM3_2_2_nigs1_times011}%
                }%
            \end{figure}%

        \end{columns}

\end{frame}%

\begin{frame}{Tomography: Experiment 20181010.032}

    \begin{columns}
        \column{0.6\textwidth}%
            \begin{figure}%
                \only<1>{%
                    \centering%
                    \color{ipp}{\textbf{\underline{$f_{rad}\sim 0.33$}}}\\%
                    \vspace{0.3cm}%
                    \includegraphics[width=1.\textwidth]%
                        {figures/content/20181010_032/%
                         tomo_20181010_032aniM4__2_035__nT13_nW2_%
                         reduced_nigs1_sN8_30x20x150_135_065_mfr1D}%
                }%
                \only<2>{%
                    \centering%
                    \color{ipp}{\textbf{\underline{$f_{rad}\sim 0.66$}}}\\%
                    \vspace{0.3cm}%
                    \includegraphics[width=1.\textwidth]%
                        {figures/content/20181010_032/%
                         tomo_20181010_032aniM4__2_035__nT13_nW2_%
                         reduced_nigs1_sN8_30x20x150_135_165_mfr1D}%
                }%
                \only<3>{%
                    \centering%
                    \color{ipp}{\textbf{\underline{$f_{rad}\sim 0.9$}}}\\%
                    \vspace{0.3cm}%
                    \includegraphics[width=1.\textwidth]%
                        {figures/content/20181010_032/%
                         tomo_20181010_032aniM4__2_035__nT13_nW2_%
                         reduced_nigs1_sN8_30x20x150_135_245_mfr1D}%
                }%
                \only<4>{%
                    \centering%
                    \color{ipp}{\textbf{\underline{$f_{rad}\sim 1.0$}}}\\%
                    \vspace{0.3cm}%
                    \includegraphics[width=1.\textwidth]%
                        {figures/content/20181010_032/%
                         tomo_20181010_032aniM4__2_035__nT13_nW2_%
                         reduced_nigs1_sN8_30x20x150_135_305_mfr1D}%
                }%
            \end{figure}%

        \column{0.45\textwidth}%
        
            \vspace*{-0.4cm}
            \begin{figure}%
                \only<1-4>{%
                    \centering%
                    \includegraphics[width=.85\textwidth]%
                        {figures/content/20181010_032/rad/new/%
                         tomo_20181010_032aniM4_2_035_nT13_nW2_%
                         reduced_nigs1_sN8_30x20x150_135_065_%
                         mfr1D_XP_radial_profile}
                }%
                \only<2-4>{%
                    \centering%
                    \includegraphics[width=.85\textwidth]%
                        {figures/content/20181010_032/rad/new/%
                         tomo_20181010_032aniM4_2_035_nT13_nW2_%
                         reduced_nigs1_sN8_30x20x150_135_165_%
                         mfr1D_XP_radial_profile}\\%
                }%
                \only<3-4>{%
                    \centering%
                    \includegraphics[width=.85\textwidth]%
                        {figures/content/20181010_032/rad/new/%
                         tomo_20181010_032aniM4_2_035_nT13_nW2_%
                         reduced_nigs1_sN8_30x20x150_135_245_%
                         mfr1D_XP_radial_profile}\\%
                }%
                \only<4>{%
                    \centering%
                    \includegraphics[width=.85\textwidth]%
                        {figures/content/20181010_032/rad/new/%
                         tomo_20181010_032aniM4_2_035_nT13_nW2_%
                         reduced_nigs1_sN8_30x20x150_135_305_%
                         mfr1D_XP_radial_profile}\\%
                }%
            \end{figure}%
        \end{columns}%

\end{frame}%

\begin{frame}{Conclusions}%

    \only<1>{%
        \begin{variableblock}{Sensitivity Study}{%
                fg=black, bg=orange!30!white}{%
                fg=black, bg=orange!50!white}%
            \begin{itemize}%
                \item[$>$]{%
                    benchmarks on different %
                    experiments, cost metrics and camera/channels %
                    subsets (up to $n=9$) show similar results}
                \item[$>$]{%
                    Bolometer most sensitive to changes %
                    in radiation distribution along separatrix %
                    and SOL}%
            \end{itemize}%
        \end{variableblock}%
    }%

    \only<1>{%
        \begin{variableblock}{Simulation and Inversion}{%
                fg=black, bg=orange!30!white}{%
                fg=black, bg=orange!50!white}%
            \begin{itemize}%
                % \item[$>$]{%
                %     sensitivity analysis in STRAHL input %
                %     parameters other than separatrix temperature %
                %     and densitiy yields small changes in $P_{rad}$ profile}%
                \item[$>$]{%
                    STRAHL shows strong radial dependence of %
                    intrinsic impurity radiation regarding %
                    separatrix electron temperature and density}%
                \item[$>$]{%
                    $C^{X+}$ radiation close to seperatrix %
                    indicator for regimes of detachment}%
                \item[$>$]{%
                    need 2D inversion of radiation distribution %
                    to validate results and to understand %
                    profiles}%
            \end{itemize}%
        \end{variableblock}
    }%

\end{frame}%

\begin{frame}
    \backgroundlogo%
    \centering%
    \color{ipp}{%
        Thank you for your attention!}%
\end{frame}

\begin{frame}%
    \backgroundlogo%
\end{frame}

% BACKUP
\appendix
\backupbegin

\begin{frame}{Bibliography}
    \backgroundlogo%

    \only<1>{\footnotesize%
        \begin{thebibliography}{}%
            \bibitem{Zhang2010} "Design Criteria of the Bolometer diagnostic %
            for steady-state operation of the W7-X %
            stellarator"; %
            Zhang, D. et al.; %
            Review of Scientific Instruments, %
            Jan 1st, 2010; DOI:10.1063/1.3483194
            \bibitem{Zhang2016} "The bolometer diagnostic at stellarator %
            Wendelstein 7-X and its first results in the %
            initial campaign"; %
            D. Zhang, et al. %
            and the W7-X Team; Stellarator-New 2017
            \bibitem{Mast1991} "A low noise highly integrated bolometer %
            array %
            for absolute measurement of VUV and soft x %
            radiation"; %
            K. F. Mast et. al; %
            Review of Scientific Instruments 62, 744 %
            (1991);
            DOI: 10.1063/1.11.42078%
            \bibitem{VMEC} "Steepest descent moment method for three %
            dimensional magnetohydrodynamic equilibria"; %
            Hirshman, S.P. et al.; %
            Physics of Fluids 26, 3553, (1983); %
            DOI: 10.1063/1.864116%
            \bibitem{Wesson} "Tokamaks"; Wesson, J.; Clarendon Press, Oxford; %
            1987%
        \end{thebibliography}%
    }

    \only<2>{\footnotesize%
        \begin{thebibliography}{}%
            \bibitem{Feng} "Numerical investigation of plasma edge %
            transport and limiter heat fluxes in %
            Wendelstein 7-X startup plasmas with %
            EMC3-EIRENE"; %
            Effenberg, F., Feng, Y. et al. %
            Nucl. Fusion 57 (2017) 036021 (15pp); %
            DOI: 10.1088/1741-4326/aa4f83%
            \bibitem{Gianone2002} "Derivation of bolometer equations %
            relevant to %
            operation in fusion experiments"; %
            Gianone, L. et al.; %
            Review of Scientific Instruments; %
            20th of November, 2002; %
            DOI: 10.1063/1.1498906%
            \bibitem{Zhang2018} "Results of the bolometer diagnostic at %
            OP 1.a of W7-X"; %
            internal review of the physics plan during %
            the %
            second operational phase at the stellarator %
            W7-X; 28.02.20i18%
            \bibitem{Yamada2005} "Characterization of energy confinement in %
            net-current free plasmas using the %
            extended International Stellarator %
            Database"; %
            H. Yamada et al.; %
            INSTITUTE OF PHYSICS PUBLISHING and %
            INTERNATIONAL ATOMIC ENERGY AGENCY; %
            Nucl. Fusion 45 (2005) 1684¿1693%
            \bibitem{FIFOSource} ``Introduction to the Queue Data Structure %
            – Array Implementation''; %
            URL: https://www.thecodingdelight.com/%
            queue-data-structure-array-implementation/
        \end{thebibliography}%
    }

\end{frame}%

\begin{frame}{}%
    \backgroundlogo%
\end{frame}%

\backupend

\end{document}
