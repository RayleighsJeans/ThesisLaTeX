% presentation template
% works only with pdflatex

\documentclass{beamer}

% use predefined IPP style
% options: St,noSt : with/without stellarator theory logo
% options: Eurofusion, noEurofusion : with/without Eurofusion logo
\useoutertheme[Eurofusion, w7x]{ippw7x}

% don't show navigation symbols
\beamertemplatenavigationsymbolsempty
% show navigation symbols
%\setbeamertemplate{navigation symbols}[only frame symbol]{}

% extra packages
\usepackage[english]{babel}
\usepackage{amsmath,amsfonts,amssymb,stackrel}
\usepackage{changepage}
\usepackage{tikz}
\usetikzlibrary{positioning}

\usepackage{units}
\usepackage{siunitx}
\usepackage{xcolor}

\usepackage{caption}
\usepackage{subcaption}
\captionsetup{labelformat=empty,labelsep=none}

\usepackage[%
      style=authortitle,%
%     style=verbose,%
%     autocite=footnote,%
      maxbibnames=15,%
      maxcitenames=15,%
%     babel=hyphen,%
%     hyperref=true,%
%     abbreviate=false,%
      backend=bibtex%
%     mcite%
%     labelyear
]{biblatex}
\renewcommand*{\bibfont}{\footnotesize}

\newcommand{\diff}{\text{d}}
\newcommand{\tenpo}[1]{\cdot 10^{#1}}
\newcommand{\ix}[1]{_\text{#1}}
\newcommand{\imag}{\mathbf{i}}
\newcommand{\fett}[1]{\textbf{#1}}
\newcommand{\tilt}[1]{\textit{#1}}
\newcommand\inlineeqno{\stepcounter{equation}\ \quad\quad(\theequation)}

\newcommand{\backupend}{\setcounter{framenumber}{\value{finalframe}}}
\newcommand{\backupbegin}{\newcounter{finalframe}%
  \setcounter{finalframe}{\value{framenumber}}}

\newcommand{\backgroundlogo}{%
  \tikz[overlay,remember picture]{%
  \node[at=(current page.west)] (source) {};%
  \node[opacity = 0.04, %
        below left= -0.4\paperheight and -0.6\paperheight of source] {%
    \includegraphics[height=0.85\paperheight]%
      {figures/minerva_gruen_ganz_ohne_hintergrund.png}%
    }%
  }
}

\begin{document}
  % title of the presentation
  % short title will be shown in the footer
  \title[The Bolometer diagnostic at W7-X]%
        {The bolometer diagnostic at the stellarator Wendelstein 7-X}

  % authors of the presentation
  % lecturer (and maybe place and date) will be shown in the footer
  \author[P. Hacker; May 14th, 2018]%
         {P. Hacker\inst{1,}\inst{2},\and D. Zhang\inst{1},\and%
          R. Burhenn\inst{1}}

  % institutes of the authors
  \institute{\tiny%
             \inst{1}Max Planck Institute for Plasma Physics, %
                     Wendelsteinstr. 1, D-17491 Greifswald, Germany \and%
             \inst{2}Ernst-Moritz-Arndt University Greifswald, Domstr. 11, %
                     D-17489 Greifswald, Germany}

  % set date of the talk
  \date{May 14th, 2018}

  % first frame 
  \begin{frame}
    \backgroundlogo%
    % show title of talk and authors
    \titlepage%
    % show acknoledgement from EUROfusion
    \acknowledgement%
  \end{frame}

  % new frame
  \begin{frame}{Contents}
    % display all sections
    \only<1>{%
      \tableofcontents%
      \backgroundlogo%
    }%
  \end{frame}%

  \section{Bolometer}
  % new frame
  \begin{frame}
    % TMP
    \only<1>{
      \frametitle{Bolometer}%
      \backgroundlogo%
      \tableofcontents[currentsection]
    }
    % GOALS
    \only<2>{%
      \frametitle{Goals}%
      \flushleft{\color{ipp}{What is the bolometer at W7-X?}}\linebreak%
        -- a two camera system using metal film resistive detector arrays %
        to measure plasma radiation based on thermal effects%
      \vspace*{1.0cm}
      \flushleft{\color{ipp}\underline{Goals:}}
      \begin{itemize}%
        \item[1. -]{globale power balance: %
                    investigation of total radiation power loss, mainly from %
                    impurities, and its distribution}%
        \item[2. -]{local power balance: %
                    radiation profiles for transport studies (using tomographic inversion)}%
      \end{itemize}%
    }%
  \end{frame}

  \section{Construction}
  % new frame
  \begin{frame}
    % TMP
    \only<1>{%
      \backgroundlogo%
      \tableofcontents[currentsection]
    }
    % PERFORMANCE
    \only<2-4>{%
      \frametitle{Construction}%
      \begin{block}{}
        \begin{itemize}%
          \item{multi-camera system with 113 channels combined:\linebreak%
                32@HBCm and 2 x 24@VBC subdetectors (+ differently %
                coated/filtered channels)}
        \end{itemize}%
      \end{block}
      \vspace*{0.2cm}%
    }
    \only<2>{%
      \begin{columns}%
        \column{0.4\textwidth}%
          \centering{%
            \color{ipp}%
            \textbf{\underline{HBCm}}%
          }%
        \column{0.4\textwidth}%
          \centering{%
            \color{ipp}%
            \textbf{\underline{VBCr/VBCl}}%
          }%
      \end{columns}%
      \begin{figure}%
        \centering{%
          \includegraphics[width=1.0\textwidth]%
           {figures/content/linesofsight_in_vessel.pdf}%
          \caption{\footnotesize%
                   (Lines of sight with individual apertures, retracted into %
                    the vacuum vessel)}%
        }%
      \end{figure}%
    }%
    \only<3>{%
      \begin{figure}%
        \centering{%
        \includegraphics[width=0.9\textwidth]%
          {figures/content/torus_full_banana.pdf}%
        \caption{\footnotesize%
                 (W7-X equilibrium fluxsurfaces from VMEC)}%
        }%
      \end{figure}
    }%
    \only<4>{%
      \begin{figure}%
        \centering{%
        \includegraphics[width=0.7\textwidth]%
          {figures/content/component_3d.pdf}%
        \caption{\footnotesize%
                 (Camera head (VBCl/r) construction)}%
        }%
      \end{figure}
    }%
    \only<5-6>{%
      \frametitle{Performance}%
      \begin{block}{}
        \begin{itemize}
          \item{spatial resolution of \SI{5}{\centi\meter} %
                \& temporal resolution of \SIrange{0.8}{1.6}{\milli\second}}%
                %spectral range between $\SIrange{600}{0.2}{\nano\meter}}%
                %\pm$ \SI{3}{\micro\second}}%
          \item{detectors of \SI{5}{\micro\meter} thick gold-foil absorbers %
                with carbon coating on \SI{1.5}{\micro\meter} Si$_{3}$N$_{4}$ %
                substrate and gold meander for \SI{200}{\nano\watt} %
                resolution between \SIrange{600}{0.2}{\nano\meter}}
        \end{itemize}
      \end{block}
    }
    \only<5>{%
      \begin{figure}%
        \centering{%
        \includegraphics[width=0.6\textwidth]%
          {figures/content/goldmeander_new.pdf}%
        \caption{\footnotesize%
                 (Single detector channel scheme with holder)}%
        }%
      \end{figure}
    }
    \only<6>{%
      \begin{figure}%
        \centering{%
        \includegraphics[width=0.6\textwidth]%
          {figures/content/wheatstone_mitschema.pdf}%
        \caption{\footnotesize%
                 ([Left] Detector array of one head with reference %
                  [Right] Wheatstone bridge, radiated (green) and un-exposed %
                  (yellow) parts)}%
        }%
      \end{figure}
    }
  \end{frame}%

  \section{Calibration \& Calculation}
  % new frame
  \begin{frame}{Calibration \& Calculation}
    % TMP
    \only<1>{%
      \backgroundlogo
      \tableofcontents[currentsection]
    }%
    \only<2>{%
      \frametitle{Design criteria}
      \begin{block}{}
        \begin{itemize}
          \item{water cooled elements (front plates) for \SI{30}{\minute} steady state operation}
          \item{ECRH stray radiation ($\sim$ %
          \SI{10}{\kilo\watt\per\square\meter}) screening by wire-mesh and %
          absorption by ceramic coating inside the camera enclosure}
        \end{itemize}
      \end{block}
      \begin{columns}
        \column{0.5\textwidth}
          \begin{figure}%
            \centering{%
            \includegraphics[width=\textwidth]%
              {figures/content/coating_results.pdf}%
            \caption{\footnotesize%
                     (Bolometer prototype in strong microwave %
                     background of MISTRAL)}%
            }
          \end{figure}
        \column{0.5\textwidth}%
          \vspace*{-1.0cm}
          \begin{block}{}
            \centering%
            \color{ipp}%
            $\ll$ 1\% microwave power flux\linebreak%
            \&\linebreak%
            53\% optical transmissivity
          \end{block}
      \end{columns}
    }
    \only<3-6>{%
      \frametitle{Equations}
      \begin{block}{Plasma radiation}
        \only<3-6>{%
          \begin{align}
            P_{rad,bolo}\propto\sum_{Z} n_{e}\cdot n_{Z}\cdot L_{Z}\nonumber%
          \end{align}
        }%
        \only<4-6>{%
          \vspace*{-0.5cm}
          \begin{align}
            \dots\,\,=\frac{V_{P,tor}}{V_{cam}}\cdot%
            \sum_{ch}\frac{V_{ch}}{K_{ch}}\cdot\frac{P_{ch}}{f_{OT}}%
            \nonumber
          \end{align}
        }%
        \only<5-6>{%
          \vspace*{-0.5cm}
          \flushleft{%
            \color{ippdark}\underline{%
              Bolometer equation:}}%
          \begin{align}
            P_{ch}=\frac{2}{U_{eff}}\cdot%
                   \left(R_{ch}+2R_{C}\right)\cdot\kappa_{ch}%
                   \sqrt{g_{C}}\cdot\left(\tau_{ch}%
                   \frac{\diff(\Delta U)}{\diff t}+%
                   f_{\tau}\cdot(\Delta U)\right)%
                   \nonumber
          \end{align}
        }%
      \end{block}
        \only<3>{%
          \begin{itemize}
            \item{$L_{Z}$: line radiation function of impurity $Z$\linebreak%
            \dots $=f\left(T_{e},\,\,T_{i},\,\,T_{Z},\,\,wall\,\,material/%
                     conditions,\,\,\dots\right)$}%
          \end{itemize}
        }%
        \only<4>{%
          \begin{itemize}%
            \item{$V_{ch}$: polygon volume of detector $ch$}%
            \item{$V_{P, tor}$: estimated plasma volume (EMC3 simulation)}%
            \item{$K_{ch}$: geometrical factor for channel $ch$}%
          \end{itemize}
        }
        \only<5>{%
          \begin{itemize}%
            \item{$\Delta U\propto\Delta T$ the change in absorber %
                  temperature}%
            \item{$U_{eff}$: voltage divider between foil and reference}%
          \end{itemize}
        }
        \only<6>{%
          \begin{itemize}%
            \item{$\tau,\,\,\kappa,\,\,R$: cooling time, heat capacity and %
                  resistance of foil}%
            \item{$R_{C},\,\,C_{cab}$: electrical connector properties}%
            \item{$f_{bridge}$: Wheatstone bridge scaling factor}%
          \end{itemize}
        }
    }
  \end{frame}

  \section{Preliminary results (OP1.2a)}
  % new frame
  \begin{frame}{Preliminary results (OP1.2a)}
    % TMP
    \only<1>{%
      \tableofcontents[currentsection]%
    }%
    \only<2-7>{%
      \frametitle{steady state vs. collapse}
    }
    \only<2>{
      \centering{\color{ipp}%
        Channel signals (steady state)}%
      \vspace*{0.15cm}
      \begin{columns}%
        \column{0.4\textwidth}%
          \centering{%
            \color{ipp}%
            \textbf{\underline{voltage}}%
          }%
        \column{0.4\textwidth}%
          \centering{%
            \color{ipp}%
            \textbf{\underline{power}}%
          }%
      \end{columns}%
      \vspace*{0.25cm}
      \includegraphics[width=1.0\textwidth]%
        {figures/content/innerchannel_steady_vbc_20171114_52_combined.pdf}%
      \vspace*{0.25cm}
      \linebreak%
      \footnotesize{Discharge XP.20171114.52: Selection of channels from %
        the VBCl (innermost part), both raw voltage and %
        calculated power $P_{ch}$}%
    }
    \only<3>{%
      \centering{\color{ipp}%
        Total radiated power (steady state)}%
      \vspace*{0.25cm}
      \includegraphics[width=0.55\textwidth]%
        {figures/content/prad_handvbc20171114052.pdf}%
      \vspace*{0.25cm}
      \linebreak%
      \footnotesize{Discharge XP.20171114.52: Full radiated power of both %
        cameras/camera arrays according to previous equations.}%
    }
    \only<4>{%
      \centering{\color{ipp}%
        Discharge overview (steady state)}\linebreak%
      \hspace*{-0.8cm}%
      \includegraphics[width=1.15\textwidth]%
        {figures/content/overview20171114052.pdf}%
      \vspace*{0.25cm}
      \linebreak%
      \footnotesize{Discharge XP.20171114.52: Experiment parameters for the %
        same discharge (2.7s, terminated).}%
    }
    \only<5>{%
      \centering{\color{ipp}%
        Discharge overview (radiative collapse)}\linebreak%
      \hspace*{-0.8cm}%
      \includegraphics[width=1.15\textwidth]%
        {figures/content/overview20171121036.pdf}%
      \vspace*{0.25cm}
      \linebreak%
      \footnotesize{Discharge XP.20171121.36: Discharge properties. Similar %
        setup as for previous discharge, but with frozen $H_{2}$ pellet %
        injection. Collapse around 2.9s.}%
    }
    \only<6>{%
      \centering{\color{ipp}%
        Radiation profile over effective radius (steady state)}\linebreak%
      \vspace*{0.25cm}%
      \includegraphics[width=1.\textwidth]%
        {figures/content/reffs20171114052.pdf}%
      \vspace*{0.25cm}
      \linebreak%
      \footnotesize{Discharge XP.20171114.52: Channel power vs. effetive plasma radius, e.g. $\rho{eff}=\sqrt{\Psi}$ the distance from the %
      magnetic field center, both cameras.}%
    }
    \only<7>{%
      \centering{\color{ipp}%
        Radiation profile over effective radius (collapse)}\linebreak%
      \vspace*{0.25cm}%
      \includegraphics[width=1.\textwidth]%
        {figures/content/reffs20171121036.pdf}%
      \vspace*{0.25cm}
      \linebreak%
      \footnotesize{Discharge XP.20171121.26: Channel power vs. effetive plasma radius with radiative collapse.}%
    }
    \only<8>{
      \frametitle{Outlook}
      \flushleft{%
        \color{ipp}\underline{%
          Goals}}%
      \begin{itemize}
        \item[1.-]{Verifying and improve the evaluation of measurement data}%
        \item[2.-]{instantaneous/direct tomographic inversion after %
          discharge\linebreak%
          \color{ipp}{$\Rightarrow$} investigation of impurity %
          transport processes and discharge feedback}%
        \item[3.-]{providing feedback signal for other diagnostics/%
          CoDaC through fast (\SI{5}{\milli\second}) calculation of $P_{rad}$ %
          estimate}%
      \end{itemize}
    }
    \only<9>{%
      \backgroundlogo%
      \frametitle{End}
      \begin{center}%
        \large%
        Thank you for your attention!\linebreak%
        Now: questions.%
      \end{center}
    }
  \end{frame}

  \begin{frame}{Bibliography}
    \tikz[overlay,remember picture]{%
    \node[at=(current page.west)] (source) {};%
    \node[opacity = 0.04, %
          below left= -0.4\paperheight and -0.6\paperheight of source] {%
      \includegraphics[height=0.85\paperheight]%
        {figures/minerva_gruen_ganz_ohne_hintergrund.png}%
      }%
    }%
    % BIBLIOGRAPHY
    \only<1>{\footnotesize%
      \begin{thebibliography}{}%
        \bibitem{Zhang2010} "Design Criteria of the Bolometer diagnostic %
                            for steady-state operation of the W7-X %
                            stellarator"; %
                            Zhang, D. et al.; %
                            Review of Scientific Instruments, %
                            Jan 1st, 2010; DOI:10.1063/1.3483194
        \bibitem{Zhang2016} "The bolometer diagnostic at stellarator %
                            Wendelstein 7-X and its first results in the %
                            initial campaign"; %
                            D. Zhang, et al. %
                            and the W7-X Team; Stellarator-New 2017
        \bibitem{Mast1991} "A low noise highly integrated bolometer array %
                            for absolute measurement of VUV and soft x %
                            radiation"; %
                            K. F. Mast et. al; %
                            Review of Scientific Instruments 62, 744 (1991);
                            DOI: 10.1063/1.11.42078%
             \bibitem{VMEC} "Steepest descent moment method for three %
                            dimensional magnetohydrodynamic equilibria"; %
                            Hirshman, S.P. et al.; %
                            Physics of Fluids 26, 3553, (1983); %
                            DOI: 10.1063/1.864116%
           \bibitem{Wesson} "Tokamaks"; %
                            Wesson, J.; %
                            Clarendon Press, Oxford; %
                            1987%
      \end{thebibliography}%
    }
    \only<2>{\footnotesize%
      \begin{thebibliography}{}%
             \bibitem{Feng} "Numerical investigation of plasma edge %
                            transport and limiter heat fluxes in %
                            Wendelstein 7-X startup plasmas with %
                            EMC3-EIRENE"; %
                            Effenberg, F., Feng, Y. et al. %
                            Nucl. Fusion 57 (2017) 036021 (15pp); %
                            DOI: 10.1088/1741-4326/aa4f83%
      \bibitem{Gianone2002} "Derivation of bolometer equations relevant to %
                            operation in fusion experiments"; %
                            Gianone, L. et al.; %
                            Review of Scientific Instruments; %
                            20th of November, 2002; %
                            DOI: 10.1063/1.1498906%
        \bibitem{Zhang2018} "Results of the bolometer diagnostic at %
                            OP 1.a of W7-X"; %
                            internal review of the physics plan during the %
                            second operational phase at the stellarator %
                            W7-X; 28.02.20i18%
       \bibitem{Yamada2005} "Characterization of energy confinement in %
                            net-current free plasmas using the %
                            extended International Stellarator Database"; %
                            H. Yamada et al.; %
                            INSTITUTE OF PHYSICS PUBLISHING and %
                            INTERNATIONAL ATOMIC ENERGY AGENCY; %
                            Nucl. Fusion 45 (2005) 1684¿1693%
      \end{thebibliography}%
    }
  \end{frame}%

  % EMPTY FRAME
  \begin{frame}{}
    \backgroundlogo
  \end{frame}

  % BACKUP
  \appendix
  \backupbegin

  \begin{frame}{}
    \begin{figure}
      \centering
      \includegraphics[width=0.75\textwidth]%
        {figures/content/backup/20171207_049_stddiv.pdf}
    \end{figure}
  \end{frame}

  \begin{frame}{}
    \begin{figure}
      \centering
      \includegraphics[width=0.75\textwidth]%
        {figures/content/backup/20171207_049_linoffs.pdf}
    \end{figure}
  \end{frame}

  \begin{frame}{}
    \begin{figure}
      \centering
      \includegraphics[width=0.75\textwidth]%
        {figures/content/backup/fs_EIM_beta000_3d.pdf}
    \end{figure}
  \end{frame}

  \begin{frame}{}
    \begin{figure}
      \centering
      \includegraphics[width=0.75\textwidth]%
        {figures/content/backup/ez_lofs2d.pdf}
    \end{figure}
  \end{frame}

  \begin{frame}{}
    \begin{figure}
      \centering
      \includegraphics[width=0.75\textwidth]%
        {figures/content/backup/ez_lofs3d.pdf}
    \end{figure}
  \end{frame}

  \begin{frame}{}
    \begin{figure}
      \centering
      \includegraphics[width=0.75\textwidth]%
        {figures/content/backup/fsez_lofs2d.pdf}
    \end{figure}
  \end{frame}

  \begin{frame}{}
    \begin{figure}
      \centering
      \includegraphics[width=0.75\textwidth]%
        {figures/content/backup/fsez_lofs3d.pdf}
    \end{figure}
  \end{frame}

  \begin{frame}{}
    \begin{figure}
      \centering
      \includegraphics[width=0.75\textwidth]%
        {figures/content/backup/mesh2dEIM_beta000_5000_full.pdf}
    \end{figure}
  \end{frame}

  \begin{frame}{}
    \begin{figure}
      \centering
      \includegraphics[width=0.75\textwidth]%
        {figures/content/backup/lines_of_sight_mesh.pdf}
    \end{figure}
  \end{frame}

  \begin{frame}{}
    \begin{figure}
      \centering
      \includegraphics[width=0.75\textwidth]%
        {figures/content/backup/lines_of_sight_mesh_cart.pdf}
    \end{figure}
  \end{frame}

  \begin{frame}{}
    \begin{figure}
      \centering
      \includegraphics[width=0.75\textwidth]%
        {figures/content/backup/crosspoints_HBC.pdf}
    \end{figure}
  \end{frame}

  \begin{frame}{}
    \begin{figure}
      \centering
      \includegraphics[width=0.75\textwidth]%
        {figures/content/backup/crosspoints_HBC_cartesian.pdf}
    \end{figure}
  \end{frame}

  % BACKUPEND
  \backupend

\end{document}