\documentclass{beamer}

% use predefined IPP style
% options: St,noSt : with/without stellarator theory logo
% options: Eurofusion, noEurofusion : with/without Eurofusion logo
\useoutertheme[Eurofusion, w7x]{ippw7x}

% don't show navigation symbols
\beamertemplatenavigationsymbolsempty
% show navigation symbols
%\setbeamertemplate{navigation symbols}[only frame symbol]{}

% extra packages
\usepackage[english]{babel}
\usepackage{amsmath,amsfonts,amssymb,stackrel}
\usepackage{changepage}
\usepackage{tikz}
\usepackage{array}

\newcommand{\backgroundlogo}{%
    \tikz[overlay,remember picture]{%
    \node[at=(current page.west)] (source) {};%
    \node[opacity = 0.02] {%
    \includegraphics[height=1.\paperheight]%
        {figures/header/minerva}%
    }%
  }
}

\usepackage{units}
\usepackage{siunitx}
\usepackage{enumitem}
\usepackage{xcolor}

\usepackage{caption}
\usepackage{subcaption}
\captionsetup{labelformat=empty,labelsep=none}

\usepackage[%
    style=authortitle,%
    % style=verbose,%
    % autocite=footnote,%
    maxbibnames=15,%
    maxcitenames=15,%
    % babel=hyphen,%
    % hyperref=true,%
    % abbreviate=false,%
    backend=bibtex%
    % mcite%
    % labelyear
    ]{biblatex}
\renewcommand*{\bibfont}{\footnotesize}

\newcommand{\backupend}{\setcounter{framenumber}{\value{finalframe}}}
\newcommand{\backupbegin}{\newcounter{finalframe}%
  \setcounter{finalframe}{\value{framenumber}}}

\newcommand{\diff}{\text{d}}
\newcommand{\tenpo}[1]{\cdot 10^{#1}}
\newcommand{\ix}[1]{_\text{#1}}
\newcommand{\imag}{\mathbf{i}}
\newcommand{\fett}[1]{\textbf{#1}}
\newcommand{\tilt}[1]{\textit{#1}}
\newcommand\inlineeqno{\stepcounter{equation}\ \quad\quad(\theequation)}

\newenvironment{variableblock}[3]{%
    \setbeamercolor{block body}{#2}
    \setbeamercolor{block title}{#3}
    \begin{block}{#1}}{\end{block}}

\begin{document}

% title of the presentation
% short title will be shown in the footer
\title[HEPP progress talk]{%
    Real time feedback on plasma radiation at W7-X}

% authors of the presentation
% lecturer (and maybe place and date) will be shown in the footer
\author[P.Hacker]{%
    P.Hacker\inst{1, 2}, F.Reimold\inst{1},%
    D.Zhang\inst{1}, R.Burhenn\inst{1}, T.Klinger\inst{1}}

% institutes of the authors
\institute[MPI for Plasmaphysics Greifswald]{%
    \inst{1}%
        Max-Planck-Institute for Plasmaphysics, %
        Wendelsteinstr. 1, Greifswald, Germany \and%
    \inst{2}%
        University of Greifswald, Rubenowstr. 6, Greifswald, Germany}

% set date of the talk
\date{20th of May, 2019}


    % first frame
    \begin{frame}
        % show title of talk and authors
        \titlepage

        % show Logos Helmholtz, Max-Planck and EUROfusion
        \begin{minipage}[]{0.35\textwidth}
            \includegraphics[height=6ex]%
                {figures/header/2017_H_Logo_CMYK_untereinander_EN}
        \end{minipage}
            \hfill
        \begin{minipage}[]{0.2\textwidth}
            \begin{center}
                \includegraphics[height=4ex]{figures/header/minerva}
            \end{center}
        \end{minipage}
        \hfill
        \begin{minipage}[]{0.35\textwidth}
            \begin{flushright}
                \includegraphics[height=5ex]%
                    {figures/header/EUROfusion-LOGO-PANTONE_REFL_BLUE}
            \end{flushright}
        \end{minipage}

        % show acknoledgement from EUROfusion
        \acknowledgement
    \end{frame}

    % \begin{frame}{Contents}
    %     % display all sections
    %     \backgroundlogo%
    %     \tableofcontents%
    % \end{frame}%

    \section{Quick recap}

        \begin{frame}

            % GOALS
            \only<1>{%
                \frametitle{Recap}%
                \flushleft{\color{ipp}{What's the bolometer at W7-X?}}%
                \linebreak -- a two camera system using metal film resistive %
                detector arrays to measure plasma radiation based on thermal %
                effects%
                \vspace*{1.0cm}
                \flushleft{\color{ipp}\underline{Goals:}}
                \begin{itemize}%
                    \item[1. -]{global power balance: %
                                investigation of total radiation power loss, %
                                mainly from impurities, and its distribution}
                    \item[2. -]{local power balance: %
                                radiation profiles for transport studies %
                                (using tomographic inversion)}%
                \end{itemize}
            }%
        \end{frame}

        \begin{frame}

            \only<1-2>{%
            	\frametitle{Recap: Construction}%
                \begin{block}{}
                    \begin{itemize}%
                    \item[+]{multi-camera system with 113 channels combined:%
                          \linebreak 32@HBCm and 2 x 24@VBC subdetectors %
                          (+ differently coated/filtered channels)}
                    \end{itemize}%
                \end{block}
                \vspace*{0.2cm}%
            }

            \only<1>{%
                \begin{columns}%
                    \column{0.4\textwidth}%
                        \centering{%
                            \color{ipp}%
                            \textbf{\underline{HBCm}}%
                      }%
                    \column{0.4\textwidth}%
                        \centering{%
                            \color{ipp}%
                            \textbf{\underline{VBCr/VBCl}}%
                      }%
                \end{columns}%
                \begin{figure}%
                    \centering{%
                        \includegraphics[width=1.0\textwidth]%
                            {figures/content/linesofsight_in_vessel}%
                        \caption{\tiny{%
                                 (Lines of sight with individual apertures, %
                                 retracted into the vacuum vessel)}%
                        }
                    }%
                \end{figure}%
            }%

            \only<2>{%
              \begin{figure}%
                \centering{%
                    \includegraphics[width=0.9\textwidth]%
                        {figures/content/torus_full_banana}%
                    \caption{\tiny{%
                             (W7-X equilibrium fluxsurfaces from VMEC)}%
                    }
                }%
              \end{figure}
            }%

            \only<3>{%
                \frametitle{Recapitulation: Performance}%
                \begin{block}{}
                    \begin{itemize}
                        \item[+]{%
                            spatial resolution of \SI{5}{\centi\meter} %
                            \& temporal resolution of %
                            \SIrange{0.8}{6.4}{\milli\second}, spectral %
                            range between \SIrange{600}{0.2}{\nano\meter}}%
                        \item[+]{%
                            detectors of gold-foil absorbers with carbon %
                            coating on silicon-nitrate substrate and gold %
                            meander for \SI{200}{\nano\watt} resolution}
                    \end{itemize}
                \end{block}
                \begin{figure}%
                    \centering{%
                        \includegraphics[width=0.8\textwidth]%
                            {figures/content/detector_me}%
                        \caption{\tiny{%
                                 (Single detectior channel scheme with %
                                 holder), D.~Zhang,~P.~Hacker~et~al.}%
                        }
                    }%
              \end{figure}
            }

        \end{frame}%

    \section{Calculations}

        \begin{frame}{Calculations}

            \only<1>{%
              \backgroundlogo
              \tableofcontents[currentsection]
            }%

            \only<2-6>{%
                \frametitle{Equations}
                \begin{block}{Plasma radiation}
                    \only<2-4>{%
                        \begin{align}
                            P_{rad,bolo}\propto\sum_{Z} %
                                n_{e}\cdot n_{Z}\cdot L_{Z}\nonumber%
                        \end{align}
                    }%

                    \only<3-4>{%
                        \vspace*{-0.5cm}
                        \begin{align}
                            \dots\,\,=\frac{V_{P,tor}}{V_{cam}}\cdot%
                                \sum_{ch}\frac{V_{ch}}{K_{ch}}\cdot%
                                \frac{P_{ch}}{f_{OT}}\nonumber%
                        \end{align}
                    }%

                    \only<4>{%
                        \vspace*{-0.5cm}
                        \flushleft{%
                            \color{ippdark}\underline{%
                            Bolometer equation:}}%
                        \begin{align}
                            P_{ch}=F_{ch}\cdot\left(\tau_{ch}%
                                \frac{\diff(\Delta U)}{\diff t}+%
                                f_{\tau}\cdot(\Delta U)\right)\nonumber
                        \end{align}
                    }%

                    \only<5-6>{%
                        \flushleft{%
                            \color{ippdark}\underline{%
                            Bolometer equation:}}%
                        \begin{align}
                            P_{ch}&=F_{ch}\cdot\left(\tau_{ch}%
                                \frac{\diff(\Delta U)}{\diff t}+%
                                f_{\tau}\cdot(\Delta U)\right)\nonumber\\%
                                &\approx%
                                25.737\cdot\frac{\text{W}}{\text{V}}\left(%
                                \diff(\Delta U_{ch})+0.014\cdot\Delta %
                                U_{ch}\right)\nonumber%
                        \end{align}
                    }

                \end{block}
            }

            \only<2>{%
                \begin{itemize}
                    \item{$L_{Z}$: line radiation function of impurity %
                          $Z$\linebreak%
                          \dots $=f\left(T_{e},\,\,T_{i},\,\,T_{Z},\,\,wall%
                          \,\,material/conditions,\,\,\dots\right)$}%
                \end{itemize}
            }%

            \only<3>{%
                \begin{itemize}%
                    \item{$V_{ch}$: polygon volume of detector $ch$}%
                    \item{$V_{P, tor}$: estimated plasma volume (EMC3/VMEC %
                          simulation)}%
                    \item{$K_{ch}$: geometrical factor for channel $ch$}%
                \end{itemize}
            }

            \only<4>{%
                \begin{itemize}%
                    \item{$\Delta U\propto\Delta T$ the change in absorber %
                          temperature}%
                    \item{$F_{ch}=f\left(\text{detector electrical %
                          properties, excitation, cables~\dots}\right)$}
                    \item{$f_{\tau}=f\left(\text{detector impedance, %
                          heat capacity~\dots}\right)$}
              \end{itemize}
            }

            \only<5>{%
                \begin{itemize}
                    \item{channel \& cabel resistances %
                          $R_{ch}\approx\text{\SI{1}{\kilo\ohm}}$, %
                          $R_{cab}\approx\text{\SI{40}{\ohm}}$}%
                    \item{cooling/relaxation time of the gold foil %
                          $\tau_{ch}\approx\text{\SI{110}{\milli\second}}$}%
                    \item{heat capacity $\kappa_{ch}\approx%
                          \text{\SI{0.8}{\milli\watt/\kilo\ohm}}$}%
                    \item{scaling $f_{\tau}\approx1$}%
                    \item{temporal sampling in measurement %
                          \SI{1.6}{\milli\second}}
                \end{itemize}
            }

            \only<6>{%
                \vspace*{0.5cm}
                \centering{%
                    \color{ipp}{\textbf{%
                        $\Rightarrow\,\,$%
                        for non-collapsing plasma scenarios: %
                        $\Delta U\approx\text{\SI{e-3}{\volt}}$\\%
                        signal derivative for sampling time %
                        $\diff(\Delta U)\approx\text{\SI{e-5}{\volt}}$%
                        }%
                    }%
                }%
            }%

        \end{frame}%

    \section{Near real time feedback during OP1.2b}%

        \begin{frame}{Previously on ...}%

            \only<1>{%
              \backgroundlogo%
              \tableofcontents[currentsection]%
            }%

            \only<2-4>{%
                \begin{variableblock}{}{%
                    bg=gray,fg=white}{bg=gray,fg=white}%
                    \flushleft{%
                        \underline{Goals}}%
                    \begin{itemize}[label=\textcolor{white}{\textbullet}]
                        \item[1.-]{Verifying and improve the evaluation of %
                                   measurement data}%
                        \item[2.-]{instantaneous/direct tomographic %
                                   inversion after discharge\linebreak%
                                   $\Rightarrow$ investigation %
                                   of impurity transport processes and %
                                   discharge feedback}%

                        \only<2>{%
                            \item[3.-]{providing feedback signal for other %
                                       diagnostics/CoDaC through fast %
                                       (\SI{5}{\milli\second}) calculation %
                                       of $P_{rad}$ estimate}%
                        }

                        \only<3-4>{%
                            \item[\textcolor{black}{3.-}]{\color{black}{%
                                providing feedback signal for other %
                                diagnostics/CoDaC through fast %
                                (\SI{5}{\milli\second}) calculation %
                                of $P_{rad}$ estimate}}%
                        }

                    \end{itemize}
                \end{variableblock}%
            }

            \only<4>{%
                \begin{block}{Feedback}
                    \begin{itemize}
                        \item[1.-]{adjust heat loads, investigate radiation %
                                   regimes, maybe improved detachment}
                        \item[2.-]{find gas puff to radiation scaling law, %
                                   i.e.~importance of intrinsic/extrinsic %
                                   impurities}
                    \end{itemize}
                \end{block}
            }

        \end{frame}
        
		\begin{frame}{Feedback: Example}

            \begin{minipage}{0.5\textwidth}%
                \only<1-6>{%
                    \begin{block}{Dataflow}
                        \centering{%
							\only<1>{\textcolor{ipp}{%
								plasma radiation \& DAQ\\}}%
							\only<2->{%
								plasma radiation \& DAQ\\}%
                        	$\downarrow$\\%
							\only<2>{\textcolor{ipp}{%
								calculation, i.e. FIFO, filter\\}}%
							\only<1,3->{%
								calculation, i.e. FIFO, filter\\}%
                        	$\downarrow$\\%
							\only<3>{\textcolor{ipp}{%
	                        	single/multi-channel feedback\\}}%
							\only<1-2,4->{%
								single/multi-channel feedback\\}%
                        	$\downarrow$\\%
							\only<4>{\textcolor{ipp}{%
								two BNC/glas fiber lines to thermal helium beam diagnostic\\}}%
							\only<1-3,5->{%
								two BNC/glas fiber lines to thermal helium beam diagnostic\\}%
                        	$\downarrow$\\%
							\only<5>{\textcolor{ipp}{%
								plasma feedback\\}}%
							\only<1-4,6->{%
                        		plasma feedback\\}%
                        	$\downarrow$\\%
							\only<6>{\textcolor{ipp}{%
								change in radiation, density etc.}}
							\only<1-5,7->{%
								change in radiation, density etc.}
						}
                    \end{block}
                }

            \end{minipage}\hfill%
            \begin{minipage}{0.45\textwidth}

				\only<2-6>{%
					\begin{variableblock}{}{%
					  bg=ipp,fg=white}{bg=ipp,fg=white}%

                		\only<2-3>{%
                		    \begin{figure}
                		        \includegraphics[width=.8\textwidth]%
                		            {figures/content/fsez_lofs2d.pdf}
                		        \vspace*{-0.25cm}
                		        \caption{\tiny{%
                		            P.Hacker~et~al.}
                		        }
                		    \end{figure}
                		    \vspace*{-0.75cm}
                		    \begin{align}
                		        P_{rad}=f%
                		        \frac{V_{P,tor}}{V_{cam}}\sum_{S}%
                		        \frac{V_{ch}}{K_{ch}}%
                		        \frac{P_{ch}}{53\%}\nonumber%
                		    \end{align}
                		}

                		\only<4-5>{%
                		    \begin{figure}
                		        \includegraphics[width=.8\textwidth]%
                		            {figures/content/hebeambox}
                		        \caption{\tiny{%
                		            (Thermal helium beam diagnostic %
                		            with piezo valves) %
                		            M.Krychowiak~et~al.}
                		        }
                		    \end{figure}
                		}

                		\only<6>{%
                		    \begin{figure}
                		        \includegraphics[width=0.8\textwidth]%
                		            {figures/content/overview20181010032}
                		        \vspace*{-.8cm}
                		        \caption{\tiny{%
                		            P.Hacker~et~al.}
                		        }
                		    \end{figure}
                		}

					\end{variableblock}
				}
            \end{minipage}

        \end{frame}

        \begin{frame}{Implementation}

                \only<1-3>{%
                    \begin{variableblock}{Sanity check~\dots}%
                        {bg=gray,fg=white}{bg=gray,fg=white}%

                        \only<1>{%
                            \begin{figure}
                                \includegraphics[width=0.7\textwidth]%
                                    {figures/content/oscilloscope}
                                \caption{\tiny{%
                                    (Testing feedback system internally %
                                    prior to campaign, using an %
                                    oscilloscope and closely defined %
                                    \SI{7}{\milli\watt} laser pulse)%
                                }}
                            \end{figure}%
                        }%

                        \only<2-3>{%
                            \begin{itemize}
                                \item[+]{%
                                    defined solid-state laser pulses with %
                                    cycle frequency \SI{1}{\hertz}~-%
                                    ~\SI{1}{\kilo\hertz} and %
                                    \SI{7}{\milli\watt} power onto detector
                                }%
                                \item[+]{%
                                    limit the system to a temporal response %
                                    of $\ge$\SI{14}{\milli\second}
                                }%
                                \item[+]{%
                                    mainly $F\cdot\Delta t$, %
                                    $\Delta t$ is sample rate and $F$ the %
                                    number of \textit{FIFO} array elements
                                }%
                                \item[+]{%
                                    shallow slope due to detector response
                                }
                            \end{itemize}
                        }
                    \end{variableblock}
                }

                \only<3>{%
                    \begin{figure}
                        \includegraphics[height=0.4\textheight]%
                            {figures/content/FIFO}
                        \vspace*{-0.3cm}
                        \caption{\tiny{%
                            https://www.thecodingdelight.com/}
                        }
                    \end{figure}
                }

        \end{frame}

    \section{Scaling analysis}%

        \begin{frame}{Scaling analysis}%

            \only<1>{%
              \backgroundlogo%
              \tableofcontents[currentsection]%
            }%

            \only<2-3>{%
                \begin{variableblock}{}%
                    {bg=red,fg=white}{bg=red,fg=white}
                    \centering{\large{%
                        In the aftermath: Who said that we need to %
                        ``calculate'' $P_{rad}$?}}
                \end{variableblock}
            }

            \only<3>{%
                \begin{block}{Semi-experimental scaling}
                    \begin{itemize}
                        \item[+]{%
                            find scaling between ECRH, density, %
                            fueling and radiation, i.e.~instead of %
                            expensive \& slow feedback}%
                        \item[+]{%
                            making simple 3 parameter %
                            $\lbrace a,b,c\rbrace$ inference %
                            assumption like:%
                            \begin{align}
                                P_{rad}[\text{MW}]&\propto a%
                                    \lbrace n_{e}%
                                    [10^{19}\text{m}^{-3}]\rbrace^{b}%
                                    \lbrace P_{ECRH}[\text{MW}]%
                                    \rbrace^{c}\nonumber\\
                                \text{or}&\nonumber\\
                                &\propto a%
                                    \lbrace f_{H2}%
                                    [\text{mbar\,s/l}]\rbrace^{b}%
                                    \lbrace P_{ECRH}[\text{MW}]%
                                    \rbrace^{c}\nonumber
                            \end{align}
                            }%
                    \end{itemize}%
                \end{block}
            }

        \end{frame}

        \begin{frame}{To find or not to find...}

            \only<1-2>{%
                \begin{block}{Results: manual selection on main gas valve}%

                    \only<1>{%
                        \begin{figure}
                            \centering%
                            \includegraphics[width=\textwidth]%
                                {figures/content/overview20180822020}%
                            \caption{\tiny{%
                                Example of discharges in dataset for %
                                analysis, P.Hacker~et~al.}
                            }
                        \end{figure}
                    }

                    \only<2>{%
                        \begin{figure}
                            \centering%
                            \includegraphics[height=0.65\textheight]%
                                {figures/content/scaling_test}%
                            \caption{\tiny{%
                                Possible scaling between ECRH, density/main gas fueling in H$_{2}$ and radiation loss, %
                                P.Hacker~et~al.}
                            }
                        \end{figure}
                    }

                \end{block}
            }

            \only<3>{%
                \begin{variableblock}%
                    {}{bg=black,fg=white}{bg=black,fg=white}%
                    \begin{itemize}
                        \item[+]{%
                            no common scaling found, parameters all over %
                            the place
                        }
                        \item[+]{%
                            further analysis would maybe require to find %
                            additional dependencies, but that would be %
                            the ``needle in the hay stack''
                        }
                        \item[+]{%
                            originally forseen scaling based off of thermal %
                            helium gas inlet too complicated for now
                        }
                        \item[--]{%
                            ALSO: differentiation between intrinsic/%
                            extrinsic impurities totally neglected here, but %
                            seriously important anyhow
                        }
                    \end{itemize}
                \end{variableblock}
            }

        \end{frame}

    \section{New agenda}

        \begin{frame}{New agenda: construction ...}

            \only<1>{%
              \backgroundlogo%
              \tableofcontents[currentsection]%
            }%

            \only<2>{%
                \begin{block}{Correlation analysis}%

                    \begin{itemize}%
                        \item[+]{%
                            find most relevant channel combination to %
                            predict P$_{rad}$, i.e. the best channels %
                            for divertor gas insertion experiments
                        }
                        \item[+]{%
                            localisation and sensitivity of channels in %
                            response divertor valves, maybe %
                            n$_{e}$(P$_{rad}$), P$_{rad}$(n$_{e}$)\\%
                            $\Rightarrow$ spatial sensitivity for tomography?
                        }
                    \end{itemize}
                    \vspace*{-0.25cm}
                    \begin{align}
                        P_{prediction}&=\frac{V_{P,tor}}{V_{S}}\cdot%
                            \sum_{ch}^{S}\frac{V_{ch}}{K_{ch}}\cdot%
                            \frac{P_{ch}}{53\%}\nonumber\\%
                        V_{S}&=\sum_{ch}^{S}V_{ch}\nonumber%
                    \end{align}

                \end{block}
            }%

            \only<3>{%
                \begin{block}{Example: ``Standard deviation''-method}%
                    \centering%
                    \vspace*{.75cm}%
                    $d_{diff}(t)=\|P_{rad}(t) - P_{prediction}(t)\|$\\%
                    \vspace*{.75cm}%
                    $\varepsilon(t)=%
                        \left\{\begin{array}{ll}%
                        1-\frac{d_{diff}(t)}{P_{rad}(t)}&,%
                            \,\,d_{diff}<P_{rad}\\%
                        0&,\text{ else}
                        \end{array}\right\}$\\%
                    \vspace*{.75cm}%
                    $\vartheta=\overline{\varepsilon(t)}$%
                    \vspace*{.75cm}%
                \end{block}
            }%

            \only<4>{%
                \begin{block}{Example: testing against P$_{rad}$(HBCm)}%
                    \centering%
                    \begin{minipage}{0.47\textwidth}%
                        \includegraphics[width=1.0\textwidth]%
                            {figures/content/std_dev[_0_15_30]}%
                    \end{minipage}%
                    \hfill%
                    \begin{minipage}{0.47\textwidth}%
                        \includegraphics[width=1.0\textwidth]%
                            {figures/content/std_dev_spectrum}%
                    \end{minipage}%
                \end{block}%
            }%

        \end{frame}

        \begin{frame}{Back on the envelope ...}%

            \only<1>{%
                \begin{block}{}
                    \flushleft{%
                        \underline{Goals}}%
                    \begin{itemize}[label=\textcolor{white}{\textbullet}]
                        \item[x]{%
                            Verifying and improve the evaluation of %
                            measurement data}%
                        \item[\textcolor{red}{1.-}]{%
                            \textcolor{red}{%
                            instantaneous/direct tomographic %
                            inversion after discharge\linebreak%
                            $\Rightarrow$ investigation %
                            of impurity transport processes and %
                            discharge feedback}}%
                        \item[x]{%
                            providing feedback signal for other %
                            diagnostics/CoDaC through fast %
                            (\SI{5}{\milli\second}) calculation %
                            of $P_{rad}$ estimate}%
                        \item[\textcolor{orange}{2.-}]{\color{orange}{%
                            local sensitivity analysis of bolometer %
                            system with regards to feedback and %
                            configuration/scenarios}}%
                    \end{itemize}
                \end{block}%
            }%

            \only<2>{%
                \backgroundlogo%
                \frametitle{End}
                \begin{center}%
                    \large%
                    Thank you for your attention.\linebreak%
                    Now: questions.%
                \end{center}
            }

        \end{frame}

    % BACKUP
    \appendix
    \backupbegin

    \begin{frame}{Bibliography}
        \backgroundlogo%

        \only<1>{\footnotesize%
          \begin{thebibliography}{}%
            \bibitem{Zhang2010} "Design Criteria of the Bolometer diagnostic %
                                for steady-state operation of the W7-X %
                                stellarator"; %
                                Zhang, D. et al.; %
                                Review of Scientific Instruments, %
                                Jan 1st, 2010; DOI:10.1063/1.3483194
            \bibitem{Zhang2016} "The bolometer diagnostic at stellarator %
                                Wendelstein 7-X and its first results in the %
                                initial campaign"; %
                                D. Zhang, et al. %
                                and the W7-X Team; Stellarator-New 2017
            \bibitem{Mast1991} "A low noise highly integrated bolometer %
                                array %
                                for absolute measurement of VUV and soft x %
                                radiation"; %
                                K. F. Mast et. al; %
                                Review of Scientific Instruments 62, 744 %
                                (1991);
                                DOI: 10.1063/1.11.42078%
            \bibitem{VMEC} "Steepest descent moment method for three %
                           dimensional magnetohydrodynamic equilibria"; %
                           Hirshman, S.P. et al.; %
                           Physics of Fluids 26, 3553, (1983); %
                           DOI: 10.1063/1.864116%
           \bibitem{Wesson} "Tokamaks"; Wesson, J.; Clarendon Press, Oxford; %
                            1987%
            \end{thebibliography}%
        }

        \only<2>{\footnotesize%
            \begin{thebibliography}{}%
                \bibitem{Feng} "Numerical investigation of plasma edge %
                               transport and limiter heat fluxes in %
                               Wendelstein 7-X startup plasmas with %
                               EMC3-EIRENE"; %
                               Effenberg, F., Feng, Y. et al. %
                               Nucl. Fusion 57 (2017) 036021 (15pp); %
                               DOI: 10.1088/1741-4326/aa4f83%
            \bibitem{Gianone2002} "Derivation of bolometer equations %
                                  relevant to %
                                  operation in fusion experiments"; %
                                  Gianone, L. et al.; %
                                  Review of Scientific Instruments; %
                                  20th of November, 2002; %
                                  DOI: 10.1063/1.1498906%
            \bibitem{Zhang2018} "Results of the bolometer diagnostic at %
                                OP 1.a of W7-X"; %
                                internal review of the physics plan during %
                                the %
                                second operational phase at the stellarator %
                                W7-X; 28.02.20i18%
            \bibitem{Yamada2005} "Characterization of energy confinement in %
                                 net-current free plasmas using the %
                                 extended International Stellarator %
                                 Database"; %
                                 H. Yamada et al.; %
                                 INSTITUTE OF PHYSICS PUBLISHING and %
                                 INTERNATIONAL ATOMIC ENERGY AGENCY; %
                                 Nucl. Fusion 45 (2005) 1684¿1693%
            \bibitem{FIFOSource} ``Introduction to the Queue Data Structure %
                                 – Array Implementation''; %
                                URL: https://www.thecodingdelight.com/%
                                queue-data-structure-array-implementation/
            \end{thebibliography}%
        }

    \end{frame}%

    \begin{frame}{}%
        \backgroundlogo%
    \end{frame}%

    \begin{frame}{Correlation: Cross correlation}
        \only<1>{%
            \section{Correlation: Cross correlation}
            \begin{block}{Cross correlation-method}%
                \centering%
                \vspace*{.75cm}%
                $C_{corr}=%
                    \int (P_{rad}*P_{prediction})%
                    (\tau)\diff\tau$\\%
                    \vspace*{.75cm}\hspace*{1.6cm}%
                    $=\iint P_{rad}(t)P_{prediction}(t+\tau)\diff t\diff\tau$%
                \vspace*{.75cm}%
            \end{block}
        }
        \only<2>{%
            \begin{block}{Example: testing against P$_{rad}$(HBCm)}%
                \centering%
                \includegraphics[height=.7\textheight]%
                    {figures/content/cross_corr_spectrum}%
            \end{block}%
        }%
    \end{frame}

    \begin{frame}{Correlation: Coherence}
        \only<1>{%
            \section{Correlation: Coherence}
            \begin{block}{Coherence}%
                \centering%
                \vspace*{.75cm}%
                $C_{x,y}=\frac{\|(P_{x,y}\|^{2}}{P_{x,x}\cdot P_{y,y}}$\\%
                \vspace*{.5cm}%
                $P_{x,x}$ and $P_{y,y}$ are power spectral density estimates %
                of $X=P_{rad}$ and $Y=P_{prediction}$,\\%
                and $P_{x,y}$ is the cross spectral density estimate of X,Y%
                \vspace*{.75cm}%
            \end{block}
        }
        \only<2>{%
            \begin{block}{Example: testing against P$_{rad}$(HBCm)}%
                \centering%
                \includegraphics[height=.8\textheight]%
                    {figures/content/coherence[_0_15_30]}%
            \end{block}%
        }%
    \end{frame}

    \begin{frame}{XPID: 20181010.032}
        \begin{figure}
            \centering
            \includegraphics[width=\textwidth]%
                {figures/content/HBC_surf_20181010032}
            \caption{Horizontal camera radiation profile}
        \end{figure}
    \end{frame}

    \begin{frame}{XPID: 20181010.032}
          \begin{figure}
            \centering
            \includegraphics[width=\textwidth]%
                {figures/content/VBC_surf_20181010032}
            \caption{Vertical camera radiation profile}
          \end{figure}
    \end{frame}

    \begin{frame}{XPID: 20181010.032}
        \begin{figure}
            \centering
            \includegraphics[width=\textwidth]%
                    {figures/content/powsum_20181010032}
            \caption{Line integrated power for each camera}
        \end{figure}
    \end{frame}

    \begin{frame}{System schematics}
            \only<1>{
                \begin{figure}
                    \includegraphics[width=.9\textwidth]%
                        {figures/content/mm_DAQ_boxless_top}
                \end{figure}%
            }

            \only<2>{%
                \begin{figure}
                    \includegraphics[width=.9\textwidth]%
                        {figures/content/mm_DAQ_boxless_bottom}
                \end{figure}%
            }

            \only<3>{%
                \begin{figure}
                    \includegraphics[width=1.\textwidth]%
                        {figures/content/mm_feedback_boxless_top}
                \end{figure}%
            }

            \only<4>{%
                \begin{figure}
                    \includegraphics[width=1.\textwidth]%
                        {figures/content/mm_feedback_boxless_bottom}
                \end{figure}%
            }

    \end{frame}

    \backupend

\end{document}
