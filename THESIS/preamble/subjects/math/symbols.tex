%%% math
%% constants
\newcommand{\eu}{\ensuremath{\makeup{e}}} % upright e	
\newcommand{\iu}{\ensuremath{\makeup{i}}} % upright i
\newcommand{\piu}{\ensuremath{\makeup{\pi}}} % upright pi

\DeclareMathOperator{\const}{const.}

%% sets
\newcommand{\complex}{\ensuremath{\mathbb{C}}}
\newcommand{\real}{\ensuremath{\mathbb{R}}}
\newcommand{\rational}{\ensuremath{\mathbb{Q}}}
% \renewcommand{\natural}{\ensuremath{\mathbb{N}}}  % this collides with another symbol from amsmath

%% operators
\newcommand{\vect}[1]{\ensuremath{\vec{#1}}} % vector command
\newcommand{\matr}[1]{\ensuremath{\makebf{#1}}}

% derivatives
\makeatletter
\newcommand{\Grad}[2][\@nil]{%
    \def\tmp{#1}%
    \ifx\tmp\@nnil
    	\ensuremath{\vect{\nabla} #2}
    \else
    	\ensuremath{\vect{\nabla}_{\!\!#1} #2}
    \fi}
\makeatother
\newcommand{\Div}[1]{\ensuremath{\vect{\nabla} \cdot #1}}
\newcommand{\Rot}[1]{\ensuremath{\vect{\nabla} \times #1}}
\DeclareMathOperator{\sgn}{sgn}
\newcommand{\laplace}{\Delta}
\newcommand{\dalembert}{\ensuremath{\Box}\xspace}
% defines derivative: \dx[# of derivative]{derive this}{in that}
\newcommand{\dx}[3][\empty]
		{
			\if{#1}\equal{\empty}
				\frac{\mathrm{d}#2}{\mathrm{d}#3}
			\else
				\frac{\mathrm{d}^{#1}#2}{\mathrm{d}#3^{#1}}
		}
% defines partial derivative: \pdx[# of derivative]{derive this}{in that}
\newcommand{\pdx}[3][\empty]
		{
			\if{#1}\equal{\empty}
				\frac{\partial#3}{\partial#2}
			\else
				\frac{\partial^{#1}#3}{\partial#2^{#1}}
		}
% copied from somewhere on the net, this allows the use of \diff in integrals, \int_A \diff x \sin(x)
\makeatletter
\providecommand*{\diff}{\@ifnextchar^{\DIfF}{\DIfF^{}}}
\def\DIfF^#1{\mathop{\mathrm{\mathstrut d}}\nolimits^{#1}\gobblespace}
\def\gobblespace{\futurelet\diffarg\opspace}
\def\opspace
		{
			\let\DiffSpace\!
			\ifx\diffarg(
				\let\DiffSpace\relax
			\else
				\ifx\diffarg[%
					\let\DiffSpace\relax
				\else
					\ifx\diffarg\{%
						\let\DiffSpace\relax
					\fi
				\fi
			\fi
			\DiffSpace
		}
\makeatother

\newcommand{\Ham}{\ensuremath{\mathcal{H}}\xspace} % Hamilton

%% statistics
\newcommand{\expect}[1]{\ensuremath{\left\langle{#1}\right\rangle}} % expectancy value
\newcommand{\erw}{\ensuremath{\mathbb{E}}}
\newcommand{\mean}{\ensuremath{\mu}}
\newcommand{\std}{\ensuremath{\sigma}}
\newcommand{\orderof}[1]{\ensuremath{\mathcal{O}(#1)}}
\newcommand{\ChiNdf}{\ensuremath{\chi^{2}/\mathrm{ndf}}}
\DeclareMathOperator*{\argmin}{arg\,min}
\DeclareMathOperator*{\argmax}{arg\,max}
% \newglossaryentry{chi2red}{%
    % name=\ensuremath{\Chi^2_\nu},
	% type=symbols,
	% sort=statistics,
	% description={$\Chi^2$ per degree of freedom},
	% symbol=,
% }

%% typography
\newcommand{\apex}[1]{\ensuremath{^{\textrm{#1}}}} % write short topscripts
\newcommand{\pedex}[1]{\ensuremath{_{\textrm{#1}}}} % write short subscripts
\newcommand{\apedex}[2]{\ensuremath{^{\textrm{#2}}_{\textrm{#1}}}} % write short top/subscripts
\newcommand{\tightoverset}[2]{\mathop{#2}\limits^{\vbox to -.5ex{\kern-0.75ex\hbox{$\! #1$}\vss}}} % for example defined to be equal is \tightoverset{!}{=}
\newcommand{\customOverset}[2]{\mathrel{\overset{\makebox[0pt]{\mbox{\normalfont\tiny\sffamily #1}}}{#2}}}
\DeclareSIUnit[number-unit-product = \,]{\permille}{\textperthousand}
