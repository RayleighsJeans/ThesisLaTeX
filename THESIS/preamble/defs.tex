
%%% convenience methods

%%% provides functionality similar to \rlap, see e.g. https://www.tug.org/TUGboat/tb22-4/tb72perlS.pdf
\def\clap#1{\hbox to 0pt{\hss#1\hss}}
\def\LW{\dimexpr.25\linewidth-.5em} % linewidth for subfigure

%%% Gives the complementary to 1ex (height of a small x) for a capital X
\newlength{\heightOfX}
\AtBeginDocument{\settoheight{\heightOfX}{X}} 

%%% Get the current font size
\newlength{\fsize}
\makeatletter
    \setlength{\fsize}{\f@size pt}
\makeatother

% own definitions
\def\-{\raisebox{.75pt}{-}}%
\newcommand{\textgreek}[1]{%
    \begingroup%
    \fontencoding{LGR}%
    \selectfont#1\endgroup}%
\newcommand{\diff}{%
    \textnormal{d}}%
\newcommand{\tenpo}[1]{%
    10^{#1}}%
\newcommand{\greek}[1]{%
    \greektext#1\latintext}
\newcommand{\ix}[1]{%
    _{\text{#1}}}%
\newcommand{\head}[1]{%
    ^{\text{#1}}}%
\newcommand{\imag}{%
    \mathbf{i}}%
\newcommand{\tilt}[1]{%
    \textit{#1}}%
\newcommand{\divergenz}[1]{%
    \textit{div}\left(#1\right)}
\newcommand{\euler}{%
    \mathnormal{e}}%
\newcommand{\fett}[1]{%
    \textbf{#1}}%
\newcommand{\inexample}{%
    \text{e.g.}}%
\newcommand{\kett}{%
    \rangle}%
\newcommand{\bra}{%
    \langle}%
\newcommand{\grad}[1]{%
    \textit{grad}\left(#1\right)}

%%% citing related
% stolen from Will Robertson, prevents linebreaks before citations
\newcommand{\nobreakbefore}
    {%
        \relax
        \ifvmode
        \else
            \ifhmode
                \ifdim\lastskip > 0pt\relax
                    \unskip\nobreakspace
                \fi
            \fi
          \fi
    }
\let\oldcite\cite
\renewcommand{\cite}{\nobreakbefore\oldcite}
\let\oldref\ref
\renewcommand{\ref}{\nobreakbefore\oldref}

\newcommand{\colorboxwithoutxtraspace}[2]{\smash{\rlap{\colorbox{#1}{\phantom{#2}}}}#2}

%%% disable typographically bad settings for toc,lof etc.
\newcommand{\disabledprotrusion}[1]
        {
            \ifKOMA
                \ifthenelseproperty{compilation}{fontspec}{%
                    \begingroup
                        \addfontfeatures{Numbers={Lining,Monospaced}}
                }{}
            \fi
            \ifthenelseproperty{compilation}{microtype}{%
                \microtypesetup{protrusion=false}
            }{}
            #1
            \ifthenelseproperty{compilation}{microtype}{%
                \microtypesetup{protrusion=true}
            }{}
            \ifKOMA
                \ifthenelseproperty{compilation}{fontspec}{%
                    \endgroup
                }{}
            \fi
        }

%%% remove next character (don't remember for what this was good or where it is from)
\makeatletter
\newcommand{\removeifnextchar}[3]
        {
            \begingroup
            \ltx@LocToksA{\endgroup#2}
            \ltx@LocToksB{\endgroup#3}
            \ltx@ifnextchar{#1}
                {
                    \def\next{\the\ltx@LocToksA}
                    \afterassignment\next
                    \let\scratch= %
                }{
                    \the\ltx@LocToksB
                }
        }
\makeatother
\newcommand*{\removeDot}
        {
            \removeifnextchar{.}{}{}
        }


%%% change the appearance of names
\newcommand{\nameofperson}[1]{\textsc{#1}}

% \RequirePackage{tabularx}
%%% provides a signature command
\newcommand{\Signature}[2]% provides a labeling field. \Signature{Place}{Name}
        {%
            \vspace{2cm}%
            \noindent%
            \begin{tabular*}{\textwidth}{@{\extracolsep{0pt}} l @{\extracolsep{\fill}} r @{\extracolsep{0pt}}}%
#1, \getproperty{document}{date}	& \rule{0.33\textwidth}{1pt} 	\\
& \raggedleft{\textsc{#2}}
            \end{tabular*}
            \vspace{1cm}
        }

\AtBeginDocument{%
    \newif\ifKOMAandFancyChapterHeadings%
    \KOMAandFancyChapterHeadingsfalse%
    \ifKOMA%
        \ifthenelseproperty{compilation}{fancychapterheadings}{%
            \RedeclareSectionCommand[
                    beforeskip=\dimexpr3.3\baselineskip+1\parskip\relax,
                    innerskip=0pt,% <- space between chapter number and chapter title
                    afterskip=1.725\baselineskip plus .115\baselineskip minus .192\baselineskip,
                    afterindent=false
                ]%
                {chapter}%
            \renewcommand\raggedchapter{\raggedleft}%
            \renewcommand\chapterformat{{\fontsize{80pt}{80pt}\selectfont\textcolor{gray!25}{\thechapter}}}%
            \KOMAandFancyChapterHeadingstrue%
        }{}%
    \fi%
    \ifKOMAandFancyChapterHeadings%
        \renewcommand\chapterlineswithprefixformat[3]{#2#3\vspace*{-.5\baselineskip}\rule{\textwidth}{.4pt}\par\nopagebreak}%
    \else%
    \fi%
}

%%% float positioning
% alter some LaTeX defaults for better treatment of figures:
% see p.105 of "TeX Unbound" for suggested values.
% see pp. 199-200 of Lamport's "LaTeX" book for details.
% general parameters, for ALL pages:
\renewcommand{\topfraction}{0.9} % max fraction of floats at top
\renewcommand{\bottomfraction}{0.5} % max fraction of floats at bottom

% parameters for TEXT pages:
\setcounter{topnumber}{2}
\setcounter{bottomnumber}{2}
\setcounter{totalnumber}{4} % 2 may work better
\setcounter{dbltopnumber}{2} % for 2-column pages
\renewcommand{\dbltopfraction}{0.9} % fit big float above 2-col. text
\renewcommand{\textfraction}{0.1} % allow minimal text w. figs

% parameters for FLOAT pages (not text pages):
% nota bene: floatpagefraction MUST be less than topfraction !!
\renewcommand{\floatpagefraction}{0.8} % require fuller float pages
\renewcommand{\dblfloatpagefraction}{0.8} % require fuller float pages

% hidden parameters for float pages:
% \@fptop % space at top of float page
% \@fpbot % space at bottom of float page
% \@fpsep % space between floats on a float page

%%% adjust TOC
\ifKOMA
    \ifthenelseproperty{compilation}{fontspec}{%
        \DeclareTOCStyleEntry[%
            entryformat={%
                \addfontfeatures{Numbers={Lining,Monospaced}}\bfseries\sffamily\color{\getproperty{color}{main}}
            },
        pagenumberformat={%
            \addfontfeatures{Numbers={Lining,Monospaced}}%
            \bfseries\sffamily
        },
        ]
        {default}
        {chapter}
        \DeclareTOCStyleEntry[%
            entryformat={%
                \addfontfeatures{Numbers={Lining,Monospaced}}
            },
            pagenumberformat={%
                \addfontfeatures{Numbers={Lining,Monospaced}}
            },
        ]
        {default}
        {section}
        \DeclareTOCStyleEntry[%
            entryformat={%
                \addfontfeatures{Numbers={Lining,Monospaced}}
            },
            pagenumberformat={%
                \addfontfeatures{Numbers={Lining,Monospaced}}
            },
        ]
        {default}
        {subsection}
        \DeclareTOCStyleEntry[%
            entryformat={%
                \addfontfeatures{Numbers={Lining,Monospaced}}
            },
            pagenumberformat={%
                \addfontfeatures{Numbers={Lining,Monospaced}}
            },
        ]
        {default}
        {subsubsection}
        \DeclareTOCStyleEntry[%
            entryformat={%
                \addfontfeatures{Numbers={Lining,Monospaced}}
            },
            pagenumberformat={%
                \addfontfeatures{Numbers={Lining,Monospaced}}
            },
        ]
        {tocline}
        {figure}
        \DeclareTOCStyleEntry[%
            entryformat={%
                \addfontfeatures{Numbers={Lining,Monospaced}}
            },
            pagenumberformat={%
                \addfontfeatures{Numbers={Lining,Monospaced}}
            },
        ]
        {tocline}
        {table}
    }{}
\fi


%%% Chapter without head. Usefull for includepdf
\makeatletter
\newcommand{\headlesschapter}[1]{%
  \begingroup
  \let\@makechapterhead\@gobble % make \@makechapterhead do nothing
  \chapter{#1}
  \endgroup
}
\makeatother


%%% graphic includes
\makeatletter
\newcommand\appendgraphicspath[1]{%
  \g@addto@macro\Ginput@path{#1}%
}
\makeatother
% \appendgraphicspath{{./content/figures/}}
\graphicspath{{content/figures/}}


%%% igraph: IGRAPH /igraph IGRAPH definition
\RequirePackage{xkeyval}
\RequirePackage{xstring}
\RequirePackage{xfp}
\RequirePackage{ifthen}
\RequirePackage{lipsum}
\RequirePackage{tikzscale} % for pgffigure rescaling with native font
\makeatletter
\define@key{igraph}{width}{\def\igraph@width{#1}}
\define@key{igraph}{svgwidth}{\def\igraph@svgwidth{#1}}
% Options that only work with includegraphics forwarding
\define@key{igraph}{angle}{\def\igraph@angle{#1}}
\define@key{igraph}{draft}{\def\igraph@draft{#1}}
\define@key{igraph}{scale}{\def\igraph@scale{#1}}
% svgwidth is a hack to change the size of text afterwards.
% The optimal way to go is to use scaletext=false as is the default and build your picture suitable for the publication width.
\define@key{igraph}{height}{\def\igraph@height{#1}}
\define@boolkey{igraph}{scaletext}{\def\igraph@scaletext{#1}}
\newcommand{\igraph}[2][]{%
    \filename@parse{#2}
    \setkeys{igraph}{width=\linewidth} % set default
    \setkeys{igraph}{svgwidth=\linewidth} % set default
    \setkeys{igraph}{height=\empty} % set default
    \setkeys{igraph}{scaletext=false} % set default
    \setkeys{igraph}{#1}
    \ifthenelse{%
        \equal{\filename@ext}{pgf}%
    }{%
        %--------------------
        % pgf
        %--------------------
        \typeout{----- INCLUDE pgf @ #2 ------}%
        \let\pgfimageWithoutPath\pgfimage%
        \renewcommand{\pgfimage}[2][]{\typeout{----- INCLUDE pgfimgage @ ##2 ------}\pgfimageWithoutPath[##1]{\filename@area/##2}}%
        \resizebox{\igraph@width}{!}{\input{#2}}%
        % \let\pgfimage\pgfimageWithoutPath  % reset to old definition
    }{%
        \ifnum\pdfstrcmp{\filename@ext}{pdf_tex}=0
            %--------------------
            % pdf_tex
            %--------------------
            \typeout{----- INCLUDE pdf_tex @ #2 ------}%
            \IfSubStr{#1}{height}{%
                \PackageError{igraph}{pdf_tex does not allow the height attribute}{}%
                }{}%
            \ifthenelse{%
                \boolean{\igraph@scaletext}%
            }{%
                \def\svgwidth{\igraph@svgwidth}%
                \resizebox{\igraph@width}{!}{%
                    \import{\filename@area}{\filename@base.\filename@ext}%
                }%
            }{%
                \def\svgwidth{\igraph@width}%
                \import{\filename@area}{\filename@base.\filename@ext}%
            }%
        \else
            \ifthenelse{%
                \equal{\filename@ext}{svg}%
            }{%
                %--------------------
                % svg
                %--------------------
                % This requires inkscape installation. You can set the inkscape path by
                % \usepackage{svg}
                % \setsvg{
                %     inkscape = flatpak run --branch=master --arch=x86_64 --command=inkscape % set the correct path to inkscape
                % }
                \typeout{----- INCLUDE svg @ #2 ------}%
                % \appendgraphicspath{\filename@area}
                % \includesvg[width=\igraph@width]{#2}%
                \PackageError{igraph}{svg not implemented!}{}%
            }{%
                \ifthenelse{%
                    \equal{\filename@ext}{tikz}%
                }{%
                    %--------------------
                    % tikz
                    %--------------------
                    \typeout{----- INCLUDE tikz @ #2 ------}%
                    \tikzsetnextfilename{\filename@base}%
                    \begin{minipage}{\igraph@width}%
                        % this way, we set textwidth to width and thus scale the tikz picture.
                        \input{#2}%
                    \end{minipage}%
                }{%
                    %--------------------
                    % else
                    %--------------------
                    \typeout{----- INCLUDE with includegraphics @ #2 ------}%
                    \IfSubStr{#1}{height}{%
                        \includegraphics[#1]{#2}%
                    }{%
                        \includegraphics[width=\igraph@width, #1]{#2}%
                    }%
                }%
            }%
        \fi
    }%
}%
\makeatother


%%% setimagedimensions: Retrieve image dimensions inspired by https://tex.stackexchange.com/questions/3657/get-the-size-that-a-figure-is-being-rendered
\newlength{\imageh}
\newlength{\imaged}
\newlength{\imagew}
\newcommand{\setimageh}[1]{
 \settoheight{\imageh}{\usebox{#1}}
}
\newcommand{\setimagew}[1]{
 \settowidth{\imagew}{\usebox{#1}}
}
\newcommand{\setimaged}[1]{
 \settodepth{\imaged}{\usebox{#1}}
}
\newsavebox{\imagesavebox}
\newcommand{\setimagedimensions}[1]{
    \savebox{\imagesavebox}{\igraph{#1}}
    \setimageh{\imagesavebox}
    \setimagew{\imagesavebox}
    \setimaged{\imagesavebox}
    %    The height of the image is : \the\imageh\\
    %    The width of the  image is : \the\imagew\\
    %    The depth of the  image is : \the\imaged\\
}


%%% storedata / getdata implemenataion gotten from https://tex.stackexchange.com/questions/215563/storing-an-array-of-strings-in-a-command
% example: \storedata{mydata}{{one}{two}{three}}
%          \storedata{mylongdata}{
%            {one} {two} {three} {four} {five} {six} {seven} {eight} {nine} {ten}
%            {eleven} {twelve} {thirteen} {fourteen} {fifteen} {sixteen}
%            {seventeen} {eighteen} {nineteen} {twenty} {twenty-one} {twenty-two}
%            {twenty-three} {twenty-four}
%          }
%          \getdata{1}{mylongdata}  -> one
%          \getdata{0}{mylongdata}  -> len(mylongdata)
\makeatletter
\newcommand*{\storedata}[2]{%
  \count@=0 %
  \@tfor\@tmp:=#2\do{%
    \advance\count@\@ne
    \expandafter\let\csname data:\the\count@:#1\endcsname\@tmp
  }%
  \expandafter\edef\csname data:0:#1\endcsname{\the\count@}%
}
\newcommand*{\getdata}[2]{%
  \@ifundefined{data:0:#2}{%
    \@latex@error{Undefined data `#2'}\@ehc
  }{%
    \expandafter\@getdata\expandafter{%
      \the\numexpr
        \ifnum\numexpr(#1)<\z@
          \@nameuse{data:0:#2}+1+%
        \fi
        (#1)%
      \relax
    }{#2}{#1}%
  }%
}
\newcommand*{\@getdata}[3]{%
  \ifnum#1<\z@
    \@getdata@error{\the\numexpr(#3)\relax}{#2}%
  \else
    \ifnum#1>\@nameuse{data:0:#2} %
      \@getdata@error{#1}{#2}%
    \else
      \@nameuse{data:#1:#2}%
    \fi
  \fi
}
\newcommand*{\@getdata@error}[2]{%
  \@latex@error{%
    Wrong data selector #1 for `#2',\MessageBreak
    which only contains \@nameuse{data:0:#2} item(s)%
  }\@ehc
}
\makeatother

\storedata{mydata}{{one}{two}{three}}
\storedata{mylongdata}{
  {one} {two} {three} {four} {five} {six} {seven} {eight} {nine} {ten}
  {eleven} {twelve} {thirteen} {fourteen} {fifteen} {sixteen}
  {seventeen} {eighteen} {nineteen} {twenty} {twenty-one} {twenty-two}
  {twenty-three} {twenty-four}
}


%%% multigraph: Same height graphs
% Example:
% \begin{figure}
%     \centering
%     \multigraph[labels=
%         {fig:first-label}
%         {fig:second-label}
%     ]{%
%         data/figures/example.pdf;
%         data/figures/example.pgf}{%
%         {My pdf example.}
%         {My tikz example.}
%     }
%     \caption{%
%         Example of a \\multigraph command
%     }
%     \label{fig:multigraph}
% \end{figure}
% Note: Since _ is catcode translated you should use \sb in equations for sub (_)
\makeatletter
% \RequirePackage{forarray}
\RequirePackage{calc}
\RequirePackage{listings}
\newlength\figratiosum
\newlength\figratio
\newlength\figheight
\newlength\figwidth
\newcounter{npaths}
\newcommand{\multigraph}{\begingroup
  \catcode`_=12 \domultigraph}
\newcommand{\catcodeigraph}{\begingroup
  \catcode`_=11 \igraph}
\define@key{multigraph}{width}{\def\multigraph@width{#1}}
\define@key{multigraph}{labels}{\def\multigraph@labels{#1}}
\newcommand{\domultigraph}[3][]{
    % #1: width=, labels=. . .
    % #2: semicolon seperated file names
    % #3: semicolon seperated subcaptions
    \setkeys{multigraph}{width=\linewidth - 1em} % set default
    \setkeys{multigraph}{labels=\empty} % set default
    \setkeys{multigraph}{#1}

    % get number of paths
    % \storedata{paths}{#2}
    % \getndata{paths}
    \setcounter{npaths}{0}
    \ForEachX{;}{%
        \stepcounter{npaths}
    }{#2}

    % put labels in pgflist \labels
    \IfSubStr{#1}{labels}{%
        \def\subfigurelabels{}
        \ForEachX{;}{%
            % set the path variable to an array
            \edef\subfigurelabels{\subfigurelabels"\thislevelitem"}
            \ifnum\thislevelcount=\value{npaths}
            \else
                \edef\subfigurelabels{\subfigurelabels,}
            \fi
        }{\multigraph@labels}
    }{}

    % put paths in pgflist \paths
    \def\paths{}
    \setlength\figratiosum{0pt}
    \def\figratios{}
    \ForEachX{;}{%
        % set the path variable to an array
        \edef\paths{\paths"\thislevelitem"}
        \ifnum\thislevelcount=\value{npaths}
        \else
            \edef\paths{\paths,}
        \fi

        % calulate ratio hi / wi and sum of the ratios
        \setimagedimensions{\thislevelitem}
        \setlength\figratio{1pt*\ratio{\imagew}{\imageh}}
        \addtolength{\figratiosum}{\figratio}
        \edef\figratios{\figratios"\the\figratio"}
        \ifnum\thislevelcount=\value{npaths}
        \else
            \edef\figratios{\figratios,}
        \fi
    }{#2}

    % put captions in pgflist \subcaptions
    \storedata{subcaptions}{#3}
    % \def\subcaptions{}
    % \ForEachX{;}{%
    %     % set the subcaption variable to an array
    %     \edef\subcaptions{\subcaptions"\thislevelitem"}
    %     \ifnum\thislevelcount=\value{npaths}
    %     \else
    %         \edef\subcaptions{\subcaptions,}
    %     \fi
    % }{#3}

    % calculate figheight = 1*completewidth/sum(figratio_i}
    \setlength{\figheight}{1pt*\ratio{\multigraph@width}{\figratiosum}}

    \addtocounter{npaths}{-1}  % CAREFUL: npaths is reduced by one for the use in foreach!
    \foreach \i in {0,...,\value{npaths}}{%
        % calculate figwidth = figheight / figratio
        \pgfmathparse{{\figratios}[\i]}
        \setlength\figwidth{\figheight}
        \pgfmathparse{\figwidth * \pgfmathresult}
        \setlength{\figwidth}{\pgfmathresult pt}
        \begin{subfigure}{\figwidth}
            \centering
            \pgfmathparse{{\paths}[\i]}
            \typeout{----- Width for multigraph figure \pgfmathresult : \the\figwidth ------}
            % \igraph{\pgfmathresult}
            \catcodeigraph{\pgfmathresult}\endgroup
            % subcaption
            % \pgfmathparse{{\subcaptions}[\i]}
            % \subcaption{\pgfmathresult}
            \subcaption{\getdata{\i+1}{subcaptions}}
            % label
            \IfSubStr{#1}{labels}{%
                \pgfmathparse{{\subfigurelabels}[\i]}
                \label{\pgfmathresult}
            }{}
        \end{subfigure}
    }
    \endgroup
}
\makeatother


% upright in math mode
\newcommand{\makeup}[1]{%
    \ensuremath{
        \ifluatex
            \symup{#1}
        \else
            \mathrm{#1}
        \fi
    }
}
\newcommand{\makebf}[1]{%
    \ensuremath{
        \ifluatex
            \symbf{#1}
        \else
            \mathbf{#1}
        \fi
    }
}
%%% math
%% constants
\newcommand{\eu}{\ensuremath{\makeup{e}}} % upright e	
\newcommand{\iu}{\ensuremath{\makeup{i}}} % upright i
\newcommand{\piu}{\ensuremath{\makeup{\pi}}} % upright pi

\DeclareMathOperator{\const}{const.}

%% sets
\newcommand{\complex}{\ensuremath{\mathbb{C}}}
\newcommand{\real}{\ensuremath{\mathbb{R}}}
\newcommand{\rational}{\ensuremath{\mathbb{Q}}}
% \renewcommand{\natural}{\ensuremath{\mathbb{N}}}  % this collides with another symbol from amsmath

%% operators
\newcommand{\vect}[1]{\ensuremath{\vec{#1}}} % vector command
\newcommand{\matr}[1]{\ensuremath{\makebf{#1}}}

% derivatives
\makeatletter
\newcommand{\Grad}[2][\@nil]{%
    \def\tmp{#1}%
    \ifx\tmp\@nnil
    	\ensuremath{\vect{\nabla} #2}
    \else
    	\ensuremath{\vect{\nabla}_{\!\!#1} #2}
    \fi}
\makeatother
\newcommand{\Div}[1]{\ensuremath{\vect{\nabla} \cdot #1}}
\newcommand{\Rot}[1]{\ensuremath{\vect{\nabla} \times #1}}
\DeclareMathOperator{\sgn}{sgn}
\newcommand{\laplace}{\Delta}
\newcommand{\dalembert}{\ensuremath{\Box}\xspace}
% defines derivative: \dx[# of derivative]{derive this}{in that}
\newcommand{\dx}[3][\empty]
		{
			\if{#1}\equal{\empty}
				\frac{\mathrm{d}#2}{\mathrm{d}#3}
			\else
				\frac{\mathrm{d}^{#1}#2}{\mathrm{d}#3^{#1}}
		}
% defines partial derivative: \pdx[# of derivative]{derive this}{in that}
\newcommand{\pdx}[3][\empty]
		{
			\if{#1}\equal{\empty}
				\frac{\partial#3}{\partial#2}
			\else
				\frac{\partial^{#1}#3}{\partial#2^{#1}}
		}
% copied from somewhere on the net, this allows the use of \diff in integrals, \int_A \diff x \sin(x)
\makeatletter
\providecommand*{\diff}{\@ifnextchar^{\DIfF}{\DIfF^{}}}
\def\DIfF^#1{\mathop{\mathrm{\mathstrut d}}\nolimits^{#1}\gobblespace}
\def\gobblespace{\futurelet\diffarg\opspace}
\def\opspace
		{
			\let\DiffSpace\!
			\ifx\diffarg(
				\let\DiffSpace\relax
			\else
				\ifx\diffarg[%
					\let\DiffSpace\relax
				\else
					\ifx\diffarg\{%
						\let\DiffSpace\relax
					\fi
				\fi
			\fi
			\DiffSpace
		}
\makeatother

\newcommand{\Ham}{\ensuremath{\mathcal{H}}\xspace} % Hamilton

%% statistics
\newcommand{\expect}[1]{\ensuremath{\left\langle{#1}\right\rangle}} % expectancy value
\newcommand{\erw}{\ensuremath{\mathbb{E}}}
\newcommand{\mean}{\ensuremath{\mu}}
\newcommand{\std}{\ensuremath{\sigma}}
\newcommand{\orderof}[1]{\ensuremath{\mathcal{O}(#1)}}
\newcommand{\ChiNdf}{\ensuremath{\chi^{2}/\mathrm{ndf}}}
\DeclareMathOperator*{\argmin}{arg\,min}
\DeclareMathOperator*{\argmax}{arg\,max}
% \newglossaryentry{chi2red}{%
    % name=\ensuremath{\Chi^2_\nu},
	% type=symbols,
	% sort=statistics,
	% description={$\Chi^2$ per degree of freedom},
	% symbol=,
% }

%% typography
\newcommand{\apex}[1]{\ensuremath{^{\textrm{#1}}}} % write short topscripts
\newcommand{\pedex}[1]{\ensuremath{_{\textrm{#1}}}} % write short subscripts
\newcommand{\apedex}[2]{\ensuremath{^{\textrm{#2}}_{\textrm{#1}}}} % write short top/subscripts
\newcommand{\tightoverset}[2]{\mathop{#2}\limits^{\vbox to -.5ex{\kern-0.75ex\hbox{$\! #1$}\vss}}} % for example defined to be equal is \tightoverset{!}{=}
\newcommand{\customOverset}[2]{\mathrel{\overset{\makebox[0pt]{\mbox{\normalfont\tiny\sffamily #1}}}{#2}}}
\DeclareSIUnit[number-unit-product = \,]{\permille}{\textperthousand}




%%% remarks
% hyphenation:
% If words shall not be hyphenated, write them in \hyphenation{...} without -, mark hyphenation points with - in \hyphenation{...}
% Rules: words that shall be hyphenated should be at least 5 letters long, hyphenation points result from syllables or etymology
% Rules: personal names, numbers, acronyms, values and their units don't get hyphenated
% Rules: There should be no more than 3 hyphenations in a row, because hyphenations disturb the reader
% example:
% \hyphenation{Un-ter-su-chung}
% \babelhyphenation[british]{cy-lin-der}
% \babelhyphenation[ngerman]{Sol=Gel=Tauch-be-schich-tungs-pro-zess}
