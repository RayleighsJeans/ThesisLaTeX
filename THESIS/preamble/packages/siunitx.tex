\usepackage
        [
            % choose language
            locale=UK,
            % uncertainty of the format (x \pm y) unit
            separate-uncertainty,
            % Use a fraction to separate units if \per is used 
            % per-mode=symbol-or-fraction,
            % display 10^0 respectively e0
            % retain-zero-exponent=true,
            binary-units, % have byte and bit available
            per-mode=symbol
        ]
        {siunitx%
        % use older (deprecated) version for backwards compatibility
        }[=v2]

% \sisetup{mode=text,range-phrase = {\text{~to~}}}
% see https://tex.stackexchange.com/questions/34997/how-to-use-sirange-in-math-mode
\newcommand{\SIrangemath}[3]{%
    \text{\SIrange{#1}{#2}{#3}}}

% Make nice intervals. Could also be done by reconfiguring \SIrange
\newcommand{\SIinterval}[3]{%
    \ensuremath{\lbrack\num{#1}{,}\ %
    \num{#2}\rbrack\,\si{#3}}}

%%% Util definitions
% au
\DeclareSIUnit{\arbitraryunit}{a.\,u.}
\DeclareSIUnit{\atom}{atoms}

% per mille
\DeclareSIUnit{\permille}{‰}

% samples
\DeclareSIUnit{\sample}{S}

% We need that as for fW, fF etc. the letters overlap, which souldn't happen here. Don't use for e.g. fA !
\DeclareSIPrefix{\Femto}{f\kern0.1ex}{-15}

% barn related shortcuts
\DeclareSIUnit{\pb}{\pico\barn}
\DeclareSIUnit{\fb}{\femto\barn}
\DeclareSIUnit{\nb}{\nano\barn}

\sisetup{range-phrase=--}
\sisetup{range-units=single}
%\sisetup{per-symbol=symbol}

% Example Code
% \sisetup{...}
% \num{9.99 + 3.7i}
% \num[parse-numbers=false]{9{,}99 + \sqrt{2}i}
% -- Option necessary if one uses \sqrt e.g.
% \num{375628375823765}
% \SIrange{1e3}{3e3}{\meter}
% \SI{1 \pm 2 e3}{\meter}

% Remarks
% Use \tothe{} for exponents!
