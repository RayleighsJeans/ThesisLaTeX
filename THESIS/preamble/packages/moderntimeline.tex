% The following 4 lines are for compatibility with the koma-moderncvclassic package
\definecolor{color0}{rgb}{0,0,0}% main default color, normally left to black
\definecolor{color1}{rgb}{0,0,0}% primary scheme color
\definecolor{color2}{rgb}{0,0,0}% secondary scheme color
\definecolor{color3}{rgb}{0,0,0}% tertiary scheme color
\usepackage{moderntimeline}

\renewcommand{\tltextstart}[2][base west]{%
   \tikzset{%
       tl@startyear/.style={%
           font=#2,
           name=tl@startyear,
           above={0.8ex+1pt},
           inner xsep=0pt,
           anchor=#1,
       }
   }
}
\tltext{\scriptsize}



\makeatletter
\@ifclassloaded{moderncv}{%
    %%% The following definitions introduce some small changes
    
    % define reference column
    \newcommand{\cvreferencecolumn}[2]{%
      \cvitem[0.8em]{}{%
        \begin{minipage}[t]{\listdoubleitemmaincolumnwidth}#1\end{minipage}%
        \hfill%
        \begin{minipage}[t]{\listdoubleitemmaincolumnwidth}#2\end{minipage}%
        }%
    }
    
    % contact
    \newcommand{\cvreference}[8]{%
        \textbf{#1}\newline% Name
        \ifthenelse{\equal{#2}{}}{}{\addresssymbol~#2\newline}%
        \ifthenelse{\equal{#3}{}}{}{#3\newline}%
        \ifthenelse{\equal{#4}{}}{}{#4\newline}%
        \ifthenelse{\equal{#5}{}}{}{#5\newline}%
        \ifthenelse{\equal{#6}{}}{}{\emailsymbol~\texttt{\href{mailto:#6}{\nolinkurl{#6}}}\newline}%
        \ifthenelse{\equal{#7}{}}{}{\phonesymbol~#7\newline}%
        \ifthenelse{\equal{#8}{}}{}{\mobilephonesymbol~#8}}
    
    % allow the use of \social[orcid][link]{Bla}
    \newcommand*{\orcidsocialsymbol}{\textnormal{\aiOrcid}}
    
    % adjust cventry
    \renewcommand*{\cventry}[7][.25em]{%
    	\cvitem[#1]{#2}{%
    		{\bfseries#3}%
    		\ifthenelse{\equal{#4}{}}{}{, {\slshape#4}}% ... into this one (without comma).
    		\ifthenelse{\equal{#5}{}}{}{, #5}%
    		\ifthenelse{\equal{#6}{}}{}{, #6}%
    		\ifthenelse{\equal{#7}{}}{}{:}\strut%
    		\ifx&#7&%
    		\else{\newline{}\begin{minipage}[t]{\linewidth}\small#7\end{minipage}}\fi
    	}
    }
    
    
    \newcommand{\posUpperCV}{center}
    \newcommand{\posLowerCV}{center}
    
    
    \newcounter{AmountOfLines}
    \setcounter{AmountOfLines}{20} % Default. Can be changed
    \setlength{\hintscolumnwidth}{2.75cm} % Default. Can be changed
    
    \usetikzlibrary{shapes.geometric}
    \makeatletter
    \renewcommand{\tldatecventry}[7][color1]{%
        \pgfmathsetmacro\tl@endyear{\tl@lastyear}%
        \pgfmathsetmacro\tl@diffyears{\tl@lastyear-\tl@firstyear}
        \pgfmathsetmacro\tl@startfraction{(#2-\tl@firstyear)/(\tl@lastyear-\tl@firstyear)}%
        \pgfmathsetmacro\tl@endfraction{(\tl@endyear-\tl@firstyear)/(\tl@lastyear-\tl@firstyear)}%
        \cventry{
            \tikz[baseline=0pt]{
                \useasboundingbox (0,-1.5ex) rectangle (\hintscolumnwidth,1ex);
                \fill [\tl@runningcolor,draw,semithick,line cap=rect] (0,0) -- (\hintscolumnwidth,0);
        	    	\foreach \x in {0,...,\theAmountOfLines} \draw [\tl@runningcolor,semithick] (\x*\hintscolumnwidth/\theAmountOfLines,0pt) -- (\x*\hintscolumnwidth/\theAmountOfLines,2pt); %---- If too many lines add a fitting prefactor in front of \tl@diffyears and adjust the x-values by multiplying with the inverse
        	    	%\foreach \x in {0,...,\tl@diffyears} \draw [\tl@runningcolor,semithick] (\x*\hintscolumnwidth/\tl@diffyears,0pt) -- (\x*\hintscolumnwidth/\tl@diffyears,2pt); %---- If too many lines add a fitting prefactor in front of \tl@diffyears and adjust the x-values by multiplying with the inverse
                \fill [#1] (0,0)
                   ++(\tl@startfraction*\hintscolumnwidth,0pt)
                   node [yshift=0.5ex,tl@singleyear] {#2\vphantom{g}}
                   node [draw,fill,shape border rotate=180,regular polygon,regular polygon sides=3,inner sep=1pt,anchor=corner 1] {};
            }%
        }%
        {#3}{#4}{#5}{#6}{#7}%
    }
    \makeatother
    
    
    \makeatletter
    \renewcommand{\tlcventry}[8][color1]{%
    \pgfmathsetmacro\tl@endyear{ifthenelse(#3==0,\tl@lastyear,#3)}%
    \pgfmathsetmacro\tl@startfraction{(#2-\tl@firstyear)/(\tl@lastyear-\tl@firstyear)}%
    \pgfmathsetmacro\tl@endfraction{(\tl@endyear-\tl@firstyear)/(\tl@lastyear-\tl@firstyear)}%
    \pgfmathsetmacro\tl@diffyears{\tl@lastyear-\tl@firstyear}
    \pgfmathsetlength{\pgf@xa}{#3} \ifdim\pgf@xa=0pt \def\tl@startlabel{\tl@since #2} \else \def\tl@startlabel{#2} \fi
     \cventry{\tikz[baseline=0pt]{
        \useasboundingbox (0,-1.5ex) rectangle (\hintscolumnwidth,1ex);
        \fill [#1] (0,0) ++ (\tl@startfraction*\hintscolumnwidth,0)
           node [tl@startyear,anchor=\posUpperCV,yshift=1ex] {\tl@startlabel\vphantom{g}}
           rectangle (\tl@endfraction*\hintscolumnwidth,\tl@width-1pt)
           node [tl@endyear,anchor=\posLowerCV,yshift=-1.75ex] {\pgfmathparse{ifthenelse(#3==0,,#3)}\pgfmathresult\vphantom{h}}
           (\hintscolumnwidth,0pt);
    		\draw [#1,thick] (0,0) ++(\tl@startfraction*\hintscolumnwidth,0) -- (\tl@startfraction*\hintscolumnwidth,\tl@width+2pt);
    		%\pgfextra{\ifthenelse{\tl@endyear==\tl@lastyear}{}{\draw [#1,thick] (0,0) ++(\tl@endfraction*\hintscolumnwidth,\tl@width) -- (\tl@endfraction*\hintscolumnwidth,-2pt); }} %------  
    			%
    			\pgfmathifthenelse{\tl@endyear==\tl@lastyear}{}{"\noexpand\draw [#1,thick] (0,0) ++(\tl@endfraction*\hintscolumnwidth,\tl@width) -- (\tl@endfraction*\hintscolumnwidth,-2pt);"}\pgfmathresult
    			%
        \pgfmathsetlength{\pgf@xa}{#3} \ifdim\pgf@xa=0pt
           \shade [left color=#1] (\tl@startfraction*\hintscolumnwidth,0)
               rectangle (\tl@endfraction*\hintscolumnwidth,\tl@width);
        \else
           \fill [#1] (\tl@startfraction*\hintscolumnwidth,0)
               rectangle (\tl@endfraction*\hintscolumnwidth,\tl@width);
        \fi
    		\fill [\tl@runningcolor,semithick,draw,line cap=rect] (0,0) -- (\hintscolumnwidth,0);
    		\foreach \x in {0,...,\tl@diffyears} \draw [\tl@runningcolor,semithick] (\x*\hintscolumnwidth/\tl@diffyears,0pt) -- (\x*\hintscolumnwidth/\tl@diffyears,2pt); %---- If too many lines add a fitting prefactor in front of \tl@diffyears and adjust the x-values by multiplying with the inverse
        }%
    }%
    {#4}{#5}{#6}{#7}{#8}%
    }
    \makeatother
    
    \newboolean{useFancyCitation}
    \setboolean{useFancyCitation}{false}
    \newcommand{\useFancyCitation}{\setboolean{useFancyCitation}{true}}
    
    \newcommand{\fancyCitation}{
    	\ifthenelse{\boolean{useFancyCitation}}{
    		\defbibenvironment{bibliography}{
    			\list{%
    				\IfInteger{\thefield{cited}}{%
    				\pgfplotscolormapaccess[0:\MaxCitation]{\thefield{cited}}{mal-map}%
        			\def\TEMP{\definecolor{my color}{rgb}}%
        			\expandafter\TEMP\expandafter\pgfmathresult%
    				\color{my color}%
    			}{}%
    			\printtext[labelnumberwidth]{%
    				\printfield{labelnumber}}
    			}{
    				\setlength{\labelwidth}{\labelnumberwidth}%
          			\setlength{\leftmargin}{\labelwidth}%
          			\setlength{\labelsep}{\biblabelsep}%
          			\addtolength{\leftmargin}{\labelsep}%
          			\setlength{\itemsep}{\bibitemsep}%
          			\setlength{\parsep}{\bibparsep}
          		}%
          		\renewcommand*{\makelabel}[1]{\hss##1}
          	}
      		{\endlist}
      		{\item}
    	
    		\cvitem{Citations}{
    			\begin{minipage}[c]{5.5cm}
    				\pgfplotsset{
    					colormap={mal-map}{rgb255=(0,0,0) rgb255=(255,0,0)}
    				}
    				\begin{tikzpicture}
    					\begin{axis}
    						[
    						colormap name={mal-map},
    						colorbar horizontal,
    						scale only axis,
    						height=0pt,
    						width=0pt,
    						colorbar style={title=,
    										width=5cm,
    										height=0.5cm,
    										title style={at={(xticklabel* cs:0)},anchor=east}},
    						hide axis,
    						point meta min=0,
    						point meta max=\MaxCitation,
    						]
    						\addplot [draw=none] coordinates {(0,0) (1,1)};
    					\end{axis}
    				\end{tikzpicture}
    			\end{minipage}
    		}
    	}{}
    }
    
    \makeatletter
    \newcommand{\useFullLines}{
    	\RenewDocumentCommand{\section}{sm}{%
    		\par\addvspace{2.5ex}%
    		\phantomsection{}% reset the anchor for hyperrefs
    		\addcontentsline{toc}{section}{##2}%
    		\parbox[t]{\hintscolumnwidth}{\strut\raggedleft\raisebox{\baseletterheight}{\color{color1}\rule{\hintscolumnwidth}{0.95ex}}}%
    		\hspace{\separatorcolumnwidth}%
    		\parbox[t]{\maincolumnwidth}{\strut\sectionstyle{##2}\hspace{\separatorcolumnwidth}%
    			\strut\raisebox{\baseletterheight}{\color{color1}\rule{\maincolumnwidth-\widthof{\sectionstyle{##2}}-\separatorcolumnwidth}{0.95ex}}}%
    		\par\nobreak\addvspace{1ex}\@afterheading
    	}% to avoid a pagebreak after the heading
    }
    \makeatother

}{}
\makeatother 
