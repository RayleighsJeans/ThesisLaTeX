% NOTE: glossaries-extra argument atuomake=immediate requires a latex version later than 2019/01/04
\makeatletter
% \@ifpackagelater{glossaries-extra}{2019/01/04}{%
	\PassOptionsToPackage{automake=immediate}{glossaries-extra}
% }{%
% 	\PassOptionsToPackage{automake}{glossaries-extra}
% }
\makeatother


% allows the easy use of glossaries
\usepackage
		[
			nonumberlist, % don't display numbers
			acronym, % make an acronym
			toc, % put the glossaries in the toc
% 			ucmark, % uppercase mark
% 			counter=equation, % use equations instead of pagenumbers
% 			nomain, % as we don't want an index at the moment
% 			nogroupskip,
		]
		{glossaries-extra}

\usepackage{glossary-mcols}
\renewcommand{\glsmcols}{2}
\renewcommand*{\glsnamefont}[1]{\textsf{#1}}
\setabbreviationstyle[acronym]{long-short}

% \makeatletter
% \@ifpackagelater{glossaries-extra}{2019/01/04}{}{%
% 	\patchcmd\@gls@automake{\write18}{\DelayedShellEscape}{}{\fail}
% }
\makeatother

%%% settings
\setlength{\glsdescwidth}{0.685\textwidth} % width of the description field
\setlength{\glspagelistwidth}{0.1\textwidth} % width of the pagelist field

\newglossary[slg]{symbols}{syi}{syg}{List of Symbols} % new glossary for symbols


\makeglossaries % make the glossaries

\renewcommand{\entryname}{\textsf{Symbol}} % Redefine headings for list of symbols (not the one in toc)
\renewcommand{\descriptionname}{\textsf{Description}}
\renewcommand{\symbolname}{\textsf{Unit}}


%%% custom glossary styles
\newglossarystyle{customGlossaryStyleForListOfSymbols}{%
    \renewenvironment{theglossary}{%
        \begin{longtable}{p{0.12\textwidth}p{\glsdescwidth}p{\glspagelistwidth}}
    }{%
        \end{longtable}
    }
    \renewcommand*{\glossaryheader}{}
    \renewcommand*{\glsgroupheading}[1]{}
    \renewcommand{\glossentry}[2]{%
        \glsentryitem{##1}\glstarget{##1}{\glossentryname{##1}} & \glossentrydesc{##1}. &  \glossentrysymbol{##1}\tabularnewline
    }
    \renewcommand{\subglossentry}[3]{%
        & \glssubentryitem{##2}\glstarget{##2}{\strut}\glossentrydesc{##2} & \glossentrysymbol{##2}\tabularnewline
    }
    \renewcommand*{\glsgroupskip}{%
        \ifglsnogroupskip
        \else
            & &\tabularnewline
        \fi
    }
}

\newglossarystyle{customListOfSymbols}{% make a new glossarystyle
    \setglossarystyle{customGlossaryStyleForListOfSymbols}
    \renewcommand*{\glossaryheader}{%
        % For some reason the commands \entryname and \symbolname were not set.
        % \bfseries\entryname&\bfseries\descriptionname&\bfseries \symbolname\\\toprule[1.5pt]\endhead
        \bfseries\textsf{Symbol}&\bfseries\textsf{Description}&\bfseries \textsf{Unit}\\\toprule[1.5pt]\endhead
    }
    \renewenvironment{theglossary}{%
        \begin{longtable}{p{0.12\textwidth}p{\glsdescwidth}p{\glspagelistwidth}}
    }{%
        \end{longtable}
    }
}

\newglossarystyle{customListOfAbbreviations}{%
    \renewenvironment{theglossary}{%
        \begin{longtable}{lp{\glsdescwidth}}
    }{%
        \end{longtable}
    }
    \renewcommand*{\glossaryheader}{}
    \renewcommand*{\glsgroupheading}[1]{}
    \renewcommand{\glossentry}[2]{%
        \textsf{\textbf{\glsentryitem{##1}\glstarget{##1}{\glossentryname{##1}}}} & \glossentrydesc{##1}\glspostdescription\space ##2\tabularnewline[1ex]
    }
    \renewcommand{\subglossentry}[3]{%
        & \glssubentryitem{##2}\glstarget{##2}{\strut}\glosentrydesc{##2}\glspostdescription\space##3\tabularnewline
    }
    \renewcommand*{\glsgroupskip}{%
        \ifglsnogroupskip
        \else
            & \tabularnewline
        \fi
    }
}

\makeatletter
\newglossarystyle{customIndex}{% this gives a goodlooking index
    \renewenvironment{theglossary}{%
        \setlength{\parindent}{0pt}
        \setlength{\parskip}{0pt plus 0.3pt}
        \let\item\@idxitem
    }{%
    }
    \renewcommand*{\glossaryheader}{}
    \renewcommand*{\glsgroupheading}[1]{}
    \renewcommand*{\glossaryentryfield}[5]{%
    \item\glstarget{##1}{##2}
        \ifx\relax##4\relax
        \else
            \space(##4)
        \fi
        \dotfill ##3\glspostdescription \space ##5
    }
    \renewcommand*{\glossarysubentryfield}[6]{%
        \ifcase##1\relax
                                % level 0
        \item
        \or
                                % level 1
            \subitem
        \else
                                % all other levels
            \subsubitem
        \fi
        \glstarget{##2}{##3}
        \ifx\relax##5\relax
        \else
            \space(##5)
        \fi
        \dotfill ##4\glspostdescription\space ##6
    }
    \renewcommand*{\glsgroupskip}{\indexspace}
    \renewcommand*{\glsgroupheading}[1]{%
        \item\textbf{\glsgetgrouptitle{##1}}\indexspace
    }
}
\makeatother


\newglossarystyle{documentationLabelStyle}{%  adapted from https://tex.stackexchange.com/questions/555612/glossariesextra-document-available-labels/565083#565083
    \renewenvironment{theglossary}{%
        \begin{longtable}{p{.2\textwidth}p{.3\textwidth}p{.5\textwidth}}
    }{%
        \end{longtable}
    }%
    \renewcommand*{\glossaryheader}{%                   setting the table's header
        \bfseries Acronym & \bfseries Label & \bfseries Description
        \\\tabularnewline\endhead
    }%
    \renewcommand*{\glossentry}[2]{%                    main entries displayed 
        \typeout{GLOSS: ##1}
        \ifglshaschildren{##1}{%
            \tabularnewline%                                break line before new section
            \multicolumn{3}{l}{%
                %         \ifglshasparent{\glsentryparent{\glsentryparent{##1}}}{%
                %             \ifglshasparent{\glsentryparent{\glsentryparent{\glsentryparent{##1}}}}{%
                %                 ---→
                %             }{--→}
                %         }{-→}
                %     }{→}
                % }{}
                \textbf{\glstarget{##1}{\glossentryname{##1}}}
            }
            \tabularnewline%                                end of row
        }{%
            \glstarget{##1}{\glossentryname{##1}}%          name
            & \glscurrententrylabel%                        label
            & \glossentrydesc{##1}%                         description
            \tabularnewline%                                end of row
        }
    }%
    \renewcommand{\subglossentry}[3]{%
        \typeout{SUBGLOSS: ##2}
        \glssubentryitem{##2}%                          increment the subentrycounter
        \glstarget{##2}{\glossentryname{##2}}%          name
        & \glscurrententrylabel%                        label
        & \glossentrydesc{##2}%                         description
        \tabularnewline%                                end of row
    }%
    \renewcommand*{\glsgroupskip}{%
        \ifglsnogroupskip
        \else
            \tabularnewline
        \fi
    }
}


\newglossarystyle{documentationSymbolsStyle}{%  adapted from https://tex.stackexchange.com/questions/555612/glossariesextra-document-available-labels/565083#565083
    % \setglossarystyle{tree}  % inherit the tree style
    \renewcommand*{\glsgroupskip}{}%
    \renewenvironment{theglossary}{%
        \begin{longtable}{p{0.12\textwidth}p{.2\textwidth}p{\glsdescwidth}p{\glspagelistwidth}}
    }{%
        \end{longtable}
    }
    \renewcommand*{\glossaryheader}{%
        % For some reason the commands \entryname and \symbolname were not set.
        % \bfseries\entryname&\bfseries\descriptionname&\bfseries \symbolname\\\toprule[1.5pt]\endhead
        \bfseries\textsf{Symbol}&\bfseries\textsf{Label}&\bfseries\textsf{Description}&\bfseries \textsf{Unit}\\\toprule[1.5pt]\endhead
    }
    \renewcommand*{\glossentry}[2]{%                    main entries displayed 
        \ifglshaschildren{##1}{%
            \tabularnewline%                                break line before new section
            \multicolumn{4}{l}{%
                % \ifglshasparent{##1}{% indentation according to number of parents
                %     \ifglshasparent{\glsentryparent{##1}}{%
                %         \ifglshasparent{\glsentryparent{\glsentryparent{##1}}}{%
                %             \ifglshasparent{\glsentryparent{\glsentryparent{\glsentryparent{##1}}}}{%
                %                 ---→
                %             }{--→}
                %         }{-→}
                %     }{→}
                % }{}
                \textbf{\glstarget{##1}{\glossentryname{##1}}}
            }
            \tabularnewline%                                end of row
        }{%
            \glstarget{##1}{\glossentryname{##1}}%          name
            & \glscurrententrylabel%                        label
            & \glossentrydesc{##1}%                         description
            & \glossentrysymbol{##1}%                       symbol
            \tabularnewline%                                end of row
        }
    }%
    \renewcommand{\subglossentry}[3]{%
        \glssubentryitem{##2}%                                  increment the subentrycounter
        \glstarget{##2}{\glossentryname{##2}}%                  name          
        & \glscurrententrylabel%                                label
        & \glossentrydesc{##2}\glspostdescription\space##3
        & \glossentrysymbol{##2}%                               symbol
        \tabularnewline
    }%
    \renewcommand*{\glsgroupskip}{%
        \ifglsnogroupskip
        \else
            \tabularnewline
        \fi
    }
}


%%% example code
% \newglossaryentry{Ditt}
% 			{
% 				type=symbols,
% 				name=Dichte,
% 				sort=dichte,
% 				description={}
% 			}
%
% \newglossaryentry{categorical}
% 			{
% 				type=symbols,
% 				name={-funktionaltheorie},
% 				description={},
% 				parent=Ditt
% 			}
%
% \newglossaryentry{Dichte}
% 			{
% 				type=symbols,
% 				name={$\varrho$},
% 				sort=deltafunktion,
% 				description={Die Delta-Distribution wertet eine beliebig oft differenzierbare Funktion $f$ an der Stelle $0$ aus, also gilt $\mathrm{\delta}(f\,)=f(0)$ wobei $f(0) \in \mathbb{C}$.},
% 				symbol={\si{\newton\meter}}
% 			}
% use like: \gls{Dichte}
