\usepackage
        [
            hyperref=true,
            backend=biber, % modern backend, if missing google for biber.exe, arguments for the compiler: "%tm"
            url=true, % urls in bibliography?
            abbreviate=false,
            isbn=false, % isbn-number in bibliography?
            doi=true, % doi in bibliography?
            backref=true, % if true show on which sites the cited source is used
            urldate=long, % specifies how to display the urldate. Options are short, long, terse, comp, iso8601
            style=numeric,  % alphabetic,  %numeric,  % -comp,
            citestyle=numeric-comp,  %science, numeric-comp,
            maxbibnames=3, % 7 is enough, then et. al should be inserted
            giveninits=true,
            block=none,
            sorting=none,
            defernumbers=true,
            %dashed=false, % removes the dashes on multiple consecutive copies of the same authors %this option is incompatible with style=numeric*
        ]
        {biblatex}

\DefineBibliographyStrings{english}{%
  backrefpage={page},  % originally "cited on page"
  backrefpages={pages},  % originally "cited on pages"
}

%%% settings
% PUBLISH:START
\IfFileExists{content/bibliography/bibliography.bib}{%
    \typeout{----- FOUND BIBLIOGRAPHY @ content/data/bibliography/bibliography.bib -----}
    \addbibresource{content/bibliography/bibliography.bib}
}{%
    \IfFileExists{content/library.bib}{%
        \typeout{----- FOUND BIBLIOGRAPHY @ content/library.bib -----}
        \addbibresource{/library.bib}
    }{%
        \typeout{----- NO BIBLIOGRAPHY FOUND -----}
        \addbibresource{bibliography/dummy.bib}
    }
}
% PUBLISH:PYTHON
% if os.path.exists('content/data/bibliography/bibliography.bib'):
%     tex = r'\addbibresource{content/data/bibliography/bibliography.bib}'
% elif os.path.exists('content/library.bib'):
%     tex = r'\addbibresource{content/library.bib}'
% else:
%     tex = r'\addbibresource{bibliography/dummy.bib}'
% PUBLISH:END

\makeatletter
\@ifpackageloaded{seqsplit}{
    \DeclareFieldFormat{eid}{Art.\addnbspace\seqsplit{#1}}
}{}
\makeatother

% use either doi or url
\DeclareSourcemap{
  \maps[datatype=bibtex]{
    \map{
      \step[fieldsource=doi,final]
      \step[fieldset=url,null]
    }
  }
}

\setcounter{biburllcpenalty}{7000} % set to a value greather than 0 to allow url breaking after lowercase (lc) values
\setcounter{biburlucpenalty}{7000} % set to a value greather than 0 to allow url breaking after uppercase (uc) values
\setcounter{biburlnumpenalty}{7000} % set to a value greather than 0 to allow url breaking after numerical (num) values

\DeclareFieldFormat{journaltitle}{\mkbibemph{#1},} % italic journal title with comma
\DeclareFieldFormat[inbook,thesis]{title}{\mkbibemph{#1}\addperiod} % italic title with period
\DeclareFieldFormat[article,misc]{title}{\enquote{#1}} % title of journal article is printed as normal text
\DeclareFieldFormat[article,inproceedings,incollection]{volume}{Vol.~#1} % makes volume of journal bold and adds colon
\DeclareFieldFormat{edition}{\ifinteger{#1}{\mkbibordedition{#1}\addnbthinspace{}ed.}{#1\isdot}}

% use eprint for urn: example fields (bib-file)
% % eprint = {urn:nbn:de:bsz:100-opus-5191},
% % eprinttype = {urn},							
\DeclareFieldFormat{eprint:urn}{\textsc{URN}\addcolon\space\ifhyperref{\href{http://www.nbn-resolving.org/#1}{\nolinkurl{#1}}}{\nolinkurl{#1}}}

% make names with small capitals
\renewcommand{\mkbibnamegiven}[1]{\textsc{#1}}
\renewcommand{\mkbibnamefamily}[1]{\textsc{#1}}
\renewcommand{\mkbibnameprefix}[1]{\textsc{#1}}
\renewcommand{\mkbibnamesuffix}[1]{\textsc{#1}}

\DeclareLanguageMapping{english}{bibliography/custom-british-ordinal-sscript}
\DeclareLanguageMapping{british}{bibliography/custom-british-ordinal-sscript}

% last entries first
\DeclareSortingTemplate{ndymdt}{
     \sort{
        \field{presort}
    }
    \sort[final]{
        \field{sortkey}
    }
    \sort[direction=descending]{
        \field{sortyear}
        \field{year}
        \literal{9999}
    }
    \sort[direction=descending]{
        \field[padside=left,padwidth=2,padchar=0]{month}
        \literal{99}
    }
    \sort[direction=descending]{
        \field[padside=left,padwidth=2,padchar=0]{day}
        \literal{99}
    }
}

%%%%%%%%%%%%
% useful for ListOfPublications at end of thesis
\newboolean{bibCV}
\setboolean{bibCV}{false}

\ifluatex
    \def\getAuthorBibCv{\FirstChar{\getproperty{author}{firstname}}.~\getproperty{author}{familyname}}
\else
    \def\getAuthorBibCv{\getproperty{author}{firstname}~\getproperty{author}{familyname}}
\fi

\newboolean{isCoauthorList}
\setboolean{isCoauthorList}{false}

\DefineBibliographyStrings{english}{andothers={\ifthenelse{\boolean{isCoauthorList}}
                                                        {et\addabbrvspace al\adddot\addcomma\addspace amongst them \textbf{\textsc{\getAuthorBibCv}}}
                                                        {et\addabbrvspace al\adddot}}}
%%%%%%%%%%%%
% useful for Author contribution e.g. at the end of a cumulative thesis
% Careful: This command does not by generate a bibligraphy!!! (dontbib category)
% See https://tex.stackexchange.com/questions/113996/how-to-cite-without-an-entry-in-the-bibliography
\DeclareBibliographyCategory{dontbib}
\let\printbibliographyold\printbibliography%
\renewcommand{\printbibliography}[1][]{\typeout{----- printbibliography ------}\printbibliographyold[notcategory=dontbib,#1]}%
\makeatletter
\newcommand{\tempmaxup}[1]{\def\blx@maxcitenames{99}#1}
\DeclareCiteCommand{\fullcitecontribution}[\tempmaxup]{% precode
    \usebibmacro{prenote}
    \addtocategory{dontbib}{\thefield{entrykey}}
}{% loopcode
    \textbf{\usebibmacro{maintitle+title}}
    \newline\nopunct\newblock
    \usebibmacro{author}
    \newline\nopunct\newblock
    \usebibmacro{journal+issuetitle}, \usebibmacro{doi+eprint+url} \usebibmacro{addendum+pubstate}
}{% sepcode
    \multicitedelim
}{% postcode
    \usebibmacro{postnote}
}
\makeatother

%%%%%%%%%%%%
