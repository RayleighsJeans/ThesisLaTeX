\ifthenelseproperty{compilation}{clsdefineschapter}{%
	\ifKOMA
		\addchap[Abstract]{Abstract}
	\else
    	\chapter[Abstract]{Abstract}
    \fi
}{%
	\ifKOMA
		\addsec[Abstract]{Abstract}
	\else
    	\section[Abstract]{Abstract}
    \fi
}
% \lettrine[lhang=0.25,findent=2pt,nindent=0pt]{M}{agnetic nanoparticles} possess a multitude of fields of application, for example in biotechnology and utilization as (magnetically) easily separable catalysts.
Among the diverse fabrication methods that allow the production of nanoparticles with magnetic properties respectively a chemical composition tailored for a specific task one can find the bottom-up sol-gel dip-coating technique, with which the Co$_{\textit{x}}$Ni$_{\text{\num{1}−\,\textit{x}}}$ nanoparticles embedded in silica were created.
Those nanoparticles exhibit a wavelength dependent coercivity if irradiated with laser light.

\ifthenelseproperty{compilation}{clsdefineschapter}{%
	\ifKOMA
		\addchap[Zusammenfassung]{Zusammenfassung}
	\else
    	\chapter[Zusammenfassung]{Zusammenfassung}
    \fi
}{%
	\ifKOMA
		\addsec[Zusammenfassung]{Zusammenfassung}
	\else
    	\section[Zusammenfassung]{Zusammenfassung}
    \fi
}
% \lettrine[lhang=0.25,findent=2pt,nindent=0pt]{M}{agnetische Nanopartikel} 
\begin{otherlanguage}{ngerman} % breaks if before \lettrine :-/
besitzen einen Durchmesser, welcher unter \SI{100}{\nano\meter} liegt und die Fabrikation entsprechend anspruchsvoll gestaltet.
Trotz der damit einhergehenden Herausforderungen gelang es in den letzten Jahren die Herstellungsmethoden soweit zu verbessern, dass sich die magnetischen Eigenschaften wie auch die chemische Zusammensetzung anwendungsspezifisch maßschneidern lassen.
\end{otherlanguage}
