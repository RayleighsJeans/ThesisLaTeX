% important package, warns about old packages and commands. Needs to be loaded before the documentclass.
\RequirePackage
		[
			abort,
			l2tabu,
			orthodox,
% 			experimental
		]
		{nag}

\input{preamble/packages/import}
% \RequirePackage{ifthen}

% lisp like object oriented property definition,
% see https://tex.stackexchange.com/questions/37094/
% what-is-the-recommended-way-to-assign-a-value-
% to-a-variable-and-retrieve-it-for
\makeatletter
\def\ece#1#2{\expandafter#1\csname#2\endcsname}%

% \setproperty{ATOM}{PROPNAME}{VALUE} defines the property PROPNAME on the
% ``atom'' ATOM to have VALUE.
\def\setproperty#1#2#3{\ece\protected@edef{#1@p#2}{\unexpanded{#3}}}%

% \getproperty{ATOM}{PROPNAME} expands to the value of the property
% PROPNAME on ATOM, or to nothing (i.e., \empty), if the property isn't
% present.
\def\getproperty#1#2{%
  \expandafter\ifx\csname#1@p#2\endcsname\relax
  % then \empty
  \else \csname#1@p#2\endcsname
  \fi
}%

\def\ifthenelsepropertydefined#1#2#3#4{%
  \expandafter\ifx\csname#1@p#2\endcsname\relax
  #4
  \else#3
  \fi
}%

\def\ifpropertydefined#1#2#3{%
    \ifthenelsepropertydefined{#1}{#2}{#3}{}
}

\def\ifpropertyundefined#1#2#3{%
    \ifthenelsepropertydefined{#1}{#2}{}{#3}
}

\def\raiseifpropertyundefined#1#2#3{%
    \ifpropertyundefined{#2}{#3}{\PackageError{#1}{%
        Property #2 #3 needs to be defined. Put %
        \@backslashchar setproperty{#2}{#3} %
        to your settings file}{Grep for your property :)}}
}

% Only set the property if it has not been set yet. 
% This should be used in the definition of default properties in 
% preamble/properties.tex. It allows to define a
% property in documentcls without beeing overwritten by 
% setters in properties.tex.
\def\setpropertyifundefined#1#2#3{%
    \ifpropertyundefined{#1}{#2}{%
        \setproperty{#1}{#2}{#3}
    }{}
}

% Forwarding a default usage message with a value 
% proposal (#3) to setpropertyifundefined
\def\setpropertyifundefinedwithusageinfo#1#2#3{%
    \setpropertyifundefined{#1}{#2}{TODO>> %
    Put \backslash{}setproperty\{#1\}\{#2\}\{<your %
    value here, e.g. ``#3``>\} into content/settings.tex <<}
}

% \ifthenelseproperty{ATOM}{PROPNAME}{if}{else}
% For boolean properties
\def\ifthenelseproperty#1#2#3#4{\providetoggle{%
    #1@p#2}\settoggle{#1@p#2}{\getproperty{%
    #1}{#2}}\iftoggle{#1@p#2}{#3}{#4}}
% \ifthenelsepropertyequal{ATOM}{PROPNAME}{VALUE}{if}{else}

\def\ifthenelsepropertyequal#1#2#3#4#5{%
    \ifthenelse{\equal{\getproperty{#1}{#2}}{#3}}{#4}{#5}}
\makeatother

\makeatletter
% This command should be used instead of usepackage.
% It provides clashes of re-imports and sets the union of all options.
\newcommand\RequirePackageWithOption[2][]{%
    \@ifpackageloaded{#2}{%
        \PassOptionsToPackage{#1}{#2}
    }{%
        \RequirePackage[#1]{#2}
    }
}

% This command follows the hirarchy for default import defined like so:
% If you do not find the file to import in content, look in 
% templates/\getproperty{default}{<filename>}, 
% else import templates/Default/<filename>

% The values of the property default should are the name of 
% the templates which are caps Args:
%   #1: filename
\newcommand\ImportDefault[1]{
    \IfFileExists{content/#1.tex}{
        \typeout{----- LOAD OWN #1 -----}
        \import{content/}{#1}
    }{
        \typeout{----- LOAD DEFAULT #1 -----}
        \ifthenelsepropertydefined{default}{#1}{%
            \import{template/\getproperty{default}{#1}/}{#1}
        }{%
            \import{template/}{#1}
        }
    }
}

\newcommand\ImportPackage[1]{
    \IfFileExists{content/packages/#1.tex}{
        \input{preamble/packages/#1}
    }{%
        \input{preamble/packages/#1}
    }
}
\makeatother
