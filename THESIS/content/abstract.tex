%
\checkoddpage\ifoddpage\clearpage\else\cleardoublepage\fi%
%
    \chapter*{Abstract}%
%
        The importance of radiative power exhaust towards the operation of existing and the development of future fusion reactors is well-understood. Exploration of deliberately impurity seeded and feedback controlled scenarios with focus on detachment physics for heat energy dissipation and machine safety is of great interest. Reliable and stable controlled radiative cooling in the confined region, in the SOL and divertor with radiation power fractions of $f\ix{rad}\ge95\%$ is therefore necessary.\\%
        During OP1.2b at the stellarator W7-X, a real-time bolometer radiation feedback system for thermal gas injection of low-Z impurities has been designed and implemented. It achieved stable hydrogen plasma with controlled helium radiative edge cooling at $f\ix{rad}\ge85\%$. Here, reduction of target heat loads of at least a factor of two was measured, while detachment of C$^{3+}$ ionization radiation was visible already for $f\ix{rad}\sim50\%$. A limited set of at least three lines of sight was validated as a proxy for fast radiation power loss extrapolations.\\%
        Exploration of the corresponding parameter space did not present correlations for low-Z impurity seeded feedback experiments. It was established however that moderately scaled gas puffs generally and reliably performed edge cooling without terminal plasma perturbation. Rudimentary models for this injection process have been found to be capable of representing exemplary radiation measurements of feedback activations. Detectors viewing the separatrix and SOL region presented to be most viable for \textit{a priori} real-time bolometer feedback configurations. A robust selection of three-seven vertical and/or horizontal bolometer camera channels can be provided that achieves $\ge85\%$ prediction accuracy when compared to the full data set. STRAHL was used in combination with experimental data from feedback scenarios and input parameters modelled for comparison purposes. Carbon impurities were found to be the dominant contributor to SOL emissivity, while reduction of diffusivity, electron temperature and density around the separatrix provided radiation profiles similar to experimental findings, seeing an inward shift from outside to inside the separatrix for $f\ix{rad}\rightarrow1$.\\%
        Significant robustness and reliability for a radially dependent anisotropy weighting Minimum Fisher regularization algorithm in combination with the W7-X bolometer system have been established through rigorous geometry perturbation testing. A large scale benchmark thereof has suggested the existence of an ideal set of $k\ix{ani}$ for a given emissivity distribution. An intrinsic bias towards the upper SOL and separatrix area in the transmissivity matrix $\mathbf{T}$ was found. Limited interval weighting factor variations have shown $k\lessgtr1$ to correspond well to localised or smooth radiation structures in reconstruction, respectively. Evaluation of phantom images with two-dimensional quality coefficients demonstrated misalignment between those and $\chi^{2}$. For the artificial brightness profiles, presented $P\ix{rad}$ consistently underestimated the actual, total power contained in the phantom.%
%