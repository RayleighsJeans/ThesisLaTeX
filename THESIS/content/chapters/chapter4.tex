%
\chapter{Two-dimensional radiation inversion}\label{chap:inversions}%
%
    The multichannel, multicamera bolometer system is used to perform tomographic inversions to obtain two-dimensional radiation profiles in the triangular shaped plane of W7-X. In contrast to looking at chord brightness profiles of each camera individually, tomography yields a geometrically entangled resolution of the line-integrated measurements of both systems together simultaneously. As previously elaborated on in \cref{subsec:local}, the underlying mathematical problem is ill-posed, i.e. the number of free parameters, represented by the total number of pixels or voxels is far greater than the number of constraints, i.e. the line of sight intersections. Tomographic reconstructions are used to solve multidimensional geometric problems of inversion to find solutions for a system with such a finite number of information. Due to the nature of the measurement system and environment, the approach to this challenge is given by an iterative solution to the initial, in \cref{eq:tikhonov_algo} introduced \textit{Tikhonov regularisation} with more, to this particularly difficult problem advantageous algorithms. In almost all cases, but particularly for this system, $\mathbf{\widetilde{T}}$ is ill-conditioned and can not be simply inverted. Therefore, the minimization of the regularized problem for a given regularisation functional $K$ and positive coupling regularisation factor $\mu$ between $K$ and calculated solution $\vec{x}$ is much more practical. We consider $T^{\left(i,j\right)}=\widetilde{T}^{\left(i,j\right)}/\sigma^{\left(j\right)}$ and $\vec{b}^{\left(j\right)}=\mathrel{\hat{\vec{b}}^{\left(j\right)}}/\sigma^{\left(j\right)}$ for measurement uncertainty weighting purposes of absorber $j$. As pointed out before, $\mu/,K$ is used to impose \textit{a priori} knowledge of the system or desired smoothness and features on solution $\vec{x}$. For $\mu\rightarrow0$ the calculated radiation distribution is dominated by the characteristics of the measurement system, i.e. the LOS camera system geometry, while for $\mu\rightarrow\infty$ the opposite is the case and $\vec{x}$ adheres to the given regularisation functional $K$.\\%
    A simple approach to a smoothing functional $K$ is a second-order gradient minimizing regularisation using the finite-difference matrix expression for the \textit{Laplace\footnote[1]{Pierre-Simon (Marquis de) Laplace * Mar. 23, 1749 \textdagger Mar. 7, 1827} operator} $\pmb{\Delta}$.%
%
    \begin{align}%
        \begin{split}\label{eq:linearregular}%
            K=\left(\pmb{\Delta}\vec{x}\right)^{\intercal}\left(\pmb{\Delta}\vec{x}\right)\\%
            \mathbf{H}=\pmb{\Delta}^{\intercal}\,\cdot\,\pmb{\Delta}\\%
            \text{min}\,\left[\frac{1}{2}\chi^{2}+\vec{x}\,^{\intercal}\mathbf{H}\vec{x}\right]%
        \end{split}%
    \end{align}%
%
    Minimizing this finds the (a) solution with the least curvature, i.e. the smoothest gradients, which in poloidal direction already is a good fit for the expected radiation distribution. This aspect will later play a more important role when extending the \textit{Minimum Fisher regularisation} in \cref{subsec:kani}. One should note that \cref{eq:linearregular} still has to be solved iteratively to find the best $\mu$ for a provided \textit{best-fit} criteria for $\chi^{2}$. Such linear algorithms can and have been successfully extended and combined with many different methods, e.g. neural networks approaches\cite{Parker1977,Peng1993,Steriti1992}.%
%
    \section{Minimum Fisher Regularisation}\label{sec:minfisher}%
%
        At Wendelstein 7-X, the multicamera, multichannel bolometer diagnostic system performs tomography using the \textit{Minimum Fisher regularisation} and variations thereof for constructing a regularisation functional. This method has first been tested and established by Anton et al.\cite{Anton1996} and was used successfully with this system by Zhang et al.\cite{Zhang2013}. The \textit{Fisher\footnote[1]{Sir Ronald Aylmer Fisher, FRS * Feb. 17, 1890, \textdagger July 29, 1962} information} (sometimes just \textit{information}) of a probability density distribution $g\left(\vec{r}\right)$ regarding the condition of $\vec{r}$ is given by:
%
        \begin{align}%
            I\ix{F}=\int\frac{1}{g\left(\vec{r}\right)}\left(\frac{\partial\left(g\left(\vec{r}\right)\right)}{\partial\vec{r}}\right)^{2}\diff\vec{r}%
        \end{align}\label{eq:fisherinfo}%
%
        The likelihood of measuring a particular value of physical quantity in $\vec{r}$ would be described by $g\left(\vec{r}\right)$. Therefore, $I\ix{F}$ describes the amount of information such a probability distribution contains about its characteristic in $\vec{r}$. Assuming a distinctly peaked likelihood profile of $g\left(\vec{r}\right)$, one can easily see that this method finds the quasi correct $\vec{r}$ for this emissivity or yields the largest information at that specific $\vec{r}$\cite{Fisher1922}. One should note that the variance of the distribution $g\left(\vec{r}\right)$, $\sigma\ix{g}$ is bound by the inverse of the Fisher information or that its accuracy is limited by $I\ix{F}$, which is given by the \textit{Cramér–Rao\footnote[2]{Calyampudi Radhakrishna Rao, FRS * Sep. 10, 1920; Harald Cramér * Sep. 25, 1893, \textdagger Oct. 5, 1985} bound}:%
%
        \begin{align}%
            \sigma\ix{g}\ge\frac{1}{I\ix{F}}\,\,.\nonumber%
        \end{align}%
%
        In other words: the solution of the Minimum Fisher regularisation (\textit{MFR}) with the smallest information has maximum variance and is the smoothest. Furthermore, MFR has shown a good stability and robustness towards noisy data and is more suited for similarly smooth distributions, i.e. not peaked as described above\cite{Frieden1988,Frieden2012}. This is also the reason why this system can not resolve the distinctly poloidally varying emissivity in the W7-X SOL.\\%
        The very familiar approach above, especially \cref{eq:fisherinfo} is exploited to construct a tailored method for finding a smooth but well characterised radiation distribution from the one-dimensional, line integrated bolometer measurements. The second order linear regularisation of \cref{eq:linearregular} is modified to a weighted first order derivative functional by introducing a diagonal matrix $\mathbf{W}$ with $W^{\left(i,i\right)}=W^{\left(i\right)}\delta_{i,j}>0$. In the nth iterative step of the algorithm, with $\pmb{\nabla}\ix{x/y}$ the corresponding discrete vector differential operator in one dimension, $\mathbf{H}^{\left(n\right)}$ becomes:
%
        \begin{align}%
            \mathbf{H}\to\mathbf{H}^{\left(n\right)}=\pmb{\nabla}^{\intercal}\ix{x}\,\mathbf{W}^{\left(n\right)}\,\pmb{\nabla}\ix{x}+\pmb{\nabla}^{\intercal}\ix{y}\,\mathbf{W}^{\left(n\right)}\,\pmb{\nabla}\ix{y}\,\,.%
        \end{align}%
%
        Looking back at \cref{eq:fisherinfo}, one can also assume that $\diff I\ix{F}=\left(g^{\prime}\left(\vec{r}\right)\right)^{2}/g\left(\vec{r}\right)=\left(\pmb{\nabla}g\left(\vec{r}\right)\right)^{2}/g\left(\vec{r}\right)$, corresponding to the originally posed first order linear regularisation with a factor of $1/g$. Therefore, the functional operator $\mathbf{H}$ for finding $g$ with the lowest Fisher information becomes:%
%
        \begin{align}%
            \begin{split}%
                n=0\,\text{:}&\qquad\mathbf{W}^{\left(0\right)}=\mathbf{1}\,,\,\,\to\,\,\mathbf{H}^{\left(0\right)}=\pmb{\Delta}^{\intercal}\,\cdot\,\pmb{\Delta}\,\,,\\%
                n\,\ge\,1\,\text{:}&\qquad \mathbf{W}^{\left(n\right)}_{i,j}=\left\{\begin{array}{ll}%
                    \left(1/g_{i}^{\left(n-1\right)}\right)\delta_{i,j}\,,&g_{i}^{\left(n-1\right)}>0\\%
                    W\ix{max}\,\delta_{i,j}\,,&g_{i}^{\left(n-1\right)}\le0%
                \end{array}\right.\,\,.%
            \end{split}\label{eq:fisher_weight}%
        \end{align}%
%
        For $n=0$, singularity is avoided by letting $\mathbf{W}$ be the unit diagonal matrix $\mathbf{1}$ and hence $\mathbf{H}$ become the second order linear regularisation functional. At and beyond $n\ge1$, the inverse of in the previous step calculated distribution $g_{i}^{\left(n-1\right)}$ yields particular smoothness for regions of small values by adding statistical weight to their corresponding second order gradients, while for larger numbers singular local structures can be amplified. For negative $g\ix{i}^{\left(n-1\right)}$, the weights are limited by an upper limit $W\ix{max}$ that is chosen beforehand on experience with the underlying reconstruction problem. Iteration of $n$ finds the solution to the initial problem in \cref{eq:linearregular} for $\vec{x}$ by substituting $K\rightarrow\mathbf{H}^{\left(n\right)}$ until a minimum desired $\chi^{2}$ is found. Let us derive a discrete and iterative set of equations and identities that can be used for a data set $\vec{b}$, provided by the bolometer camera system to calculate the two-dimensional radiation distribution $\vec{x}$ with the minimum Fisher information $I\ix{F}$:%
%
        \begin{align}%
            \begin{split}%
                x\,\to&\, r\,,i\qquad y\,\to\,\vartheta\,,j\\%
                \pmb{\nabla}\ix{x}\,\to\,\pmb{\nabla}\ix{r}=&\frac{1}{\Delta r}\left(\mathbf{D}\ix{dia}^{\left(n_{\vartheta}\right)}-\mathbf{D}\ix{dia}^{\left(0\right)}\right)\\%
                \pmb{\nabla}\ix{y}\,\to\,\pmb{\widetilde{\nabla}}_{\vartheta}= &\frac{1}{\Delta r\Delta\vartheta}\left(\mathbf{D}\ix{dia}^{\left(1\right)}-\mathbf{D}\ix{dia}^{\left(0\right)}\right)\\%
                \to\,\mathbf{H}^{\left(0\right)}=&\pmb{\nabla}\ix{r}^{\intercal}\,\mathbf{1}^{n\ix{r}\times n_{\vartheta}}\,\pmb{\nabla}\ix{r}+\pmb{\widetilde{\nabla}}_{\vartheta}^{\intercal}\,\mathbf{1}^{n\ix{r}\times n_{\vartheta}}\,\pmb{\widetilde{\nabla}}_{\vartheta}%
            \end{split}\label{eq:fisher_algo}%
        \end{align}%
%
        The above dimensions shall correspond to the commonly used notations for the cylindrical coordinate system, i.e. $\vec{r}\rightarrow\left(r,\vartheta\right)$ and their pixel and matrix indices $\left(i,j\right)$. The discrete vector differential in radial direction $\pmb{\nabla}\ix{r}$ is written as a combination of two diagonal matrices, where $\mathbf{D}\ix{dia}^{\left(k\right)}\,\in\mathbb{R}^{n\ix{r}\times n\ix{\text{$\vartheta$}}}$ corresponds to%
%
        \begin{align}%
            \left(\mathbf{D}\ix{dia}^{\left(k\right)}\right)_{q,p}=\left\{%
            \begin{array}{ll}%
                \delta_{q,p-k}\,,&k\ge0\\%
                \delta_{q-k,p}\,,&k<0%
            \end{array}\right.\,\,,\nonumber%
        \end{align}%
%
        which for $k=0$ gives the unit matrix $\mathbf{1}^{n\ix{r}\times n_{\text{$\vartheta$}}}$. A corresponding poloidal differential operator $\pmb{\nabla}_{\vartheta}$ is slightly modified by dividing with the radial bin $1/\Delta r$ to further mitigate noise along the flux surfaces in accordance to Fisher information in \cref{eq:fisherinfo}. Using \cref{eq:fisher_algo} and \cref{eq:linearregular} yields an expression for the calculated radiation distribution in the $\left(n+1\right)$-th step:
%
        \begin{align}%
            \vec{x}\,^{\left(n+1\right)}=\left(\mathbf{T}^{\intercal}\mathbf{T}+\mu\mathbf{H}^{\left(n\right)}\right)^{-1}\,\mathbf{T}^{\intercal}\vec{b}\,\,.\label{eq:fisher_iteration}%
        \end{align}%
%
        Computationally, the inversion $\left(\cdot\right)^{-1}$ is the most costly operation of this algorithm, though using mathematically optimized and performant applications such as \textit{SuperLU}\footnote[1]{general purpose library for the direct solution of large, sparse, nonsymmetric systems of linear equations (\textit{LU factorization of matrix A in a lower triangular L and an upper triangular U, A=LU})}\cite{Li2005} makes this feasible. Iterations in $n$ are performed until:%
%
        \begin{align}
            \begin{split}%
                \left(\chi^{2}\right)^{\left(n\right)}=&\left(\mathbf{T}\vec{x}-\vec{b\,}\right)^{\intercal}\,\left(\mathbf{T}\vec{x}-\vec{b}\,\right)\le\chi^{2}\ix{min}\,\,,\\%
                &\left\|\vec{x}^{\left(n\right)}-\vec{x}^{\left(n-1\right)}\right\|\le\sigma\ix{min}\,\,,%
            \end{split}%
        \end{align}
%
        where $\chi^{2}\ix{min}$ and $\sigma\ix{min}$ are threshold values for terminating the iteration process. The smaller both of these values are set to, the longer the calculation takes and theoretically the more accurate the solution in $\vec{x}$ is according to the above, however the quality of the resulting profile is, from experience, more tied to the preparation of the input data, its noise and error level and the construction of the inversion domain. This circumstance will later be thoroughly explored when benchmarking the algorithm and setup with surrogate input radiation profiles and phantom image reconstructions in \cref{sec:phantoms}. Finally, looking back at the initial proposition of minimizing the Fisher information of the solution $\vec{x}$ with an adequate regularisation in \cref{eq:linearregular}, one easily finds that
%
        \begin{align}%
            \diff I\ix{F}\sim\frac{\left(g^{\prime}\right)^{2}}{g}\qquad\leftrightarrow\qquad\vec{x}^{\intercal}\,\mathbf{H}\,\vec{x}\propto\frac{1}{g}\left(\nabla g\right)^{2}\nonumber%
        \end{align}%
%
        when identifying the sought after probability or emissivity distribution $g\left(\vec{r}\right)$ and remembering the weighting in $\mathbf{H}$ through $\mathbf{W}$.\\%
        The MFR can and was used successfully with the bolometer camera system at W7-X. However, different improvements perhaps have to be made to the regularisation functional weighting. Due to the nature of the diagnostic setup, the influence and behaviour of the measurement environment, i.e. the entirety of the plasma device and construction thereof, the input $\mathbf{T}$ is rather prone to errors. Particularly the resulting tomogram in $\vec{x}$ is very sensitive to deviations between the assumed and actual line of sight geometry, which in fact is not reflected by a larger $\chi^{2}$ or $\sigma\ix{min}$. Thorough evaluation of the sensitivity of both parameter coefficient matrix $\mathbf{T}$ and resulting reconstruction to variations in the camera geometry will be performed in \cref{sec:geomimpact}. In addition to that, further exploiting \textit{a priori} knowledge about plasma radiation distributions, especially incorporating such into \cref{eq:fisher_weight} and deriving from that will be beneficial to the overall quality of the bolometer tomography.%
%
        \subsection{Radially Dependent Anisotropy}\label{subsec:kani}%
%
            Modifications to $\mathbf{H}^{\left(n\right)}$ on the basis of \textit{radial radiation distribution anisotropy} will be made to improve the performance of the MFR in regard to extracting emissivity concentrations around X-points and magnetic islands from the bolometer measurements in the tomogram. The previous chapter in particular established their importance to the scientific challenges posed by the real time feedback experiments and the occurrence of (partial) radiation detachment. Let us write a discrete anisotropy factor $k\ix{ani}^{\left(i\right)}$ that applies varying weight to the poloidal differential operator $\pmb{\widetilde{\nabla}}_{\text{$\vartheta$}}$:
%
            \begin{align}%
                k\ix{ani}^{\left(i\right)}=\left\{\begin{array}{ll}%
                    \left[\left(\tan^{-1}{\left(\frac{10}{N\ix{S}}\left(1-\left(i-N\ix{T}\right)\right)\right)}+\frac{\pi}{2}\right)\times\right.\\%
                    \quad\left.\frac{2}{\pi}\left(\frac{k\ix{core}}{2}+k\ix{edge}\right)\right]-1&,\,k\ix{core}>k\ix{edge}\\%
                    \left[\tan^{-1}{\left(\frac{10}{N\ix{S}}\left(i-N\ix{T}\right)\right)+\frac{\pi}{2}}\right]\times\\%
                    \quad\frac{1}{\pi}\left(k\ix{edge}-k\ix{core}\right)+k\ix{core}&,\,\text{else}%
                \end{array}\right.\,\,.%
            \end{align}\label{eq:kani}%
%
            This representation applies for both favouring poloidal anisotropy inside and outside a threshold \textit{target} radial bin $N\ix{T}\in\left[0,n\ix{r}\right]$. A parameter $N\ix{S}\in\left[1,n\ix{r}-1\right]$ describes the width of the transition from smooth to anisotropic weighting and vice versa, while the values of $k\ix{core}\,,k\ix{edge}\ge0$ represent the respective desired weight in those areas. Two examples for a radial representation of $k\ix{ani}^{\left(i\right)}$ can be found in \cref{fig:kanicomparison}, where each is derived for the same core and edge factors, as well as threshold and width parameters. The profiles are, if not within one or two bins of $N\ix{T}\pm N\ix{S}/2$, generally flat and can be assumed constant. Only around the threshold $k\ix{ani}$ yields, depending on the width $N\ix{S}$, a more or less smooth transition between $k\ix{core}$ and $k\ix{edge}$. For very small $N\ix{S}\le2$, a linear progression between core and edge anisotropy profile is expected.\\%
%
            \begin{figure}[t]%
                \centering%
                \includegraphics[width=0.4\paperwidth]{%
                    content/figures/chapter4/%
                    kAni_profile_comparison.pdf}%
                \caption{Comparison between two radial anisotropy profiles $k\ix{ani}$ for $k\ix{core}>k\ix{edge}$ and $k\ix{core}<k\ix{edge}$ each. The bottom abscissa notes the radial bin number, i.e. index and the top the corresponding radius in the $1.3^{2}V\ix{P}$ inflated cross-section. A grey line indicates the separatrix location $r\ix{a}$. Both lines are calculated using \cref{eq:kani} with $k\ix{edge}=0.5$, $k\ix{edge}=10$ and vice versa, $N\ix{T}=15$ and $N\ix{S}=2$ with a domain size of $n\ix{r}=20$.}\label{fig:kanicomparison}%
            \end{figure}%
%
            The regularisation operator and functional can now be written as follows:%
%
            \begin{align}%
                \begin{split}%
                    \mathbf{K}\ix{ani}=&\left(\frac{1}{k\ix{ani}^{\left(i\right)}}\delta\ix{i,j}\right)^{\left(i,j\right)}\mathrel{\hat{=}}\,\frac{1}{k\ix{ani}}\mathbf{D}\ix{dia}^{\left(0\right)}\,\,,\\%
                    \pmb{\hat{\nabla}}_{\vartheta}=\mathbf{K}\ix{ani}\,\pmb{\widetilde{\nabla}}_{\vartheta}=&\frac{1}{\Delta r\Delta\vartheta}\,\mathbf{K}\ix{ani}\left(\mathbf{D}\ix{dia}^{\left(1\right)}-\mathbf{D}\ix{dia}^{\left(0\right)}\right)\,\,,\\%
                    \to\,\mathbf{H}^{\left(n\right)}\ix{ani}=&\,\pmb{\nabla}^{\intercal}\ix{r}\,\mathbf{W}^{\left(n\right)}\,\pmb{\nabla}\ix{r}+\pmb{\hat{\nabla}}^{\intercal}_{\vartheta}\,\mathbf{W}^{\left(n\right)}\,\pmb{\hat{\nabla}}_{\vartheta}\,\,.%
                \end{split}\label{eq:anisotropy}%
            \end{align}%
%
            The resulting $\mathbf{H}^{\left(n\right)}\ix{ani}$ differs only in an additional weight to the poloidal differential from the underlying MFR algorithm. Note that the radial operator $\pmb{\nabla}\ix{r}$ is not changed, since deliberately enforcing individual poloidal emissivity hotspots would inevitably reduce the robustness of the calculated results and question the credibility of the anisotropy method. It is assumed that, at least in the plasma core - read: not exclusively inside the separatrix - radiation is distributed smoothly along magnetic field lines. This way, the tomography algorithm is only allowed and not constrained to find localized structures in the radiation distribution for a given set of $\left(k\ix{core},k\ix{edge},N\ix{T},N\ix{S}\right)$. This approach was also applied, particularly at W7-X by Zhang et al.\cite{Zhang2021_2}, with the focus on a novel regularisation method that will be introduced below and used for comparison later. One should note that also Fuchs et al.\cite{Fuchs1994} previously proposed a local anisotropy weighting, where there is a second set of weights introduced:%
%
            \begin{align}%
                I\ix{F}=\int\frac{%
                    \left(k\ix{ani}^{r}\partial^{r}g\left(\vec{r}\right)\right)^{2}+\left(k\ix{ani}^{\vartheta}\partial^{\vartheta}g\left(\vec{r}\right)\right)^{2}}{%
                    g\left(\vec{r}\right)}\diff\vec{r}\,\,,\nonumber%
            \end{align}%
%
            of which the in \cref{eq:anisotropy} defined equations are an extension for $k\ix{ani}^{r}=1$. One could argue that the radial variation of $k\ix{ani}^{\left(i\right)}$ in \cref{eq:kani} yields the same effect. However, this will not be further explored in this work.\\%
            This concludes the introduction of a deliberately tailored tomographic reconstruction method for the purpose of finding the two-dimensional emissivity distribution using the multicamera bolometer diagnostic at Wendelstein 7-X. Its respective application will be carried out later, after the sensitivity to the underlying geometry (perturbations) is thoroughly evaluated and potential errors or artefacts can be adequately assessed. Additional algorithms and comparisons thereof will not be highlighted here but have been in previous works\cite{Anton1996,Giannone1997,Li2021}, see also \cref{apx:rgs} and following.%
%
    \section{Camera Geometry Sensitivity Towards Line of Sight Perturbations}\label{sec:geomimpact}%
%
        One of, if not the most important aspect of a tomographic inversion with the number of absorbers $n\ix{l}\ll n\ix{r}n_{\text{$\vartheta$}}$ the number of pixels or voxels, is the accuracy and robustness of the underlying geometry coefficient input matrix $\mathbf{T}\in\mathbb{R}\,^{n\ix{r}\times n_{\text{$\vartheta$}}}$. Therefore, finding the local sensitivity of each LOS for a given inversion domain, i.e. size of the plasma $V\ix{P}$ and number of radial and poloidal intersections $n\ix{r}\times n_{\text{$\vartheta$}}$ is first studied.\\%
        In \cref{subsec:local} the idea of finding the individual geometric contribution of the cameras absorbers to $\mathbf{T}$ was schematically outlined and will now be revisited and pedantically exercised in order to build an adequate foundation for the following tomography. First one has to remind themselves that both the absorber-pinhole arrangement and the construction of the bolometer cameras are far from two-dimensional and each yield a significant toroidal extension and are only projected onto a singular plane at $\varphi\ix{tor}=$\SI{108}{\degree} for sake of comprehensibility. The diagnostics geometry has already been presented in detail in \cref{subsec:losgeometry}, where it was shown that both VBCl/r and HBC are tilted in the same direction by $\sim$\SI{5}{\degree} due to the intrinsic port geometry of W7-X. In \cref{eq:etendue_woline} and \cref{eq:emissivity} the idea of infinitesimal absorber and aperture subdivisions as well as differential etendues was introduced. However, in practice this challenge can not be met with analytical means and has to be solved discretely. Therefore, both pinhole and detector are divided into $n\ix{A}$ and $n\ix{M}$ smaller fractions, respectively. Let us assume a volume for tomographic inversion of $V\ix{P}$ divided into $N\ix{r}\cdot N\ix{$\vartheta$}\cdot N\ix{$\varphi$}$ voxels, i.e. $1.3^{2}\,V\ix{LCFS}$ into $30\times120\times10$ partial volumes. Let $T^{\left(i,j,k\right)}_{M}$ be the local sensitivity of absorber $M$ to radiation in voxel $v^{\left(i,j,k\right)}$. With $L^{\left(i,j,k\right)}_{p,q}$ the LOS section length from differential absorber $\diff A_{M}^{\left(p\right)}$ through differential aperture $\diff A_{A}^{\left(q\right)}$ (see \cref{eq:diff_etendues}) inside voxel $v^{\left(i,\,j,\,k\right)}$, the geometrical contribution to pixel $p^{\left(i,j\right)}$ can be written as:
%
        \begin{align}%
            \begin{split}%
                T^{\left(i,j\right)}_{M}=&\sum_{k=1}^{N\ix{$\varphi$}}T^{\left(i,j,k\right)}_{M}=\sum_{k=1}^{N\ix{$\varphi$}}\left(\int_{M}L^{\left(i,j,k\right)}_{M}\diff\widetilde{K}_{M}\right)\\%
                =&\sum_{k=1}^{N\ix{$\varphi$}}\sum_{p,\,q}^{n\ix{M},\,n\ix{A}}L^{\left(i,j,k\right)}_{p,q}\left(\frac{\cos\left(\alpha\right)\cos\left(\beta\right)}{2\pi d^{2}}\right)^{\left(p,q\right)}\diff A_{M}^{\left(p\right)}\diff A_{A}^{\left(q\right)}\,\,.%
            \end{split}\label{eq:discrete_emissivity}%
        \end{align}%
%
        The supercript $\left(p,q\right)$ on the fraction is used for simplicity and denotes that the incident angles $\alpha$ and $\beta$ between LOS and face normals, as indicated in \cref{fig:etendue}, as well as the distance $d$ between the centers of $\diff A_{M}^{\left(p\right)}$ and $\diff A_{A}^{\left(q\right)}$ have to be calculated for each of the fractional absorber and aperture. The resulting \tilt{etendue} $\mathbf{T}_{M}^{\left(i,j\right)}$ for pixel $p^{\left(i,j\right)}$ is in units of $\left[\text{m}^{3}\right]$ and corresponds to the contribution of the local emissivity in voxel $v^{\left(i,j,k\right)}$ to absorber $M$ through its respective pinhole.%
%
        \subsection{Absorber and Aperture Segmentation}%
%
            \begin{figure}[t]%
                \centering%
                \includegraphics[height=0.3\paperheight]{%
                    content/figures/chapter4/MFR/%
                    detector_splitting.pdf}%
                \caption{Example for variations of subdividing the individual detectors and apertures with increasing number of compartments. Shown on the left is a method of rectangular splitting, on the right using a simple triangulation procedure for $N\in\left\{2,4,8\right\}$ the number of compartments. Each individual subcompartment is coloured differently.}\label{fig:detector_splitting}%
            \end{figure}%
%
           The first challenge is to find and verify the etendues for all bolometer detectors. Two common segmentation methods for dividing the absorbers and apertures into $N\in\left\{2,4,8\right\}$ parts are presented in \cref{fig:detector_splitting}. A comparison shows the varying location of the face (area) center and size of the individual subdivisions for an exemplary object, i.e. pinhole or detector in a stylised perspective. On the left, a rectangular interpolation is shown next to a common \textit{Delaunay\footnote[1]{Boris Nikolajewitsch Delone * Mar. 15, 1890 \textdagger Jul. 17, 1980} triangulation} on the right. Resulting subdivisions of the interpolation are constructed by dividing the longer and shorter - if applicable - sides of the respective face alternatingly $m$ times in that order, which yields $N=2^{m},\,m\in\mathbb{N}$ parts. The triangulation is done using built-in functions for such Delaunay algorithms, e.g. Pythons \textit{SciPy\footnote[2]{a collection of mathematical algorithms and functions built on the NumPy extension}} toolkit, limited to the same amount of segmentations as on the left for sake of comparability. Simply put, this method finds the set of triangles for a given set of discrete points $\mathbf{P}$ inside its corresponding \tilt{convex hull\footnote[3]{or convex envelope of a shape is the smallest convex set that contains it entirely}} where no circumcircle of any triangle contains another point of $\mathbf{P}$, while a maximum of all individual minimum angles of those triangles is achieved. The presented rectangular interpolation finds $N$ parts of same size and shape for a given splitting of $m$ times. However, the segmented face centres are spread rather unevenly with respect to the full absorber or pinhole, which similarly affects the set of $\left(\alpha,\beta,d\right)$. For a triangulation of $N\ge4$, which is a common setting, those are spread more uniformly, while $\diff A\ix{X}^{\left(i\right)}$ is not necessarily constant. A homogenous distribution though is expected to yield a more accurate result due to the previously discussed locally varying transmission between aperture and pinhole in \cref{fig:etendue_transmission} and \cref{eq:etendue_woline}. \cref{eq:discrete_emissivity} only computes the etendue for each fractional detector and pinhole combination once based on the individual differential area center, hence the accuracy of $\mathbf{T}$ is assumed to benefit from a diversified segmentation, which in fact will be verified below. Based off of those splitting methods, comparisons in LOS geometry, etendues, volumes and finally sensitivity matrices will be performed.\\%
%
            \begin{figure}[t]%
                \centering%
                \begin{subfigure}{0.47\textwidth}%
                    \includegraphics[width=\textwidth]{%
                        content/figures/chapter4/MFR/%
                        viewcones_ch15_comp.pdf}%
                    \caption{}%
                \end{subfigure}%
                \hfill%
                \begin{subfigure}{0.47\textwidth}%
                    \includegraphics[width=\textwidth]{%
                        content/figures/chapter4/MFR/%
                        viewcones_ch15_16_sN8.pdf}%
                    \caption{}%
                \end{subfigure}%
                \caption{\textbf{(a)} Comparison for example LOS cones for channel 14 and 16 of the HBC with rectangular splitting at (\textcolor{blue}{blue}) $N=2$ and (\textcolor{green}{green}) $N=8$ for both the detector and pinhole. Shown in both cases is the \textit{full} LOS cone in \textcolor{red}{red}, created by projecting detector corners through respective opposite side corners of the pinhole, and the sum of the individual $N^{2}$ cones from the center of the detector elements through the corners of the pinhole rectangles in a different colour. \textbf{(b)} Line of sight cones from $N=8$ split detectors and pinholes of neighbouring channels 15 and 16 of the HBC in \textcolor{red}{red} and \textcolor{blue}{blue}. In both images, all of it is enclosed by a \textcolor{gray}{grey} mesh, representing the previously introduced $1.3^{2}V\ix{P}$ hull of the extrapolated magnetic flux surfaces, wherein the LOS cones are defined and their volume calculated.}\label{fig:viewconeComparison_N2_N8}%
            \end{figure}%
%
            In \cref{fig:viewconeComparison_N2_N8} a visual comparison between different segmentation resolutions and their three-dimensional coverage is shown for a typical $1.3^{2}V\ix{P}$ inflated equilibrium flux surface domain with toroidal expansion. The left image \textbf{(a)} shows the impact of dividing both pinhole and detector into two and eight parts respectively, while projecting the different lines constructed through the individual center and corner points onto the opposite side of the torus. They form a trapezoidal polyeder whose convex hull is illustrated as a cone using different colours. Both of the red boxes are achieved by finding the full extension of the LOS from the noted absorbers and corresponding aperture, i.e. tracing each corner of the detector face through any from the pinhole onto the enclosing plane and finding their convex hull, which takes also the lowest transmissible areas or geometry of the etendue into account - see \cref{fig:etendue_transmission}. An infinitesimal splitting finds the full transmission for this construct, while one has to keep in mind that this theoretically takes $\ge 2N^{2}\cdot n\ix{$\varphi$}n\ix{r}n\ix{$\vartheta$}$ iterations to find $L^{\left(i,j,k\right)}_{p,q}$ the intersection of LOS and voxel without prior knowledge about its geometry. For $N\rightarrow10^{3}$ this alone requires $3.6\times10^{9}$ operations, which justifies the reduction of number of segments to eight each in this case. One immediately finds that splitting both absorber and aperture into eight segments each (\textcolor{green}{green}) yields a significantly improved coverage of the full LOS cone compared to $N=2$ (\textcolor{blue}{blue}). The latter has a spatial discrepancy to the enclosing projection at the opposite of the torus of $>\diff\varphi\diff\vartheta r\ix{a}$, while for $N=8$ this is reduced to $<1/2\diff\varphi\diff\vartheta r\ix{a}$. Since the depicted channels are 14 and 16, which are positioned almost perfectly symmetrical around the center detector and camera axis, it is assumed that this visual approach and conclusion are adequately supported. The left \cref{fig:viewconeComparison_N2_N8}\textbf{:(b)} shows neighbouring LOS cones of channel 15 and 16 for $N=8$. No gap is found between them, while they also do not cover the same volume twice or overlap at any point. In fact, in \cref{fig:los_grid_emiss} one can already see that the poloidal LOS coverage in the plane of the bolometer, for a set of $n\ix{r}=30$ and $n\ix{$\vartheta$}=150$ at $N=8$, particularly from the HBC is very good and features no local voids or irregularities. One remembers that the systems spatial resolution at the magnetic axis is \SI{5}{\centi\meter}, which is in agreement with this assessment of the LOS cones.\\%
%
            \begin{figure}[t]%
                \centering%
                \includegraphics[width=0.9\textwidth]{%
                    content/figures/chapter4/MFR/%
                    kfactors_volumes_interpolations_comp.pdf}%
                \caption{Collection of etendues and LOS cone volumes of all bolometer camera channels for varying interpolation methods and amounts. The left images include the HBC, while the right shows channels from the VBC, which is made up out of the left and right sub-array with individual pinholes. Rectanglular interpolation results are depicted with squares, triangulation results with triangles and different colours for $N=\left\{2,4,8\right\}$. The top images show the collective etendue per channel and the lower display the total LOS volume.}\label{fig:kfactors_comparison_splitting}%
            \end{figure}%
%
            Figure \ref{fig:kfactors_comparison_splitting} presents the impact of varying segmentation methods and $N$ on the integrated LOS etendues and cone volumes. In \cref{fig:volume_channels}, the commonly used $V\ix{M}$ and $K\ix{M}$ have been shown, which are produced for the same set of $N=8$ and $n\ix{r},n\ix{$\vartheta$},n\ix{$\varphi$}=30,150,10$, using a triangulating pinhole and absorber segmentation. The values of $V\ix{M}$ are calculated, as described above for the full expansion cone, by tracing all of the individual sub-absorbers center through the corners of each pinhole fraction, constructing their convex hull and finding its volume. Equation \ref{eq:discrete_emissivity} and following already generally describes how the individual channels sensitivity is computed, while $K\ix{M}$ is just the sum over all pixels of $\mathbf{T}\ix{M}^{\left(i,j\right)}$. Overall, both segmentation methods and all degrees of refinement are qualitatively in very good agreement with each other. In this plot, differences are almost not noticeable, besides deviations in LOS volume of the outermost HBC and inboard VBCl channels for a triangulation of $N=\left\{2,4\right\}$. Those discrepancies are small compared to their respective maximum, however at the lower $V\ix{M}$ they become non-negligible, which therefore requires a triangulation of $N\ge5$ for validity. The previously noted - see \cref{subsec:strahlchord} - minor asymmetry in the HBC detector arrays geometry can not be seen here, although a much more thorough evaluation of the camera constructions impact on $\mathbf{T}$ will be done further below.\\%
%
            \begin{figure}[t]%
                \centering%
                \includegraphics[width=0.9\textwidth]{%
                    content/figures/chapter4/MFR/%
                    compare_emissivities3D.pdf}%
                \caption{Comparison between the total etendue or local sensitivity of the LOS system of the bolometer for different interpolation resolutions and methods. For both, the absolute logarithmic value of the difference between the sensitivities is plotted on the respective mesh grid that has been used to calculate the etendues. A Poincaré plot for the associated magnetic field is superimposed, showing the last closed flux surface and enclosing magnetic islands at that toroidal location. This is enclosed by a cut through the first machine wall. The left image shows the difference between $N=2$ and $N=8$ rectangular interpolations, while the right similarly displays the contrast between the prior and triangulations of the same resolution.}\label{fig:2Detendues_comparison_splitting}%
            \end{figure}%
%
            Finally, \cref{fig:2Detendues_comparison_splitting} shows the variation between individual pixel distributions of $\mathbf{T}$ for different refinements and segmentation methods for the same grid mesh as introduced for \cref{fig:los_grid_emiss} of all bolometer camera arrays combined. The left figure presents the absolute difference on a logarithmic scale between $N=2$ and $N=8$ for a rectangular subdivision across the inversion domain. With respect to the integrated etendue and LOS volume per channel with an order of magnitude of $\sim$\SIrange{10}{e4}{\cubic\milli\meter}, a variance of \SIrange{0.5e-10}{e-14}{\cubic\milli\meter} is of no importance to the initial challenge of tomographic reconstruction of bolometer measurements. Looking back at the central MFR iterative expression for the radiation profile $x^{\left(n\right)}$ in \cref{eq:fisher_iteration}, a variation of \SI{e-11}{\cubic\milli\meter} in $\mathbf{T}$ becomes negligible when evaluating the inversion since the condition of this expression is no longer only dominated by the geometry matrix. Particularly, assuming the variation or perturbance to $\mathbf{T}$, $\Delta\mathbf{T}$ be structured and not infinitesimal, which is very much true for this example, Ghaoui et al.\cite{Ghaoui2002} find that the error in the inversion is limited by the product of condition and euclidean norm $\kappa\left(\Delta\mathbf{T}\right)\cdot\left(\left\|\Delta \mathbf{T}\right\|/\left\|\mathbf{T}\right\|\right)$. Since $\kappa\left(\Delta\mathbf{T}\right)$ is in fact potentially large but its norm significantly smaller, this upper limit underlines that such a difference in local sensitivity, i.e. variations in $N$ correspond to perturbations of appropriate dimension and do not catastrophically change the results in $x^{\left(n\right)}$. Looking at the right plot in \cref{fig:2Detendues_comparison_splitting}, which shows the difference between triangulation and rectangular subdivision at $N=8$, one finds a very similarly scaled image. Variations in colour can be attributed to its changed range of \SI{e-11}{e-15}. The same arguments as above towards the impact of changing between those segmentation methods on the reconstruction apply here as well.\\%
            The segmentation of aperture and absorber and therefore finding a discrete, pixel or voxel based representation of the geometry coefficient matrix $\mathbf{T}$ containing the etendues of all individual bolometer camera channels is adequately achieved by either triangulation or rectangular splitting with a resulting number of subdivisions $N>4$, given a common setting of reconstruction grid expansion and refinement (i.e. $1.3^{2}\times150\times20\times10$, see above). From here on, we will assume that any given etendue, LOS volume or results derived therefrom are computed using a detector and pinhole triangulation with $N=8$.\\%
%
        \subsection{Line-of-Sight Geometry Perturbations}\label{subsec:geompertub}%
%
        In the following paragraphs, a variation of perturbations to the presented geometry in \cref{subsec:losgeometry} and its effect on the corresponding parameter coefficient matrix $\mathbf{T}$ in \cref{fig:los_grid_emiss} will be discussed. This is done to test the robustness of the outlined approach to deviations in the designed construction and investigate upon the minor asymmetry in the forward calculated chord brightness profile from the otherwise ideally symmetrical STRAHL radiation distribution.%
%
        \subsubsection*{Centered Aperture and Camera Array}%
%
%           new center point HBCm aperture [-1.99965 6.15283 -1.7349]
            In a complex and large machine that experiences strong thermal effects during preparation and experiments such as Wendelstein 7-X, changes in the actual geometry of the diagnostic in comparison to the initially proposed design are not uncommon and in fact expected. Furthermore, this potential displacement is not constant either and can vary over multiple experimental campaigns. Hence, a measurement of all pinholes for HBC and VBCl/r was conducted before the very first plasma and after the last experimental campaign OP1.2b. The provided benchmark data revealed that the HBC camera aperture, i.e. its \textit{barycentre}, is displaced off the center of the machines vertical axis, which ideally is aligned also with the magnetic axis, by a few millimetres at $\vec{r}=\left\{\right.$\SI{-1.99}{\meter}, \SI{6.152}{\meter}, \SI{-1.735}{\milli\meter}$\left.\right\}$. The LOS geometry of the HBCm is assumed to be symmetrical and cover the plasma area evenly, thus potentially introducing a significant perturbation by this shift to the inversion. Hence, a first test of the robustness of $\mathbf{T}$ is conducted by finding a hypothetical, centred at $z=0$ horizontal camera that otherwise has the same geometry as the original HBC. Corresponding parameters and forward model chord brightness profiles are compiled accordingly.\\%
%
            \begin{figure}[t]%
                \centering%
                \begin{subfigure}{0.4\textwidth}%
                    \includegraphics[width=\textwidth]{%
                        content/figures/chapter4/MFR/%
                        LoS_comparison_centered_sN8.pdf}%
                    \caption{}%
                \end{subfigure}%
                \hfill%
                \begin{subfigure}{0.55\textwidth}%
                    \includegraphics[width=\textwidth]{%
                        content/figures/chapter4/MFR/%
                        compare_centered_emissivities3D.pdf}%
                    \caption{}%
                \end{subfigure}%
                \caption{\textbf{(a)} Comparison of center LOS of the (black) original HBC geometry and an adjusted (\textcolor{red}{red}) geometry, where the pinhole and respective camera array have been centred around $z=0$, the vertical dimension of the W7-X coordinate system. \textbf{(b)} Impact of the geometry changes in (a) to the collective local sensitivity of the horizontal bolometer camera, calculated and portrayed as in \cref{fig:2Detendues_comparison_splitting}.}\label{fig:geometry_change_centered}%
            \end{figure}%
%
            A comparison between the ideal and actual geometry with respect to the viewed plasma volume, as well as its impact on the local sensitivity in the triangular plane is shown in \cref{fig:geometry_change_centered}. The left set of \textcolor{red}{red} LOS rays is produced by tracing from the center of the detector through the center of the respective aperture. In this case, the new pinhole is constructed by shifting its center to $z=0$ and tilting it so that its face normal points towards $\left\{0, 0, 0\right\}$. A resulting rotation and transposition between this and the original aperture is similarly applied to each detector of the horizontal camera array. The minor differences between the black and \textcolor{red}{red} LOS are, as expected, constant across the fan. Variations in etendue between the shifted and as-designed HBC on the right are roughly of same order of magnitude as the previous comparisons for selected segmentation methods in \cref{fig:2Detendues_comparison_splitting}. However, its distribution is aligned with the orientation and geometry of the fan and features up-down symmetric, alternating structures in level of perturbation. The largest discrepancy between the original and hypothetical camera geometry coefficients can be found on both sides, closest to the aperture, along the edge of the reconstruction area and inboard magnetic islands.\\%
%
            \begin{figure}[t]%
                \centering%
                \includegraphics[width=0.5\textwidth]{%
                    content/figures/chapter4/forward_int/%
                    forward_chord_HBCm_00091_00092_00093_00094_min_EIM_beta000_centered_apt_sN8_30x20x150_1.35.png}%
                \caption{Forward integrated chord brightness for the HBC at different $f\ix{rad}$ for previously presented STRAHL results in \cref{fig:chord_forward_exp_vs_STRAHL} using the adjusted, hypothetical camera geometry from \cref{fig:geometry_change_centered}.}\label{fig:forward_intSTRAHL_centered}%
            \end{figure}%
%
            Based off of the results in \cref{fig:geometry_change_centered} and the approach presented in \cref{eq:forward_lint}, one can easily now evaluate the impact of said displacement to the chord brightness profiles measured by the horizontal bolometer. Using the same STRAHL results that have been discussed in \cref{sec:strahlmodel}, i.e. poloidally symmetrical radiation distributions and the new geometry coefficients $\mathbf{T}$ for a centred and upright HBC pinhole, this is achieved in \cref{fig:forward_intSTRAHL_centered}. As before, this image features a semi-transparent mirror image of the corresponding other half of the profile superimposed on each side for all radiation fraction levels. It has already been established that the original geometry yields asymmetric results, which is also very clearly true for this hypothetical camera arrangement. In fact, in direct comparison to the initial $\sim$\SIrange{2}{5}{\kilo\watt\per\cubic\meter} difference between the measured brightness along the separatrix before, the variation around $\pm r\ix{a}$ is significantly increased. For all $f\ix{rad}$, the profile is not only different at its peak around the LCFS but also at the center. At 100\% radiation fraction, the brightness is shifted towards $r=0$ and about \SI{30}{\kilo\watt\per\cubic\meter} higher on the lower side than on the top, i.e. $+1.0r\ix{a}$. This shift and increase can be found for all lower $f\ix{rad}$, however at a correspondingly smaller level. In contrast to \cref{fig:chord_forward_exp_vs_STRAHL}, the two local maxima around the separatrix differ also qualitatively, where the lower side peak is very sharp, and the other is more akin to the previous shape.\\%
            The portrayed changes to the camera geometry have shown to yield no more symmetric forward modelling results than the corresponding original LOS. This particular hypothetical generally yields a greater level of discrepancy between each channel and its respective counterpart than is the case for an unperturbed HBC. Therefore, the initial asymmetry in chord brightness can not be attributed to this deviation and not be corrected by its resulting transformation. However, this perturbation to the etendue in $\mathbf{T}$ has produced large changes in the brightness profiles and thereby underlined the importance of camera geometry towards the experimental measurements and the tomographic reconstruction's adequacy.%
%
        \subsubsection*{Virtual Horizontal Bolometer Camera}%
%
            \begin{figure}[t]%
                \centering%
                \begin{subfigure}{0.4\textwidth}%
                    \includegraphics[width=\textwidth]{%
                        content/figures/chapter4/MFR/%
                        LoS_comparison_artificialHBCm_sN8.pdf}%
                    \caption{}%
                \end{subfigure}%
                \hfill%
                \begin{subfigure}{0.55\textwidth}%
                    \includegraphics[width=\textwidth]{%
                        content/figures/chapter4/MFR/%
                        compare_artificialHBCm_emissivities3D.pdf}%
                    \caption{}%
                \end{subfigure}%
                \caption{\textbf{(a)} Comparison of the (black) original HBC LOS fan and (\textcolor{red}{red}) changed geometry, where the entire camera array has been rotated so that all channels lie in one plane that is vertically upright and centred around $z=0$, the vertical dimension of the W7-X coordinate system. \textbf{(b)} Impact of the geometry changes in (a) to the collective etendues of the horizontal camera.}\label{fig:geometry_change_artificial}%
            \end{figure}%
%
            Continuing the previous geometric variation, not only the intrinsic rotation within the as-designed HBCm camera array is corrected, but the entire LOS fan rotated and transpositioned to $z=0$ and into an upright orientation. The resulting artificial projections in \textcolor{red}{red} are shown in \cref{fig:geometry_change_artificial} in comparison next to the original horizontal camera in black. In addition to the shift of aperture and detector assembly to be centred around the horizontal axis of the machine, both are rotated individually into an equally perpendicular position to the latter. The new, centred fan, constructed by adjusting each absorber-pinhole trace into one plane - see the previous geometric perturbation above -, is rotated so that their two-dimensional projections in $\left(x,y\right)$ are pointing towards $\left\{0,0\right\}$. The aperture is also rotated into a position where its longer side, i.e. in direction of the detector array, is now also upright. Effectively, only a small transposition and rotation around the axis between pinhole center and device origin have been done and the size of the absorbers themselves, their nearest neighbour distance and respective position within the camera were not altered. \autoref{fig:geometry_change_artificial}:\textbf{(b)} yields a level of perturbation similar to earlier at up to \SI[per-mode=reciprocal]{e-11}{\per\cubic\milli\meter}, with similar structures and distribution as in \cref{fig:2Detendues_comparison_splitting}. The largest change in local sensitivity can be found, again, right in front of the aperture at the tip of the triangular plane and around the outboard magnetic islands.\\%
%
            \begin{figure}[t]%
                \centering%
                \includegraphics[width=0.6\textwidth]{%
                    content/figures/chapter4/forward_int/%
                    forward_chord_HBCm_00091_00092_00093_00094_min_EIM_beta000_artificialHBCm_sN8_30x20x150_1.35.png}%
                \caption{Forward integrated chord brightness for the HBC at different $f\ix{rad}$ for previously presented STRAHL results in \cref{fig:chord_forward_exp_vs_STRAHL}, using the theoretical camera geometry of \cref{fig:geometry_change_artificial}.}\label{fig:forward_intSTRAHL_artificial}%
            \end{figure}%
%
            Consistent with the earlier geometric evaluations, the forward model integrated chord brightness for the STRAHL profiles from \cref{sec:strahlmodel} for varying levels of $f\ix{rad}$ is calculated and presented in \cref{fig:geometry_forward_symmetric}. This plot is perfectly symmetric across all radiation fractions around $r=0$, underlined by no visible, semi-transparent mirror lines on either side. With respect to the initial results, the peak brightness for $f\ix{rad}=1$ is increased to \SI{260}{\kilo\watt\per\cubic\meter} and significantly more localized. Furthermore, all lower levels of radiation and their maxima in particular are shifted towards the separatrix and further inside. Otherwise, the shown profiles are qualitatively very similar to the ones before.\\%
%
        \subsubsection*{Aperture Displacement}%
%
            \begin{figure}[t]%
                \centering%
                \begin{subfigure}{0.3\textwidth}%
                    \includegraphics[width=\textwidth]{%
                        content/figures/chapter4/MFR/%
                        LoS_comparison_tilted_neg3_sN8.pdf}%
                    \caption{}%
                \end{subfigure}%
                \hspace*{0.5cm}%\hfill%
                \begin{subfigure}{0.3\textwidth}%
                    \includegraphics[width=\textwidth]{%
                        content/figures/chapter4/MFR/%
                        LoS_comparison_tilted_pos3_sN8.pdf}%
                    \caption{}%
                \end{subfigure}%
                \caption{\textbf{(a), (b)} Comparison of the (black) vertically upright and $z=0$ centred (hypothetical) HBC LOS fan - see \cref{fig:geometry_change_artificial} - and (\textcolor{red}{red}) a (a) $-2^{\circ}$ (upward) and (b) $+2^{\circ}$ (downward) respectively tilted geometry, where the array has been rotated around the normal of the LOS plane.}\label{fig:geometry_change_tilted}%
            \end{figure}%
%
            \begin{figure}[t]%
                \centering%
                \begin{subfigure}{0.47\textwidth}%
                    \includegraphics[width=\textwidth]{%
                        content/figures/chapter4/MFR/%
                        compare_tilt_neg3_emissivities3D.pdf}%
                    \caption{$-2^{\circ}$}%
                \end{subfigure}%
                \hfill%
                \begin{subfigure}{0.47\textwidth}%
                    \includegraphics[width=\textwidth]{%
                        content/figures/chapter4/MFR/%
                        compare_tilt_pos3_emissivities3D.pdf}%
                    \caption{$+2^{\circ}$}%
                \end{subfigure}%
                \caption{\textbf{(a), (b)} Impact on the collective etendues of the horizontal bolometer camera for the theoretical geometries shown in \cref{fig:geometry_change_tilted}, likewise respectively for (a) $-2^{\circ}$ and (b) $+2^{\circ}$ tilts.}\label{fig:emiss_change_tilted}%
            \end{figure}%
%
            As it was already mentioned above, the entire machine and therefore also the integrated bolometer multicamera system is subject to large scale thermal drifts and mechanical stresses, i.e. cooling, vacuum pumping and the high temperature experiments themselves, which inevitably impact their geometry. Before the very first plasma and after the last experiment in operation phase OP1.2b, during which the measurement results presented in this work were produced, such an automated, high precision measurement of the position of all HBC and VBCl/r pinholes was conducted. There it was found that, with respect to the original design and first position measurement before the first discharge at W7-X, the system showed a displacement of \SIrange{0.5}{1.5}{\milli\meter} for the individual corners of the apertures. This particular shift yields a downward - towards the lower side within the plane of the bolometer LOS fan - tilt of the camera of \SI{2}{\degree}. The absorber array is, due to the design of the bolometer, presumed to share the same displacement, though no equivalent estimation for their shift is available. Since there exist no additional information regarding the bolometer LOS perturbation, this deviation is assumed to neither be constant during an experimental campaign nor across the past operation and construction phases. A variation to the camera geometry of this order of magnitude was already performed when examining the local sensitivity of a vertically centred and aligned detector array. This most recent positional benchmark is of course taken into account for all prior and following calculations, however evaluating potential additional shifts during or in-between campaigns is crucial towards assessing the plausibility of experimental data reconstructions. Hence, a randomized transposition in the range of \SIrange{0.5}{1.5}{\milli\meter} is applied to the virtual LOS geometry that was achieved in the previous section. Two limiting cases, where this perturbation method yields $\pm$\SI{2}{\degree} tilt, i.e. in upward or downward direction of the fan, were chosen to gauge the impact of this error to the local sensitivity. For sake of comprehensibility, this will again only be examined in detail here for the HBCm, while the effect of disassociated forward integration and reconstruction camera geometries for phantom tomograms will be investigated upon later. The two upright and tilted LOS fans, based on the results in \cref{fig:geometry_change_artificial}, are shown in \cref{fig:geometry_change_tilted}.\\%
            The respective variation in local sensitivity for the above set of hypothetical geometries can be found in \cref{fig:emiss_change_tilted}. Compared to the previous differences in etendue in \cref{fig:geometry_forward_symmetric}, similar features of same order of magnitude are presented here for both sets of lines-of-sight. The absolute range in difference is insignificantly increased to \SIrange[per-mode=reciprocal]{e-11}{e-16}{\per\cubic\milli\meter}. Individual channels and their variation can be noted in the pixelated plot. Again, the largest differences appear in front of the aperture and along the separatrix and domain boundary on the top and bottom. One should note that, despite the tilt in poloidal direction, there appears no gap in coverage of the reconstruction domain, i.e. the area of $1.3^{2}V\ix{p}$ by the newly constructed cameras.\\%
%
            \begin{figure}[t]%
                \centering%
                \begin{subfigure}{0.47\textwidth}%
                    \includegraphics[width=\textwidth]{%
                        content/figures/chapter4/forward_int/%
                        forward_chord_HBCm_00091_00092_00093_00094_min_EIM_beta000_sym_dets_aptplane_tilt_cor_-3.0deg_sN8_30x20x150_1.35_fixed2.png}%
                    \caption{$-2^{\circ}$}%
                \end{subfigure}%
                \hfill%
                \begin{subfigure}{0.47\textwidth}%
                    \includegraphics[width=\textwidth]{%
                        content/figures/chapter4/forward_int/%
                        forward_chord_HBCm_00091_00092_00093_00094_min_EIM_beta000_sym_dets_aptplane_tilt_cor_3.0deg_sN8_30x20x150_1.35.png}%
                    \caption{$+2^{\circ}$}%
                \end{subfigure}%
                \caption{\textbf{(a), (b)} Forward calculated chord brightness at different $f\ix{rad}$ for previously presented STRAHL results in \cref{fig:chord_forward_exp_vs_STRAHL}, using the theoretical geometries of \cref{fig:geometry_change_tilted}, likewise for (a) $-2^{\circ}$ and (b) $+2^{\circ}$ tilts.}\label{fig:chord_change_tilted}%
            \end{figure}%
%
            Both of the presented LOS geometries and their corresponding parameter coefficients are again used to forward model integrate the chord brightness from STRAHL results in \cref{sec:strahlmodel}. However, in order to assess the impact of this specific misalignment to a measurement that is performed under the assumption of a different geometry, one has to consider the results using the ideally upright, centred and symmetrical LOS fan of \cref{fig:geometry_change_artificial} in this case. The profiles in \cref{fig:chord_change_tilted} are calculated with the new coefficients, while being presented using the information of the symmetric geometry - the abscissa of individual LOS radii is given by the prior geometries traces through the mesh, see \cref{eq:effective_radM}. Both $\pm$\SI{2}{\degree} show significant lateral shifts towards \textbf{(a)} upper and \textbf{(b)} lower side of the separatrix, i.e. $\pm r\ix{a}$. While the relative position and height of the individual lines and local extremes for all levels of $f\ix{rad}$ stay the same, the \textbf{(a)} left and \textbf{(b)} right part of the profiles are compressed, while the rest is shifted in the opposite direction. The relocated local maxima for $f\ix{rad}=1$ have been moved from $\pm0.9r\ix{a}$ to $\pm0.5r\ix{a}$ - the other plots show a similar behaviour, underlined by the semi-transparent, mirrored lines in both images for all radiation fractions.\\%
            This clearly show the importance of congruent measurement or forward calculation and reconstruction camera geometries. In this scenario - arguably the worst case -, an intrinsically and fully symmetrical emissivity distribution yields highly asymmetrical and radially transposed chord brightness profiles. An inversion performed under this premise, which will be explored in more detail in \cref{sec:phantoms}, is badly pre-conditioned since the regularisation functional, see \cref{sec:minfisher}, that does and can not account for this order of magnitude discrepancy. Later it will be found that tomography for misalignment geometries will not be able to find plausible radiation distributions, produce misleading results or even artefacts. It is of significance however to understand such effects and their impact for later experimental data reconstructions where no in-situ information on the actual LOS geometry or perturbation thereof is available.%
%
        \subsection{Additional Artificial Camera Arrays}\label{subsec:artf}%
%
           Upon investigation of possible LOS and camera geometry perturbations and their impact on the local sensitivity a forward model line-integrated measurements, the question of benefiting the tomographic inversions with additional, hypothetical and tailored detector arrays arose. This is largely motivated by the observed discrepancies during reconstructions of experimental data, which will be performed later in this chapter. While trying to find plausible radiation distributions that are in agreement with other diagnostics and intrinsically coherent, i.e. a temporal sequence of tomographic images shows comprehensible results, more cameras in particular locations were presumed to greatly improve upon their quality. Therefore, two sets of absorbers located opposite of the existing cameras were designed. One on the inboard side viewing the inside magnetic islands and X-points and one at the top of the triangular plane, viewing along the separatrix towards the outboard side and also the inboard island, both limited to fifteen LOS each.%
%
           \subsubsection*{Horizontally Mirrored Camera}%
%
                \begin{figure}[t]%
                    \centering%
                    \begin{subfigure}{0.4\textwidth}%
                        \includegraphics[width=\textwidth]{%
                            content/figures/chapter4/MFR/%
                            LoS_comparison_newCamera_MIRh_sN8.pdf}%
                        \caption{}%
                    \end{subfigure}%
                    \hspace*{0.5cm}%\hfill%
                    \begin{subfigure}{0.5\textwidth}%
                        \includegraphics[width=\textwidth]{%
                            content/figures/chapter4/MFR/%
                            LoS2D_comparison_newCamera_MIRh_sN8.pdf}%
                        \caption{}%
                    \end{subfigure}%
                    \caption{\textbf{(a)} Comparison of the (black) horizontal bolometer camera LOS geometry and an artificial, \textit{new} camera ("MIRh", \textcolor{blue}{blue}) that is arranged opposite of the prior, spanning 15 new channels that cover the full torus cross-section homogeneously. \textbf{(b)} Same geometry comparison in projection of the toroidal plane of the bolometer camera assembly, i.e. $\varphi\ix{tor}=108^{\circ}$.}\label{fig:geometry_newcam_mirh}%
                \end{figure}%
    %
                The first new, artificial camera is presented in \cref{fig:geometry_newcam_mirh} alongside the original HBCm in both three- and two-dimensional perspective similarly to the previous LOS plots. This new geometry will be called \textit{\textcolor{blue}{MIRh}} - abbreviation/simplification of \textit{MIRrored (horizontal)} (bolometer camera). The array is centred around $z=0$ and designed to cover the poloidal cross-section evenly while also having individual pseudo detectors viewing regions of particular significance, i.e. magnetic islands and X-points. It is also tilted in toroidal direction  - \SI{68.75}{\degree}, therefore with a toroidal reach of $\sim$\SI{5}{\degree} - and oriented antiparallel to its respective counterpart. Absorber size and shape within this camera are all the same and based on the respective HBCm average, while its aperture is equal to that of the latter. In contrast to the original horizontal camera, the MIRh pinhole is located much closer to the plasma and inside the enclosing first wall of the device, though the hypothetical construction puts the entire detector assembly outside said boundary, in some way being consistent with the space constraints at W7-X and the position of the VBCl/r.\\%
%
        \newline%
        The evaluation of the bolometer LOS geometry in a context of tomographic inversion is thereby concluded for now. Comparison of different discrete segmentation methods and their impact on the three-dimensional, local sensitivity in a voxel grid based plasma tube has concluded that rectangular and triangulated approaches are equally valid and applicable for their respective task. Their corresponding fineness or resolution was also found to be sufficient above four individual parts of both detector and aperture, which yields a great reduction in computational tasks, though a general setting of $N=8$ triangulated sub-detectors and -apertures has been established. Robustness of the crucial to a successful reconstruction, central geometric parameter matrix $\mathbf{T}$ and perturbation thereof were examined thoroughly with the following hypothetical camera geometries. A variety of independent and combined changes to the original construction and their respective impact on the two-dimensional, local sensitivity distribution for tomography have been presented. This included chord brightness forward model integrations from fully symmetrical STRAHL radiation profiles using said results. Throughout this it was found that the orientation of the cameras in the device introduces an intrinsic asymmetry to the LOS fans yield and hence their measurements. Beyond that, their individual deviation in local sensitivity  generally is of order of magnitude found while comparing segmentation methods and is therefore assumed to be of lesser relevance. However, high precision measurements provided large discrepancies to the designed in-device aperture positions and the conclusive application to the previous models showed significant impact on the forward calculations. Therefore, one has to assume some level of additional, unknown, non-negligible error in $\mathbf{T}$, or more specifically to the regularisation functional $\mathbf{K}$ when performing tomographic reconstructions. Finally, two entirely new cameras have been introduced, which now be used to improve the reconstruction of complex phantom radiation profiles and eventually gauge the efficacy of potential extensions to the existing multicamera system.%
%
    \section{Phantom Radiation Profiles}\label{sec:phantoms}%
%
        Tomographic reconstructions, particularly with badly conditioned - the number of LOS (intersections) is significantly smaller than the number of free parameters or pixels/voxels - detector systems are difficult to perform and potentially yield a high level of uncertainty in the final result, subject to assumptions about the underlying geometry and errors in the measurement. Furthermore, improvements to the algorithm of already proven methods like the \textit{Minimum Fisher regularisation} similar to the previously introduced \textit{radially dependent anisotropy} generally need tailoring for individual, characteristic radiation distributions, i.e. varying relative weights for smooth or localised emissivities. Practical evaluating of these algorithms with a selection of common and/or expected two-dimensional \textit{phantom radiation profiles} is crucial towards successful reconstruction of experimental data. Insights from previous investigations will be used for advantage upon improving the tomography of said phantom images.\\%
        First, a set of tools will have to be introduced for evaluating the results and their grade properly. One trivial task is to find both the two-dimensional and integrated deviation between phantom and tomogram. A \textit{mean square deviation (MSD)} is introduced as:%

        \begin{align}%
            \vec{m}\ix{var}=\frac{\sqrt{\left(\vec{x}-\vec{x}\ix{phan}\right)\left(\vec{x}-\vec{x}\ix{phan}\right)^{\intercal}}}{\|x\ix{phan}\|}\cdot\vec{A}\ix{p}^{\intercal}\,\,,\qquad%
            M\ix{var}=\|\vec{m}\ix{var}\|\,\,.%
            \label{eq:msd}%
        \end{align}%
%
        The above equation finds the relative, positive definite local deviation between the reconstruction and initial image $\vec{m}\ix{var}\in\mathbb{R}^{m}$, where $\vec{x}\ix{phan}\in\mathbb{R}^{m}$ is the radiation distribution of the phantom and $\vec{x}\in\mathbb{R}^{m}$ the resulting tomogram - remember that $m=N\ix{$\vartheta$}\cdot N\ix{r}$. The integrated \textit{mean square deviation, MSD} is given by $M\ix{var}$, while $\vec{A}\ix{p}$ is the area of the individual pixels $p^{\left(i,j\right)}$ associated with the radiation intensity in that location - for sake of simplicity let us write from here on%
%
        \begin{align}%
            \vec{x}\,\,,\vec{A}\ix{p}\quad\leftrightarrow\quad%
            x^{\left(i,j\right)}=x^{\left(i\cdot N\ix{r}+j\right)}\,,\,\,%
            A\ix{p}^{\left(i,j\right)}=A\ix{p}^{\left(i\cdot N\ix{r}+j\right)}%
            \,\,.\nonumber%
        \end{align}%
%
        In order to find $A^{\left(i,j\right)}\ix{p}$ the \textit{Heron\footnote[1]{Hero of Alexandria (Hērōn hò Alexandreús) \textdagger~60 AD; Greek mathematician and engineer from Alexandria, Egypt during the Roman era} formula} is used in \cref{eq:heron} to sepperate the pixels into two triangles that fill the initial space and calculating their individual surface area. Let the the pixel be:
%
        \begin{align}%
            p^{\left(i,j\right)}:&\quad\square\left(BCDE\right)^{\left(i,j\right)}\,\,.\nonumber%
        \end{align}%
%
        The surface area $A\ix{p}^{\left(i,j\right)}$ of pixel $p^{\left(i,j\right)}$, constructed by the mesh intersection points $\left\{B,C,D,E\right\}$ is given by:
%
        \begin{align}%
            \begin{split}%
                &A\ix{p}^{\left(i,j\right)}=A\left(\bigtriangleup\left(BCD\right)\right)+A\left(\bigtriangleup\left(CDE\right)\right)\,\,,\\%
                &A\left(\bigtriangleup\left(XYZ\right)\right)=\sqrt{p\left(p-a\right)\left(p-b\right)\left(p-c\right)}\,\,,\\%
                a=&\overline{XY}\,,\,\,b=\overline{YZ}\,,\,\,c=\overline{ZX}\,,\,\,\,p=\frac{1}{2}\left(a+b+c\right)\,\,.%
            \end{split}\label{eq:heron}%
        \end{align}%
%
        The values of $A\ix{p}^{\left(i,j\right)}$ are calculated once for a parameter set of magnetic field configuration, voxel grid resolution $N\ix{$\varphi$}\times N\ix{$\vartheta$}\times N\ix{r}$ and camera geometry - remember that the two-dimensional representation of the grid like in \cref{fig:los_grid_emiss} is achieved by finding the intersection of the camera LOS fan plane with the voxel mesh. This also makes it very easy to find the total, integrated radiation power of the phantom and tomogram:
%
        \begin{align}%
            P\ix{rad,2D}\left(\vec{x}\right)=\vec{x}^{\intercal}\cdot\vec{A}\ix{p}=%
            2\pi R\ix{maj}\,\displaystyle\sum_{i=0}^{n\ix{r}}\sum_{j=0}^{n_{\vartheta}}x^{\left(i\cdot n_{\vartheta}+j\right)}A\ix{p}^{\left(i,j\right)}\,\,.%
            \label{eq:prad2D}%
        \end{align}%
%
        Remember that $R\ix{maj}$ is the major plasma radius, i.e. the distance between center of the machine and the magnetic axis at the center of the plasma. Finally, the regularisation \textit{fitness} or \textit{robustness factor} $\chi^{2}$ - see \cref{eq:linearregular} -, which measures how well input and output forward calculated signals are in agreement, is re-introduced as:%
%
        \begin{align}%
            \chi^{2}=\frac{1}{N\ix{ch}}\left\|\left(\vec{b}\ix{phan}-\vec{b}\ix{tom}\right)\vec{\sigma}^{-1}\right\|\,\,.\label{eq:chi2}%
        \end{align}%
%
        Remember that $\sigma$ is the measurement uncertainty or standard deviation of each individual channel and $N\ix{ch}$ the total number of channels. In practice, this is dictated by an error in acquisition and calibration, while for a forward phantom integration, an experimentally motivated and randomly distributed percentage between \SIrange{1}{2.5}{\percent} of the global maximum is applied. In this case, and for all following phantom reconstructions if not stated otherwise, the error is set to%
%
        \begin{align}%
            \sigma^{\left(n\right)}=0.025\cdot\max\left(\vec{b}\right)\,\widetilde{p}\left(n,\mathcal{N}\left(0,0.5\right)\right)\,\,.\nonumber%
        \end{align}%
%
        Here, $\widetilde{p}\left(x,P\right)$ is a random sample for value $x$ of a probability distribution $P$ and $\mathcal{N}\left(\mu,\varepsilon\right)$ the \textit{Gaussian normal distribution} of width $\varepsilon$ and mean $\mu$. With $\max\left(\vec{b}\right)$ the global chord brightness peak, the error of channel $n$ is at most \SI{2.5}{\percent} of said value.\\%
        Equations \ref{eq:msd} through \ref{eq:chi2} yield the necessary toolset to initially assess the quality of the phantom image reconstruction using the \textit{Minimum Fisher regularisation tomography}. In the following pages, a comprehensive and diverse as possible set of hypothetical radiation distributions will be presented, alongside their individual tomographic reconstructions and corresponding errors, including both radial and chord brightness profiles from those, respectively. This exercise will be centred around experimentally motivated and more simplified and symmetric phantom images, from where more complex and challenging variations thereof are derived and reconstructed, while also trying to find an ideal - or suitable - set of anisotropy coefficients $\left\{k\ix{core},\,k\ix{edge}\right\}$. This method will ultimately provide one with the practical limits of this approach for performing tomography on experimental data, which will be done so conclusively. If not stated otherwise, the employed voxel grid domain is $\left\{N\ix{$\varphi$},N\ix{$\vartheta$},N\ix{r}\right\}=\left\{30,20,150\right\}$ and in standard magnetic configuration.%
%
        \subsection{Symmetrical Ring in Scrape-Off Layer}\label{subsec:phantoms_symmring}%
%
            \begin{figure}[t]%
                \centering%
                \makebox[\textwidth][c]{%
                    \includegraphics[width=1.2\textwidth]{%
                        content/figures/chapter4/MFR/new/phantom/2D/%
                        phantom_v_tomo2D_ring_R1.0_mx1.0e+06_aniM3_2.0_0.5_nigs1_times0.11.png}}%
                \caption{Example of a \textbf{(left)} \textit{phantom radiation distribution} and \textbf{(center)} its reconstruction using the Minimum Fisher tomography and standard bolometer camera geometry, including the \textbf{(right)} relative local deviation between the two prior. The phantom image is constructed as a uniform \textit{ring} along a fixed flux surface with radius $r\ix{a}$ and a normal distributed intensity in radial direction with a maximum of \SI{1}{\mega\watt\per\cubic\meter} and $\sigma=$\SI{0.25}{\meter}. The inversion domain is the magnetic standard configuration, as previously described in \cref{subsec:global}. The RDA parameters were chosen as $k\ix{core}=2$ and $k\ix{edge}=0.5$ with no smoothing in-between.}\label{fig:phantom_fsring_example}%
            \end{figure}%
%
            The first phantom radiation profile that is used for forward integration and reconstructed with the MFR algorithm is shown, alongside its tomogram and respective mean deviation from \cref{eq:msd} in \cref{fig:phantom_fsring_example}. A bright ring with a radially normal distributed emissivity is constructed around a radius of $r\ix{0}=r\ix{a}$ with a maximum of $P\ix{phan,0}=$\SI{1}{\mega\watt\per\cubic\meter} and FWHM of $\sigma\ix{r}=0.25r\ix{a}$. The latter two values are experimentally motivated and chosen to yield sensible $P\ix{rad}$ from the chord brightness profiles and two-dimensional integrated powers.%
%
            \begin{align}%
                P\ix{phan}\left(r\right)=\frac{P\ix{phan,0}}{2\pi\sigma^{2}}\,e^{\left(-\frac{1}{2}\left(\frac{r-r\ix{0}}{\sigma}\right)^{2}\right)}\nonumber%
            \end{align}%
%
            Individual values are and can only be attributed to discrete pixels on the grid. This is also due to the forward integration by $\mathbf{T}\vec{x}\ix{phan}=\vec{b}\ix{ch}$, where $\vec{b}\ix{ch}$ is the chord brightness profile - also the measurement data in an experimental environment. The radius of a pixel is given by the radius of its rectangles $p^{\left(i,j\right)}$ \textit{barycentre}. Hence, the actual maximum of the phantom emissivity is approximately \SI{0.75}{\mega\watt\per\cubic\meter} since the pixel layer's center does not necessarily align with a particular radius, as given above. All plots also indicate, like before, a half-transparent schematic of the separatrix and magnetic island structure. The middle plot shows the Minimum Fisher regularisation reconstruction for $k\ix{core}=2$ and $k\ix{edge}=0.5$. In this case, the anisotropy profile is a simple plateau, i.e. a \textit{Heaviside\footnote[1]{Oliver Heaviside, FRS *~May 18, 1850 \textdagger~Feb. 3 1925; English self-taught mathematician and physicist, developed technique for solving differential equations, vector calculus independently and rewrote Maxwell's} step function} $\Theta\ix{$N\ix{T}$}$ for $N\ix{T}=14$.%
%
            \begin{align}%
                \Theta\ix{N}\left(x\right)=\left\{\begin{array}{ll}%
                    1&,\,x>N\\%
                    0&,\,\text{else}%
                \end{array}\right.\nonumber%
            \end{align}%
%
            This particular set of coefficients was deliberately chosen to mimic a typical configuration that would be used for experimental data tomography, i.e. in order to favor smooth radiation profiles in the core and enable poloidal variation beyond the LCFS. Radial location and absolute value of the emissivity in the tomogram are very similar to its counterpart, however the distribution thereof is very asymmetric and focused strongly around the lower middle and inboard side X-point. The reconstructed emissivity is also much smoother and radially spread more than in the phantom along the entire separatrix. Gradients in poloidal direction are increased while they are significantly reduced perpendicular to the LCFS. At the inboard upper magnetic island and around the neighbouring X-points, the tomogram's radiation density is nearly constant between separatrix and domain boundary, while in the phantom no emissions can be found in the very last pixel shell towards the edge. These observations are also very much reflected in the right side \textit{MSD} plot in \cref{fig:phantom_fsring_example} - see \cref{eq:msd}. The larger the brightness or value of deviation here, the greater the discrepancy in that particular location of the mesh between phantom and tomogram emissivity distribution. Variations of \SIrange{1}{3}{\percent} are largely centred around the pixels closest to the LCFS and magnetic structures. They match best, as described above, at the lower inboard and center, as well as worst around the opposite side X-points of the triangular plane. The average deviation along the separatrix is \SIrange{2}{2.5}{\percent}.\\%
%
           \begin{figure}[t]%
               \centering%
               \includegraphics[width=1.\textwidth]{%
                   content/figures/chapter4/MFR/new/phantom/%
                   phantom_v_tomo_profiles_ring_R1.0_mx1.0e+06_aniM3_2.0_0.5_nigs1_times0.11.png}%
               \caption{Corresponding analysis for the results shown in \cref{fig:phantom_fsring_example}. \textbf{(left)} Radial profile comparison between phantom and two-dimensional reconstruction, including the relative radial and poloidal weighting factor profiles $\diff r$ and $\diff\vartheta$. Noted here are also the total integrated radiation powers for both. \textbf{(right)} Foward integrated chord brightness profiles of the individual arrays, using the standard camera geometry and the distributions from \cref{fig:phantom_fsring_example}. The corresponding values for $P\ix{rad}$, as well as the fitness factor $\chi^2$ are included.}\label{fig:phantom_fsring_example_profiles}%
           \end{figure}%
%
           In \cref{fig:phantom_fsring_example_profiles}, the individual radial and forward integration profiles are shown. The left image presents both radial emissivity distributions, which are derived using%
%
           \begin{align}%
                p\ix{r}=\,&p\ix{r$^{\left(j\right)}$}=%
                \frac{\int\int_{2\pi}P\ix{rad}\left(r,\vartheta\right)r\diff r\diff\vartheta}{\int r\diff r}%
                \mathrel{\hat{=}}%
                \frac{\sum_{j=0}^{N\ix{r}}\sum_{i=0}^{N\ix{$\vartheta$}}x^{\left(i,j\right)}A\ix{p}^{\left(i,j\right)}}{%
                    2\pi\sum_{j=0}^{N\ix{r}}A\ix{p}^{\left(i,j\right)}}\,\,,\label{eq:averageprofiles}%
           \end{align}%
%
           where one remembers that $x^{\left(i,j\right)}=\left(\vec{x}\right)^{\left(i,j\right)}$ is the phantom or tomogram radiation profile - later referred to as $p\ix{r}^{\left(phan\right)}$ and $p\ix{r}^{\left(tom\right)}$. \autoref{eq:averageprofiles} essentially yields the average emissivity along the flux surfaces. For the individual lines, local maxima positions and values are indicated with by a $\times$-marker and their corresponding peak FWHM by a dash-dotted line at that particular height and location. Included next to $p\ix{r}$ in the same plot are also the corresponding radial and poloidal weighting factors of the regularisation functional, i.e. the first order local gradients with $\diff\vartheta^{\prime}=k\ix{ani}\diff\vartheta$ the impact of the anisotropy factor. Two-dimensional integrated powers, calculated using \cref{eq:prad2D} are noted for both distributions individually. The plot on the right side features both forward integration profiles for all three bolometer cameras in a simplified presentation - results are shown over their discrete channel numbers rather than their LOS projected radii. Corresponding errors are indicated as vertical lines as well for the forward integrated profile. Included here as well are both radiation power losses $P\ix{rad,X}$ as measured by the bolometer cameras and introduced in \cref{eq:prad_total} and the corresponding fitness factor $\chi^{2}$.\\%
           Looking at the left image, a significant discrepancy between sharpness and absolute height of the radial profile peaks at $r\ix{a}$ is immediately noticeable. The position of the maximum in both the phantom and tomogram line is slightly outside the separatrix, which is indicated by a grey dotted line. Conclusively, the $p\ix{r}^{\left(tom\right)}$ profile maximum extends far more in radial direction. With a separate abscissa on the right, the regularisation weights are measured in the same plot. The radial gradient $\diff r$ and standard poloidal factor $\diff \vartheta$ are both constant across the domain, since the underlying grid mesh was designed with constant step sizes. Modified by $k\ix{ani}$ the anisotropy factor, $\diff \vartheta^{\prime}$ becomes a step at \SI{0.875}{\meter}. Despite the unfavourable weighting and discrepancy in absolute height and position, the radial profiles of tomogram and phantom generally and qualitatively match well, which is underlined by the negligibly small error in integrated power. Similar circumstances are true for the forward integration comparison on the right. Both sets of camera profiles yield a $P\ix{rad}$ within \SI{0.4}{\percent} of each other, while the fitness factor $\chi^{2}$ is also close to unity. However, besides the nearly identical and certainly congruent within the error bars VBCl/r lines, the HBCm shows noticeable disagreement between the tomogram and phantom. Particularly, if not exclusively, the lower side range of this array, i.e. channel three through ten deviate significantly, where the phantoms profile is again less sharp and lower in absolute height. Furthermore, for both profiles, the lower and upper part are distinctly different in shape and height. The vertical cameras accumulate, with a few exceptions, globally the most radiation. Overall, the characteristic of both two-dimensional radiation distributions is resembled by the structure of the individual and combined profiles, i.e. increased brightness at the separatrix, none outside and secondary emissivities at the core due to the cross-domain geometry of the LOS.\\%
           This simple phantom and reconstruction for a common set of anisotropy factors show the intricacies and particularities that come with the Minimum Fisher regularisation of such experimentally motivated emissivity profiles. Experience from this exercise is used to familiarize one with the previously introduced set of algorithms and tools, while building a foundation for the following artificial radiation distributions and their tomograms. Especially evaluating and tuning the $k\ix{ani}$ coefficients towards characteristics of the individual phantoms will be explored further in the following sections.%
%
        \subsection{Anisotropic Scrape-Off Layer Ring Combinations}\label{subsec:phantoms_anirings}%
%
            Based on the profile in \cref{fig:phantom_fsring_example_profiles}, various anisotropic two-dimensional emissivities are constructed in order to examine superimposed or convoluted poloidally symmetric and asymmetric distributions and their reconstructions. Particularly the impact of different $k\ix{ani}$ parameter sets on the tomograms quality will be of focus here, while this is still being considered from the perspective of experimental data tomographies. The set of artificial input radiation profiles is deliberately chosen to better aid later experimental data investigations.\\%
%
            \subsubsection*{SOL Ring and \SI{180}{\degree} Core Anisotropy}%
%
                The first combination phantom is constructed from a bright ring at the SOL, i.e. a maximum of \SI{1}{\mega\watt\per\cubic\meter} with a normal distributed decay in radial direction of $\sigma\ix{r}=0.25r\ix{a}$ width, and an additional, inside the separatrix core structure of 150\% intensity of the prior at $0.7r\ix{a}$ and poloidal asymmetry towards the inboard lower X-point - for reference, this will be called \SI{0}{\degree}-orientation in the following exploitations. This superimposed emissivity is derived by convoluting two normal distributions in radial and poloidal direction of different widths, i.e.%
%
                \begin{align}%
                    P\ix{phan}\left(r,\vartheta\right)=%
                    \frac{P\ix{phan,0}}{2\pi\sigma\ix{r}^{2}}\left(%
                    e^{-\frac{1}{2}\left(\frac{r-r\ix{a}}{\sigma\ix{r}}\right)^{2}}+%
                    \frac{3}{4\pi\sigma\ix{$\vartheta$}^{2}}\,e^{-\frac{1}{2}\left(\frac{r-0.7r\ix{a}}{\sigma\ix{r}}+\frac{\vartheta-\vartheta\ix{0}}{\sigma\ix{$\vartheta$}}\right)^{2}}%
                    \right)\,\,.\label{eq:fsring_asym_single_deg}%
                \end{align}%
%
                Here, the width in poloidal direction is $\sigma\ix{$\vartheta$}=$\SI{0.785}{\radian} or $\sim$\SI{15}{\degree}/$
                \pi/4$ and $\vartheta\ix{0}=$\SI{0}{\degree}, the angle of the anisotropy - this particular orientation is chosen deliberately as a starting point from experience with previous experimental data reconstructions. The resulting phantom radiation distribution can be found in the top left of \cref{fig:phantom_fsring_asym_180deg_2D}, accompanied by the reconstructions and two-dimensional error profiles for two sets of $k\ix{ani}$ coefficients. In sum, the maximum emissivity becomes \SI{2.55}{\mega\watt\per\cubic\meter} with visually clear separation between the two bright rings. The first set of anisotropy factors, again extrapolated using a Heaviside $\Theta\ix{14}\left(n\right)$ in the radial dimension, reads $k\ix{core},\,k\ix{edge}=\left\{0.3,\,0.3\right\}$ and the second $\left\{1.,5, 0.25\right\}$.\\%
%
                \begin{figure}[t]%
                    \centering%
                    \begin{subfigure}{\textwidth}%
                        \centering%
                        \makebox[\textwidth][c]{\includegraphics[width=1.2\textwidth]{%
                            content/figures/chapter4/MFR/new/phantom/2D/%
                            phantom_v_tomo2D_asym_fsR1.0_m1_offset180_mx11.0e+06_mx01.5x_aniM3_0.3_0.3_nigs1_times0.11.png}}%
                        \caption{$k\ix{core},\,k\ix{edge}=\left\{0.3,\,0.3\right\}$}%
                    \end{subfigure}%
                    \newline%
                    \begin{subfigure}{\textwidth}%
                        \centering%
                        \makebox[\textwidth][c]{\includegraphics[width=1.2\textwidth]{%
                            content/figures/chapter4/MFR/new/phantom/2D/%
                            phantom_v_tomo2D_asym_fsR1.0_m1_offset180_mx11.0e+06_mx01.5x_aniM3_1.5_0.25_nigs1_times0.11_white.png}}%
                        \caption{$k\ix{core},\,k\ix{edge}=\left\{1.5,\,0.25\right\}$}%
                    \end{subfigure}%
                    \caption{%
                        Comparison of different RDA coefficient combinations for the same phantom radiation distribution (top left), their reconstruction (center) and relative difference (right) using the Minimum Fisher tomography with standard bolometer camera geometry. The phantom image is constructed using a similarly bright ring around $r\ix{a}$ as in \cref{fig:phantom_fsring_example}, with a maximum intensity of \SI{1}{\mega\watt\per\cubic\meter}, as well as an inside structure at $0.7r\ix{a}$ with a poloidally asymmetric maximum towards the upper outboard side, close to the HBC aperture at 150\% of the rings' intensity. \textbf{(a)}: A parameter combination preferring poloidal asymmetry equally in the core and edge. \textbf{(b)}: RDA coefficients amplifying distinguishable structures in the edge and smooth radiation profiles in the core.All distributions are plotted over the standard magnetic geometry mesh at 108$^{\circ}$ toroidally. The dashed line indicates the separatrix and magnetic island structure, while the dotted line shows the LCFS.}\label{fig:phantom_fsring_asym_180deg_2D}%
                \end{figure}%
%
                The phantom in \cref{fig:phantom_fsring_asym_180deg_2D} is derived using equation \ref{eq:fsring_asym_single_deg} with $\vartheta\ix{0}=$\SI{180}{\degree}. All other parameters, including $k\ix{ani}$ are constant. In \textbf{(a)} and \textbf{(b)}, the center image shows the respective tomographic reconstruction. At the top, this shows a distinctly brighter core structure that is broadened towards the LCFS and a less intense ring along the separatrix. The latter also features small local maxima in X-points and areas where magnetic islands intersect. Particular about this plot is the continued increased brightness, almost following a straight line from that localised inside feature to the outboard upper magnetic island. Conversely, the respective SOL intensity is higher here than in the rest of the triangular plane. At the very edge of the tomographic domain, there is still significant radiation power towards the inboard upper area and boundary. The corresponding MSD on the left prominently shows this extension of the anisotropy towards the tip of the triangular plane, on the inside of the respective upper magnetic island. The inconsistency of the separatrix ring intensity in comparison to the artificial distribution is highlighted here as well.\\%
                Reconstruction and error profile in \cref{fig:phantom_fsring_asym_180deg_2D}:\textbf{(b)} present a hyper localised bright spot of greatly larger intensity than the rest of the distribution and respective tomogram for comparison. In this case, the upper inboard magnetic island and part of the separatrix feature a relatively small $0.2\times$\SI{0.2}{\meter} characteristic with a maximum that shows a very sharp decay radially, while extending noticeably in poloidal direction. With respect to its brightness, the rest of the tomogram presents no other significant contributions to the overall radiation power. However, with $<$\SI{1}{\mega\watt\per\cubic\meter} some of the previously noted structures are found here as well, i.e.the variation in intensity along the LCFS, of which the very bright spot is a part, illuminated domain boundary at the inboard top and the core anisotropy. The prominence of the upper outboard magnetic island characteristic in the tomogram is reflected in its corresponding MSD profile also, as there is an increased error at and around the separatrix in that location. Coincidentally, along the inside of the LCFS the deviation is reduced accordingly with respect to the prior.\\%
%
                \begin{figure}[t]%
                    \centering%
                    \begin{subfigure}{\textwidth}%
                        \centering%
                        \includegraphics[width=\textwidth]{%
                            content/figures/chapter4/MFR/new/phantom/%
                            phantom_v_tomo_profiles_asym_fsR1.0_m1_offset180_mx11.0e+06_mx01.5x_aniM3_0.3_0.3_nigs1_times0.11.png}%
                        \caption{$k\ix{core},\,k\ix{edge}=\left\{0.3,\,0.3\right\}$}%
                    \end{subfigure}%
                    \newline%
                    \begin{subfigure}{\textwidth}%
                        \centering%
                        \includegraphics[width=\textwidth]{%
                            content/figures/chapter4/MFR/new/phantom/%
                            phantom_v_tomo_profiles_asym_fsR1.0_m1_offset180_mx11.0e+06_mx01.5x_aniM3_1.5_0.25_nigs1_times0.11.png}%
                        \caption{$k\ix{core},\,k\ix{edge}=\left\{1.5,\,0.25\right\}$}%
                    \end{subfigure}%
                    \caption{\textbf{(a), (b)} Corresponding analysis for the results shown in \cref{fig:phantom_fsring_asym_180deg_2D}, similar to \cref{fig:phantom_fsring_example_profiles}, using the noted below anisotropy parameters for the tomographic inversion. The results are presented in the same way as in \cref{fig:phantom_fsring_example_profiles}.}\label{fig:phantom_fsring_asym_180deg_profiles}%
                \end{figure}%

                The corresponding radial, regularisation weight and forward integrated profiles for the two-dimensional distributions in \cref{fig:phantom_fsring_asym_180deg_2D} can be found in \cref{fig:phantom_fsring_asym_180deg_profiles}. Two sets of images show the results for the respective two $k\ix{ani}$ parameter combinations from the previous plots in \textbf{(a)} and \textbf{(b)}. Two-dimensional integration power values of the emissivity for both tomogram and phantom are noted in the individual plots on the left, as is the fitness $\chi^{2}$ of the chord profiles and $P\ix{rad}$ extrapolations each on the right as well. Generally, the tomograms radial profile on the top is very similar to their phantom counterparts. A slightly higher $P\ix{rad,2D}$ is produced by the reconstruction, while the radial profile matches well locally, though presents a higher core brightness and significantly less pronounced peaks with greatly decreased absolute height and hence greater FHWM. The corresponding phantom and tomogram forward integrated absorber signals on the right also show overall good agreement within the respective confidence interval for all cameras, except in detector no. eleven of the HBCm. In this location and for all other local extremes of any absorber array, the output results generally yield a more sharp and pronounced shape while transitions in the phantoms profiles are smoother. Summation and extrapolation of those produces a total radiation power less than \SI{0.5}{\percent} greater than for the input distribution.\\%
                In \cref{fig:phantom_fsring_asym_180deg_profiles}:\textbf{(b)}, the reconstruction of the same phantom emissivity using $k\ix{ani}=\left\{1.5, 0.25\right\}$ finds a total power that is lower than in the input profile and previous reconstruction. The tomograms radial profile is noticeably more pronounce here, with higher maxima and improved and better matching localization and FWHM. Forward integrated signals corresponding to this reconstruction are in stronger agreement with the respective input data compared to above. Exceptions are the global maximum in the HBCm plot in channel no. eleven, where the discrepancy remains constant, and the lower side and core part of the array, which show a minutely larger error. Particularly the vertical cameras match up significantly better, therefore the fitness is increased to $\chi^{2}=$\SI{0.896}{\arbitraryunit}. The extrapolated power loss however yields a greater discrepancy to both the actual, i.e. integrated two-dimensional power and artificial emissivity.\\%
                The varying $k\ix{ani}$ tomographies for the phantom radiation distribution in \cref{fig:phantom_fsring_asym_180deg_2D} effectively show that superimposed and localised, anisotropic emissivities of this kind are a particular challenge and require further analysis in order to achieve comparable results when compared to simpler proxies. The first reconstruction in fact also yields a core anisotropy that is clearly distinguishable from the smoother and less intense bright ring around the LCFS. In \textbf{(a)}, the balanced weighting coefficients enable the MFR to achieve both an inside asymmetry and brightness outside, though with the aforementioned drawbacks due to $k\ix{core},k\ix{edge}<1$, which prefers localisation above smoothness. In direct contrast and compared to the corresponding plot in the prior set of results, for an anisotropy profile of $k\ix{ani}=\left\{1.5, 0.25\right\}$, the second reconstruction is found to provide an arguably worse fitting emissivity and a much more localised power deposition. From the perspective of the HBCm, this anisotropy is viewed by only a few, i.e. one to three LOS and hence the sensitivity of the whole system towards such structures is constructed very unfavourably. In combination with the respective coefficient profile, this concentrates large amounts of power in this region, while also omitting the desired smoothness due to the otherwise even distribution of radiation, however at a much smaller level, in the core and along the separatrix. Taking into account the position of the asymmetry and the structure and geometry of the bolometer camera system, this generally agrees with the findings from the prior phantom reconstructions. Integrated powers from both two-dimensional and detector signal profiles deviate by varying amounts within this set of tomographies and compared to the previous results. The relationship of $P\ix{rad,2D}$ and $P\ix{rad}$ between tomogram and phantom, as well as $\chi^{2}$ is inconsistent and thereby inconclusive.%
%
            \subsubsection*{SOL Ring and \SI{270}{\degree} Core Anisotropy}%
%
                \begin{figure}[t]%
                    \centering%
                    \begin{subfigure}{\textwidth}%
                        \centering%
                        \makebox[\textwidth][c]{\includegraphics[width=1.2\textwidth]{%
                            content/figures/chapter4/MFR/new/phantom/2D/%
                            phantom_v_tomo2D_asym_fsR1.0_m1_offset270_mx11.0e+06_mx01.5x_aniM3_0.3_0.3_nigs1_times0.11.png}}%
                        \caption{$k\ix{core},\,k\ix{edge}=\left\{0.3,\,0.3\right\}$}%
                    \end{subfigure}%
                    \newline%
                    \begin{subfigure}{\textwidth}%
                        \centering%
                        \makebox[\textwidth][c]{\includegraphics[width=1.2\textwidth]{%
                            content/figures/chapter4/MFR/new/phantom/2D/%
                            phantom_v_tomo2D_asym_fsR1.0_m1_offset270_mx11.0e+06_mx01.5x_aniM3_1.5_0.25_nigs1_times0.11_white.png}}%
                        \caption{$k\ix{core},\,k\ix{edge}=\left\{1.5,\,0.25\right\}$}%
                    \end{subfigure}%
                    \caption{\textbf{(a), (b)} Similarly constructed and reconstructed phantom radiation distribution as in \cref{fig:phantom_fsring_asym_180deg_2D}, but the core structure has been rotated in clockwise direction by \SI{90}{\degree}. The tomographic inversion features the same RDA parameter combinations as before.}\label{fig:phantom_fsring_asym_270deg_2D}%
                \end{figure}%
%
                The next phantom radiation distribution is the same as before for an angle of $3/2\pi$, i.e. towards the outboard tip and lower side of the triangular plane. In \cref{fig:phantom_fsring_asym_270deg_2D}, the artificial emissivity is produced using \cref{eq:fsring_asym_single_deg} with the same set of coefficients as for \cref{fig:phantom_fsring_asym_270deg_2D}, except for $\vartheta\ix{0}=$\SI{270}{\degree} or another $\pi/2$ shift of the poloidal location of the asymmetries peak in clockwise direction. As usual, all other parameter, including the two sets of $k\ix{ani}$ are kept constant.\\%
                The localised maximum in the phantoms core in \cref{fig:phantom_fsring_asym_270deg_2D} is centred more towards the lower outboard magnetic island. In \textbf{(a)}, the MFR finds the highest emissivity in $0.7r\ix{a}$, closest to the lower edge of the latter island with a reduced poloidal extension. This inside anisotropy is not distinctly separated from the emission that is placed along the separatrix and beyond. Significant radiation can be found towards the opposite located magnetic island at the top and in the X-point on the lower inboard side, as well as the neighbouring X-points to the island closest to the anisotropy. With respect to the other tomograms of this $k\ix{ani}$ profile, this one features a noticeable gap in the brightness distribution at the inboard top and center. Correspondingly, the MSD profile on the right reflects these characteristics with elevated error values particularly in and around the separatrix, except for the lower inboard X-point and outboard upper magnetic island. The global maximum is located in the core where the input anisotropy is created.\\%
                The second reconstruction in \cref{fig:phantom_fsring_asym_270deg_2D}:\textbf{(b)} for $k\ix{ani}=\left\{1.5, 0.25\right\}$ shows three distinctly localised bright spots in the inboard lower X-point, outboard upper magnetic island and lower central X-point with increasing intensity, wherein the latter is also the global maximum. Between the individual structures, outside the LCFS and especially towards and around the location of the input emissivity, a smooth and nearly constant brightness is shown.\\%
%
                \begin{figure}[t]%
                    \centering%
                    \begin{subfigure}{\textwidth}%
                        \centering%
                        \includegraphics[width=\textwidth]{%
                            content/figures/chapter4/MFR/new/phantom/%
                            phantom_v_tomo_profiles_asym_fsR1.0_m1_offset270_mx11.0e+06_mx01.5x_aniM3_0.3_0.3_nigs1_times0.11.png}%
                        \caption{$k\ix{core},\,k\ix{edge}=\left\{0.3,\,0.3\right\}$}%
                    \end{subfigure}%
                    \newline%
                    \begin{subfigure}{\textwidth}%
                        \centering%
                        \includegraphics[width=\textwidth]{%
                            content/figures/chapter4/MFR/new/phantom/%
                            phantom_v_tomo_profiles_asym_fsR1.0_m1_offset270_mx11.0e+06_mx01.5x_aniM3_1.5_0.25_nigs1_times0.11.png}%
                        \caption{$k\ix{core},\,k\ix{edge}=\left\{1.5,\,0.25\right\}$}%
                    \end{subfigure}%
                    \caption{\textbf{(a), (b)} Corresponding analysis for the results shown in \cref{fig:phantom_fsring_asym_270deg_2D}, similar to \cref{fig:phantom_fsring_example_profiles}, using the noted below anisotropy parameters for the tomographic inversion. The results are presented in the same way as in \cref{fig:phantom_fsring_example_profiles}.}\label{fig:phantom_fsring_asym_270deg_profiles}%
                \end{figure}%
%
                A matching set of line plots is presented in \cref{fig:phantom_fsring_asym_270deg_profiles}. Regularisation weights, as well as phantom profiles are the same as before. Integration of the artificial, two-dimensional distribution yields $P\ix{rad,2D}=$\SI{19.357}{\mega\watt} and extrapolation from the similarly congruent within this comparison forward measurement results finds $P\ix{rad}=$\SI{17.150}{\mega\watt}. Again, integration of the artificial, two-dimensional distribution and camera profile extrapolations both are distinctly different to the previous values of radiation power, i.e. the latter is >\SI{1}{\mega\watt} or \SI{14.5}{\percent} larger and the prior minutely smaller than for the initial phantom image. With respect to the last artificial distribution, this is also true, though to a lesser extent. In \textbf{(a)}, a slightly increased core brightness is followed by an equally positioned and peaked inside local maximum. A minor valley and, this time located on the inside of the separatrix, second maximum conclude this radial tomogram profile. Integration finds a radiation power negligibly higher than in from the phantom. On the right, the phantoms forward integrated detector signals yield significantly higher powers in channel no. four of the VBCr. The relative intensity of the remaining measurements is therefore decreased. Left and right vertical camera arrays provide a respectively strong maximum towards the outboard side. The horizontal camera finds its maximum on the lower side.\\%
                In \textbf{(b)}, the radial profile is very similar to that on the top and akin to that in the first set of images corresponding to this weighting coefficient combination in \cref{fig:phantom_fsring_asym_180deg_profiles}. Localised towards the separatrix around $0.95r\ix{a}$ is the second peak at a slightly increased brightness and of same width. Integration of the two-dimensional distribution yields less than from the phantom, particularly when compared to the above results. Lastly, the respective forward integral detector signals on the right show comparable congruence to the phantoms results like before. Overall shape and absolute values are largely consistent to before and feature only minor changes. This ultimately produces a fitness that is overall and within the results of this set of $k\ix{ani}$ the highest matching quality.\\%
                Not only is the perceived fitness of both tomograms greater than for the last reconstruction, but all measurements are in agreement with this assessment. Still, though pointed out before already, the development of $\chi^{2}$ etc. and the subjectively worse match between phantom and tomogram for $k\ix{ani}=\left\{1.5, 0.25\right\}$ do not align here as well. Both anisotropy coefficient profiles again yield, quantitatively, entirely different radiation distributions, while the first combination produces a more adequate result compared to the latter. Comparing with the other artificial radiation distribution tomographies, one can also find similar levels of diffusion of emissivity across the triangular plane on both images here. However, localisation and structural quality of maximum intensity regions are, at least in the case of \cref{fig:phantom_fsring_asym_270deg_2D}:\textbf{(a)}, significantly improved. Analogue trends to before of increased edge anisotropy and decreased, smoother emissivity in the core can also be found in this second reconstruction. Though the ratio of both total radiation power extrapolations within this set of tomographies is congruent, the evolution of $\chi^{2}$ and MSD profile or maximum is not, which previously was also the case. Local resolution and LOS coverage are improved in this area, near to the pinhole of the HBCm, as they also are closer to the VBC cameras aperture. Additionally, the presented MFR have provided yet another noticeably different sample of measurements, i.e. $P\ix{rad}$ etc. for an almost constant artificial input distribution. Cross-correlating the fitness factor, the integral of the two-dimensional brightness distribution and radiation power from the forward calculated detector yield only supports the subjective impression of a qualitatively lesser reconstruction. However, the difference in $\chi^{2}$ conflicts with that examination, i.e. a larger error between $P\ix{rad,2D}$ and $P\ix{rad}$ seemingly does not contribute to a worse fit of input and output of the MFR.\\%
                The comparison in \cref{fig:phantom_fsring_asym_270deg_2D} and \ref{fig:phantom_fsring_asym_270deg_profiles}, though individually presenting plausible, if not good reconstruction results, further highlights a major challenge of regularisation tomographies and particularly the anisotropically weighted MFR. The first set of $k\ix{ani}=\left\{0.3, 0.3\right\}$ is explicitly not well suited or intended to reconstruct this phantom radiation distribution adequately, however its corresponding tomogram subjectively resembles the input image very well. Most certainly, this does not say that this set of anisotropy coefficients yields the qualitatively and quantitatively best reconstruction results. Finding said ideal $k\ix{ani}$ profile is a very difficult task and involves, including but not limited to, multidimensional algorithmic optimizations and incorporation of \textit{a priori} knowledge and respective constraints. Underlined by the corresponding, close-to-unity fitness factor $\chi^{2}$ and the congruence of the forward integrated camera profiles, the first set of results represents an accurate reconstruction. However, the second set of $\left\{1.5, 0.25\right\}$ produces a qualitative similarly if not quantitatively better matching tomogram, radial and individual camera profiles, while finding a subjectively far less adequate two-dimensional radiation distribution. One should note that, purely examining the impact of singular $k\ix{core}$ and $k\ix{edge}$ coefficients, the first tomogram does show improved localisation in the core compared to the second, which in fact yields a smoother characteristic there. Also, the significantly lesser edge factor actually amplifies anisotropies in that area. The second reconstructions $\chi^{2}$ and MSD both indicate a better tomography result. In a scenario of experimental input data, the fitness coefficient is the only measure by which to examine the efficacy of the MFR. This again amplifies the importance of a well-rounded and thoroughly examined set of phantom image reconstructions with a focus on the relationship between quantitative and qualitative matching. Later series of MFR tomographies will focus on this particular aspect and evaluate the influence of and relationship between $k\ix{ani}$ and $\chi^{2}$, $P\ix{rad/2D}$ etc.\\%
%
                \newline%
                In summary, this exploration of superimposed, anisotropic phantom radiation images has provided significant insight into the efficacy and performance of the \textit{RDA MFR tomography} algorithm. Of particularly interest and relevance to later examinations of more complex artificial distributions is the experienced about local sensitivity and its entanglement with $k\ix{ani}$ profiles. Especially important towards the tomography of experimental measurement data is the knowledge that explicitly opposite to the multiple camera apertures located structures or characteristics in the input distribution are more difficult to reconstruct and potentially yield quantitatively and qualitatively drastically different results for varying weighting coefficients. From the presented images and numbers it can be deduced that the congruence between input radiation profile and tomogram is primarily dependent on the distance of the emissivity and hence the local resolution and LOS coverage in that location. Fitness factor $\chi^{2}$, as well as a comparison between the integrated, two-dimensional radiation power and $P\ix{rad}$ from the individual detector signals are not necessarily indicative of a good reconstruction, even across multiple $k\ix{ani}$ combinations for the same set of input data. All the tomographies have in common that the pronunciation and absolute brightness, i.e. the sharpness of the particular localised features, of the artificial distribution is not entirely reproduced and the level of radiation is significantly reduced overall, though depending on the configuration relative trends are preserved to some extent. The selection of anisotropy parameters and their profile shape have also shown that, given the stark contrast in tomogram structures across the phantom variations, for $k\ix{core}>1>k\ix{edge}$ explicit anisotropic characteristics are lost and translated into the SOL. Here they are, for the given set of input profiles, mostly found in areas of increased LOS density - e.g. X-points and magnetic islands share areas of larger $\varepsilon$. This simultaneously also enforces smoother emissivity or less poloidal gradients in the core. Furthermore, at $1>k\ix{core}=k\ix{edge}$, the inversion from LOS data of said properties is enabled in their original localisation, though with varying detail and width. Features of larger connection lengths or smooth and symmetrical emissions are not reproduced using such coefficients. This supports the assumption of, on the one hand, unfavourable radiation distributions with respect to their structure corresponding to the bolometer camera system and, on the other, the need for specific, tailored regularisation weights for individual input data.\\%
                Employing the so far gained experience from the previous phantom emissivity tomographies, one will now further extend the data space with variations of complex, experimentally motivated structures.%
%
        \subsection{Experimentally Motivated Anisotropic Phantoms}\label{subsec:phantoms_expani}%
%
            In the following paragraphs one will explore more complex, experimentally motivated and therefore particularly challenging to reconstruct phantom radiation profiles. The focus here will be specifically localised anisotropic distributions that also feature additional extremes in positions and number of the magnetic island and intersecting X-point structure. With the knowledge and experience gained from the previous analysis, this should prepare for and build the foundation necessary to adequately and confidently Reconstructive actual experimental data later on.\\%
%
            \begin{figure}[t]%
                \centering%
                \begin{subfigure}{\textwidth}%
                    \centering%
                    \makebox[\textwidth][c]{\includegraphics[width=1.2\textwidth]{%
                        content/figures/chapter4/MFR/new/phantom/2D/%
                        phantom_v_tomo2D_symR1.1_fsR0.8_mx1.0e+06_aniM3_2.0_0.1_nigs1_times0.11.png}}%
                    \caption{$k\ix{core},\,k\ix{edge}=\left\{2,\,0.1\right\}$}
                \end{subfigure}%
                \newline%
                \begin{subfigure}{\textwidth}%
                    \centering%
                    \includegraphics[width=\textwidth]{%
                        content/figures/chapter4/MFR/new/phantom/%
                        phantom_v_tomo_profiles_symR1.1_fsR0.8_mx1.0e+06_aniM3_2.0_0.1_nigs1_times0.11.png}%
                \end{subfigure}%
                \caption{Phantom radiation distribution reconstruction, mimicking a bright plasma core with six intensity maxima around the five islands of the magnetic standard configuration. The RDA coefficients that have been used support the accurate reconstruction of the designed profiles in the tomographic algorithm. \textbf{(a, top)} Phantom, tomogram and relative deviance, similar to \cref{fig:phantom_fsring_example}. \textbf{(down)} Radial (left) and chordal (right) profile analysis, similar to \cref{fig:phantom_fsring_example_profiles}.}\label{fig:phantom_islands_6_fsring_ani3}%
            \end{figure}%
%
            The first of this set of artificial emissivity profiles in \cref{fig:phantom_fsring_example_profiles} is a superposition of a symmetrical bright ring in the core at $0.7r\ix{a}$, like before, and a distribution of six singular, localised structures around the triangular plane in the SOL, close to the magnetic islands in $1.1r\ix{a}$. This outside anisotropy is constructed \textit{up-down} symmetrically, i.e. the discrete representation of the emissivity yields $P\ix{rad}^{\left(\text{i,j}\right)}=P\ix{rad}^{\left(\text{i,n$_{\vartheta}$-j}\right)}$, where $n_{\vartheta}$ is the number of poloidal bins and $i$ and $j$ the radial and poloidal indices for pixel $p^{\left(i,j\right)}$. One should note that this is trivial for $n_{\vartheta}$ even, however for an odd number of bins we construct the emissivity so that in pixel $p^{\left(i,m\right)}$ for $m=(n_{\vartheta}-1)/2+1$ the absolute value is the average of the neighbouring cells, so essentially $P\ix{rad}^{\left(i,m\right)}=P\ix{rad}^{\left(i,m-1\right)}$. Using%
%
            \begin{align}%
                m=\left\{\,(n_{\vartheta}-1)/2+1\,\,\text{for }n_{\vartheta}\text{ odd;}\qquad n_{\vartheta}/2\,\,\text{for }n_{\vartheta}\text{ even}\right.\,\,,\nonumber\\%
                \widetilde{P}\ix{r}^{\left(i,j\right)}\left(P,r\ix{0},\sigma\right)=\frac{P}{2\pi\sigma^{2}}\exp\left(-\frac{1}{2}\left(\frac{r^{\left(i\right)}-r\ix{0}}{\sigma}\right)^{2}\right)\,\,,\nonumber%
            \end{align}%
%
            the up-down split phantom emissivity can therefore be written in two parts as%
%
            \begin{align}
                \begin{split}\label{eq:prad_anisym}%
                    j\leq m&:\quad P\ix{phan}^{\left(i,j\right)}=%
                        \widetilde{P}\ix{r}^{\left(i,j\right)}\left(\hat{P},0.7r\ix{a},\sigma\ix{r}\right)+\frac{3}{2}\widetilde{P}\ix{r}^{\left(i,j\right)}\left(\hat{P},1.1r\ix{a},\sigma\ix{r}\right)\times\\%
                        &\hspace*{3cm}\frac{1}{2\pi\sigma_{\vartheta}^{2}}\exp\left(-\frac{1}{2}\left(\frac{1+\sin\left(\vartheta^{\left(j\right)}\delta_{\vartheta}+\vartheta\ix{0}\right)}{2}\right)^{2}\right)\\%
                    j>m&:\quad P\ix{phan}^{\left(i,j\right)}=P\ix{phan}^{\left(i,m-j\right)}%
                \end{split}\,\,.%
            \end{align}%
%
            Here, the radial width is designed as before, i.e. $\sigma\ix{r}=0.25r\ix{a}$ while the poloidal dimension of the SOL anisotropies is set to $\sigma_{\vartheta}=$\SI{0.2617}{\radian} or $\pi/12$. The amplitude of the phantom is set to $\hat{P}=$\SI{1}{\mega\watt\per\cubic\meter}, which directly corresponds to the brightness of the core structure. Hence, outside the separatrix, the individual island-like spots have a maximum of 150\% of the prior intensity. Due to the already mentioned difficulties of mapping continuous emissivity distributions to such relatively large and coarse, discrete pixel grid, the actual global maximum brightness in the phantom yields \SI{1.41}{\mega\watt\per\cubic\meter}. Two new parameter $\delta_{\vartheta}$ and $\vartheta\ix{0}$ are introduced, which will also be featured in the following artificial distributions, as the \textit{poloidal frequency of the anisotropies}, i.e. the spatial periodicity of the structures (in the SOL) and an angular offset. Here, for $j\leq m$ or $\vartheta^{\left(j\right)}\leq\pi$ - the definition of the poloidal bin number and dimension is changed from before to have its origin towards the center of the HBCm pinhole - one can find three distinguishable bright spots outside the LCFS. This is equal to $\sigma_{\vartheta}=$\SI[per-mode=reciprocal]{0.9549}{\per\radian}, while the first local maximum is rotated about one fourth of this period away from $\vartheta=0$ and therefore $\vartheta\ix{0}=0$. On the opposite, inboard side at $j=m$ or $\vartheta^{\left(j\right)}=\pi$, a small minimum separates the two structures that are mirrored around a hypothetical, horizontal median through the HBCm aperture and magnetic axis. Both are aligned along the outside edge of the corresponding island and spread into the neighbouring X-points.\\%
            A first reconstruction of the above artificial radiation distribution for $k\ix{ani}=\left\{2,0.1\right\}$ can be found in the center of \cref{fig:phantom_islands_6_fsring_ani3} as per usual. The anisotropic weighting profile is again given by $\Theta\ix{$N\ix{T}$}$ for $N\ix{T}=17$, a simple step from $k\ix{core}$ to $k\ix{edge}$ (see \cref{subsec:phantoms_symmring}), and is designed to emphasize smooth structures of large angular width in the core for $r<r\ix{a}$ and anisotropic brightness profiles with smaller poloidal expansion. Immediately standing out is the now distinct asymmetry around the core between the individual mirrored, artificial peaks. Particularly the top inboard bright spot is far more intense with its global maximum, while also being shifted closer to the corresponding X-point, very close to the domain boundary. The bottom two images similarly present the radial and camera forward integrated chordal brightness profiles, including the regularisation weights across the radius.\\%
            The performed and examined reconstruction in \cref{fig:phantom_islands_6_fsring_ani3} using $k\ix{ani}=\left\{2, 0.1\right\}$ is able to find the specifically designed poloidal anisotropies with a sharp outline and close to their original position in the input phantom, while also reproducing a smooth, less bright structure in the core. However, this does not come without the introduction of a significant asymmetry in the island-like localisations around the separatrix, particularly where the LOS density and sensitivity is increased - this pattern has already been discussed before -, i.e. in front the HBCm and VBCl pinhole and the largest intersection of their LOS cones. Respective to the radial profile and its discrepancy with the aritificial input, a strong radial broadening of the inside emissivity is highlighted here, though the SOL features yield only slightly higher radiation power on average. The outstanding congruence between the individual integrated and extrapolated power values $P\ix{rad/2D}$ underlines the overall quality of the tomogram, which is supported by the good agreement of the forward calculated camera measurements and their corresponding fitness factor $\chi^{2}$. The latter describes an adequate reconstruction, however of no exceptional grade if estimated by solely with value. Again, like for all previous phantom MFR tomographies, a large variance in the singular power numbers when integrating from two- or one-dimensional profiles is found. The difference between the \textit{actual} total power from the phantom and tomographic radiation profile and the absorber signals is greatly increased relatively and in absolute terms when compared to reconstructions of similar fitness, i.e. \cref{fig:phantom_fsring_asym_270deg_profiles}.\\%
            This MFR of a more complex, experimentally motivated phantom brightness profile using the same tools and gained knowledge from previous tomographies as a first attempt has proven successful. Given the \textit{best-guess} anisotropic weighting profile and parameter combination, the achieved results adequately present both quantitative and qualitative features of the artificial input image. This is another important step towards the reconstruction of actual experimental data. Furthermore, in the following segment varying sets of $k\ix{ani}$ for the same phantom and MFR method will be explored to solidify and expand the experience for this set of tools towards such data.%
%
            \subsubsection*{Anisotropic Weighting Parameter Variation}%
%
                \begin{figure}[t]%
                    \centering%
                    \begin{subfigure}{\textwidth}%
                        \centering%
                        \makebox[\textwidth][c]{\includegraphics[width=1.2\textwidth]{%
                            content/figures/chapter4/MFR/new/phantom/2D/%
                            phantom_v_tomo2D_symR1.1_fsR0.8_mx1.0e+06_aniM4_2.0_0.3_nT15_nW2_nigs1_times0.11.png}}%
                        \caption{$k\ix{core},\,k\ix{edge}=\left\{2,\,0.3\right\}$}%
                    \end{subfigure}%
                    \newline%
                    \begin{subfigure}{\textwidth}%
                        \centering%
                        \makebox[\textwidth][c]{\includegraphics[width=1.2\textwidth]{%
                            content/figures/chapter4/MFR/new/phantom/2D/%
                            phantom_v_tomo2D_symR1.1_fsR0.8_mx1.0e+06_aniM4_20.0_3.0_nT15_nW2_nigs1_times0.11_white.png}}%
                        \caption{$k\ix{core},\,k\ix{edge}=\left\{20,\,0.3\right\}$}%
                    \end{subfigure}%
                    \newline%
                    \begin{subfigure}{\textwidth}%
                        \centering%
                        \makebox[\textwidth][c]{\includegraphics[width=1.2\textwidth]{%
                            content/figures/chapter4/MFR/new/phantom/2D/%
                            phantom_v_tomo2D_symR1.1_fsR0.8_mx1.0e+06_aniM4_2.0_0.6_nT15_nW2_nigs1_times0.11_white.png}}%
                        \caption{$k\ix{core},\,k\ix{edge}=\left\{2,\,0.6\right\}$}%
                    \end{subfigure}%
                    \caption{MFR tomography like in \cref{fig:phantom_islands_6_fsring_ani3} where the RDA coefficients have been varied and are noted below. \textbf{(a, top)} Phantom, tomogram and relative deviance, similar to \cref{fig:phantom_fsring_example}. \textbf{(b, c)} Only the reconstruction and its respective error to the phantom image like above.}\label{fig:phantom_islands_6_fsring_kanivar}%
                \end{figure}%
%
                In \cref{fig:phantom_islands_6_fsring_kanivar}, the same artificial emissivity distribution as in \cref{fig:phantom_islands_6_fsring_ani3} is reconstructed using three drastically different, yet characteristically similar $k\ix{ani}$ profiles. The goal is, in principle, to evaluate the impact of varying weighting factor ratios and orders of magnitude for such a given phantom that features anisotropic as well as smooth and uniform structures. A starting point is the already established benchmark for a set of $k\ix{core}=$\SI{2}{\arbitraryunit} and $k\ix{edge}=$\SI{0.1}{\arbitraryunit} from before. Hence, this parameter set is extended by \textbf{(a)} $k\ix{ani}=\left\{2, 0.3\right\}$, \textbf{(b)} $\left\{20, 0.3\right\}$ and \textbf{(c)} $\left\{2, 0.6\right\}$. For all of these, the regularisation weights configuration is the same as before, i.e. RDA method with a shape of $\Theta\ix{$N\ix{T}$}$ for $N\ix{T}=17$. Furthermore, though in contrast, an extended examination is not performed here and the presentation of one-dimensional profiles as well as singular power values or fitness factors omitted. The focus is, after having already proven the generally applicability of this approach, to classify the new parametric weights with regards to their capability to qualitatively reproduce the characteristic features in the artificial profile.\\%
                In the top row of \cref{fig:phantom_islands_6_fsring_kanivar}, the input emissivity distribution and corresponding reconstruction and MSD profile for the anisotropy coefficient set \textbf{(a)} are shown. This combination emphasizes comparatively three times larger connection lengths in the SOL than the prior benchmark. The second row for parameter set \textbf{(b)} $k\ix{ani}=\left\{20, 0.3\right\}$ again prefers longer characteristics in the SOL and now amplifies emission from the core compared to the benchmark and prior reconstruction. Evolution of the ratio between $k\ix{core}$ and $k\ix{edge}$ from 20:1 to 6.67:1 to 66.67:1 underlines the change in focus of the regularisation coefficients towards concentration of radiation in the edge and then inside the LCFS. Finally, the last line of plots in \cref{fig:phantom_islands_6_fsring_kanivar} presents the tomogram and MSD for $k\ix{ani}=\left\{2, 0.6\right\}$, which now yields a core to SOL ratio of 3.33:1. The focus now is changed again to smoother, greater connection lengths in the SOL and less emphasis on a radiation majority in the core, which is underlined by the singular maximum outside the separatrix and secondary minor localisations.\\%
                Of the examined MFR tomographies, none exceeds the subjective quality that has been achieved in \cref{fig:phantom_islands_6_fsring_ani3} by $k\ix{ani}=\left\{2, 0.1\right\}$. Irrespective of the quantitative agreement between artificial and reconstructed radiation distribution, the latter yields a significantly better distinction of SOL localisation, with furthermore largely reduced erroneous emissivity in-between and a continuous profile at the target location inside the LCFS at the same time. However, certain aspects of the input phantom are highlighted more in the other tomograms, i.e. a clearly separated and relatively brighter, symmetrical ring in the core for $k\ix{core}=$\SI{20}{\arbitraryunit} or individual, more focused spots on the outside. Though the opposite effect presents when the $k\ix{core}:k\ix{edge}$ ratio is shifted away from the previous sets trend, as is shown by the images in the top. Increasing the SOL regularisation weight further yields a lessened maximum intensity and stronger characteristic smoothness in that area, while the separation from the core and its relative brightness is hence more pronounced in the bottom. In conclusion, the tomographic inversion for such a - in the case of experimental data potentially - complex radiation distribution of varying symmetric and anisotropic profiles inside and outside the separatrix is benefited by a set of $k\ix{ani}$ coefficients with a ratio of 10:1 and higher but no larger than 50:1 depending on the individual focus. Judging from the given tomograms and their RDA weights, an initial, educated guess for a set that adequately suits the reconstruction of smooth emissivities in the core and short connection length features in the SOL can therefore be constructed using $k\ix{core}=$\SIrange{2}{10}{\arbitraryunit} and $k\ix{edge}=$\SIrange{e-2}{0.5}{\arbitraryunit} for similar profile shapes to \cref{eq:kani}. Trivially, the opposite case is also true for equally inverted profiles.%
%
            \subsubsection*{Artificial Camera Supported MFR Tomography}%
%
                \begin{figure}[t]%
                    \centering%
                    \begin{subfigure}{\textwidth}%
                        \centering%
                        \makebox[\textwidth][c]{\includegraphics[width=1.2\textwidth]{%
                            content/figures/chapter4/MFR/new/phantom/2D/%
                            phantom_v_tomo2D_symR1.1_fsR0.8_mx1.0e+06_ARTm_aniM3_2.0_0.1_ARTf_nigs1_times0.11.png}}%
                        \caption{$k\ix{core},\,k\ix{edge}=\left\{2,\,0.1\right\}$}%
                    \end{subfigure}%
                    \newline%
                    \begin{subfigure}{\textwidth}%
                        \centering%
                        \includegraphics[width=\textwidth]{%
                            content/figures/chapter4/MFR/new/phantom/%
                            phantom_v_tomo_profiles_symR1.1_fsR0.8_mx1.0e+06_ARTm_aniM3_2.0_0.1_ARTf_nigs1_times0.11.png}%
                        \caption{}%
                    \end{subfigure}%
                    \caption{Phantom radiation distribution reconstruction, mimicking a bright core and islands-like structures in the edge like in \cref{fig:phantom_islands_6_fsring_ani3}. In this case, the camera system is extended by an artificial, secondary horizontal camera on the opposite (poloidally) side of the HBCm - its geometry has been introduced in \cref{fig:geometry_newcam_mirh}. The chosen RDA coefficients support the reconstruction of the given phantom distribution. \textbf{(a)} Phantom, tomogram and relative deviance, similar to \cref{fig:phantom_fsring_example}. \textbf{(b)} Radial (left) and chordal (right) profile analysis, similar to \cref{fig:phantom_fsring_example_profiles}. The new camera \textit{MIRh} has been added in the chord brightness profile.}\label{fig:phantom_islands_6_fsring_ARTm}%
                \end{figure}%
%
                Remaining still with the superimposed \textit{island-like} phantom radiation distribution from \cref{fig:phantom_islands_6_fsring_ani3}, given the complexity of the image and the challenges its inversion poses, additional tomograms are produced using an extended set of cameras and therefore LOS that incorporate the previously introduced artificial \textit{MIRh} from \cref{subsec:artf}. Results for a MFR tomography using the supplementary set of fifteen artificial detectors on the inboard side, covering the entire triangular plane are shown in \cref{fig:phantom_islands_6_fsring_ARTm}. Regularisation weights and anisotropic coefficient profile $k\ix{ani}=\left\{2, 0.1\right\}$ are kept the same as in the first case. The new arrays forward calculated signals in \textbf{(b)} are indicated with ARTf (pseudonym for \textit{ARTificial}), next to the usual composition of radial, anisotropic and forward calculated detector signal profiles, accompanied by the singular two- and one-dimensional integrated power values. The shown regularisation weight and $k\ix{ani}$, as well as the phantoms poloidally averaged plots are the same as before.\\%
                With respect to the first set of results in \cref{fig:phantom_islands_6_fsring_ani3} and constant $k\ix{ani}$, the MFR employing an additional set of LOS from the artificial \textit{MIRh} presents a subjectively improved radiation profile with significantly more distinct and accurately localised spots in the SOL. No improvement can be noticed in the separation or intensity of the emissivity inside the separatrix. However, judging from the poloidally averaged and forward calculated, one-dimensional plots, including the integrated powers and fitness coefficient, a quantitative upgrade can hardly be argued for, especially given the previously examined variability in those parameters for different regularisation weights. This furthermore does not equate to a degradation in MFR results, though it can be concluded that the complementary detectors and their area of observations yield a positive effect on the tomographies quality for complex, superimposed radiation profiles of this kind. Conceptually, assuming that the point-of-view and orientation of the new camera significantly differs from the others, adding more LOS and therefore measurement points undeniably enhances the regularisation capabilities. Most critically, at least $\left(N\ix{HBC}+N\ix{VBCr}\right)$ and at most $N\times\left(N\ix{HBC}+N\ix{VBCr}\right)$ new intersections contribute to the algorithmic inversion and the $N$ new absorber to $\mathbf{T}\in\mathbb{R}^{n\times m}$ - remember that $n$ is the total number of absorbers and $\mathbf{T}$ the transmission matrix. As discussed earlier, this also makes the geometric expression of local sensitivity more robust towards perturbations, as one can trivially verify using the same argument regarding the norm of the extended $\mathbf{T}^{\prime}$ and its condition $\kappa\left(\mathbf{T}^{\prime}\right)$, see Ghaoui et al.\cite{Ghaoui2002}.%
%
        \subsection{Geometry Error Propagation for Reconstructions}\label{eq:tomo_geo_error}%
%
            \begin{figure}[t]%
                \centering%
                \begin{subfigure}{\textwidth}%
                    \centering%
                    \makebox[\textwidth][c]{\includegraphics[width=1.2\textwidth]{%
                        content/figures/chapter4/MFR/new/phantom/2D/%
                        phantom_v_tomo2D_ring_R1.1_mx1.0e+06_tilt1.0_aniM3_2.0_2.0_nigs1_times0.11.png}}%
                    \caption{$k\ix{core},\,k\ix{edge}=\left\{2,\,2\right\}$}%
                \end{subfigure}%
                \newline%
                \begin{subfigure}{\textwidth}%
                    \centering%
                    \includegraphics[width=\textwidth]{%
                        content/figures/chapter4/MFR/new/phantom/%
                        phantom_v_tomo_profiles_ring_R1.1_mx1.0e+06_tilt1.0_aniM3_2.0_2.0_nigs1_times0.11.png}%
                    \caption{}%
                \end{subfigure}%
                \caption{Phantom radiation distribution and reconstruction of a bright ring around $1.1r\ix{a}$, similar to \cref{fig:phantom_fsring_example}. The incorporated camera geometry  for the tomographic inversion was altered and \textit{rotated} (tilted) clockwise by +\SI{1}{\degree}, as it was described and presented in \cref{fig:geometry_change_tilted}. RDA coefficients equally support smooth, isotropic emissivity profiles in the core and edge. \textbf{(a)} Phantom, tomogram and relative deviance, similar to \cref{fig:phantom_fsring_example}. \textbf{(b)} Radial (left) and chordal (right) profile analysis, similar to \cref{fig:phantom_fsring_example_profiles}.}\label{fig:phantom_fsring_tilt_1deg}%
            \end{figure}%
%
            A conclusive, necessary step before statistically exploring the MFR method in more detail and applying the gained experience in tandem with the tomography of experimental bolometer data is to gauge the impact of faulty assumptions or measurement errors in the camera geometry and by extension the transmission matrix $\mathbf{T}$. The goal here is the find out if and how big of an influence the propagation of alignment errors or discrepancies in the determination of the pinhole (i.e. camera) position has on the final quality of the tomogram. Premise for this approach is the \textit{directed} deviation of all LOS, i.e. all pinholes and absorbers are shifted in the same way with respect to their assumed geometry, so that the perturbed LOS fan yields a distinct and significant rotation compared to its original orientation. Furthermore, this particular effect is to be unknown and only affects the calculation of the pseudo-measurements as input for the MFR, which is performed still with the benchmark geometry. For this very purpose, a fairly simple, symmetrical and most importantly already examined and reconstructed phantom radiation profile will be used, and a reconstruction performed using the also previously introduced LOS geometries with $\pm$\SI{1}{\degree} poloidal tilt from the \textit{as-designed} construction. Details respective their setup can be found in \cref{subsec:geompertub}. The baseline artificial brightness distribution and MFR as a benchmark will be that of \cref{fig:phantom_fsring_example} and \cref{subsec:phantoms_symmring}, a symmetrical, thin ring of radiation in $r\ix{0}=1.1r\ix{a}$ with no anisotropic variation, using a uniform $k\ix{ani}=\left\{2, 2\right\}$. In both cases, the entirety of the multicamera bolometer system has been rotated collectively by the given angle around their corresponding aperture center locations in clockwise (\textit{positive}) or counter-clockwise (\textit{negative}) direction.\\%
            The first set of results can be found in \cref{fig:phantom_fsring_tilt_1deg} and features the original phantom and reconstructed tomogram based off of the \textit{as-designed} LOS system, as well as the forward integrated signals from the phantom, achieved with the misaligned geometry and used for the reconstruction and their reverse counterpart as before. At the center of \textbf{(a)}, the tomogram places stronger asymmetries at and around the bottom inboard, on the edge of the lower outboard and in-between the upper two magnetic islands. Their individual maximum brightness is higher than the intensity of the input ring. Coincidentally, the MSD profile yields the largest error values in that area with a large, expanded feature between the two neighbouring islands. The remaining deviations are far below at $\le$\SI{2.5}{\percent} around the triangular plane, while minor variations can be found towards the edge where the emissivity in the tomogram was broadened. Like described above, in \textbf{(b)}, results from the artificial distribution are produced using the geometry with the +\SI{1}{\degree} tilt in clockwise direction, while the back-calculated profiles from the tomogram are achieved by integration with the \textit{as-designed} LOS alignment. Overall shape and characteristics of the plots are the same as in the initial MFR of this particular image, i.e. in \cref{fig:phantom_fsring_example_profiles}. As in the latter, the horizontal cameras detectors indicate a noticeable asymmetry in both profiles, though the structure is the same towards and around each portion of the array that tangentially views a part of the separatrix. With respect to the established benchmark of this artificial distribution, the discrepancy between detectors viewing the upper and lower portion of the SOL is markedly reduced. In the phantom, absorbers with LOS through the lower half of the triangular plane find lower emissivity values compared to the already reduced results from the tomogram. Although the integral power per channel is still within the corresponding error confidence interval, a relatively consistent deviation of $\sim$\SI{7.5}{\percent} can be found. The VBCr shows the reverse behaviour, where absorbers towards the outboard side, watching the SOL and LCFS yield higher brightness measurements of the same rate from channel no. five through zero. On the other side, the left vertical bolometer camera presents no significant variance between the two plots, besides a negligible decrease in the reconstructed results closer to the core, i.e. detectors no. three and below. In contrast to the visually well-matched to profiles, extrapolation of the shown absorber signals indicates a large discrepancy between phantom and tomogram respectively. The low fitness factor of $\chi^{2}=$\SI{0.513}{\arbitraryunit} underlines the latter observation.\\%
%
            \begin{figure}[t]%
                \centering%
                \begin{subfigure}{\textwidth}%
                    \centering%
                    \makebox[\textwidth][c]{\includegraphics[width=1.2\textwidth]{%
                        content/figures/chapter4/MFR/new/phantom/2D/%
                        phantom_v_tomo2D_ring_R1.1_mx1.0e+06_tilt-1.0_aniM3_2.0_2.0_nigs1_times0.11.png}}%
                    \caption{$k\ix{core},\,k\ix{edge}=\left\{2,\,2\right\}$}%
                \end{subfigure}%
                \newline%
                \begin{subfigure}{\textwidth}%
                    \centering%
                    \includegraphics[width=\textwidth]{%
                        content/figures/chapter4/MFR/new/phantom/%
                        phantom_v_tomo_profiles_ring_R1.1_mx1.0e+06_tilt-1.0_aniM3_2.0_2.0_nigs1_times0.11.png}%
                    \caption{}%
                \end{subfigure}%
                \caption{Same phantom radiation distribution and reconstruction problem as in \cref{fig:phantom_fsring_tilt_-1deg}, but for a camera geometry rotation of \SI{-1}{\degree}. \textbf{(a)} Phantom, tomogram and relative deviance, similar to \cref{fig:phantom_fsring_example}. \textbf{(b)} Radial (left) and chordal (right) profile analysis, similar to \cref{fig:phantom_fsring_example_profiles}.}\label{fig:phantom_fsring_tilt_-1deg}%
            \end{figure}%
%
            A second set of results for the corresponding rotation of the LOS fans of all three cameras by $-$\SI{1}{\degree} is shown in \cref{fig:phantom_fsring_tilt_-1deg} and presented in the same way for also the same set of $k\ix{ani}$. In the center of \textbf{(a)}, the phantom features an asymmetry and global maximum around the inside edge of the upper inboard magnetic islands that extends along the $1.1r\ix{a}$ radius also towards the inboard side. A minor, secondary maximum is located similarly to before between the lower two field structures in the SOL, while the rest of the ring is greatly reduced in brightness. Correspondingly, the MSD on the right shows large error values where the global maximum was found and smaller ones in the lower outboard magnetic island. The tomograms poloidally averaged profile in \textbf{(b)} is noticeably shifted towards the separatrix and core. All the profile inside and outside the LCFS is moved by $0.06r\ix{a}$ closer to the magnetic axis, including the location of the minutely reduced peak. The previously examined and discussed dissymmetry in the HBCm results is amplified here on the lower side viewing the SOL, where the forward integrated signals are now double that of the opposite side, no. 23 through 27 in both phantom and tomogram. Otherwise, core and overall profile are qualitatively similar compared to \cref{fig:phantom_fsring_tilt_1deg}. No significant changes with respect to the initial benchmark can be found for the vertical bolometer array detectors. Like the last set of forward calculations, deviations between tomogram and phantom plot for all cameras are limited to the provided error confidence interval, though in contrast not particularly elevated in or around extremes. Finally, again clashing with the observations above is a weak fitness of $\chi^{2}=$\SI{0.394}{\arbitraryunit}.\\%
            The criticality and importance concerning the underlying issue of this examination towards the MFR and its performance is highlighted by the variance of $P\ix{rad}$ across the different sets of results. In this case, substantial increments compared to the baseline geometry extrapolation are found from the faulty profiles. Furthermore, the reproduced two-dimensional emissivity distributions show noticeable and particular characteristics that align with the designed error in the LOS orientation. Similarly, particularly in the HBCm forward calculated measurements, a correlating effect of the rotation on the shape and intrinsic asymmetry of the profile is very prominent. A significantly worse fitness coefficient compared to the previous supports the above findings, though the detector signal integration trend towards the \textit{actual} radiation powers yield the contrary picture. From the presented impact of a distinct error in the transmission matrix $\mathbf{T}$, unfortunately no other attribute than the increased or decreased characteristic in the individual camera results for symmetrical radiation distributions definitively points towards such underlying discrepancies. In a worst case scenario like this cumulative deviation in the location measurement of the bolometer or changes in its construction alignment, a divergence of $\ge$\SI{10}{\percent} in both singular absorber data and tomogram values are to be expected and in an experimental context nearly impossible to account for.%
%
        \subsection{Reconstructive Limits}\label{subsec:limits}%
%
            A quantitative, i.e. statistical evaluation of phantom emissivity distributions shall conclude this section. Here one will focus on systematic iteration of the anisotropic regularisation weights $k\ix{core}$ and $k\ix{edge}$ individually across larger spectra than before and recording the findings of the reconstruction for simple artificial images, i.e. \textit{a bright ring} like before in \cref{subsec:phantoms_symmring}. For this purpose, the previously introduced and examined characteristics, i.e. extremes location and FWHM, \textit{mean squared deviation} (MSD) of \cref{eq:msd}, integrated two-dimensional power values from the phantom and tomogram and the fitness factor $\chi^{2}$. At this opportunity, an additional measure is introduced to gauge the agreement between input and output of the algorithm. The \textit{Pearson\footnote[1]{Karl Pearson, FRS, FRSE * Mar 27, 1857 \textdagger Apr. 17, 1936; English mathematician and biostatistician} correlation coefficient (PCC)} $\rho\ix{c}$ or $\rho\ix{x,y}$ between quantities $x$ and $y$ measures their linear correlation. The number $\rho\ix{c}$ is the ratio of covariances and product of their respective standard deviations, therefore its result is binned by $\left[-1, 1\right]$. Let $E\left(\vec{x}\right)$ be the \textit{expected value} of variable $\vec{x}$, i.e. the vectorized representation of the phantom or tomogram like in \cref{eq:fisher_algo} and \cref{subsec:phantoms_expani}. The Pearson coefficient hence can be written as:%
%
            \begin{align}%
                \begin{split}\label{eq:pearson}%
                    E\left(\vec{x}\right)&=\frac{1}{n\ix{r}n_{\vartheta}}\displaystyle\sum_{i=0}^{n\ix{r}n_{\vartheta}}x^{\left(i\right)}\overset{!}{=}\overline{x}\,\,,\\%
                    \rho\ix{x,y}=\rho\ix{C}\left(\vec{x},\,\vec{y}\right)&=\frac{E\left(\left(\vec{x}-E\left(\vec{x}\right)\cdot\vec{1}\right)\,\left(\vec{y}-E\left(\vec{y}\right)\cdot\vec{1}\right)\right)}{\sqrt{\vec{x}-E\left(\vec{x}\right)\cdot\vec{1}}\sqrt{\vec{y}-E\left(\vec{y}\right)\cdot\vec{1}}}\,\,,\\
                    &\overset{!}{=}\frac{1}{\sigma_{\vec{x}}\sigma_{\vec{y}}}\text{cov}\left(\vec{x},\,\vec{y}\right)\,\,.%
                \end{split}%
            \end{align}%
%
            \subsubsection*{Isotropic Rings}%
%
                \begin{figure}[t]%
                    \centering%
                    \begin{subfigure}{0.45\textwidth}%
                        \centering%
                        \includegraphics[width=\textwidth]{%
                            content/figures/chapter4/MFR/scan/%
                            phantom_scan_fs_reff_ring_kcore_scan_old.png}%
                        \caption{}%
                    \end{subfigure}%
                    \begin{subfigure}{0.45\textwidth}%
                        \centering%
                        \includegraphics[width=\textwidth]{%
                            content/figures/chapter4/MFR/scan/%
                            phantom_scan_fs_reff_ring_kedge_scan_old.png}%
                        \caption{}%
                    \end{subfigure}%
                    \caption{MFR scan of phantom radiation distribution consisting of an isotropic ring at $0.5r\ix{a}$ with \SI{0.1}{\meter} FWHM. \textbf{(a)} Core and \textbf{(b)} edge RDA coefficients are varied independently. In both, top to bottom: radial position of the maximum, FWHM, MSD for core, SOL and in total, total and core integrated 2D powers and fitness scales $\chi^{2}$ and \textit{Pearson coefficient} $\rho\ix{c}$.}\label{fig:phantom_scans_fsring_kani_coreEdge}%
                \end{figure}%
%
                In \cref{fig:phantom_scans_fsring_kani_coreEdge}, these particular sets of attributes are collected for a simple artificial radiation distribution consisting of a singular bright ring at $0.5r\ix{a}$ after the example in \cref{fig:phantom_fsring_example}. The left and right plot series are split into the individual $k\ix{core}$ and $k\ix{edge}$ variations across ranges of $\left[0.65, 1\right]\text{a.u.}$ and $\left[1.25,1.8\right]\text{a.u.}$ respectively. While one of those is altered, the other is kept at unity, i.e. $k\ix{ani}=$\SI{1}{\arbitraryunit}, yielding no change in regularisation. From top to bottom, the presented images show: the global maximum brightness radial location, its full width at half maximum height (FWHM), individual core, SOL and total MSD according to \cref{eq:msd}, the core and absolute integrated, two-dimensional power from \cref{eq:prad2D} and finally the fitness factor $\chi^{2}$ and Pearson coefficient $\rho\ix{c}$ in the bottom figure. Each plot contains two sets of lines for phantom and tomogram separately, except the very last in which the quantities are derived from both together. If applicable, multiple abscissa are indicated where the order of magnitude between the profiles is distinctly different.\\%
                Focusing on one column of plots at a time, the vertical alignment for individual $k\ix{ani}$ of features in the respective plots reveals an intuitive and characteristic behaviour across that spectrum. On the left, the fitness is closest to unity, while the Pearson coefficient is largest for $k\ix{core}=$\SIrange{0.9}{0.95}{\arbitraryunit}. At the same time, the cumulative core, SOL and absolute deviation in total are lowest and the error in integrated power is relatively constant at around \SI{0.2}{\mega\watt}. However, though the peak localisations are comparatively improved for those values, the FWHM deviate greatly here with respect to the lower $k\ix{core}$. Similarly, on the right, the lowest individual and total two-dimensional variance is found for $k\ix{edge}=$\SI{1.45}{\arbitraryunit}. At the same time, $\rho\ix{c}$ is largest and $\chi^{2}$ is very close to unity, while the reconstructed structures are radially close to the location of the input profile. The corresponding widths are also in stark contrast here and $P\ix{rad,2D}$ yields again an equivalent error, though a singular outlier for a smaller $k\ix{edge}$ does not align with those observations.\\%
                Conclusively, for this kind of simple and symmetric radiation distribution, there do exist ideal $k\ix{core}$ and $k\ix{edge}$, individually given a fixed counterpart. In this case, the quality parameters shown are largely in strong agreement across the spectra and no significant characteristic is found that is in contradiction. However, a large drawback of this particular approach is the large parameter range that has not been explored by this relatively expensive method. For only one phantom image, a small spectrum of anisotropic weighting coefficients has been evaluated separately. For $N$ the number of variations in this small window, one would have to perform at least $N\times N-2N$ more reconstructions to find a definitive answer to what the actual optimum $k\ix{ani}$ profile is. Another possible ansatz is the variation of the artificial brightness distribution for constant reconstruction settings, i.e. changing the radius of the ring in the previous data set instead of $k\ix{core}$ and $k\ix{edge}$ each.\\%
%
                \begin{figure}[t]%
                    \centering%
                    \begin{subfigure}{0.45\textwidth}%
                        \centering%
                        \includegraphics[width=\textwidth]{%
                            content/figures/chapter4/MFR/scan/%
                            phantom_scan_single_ring_radius_scan.png}%
                        \caption{single ring radius scan}%
                    \end{subfigure}%
                    \begin{subfigure}{0.45\textwidth}%
                        \centering%
                        \includegraphics[width=\textwidth]{%
                            content/figures/chapter4/MFR/scan/%
                            phantom_scan_dfs_R1_05_inner_radius_scan_old.png}%
                        \caption{inside ring radius scan}%
                    \end{subfigure}%
                    \caption{MFR scan of phantom images consisting of \textbf{(a)} one or \textbf{(b)} two concentric, isotropic, equally bright rings. In \textbf{(b)} only the inside rings' and in \textbf{(a)} both rings' radii have been varied, while the same RDA coefficients $k\ix{core}, k\ix{edge}=\left\{2, 2\right\}$ are used throughout. The layout of the results is similar to \cref{fig:phantom_scans_fsring_kani_coreEdge}.}\label{fig:phantom_scans_fsring_doubleRings}%
                \end{figure}%
%
                This procedure has been applied in \cref{fig:phantom_scans_fsring_doubleRings}, where two simple phantom radiation profiles have been varied for a set of constant $k\ix{ani}=\left\{2, 2\right\}$. The same set of quantities as in \cref{fig:phantom_scans_fsring_kani_coreEdge} have been measured for each of the results. In \textbf{(a)}, the radial position of a magnetic axis concentric, isotropic singular thin, bright ring is altered between $0.35r\ix{a}--1.3r\ix{a}$, while at the two very last iterations the actual maximum emissivity in the phantom is close to or outside the regularisation domain. The presented data indicate that for such a constant set of regularisation parameters, reconstructions perform best for a concentrated emissivity in the range of $0.9--1.1r\ix{a}$, i.e. in and around the LCFS. Although fitness and Pearson coefficient both produce maxima at opposite ends of the spectrum and are not at unity at those radii, they yield sufficient values indicating good overall agreement between the forward and two-dimensional profiles. This is also true for the presented individual and absolute variance, as well as all the shown power numbers. With respect to the first plots of extremes location and shape, the largest diameter have to be excluded or at least treated differently in terms of anisotropic weighting and statistic evaluation, further supporting the optimum up until that range.\\%
                \autoref{fig:phantom_scans_fsring_doubleRings}:\textbf{(b)} shows similar variations for a combination of two constant, thin, bright concentric rings, in which one is held at $1.05r\ix{a}$ and the other altered in radius between $\left(0.4--1\right)r\ix{a}$. For the given configuration of $k\ix{ani}=\left\{2, 2\right\}$ and ring positions, i.e. a fixed larger concentric structure and a variable smaller inside, this tomography yields generally acceptable results. In contrast to on the left, no significant deviations in reconstruction quality across the radial spectrum of the inner ring can be observed - fitness and Pearson coefficient are within 2\% and 8\% respectively. Location and shape of the rings are consistently and accurately represented, while the overall power is congruent throughout. Both absolute variance and fitness factor $\chi^{2}$ however indicate a worse match of the tomogram at an inner radius of $0.8r\ix{a}$, though beyond and hence for a smaller distance between the profiles the error decreases and $\chi^{2}$ gets closer to unity again. On the other hand, $\rho\ix{c}$ is smallest for the greater radial differences and closest to \SI{1}{\arbitraryunit} in that range at and beyond $0.8r\ix{a}$. Coincidentally and almost trivially, the SOL deviation also peaks here and the individual peak detection eventually fails. Conclusively, the given radiation structures are reconstructed with the presented configuration nearly uniformly well, except a threshold value and negligible spacings. The prior appears to, judging from the shown results, not correlate with known characteristic of this system.%
%
            \subsubsection*{Anisotropic Island Combinations}%
%
            \begin{figure}[t]%
                    \centering%
                    \captionsetup{width=.37\textwidth}%
                    \begin{minipage}[c]{0.37\textwidth}%
                        \centering%
                        \caption{%
                           MFR scan of a phantom image consisting of \textit{up-down} symmetrical chain of island-like radiation spots at $1.1r\ix{a}$ of \SI{0.1}{\meter} FWHM (exemplary artificial profile can be found in \cref{fig:phantom_islands_6_fsring_ani3}). Edge RDA coefficients have been varied independently. The layout of the results is similar to \cref{fig:phantom_scans_fsring_kani_coreEdge}.}\label{fig:phantom_scans_sym5_kani_edge}%
                    \end{minipage}%
                    \hfill%
                    \begin{minipage}[c]{0.53\textwidth}%
                        \centering%
                        \includegraphics[width=\textwidth]{%
                            content/figures/chapter4/MFR/scan/%
                            phantom_scan_sym_R1_1_m5_kedge_scan.png}%
                    \end{minipage}%
                \end{figure}%
%
                Of much greater relevance towards the reconstruction of potentially more complex and anisotropic experimental data is the individual $k\ix{core}$ and/or $k\ix{edge}$ variation for an \textit{island chain-like} artificial profile like in \cref{fig:phantom_scans_sym5_kani_edge}. Here, a ring of five island-like structures is placed, like before in \cref{fig:phantom_scans_fsring_kani_coreEdge}, around the triangular plane at $1.1r\ix{a}$ with a FWHM of $\SI{0.1}{\meter}$. For sake of comprehensibility, and more akin to a hypothetical, experimental approach where the actual emissivity distribution is not known, only $k\ix{edge}$ is varied in the range of $\left[0.1, 6\right]\text{a.u.}$, while $k\ix{core}=$\SI{1}{\arbitraryunit} is kept constant. Presented results initially yield contradictory conclusions: the functional relationship for the RDA should produce more anisotropic patterns for smaller $k\ix{ani}$, as is desired here, particularly $k\ix{ani}<$\SI{1}{\arbitraryunit}, however the opposite appears to be the case for this set of plots. For $k\ix{edge}<$\SI{2}{\arbitraryunit}, the total and core localized variance is largest at an approximately constant integrated SOL error. Furthermore, with the exception of $k\ix{edge}=$\SI{0.25}{\arbitraryunit}, fitness and Pearson coefficient also only increase with $k\ix{edge}$ towards unity. Comparisons in individual power values or radial emissivity peak analysis show no particular behaviour, although the latter features secondary structures in addition to the match at $1.05r\ix{a}$, with one consistently at the very core.\\%
                One can assume that the large core variance also leads to a larger total error due to the dark center of the underlying phantom radiation distribution, i.e. $E\left(\vec{x}\right)_{\mu\rightarrow0}>>1$, hence skewing the results if any emissivity is erroneously reconstructed for $r<r\ix{a}$. The small local and global maxima respectively for $\chi^{2}$ and $\rho\ix{c}$ at $k\ix{edge}=$\SI{0.25}{\arbitraryunit} indicate that this setting achieves a good anisotropy in the resulting tomogram, though some smoothing still has to occur given their distance from unity and coincidental aforementioned variance, as well as inconclusive power integrals and radial pattern match. Underlined by the higher fitness, Pearson coefficient and smaller total error for larger $k\ix{ani}$ here, blurred out localized structures of this size and absolute brightness appear to be more easily misidentified. This scan shows that, assuming an anisotropic radiation distribution, a delicate approach to dialling in $k\ix{ani}$ is absolutely necessary and very well depends on its respective intensity and spatial characteristics.%
%
            \subsubsection*{Composite Phantom Images}%
%
                \begin{figure}%
                    \centering%
                    \makebox[\textwidth][c]{%
                        \begin{subfigure}{0.55\textwidth}%
                            \centering%
                            \includegraphics[width=0.8\textwidth]{%
                                content/figures/chapter4/MFR/new/phantom/2D/%
                                phantom_x475_sigma02.png}%
                            \caption{%
                                $R\ix{rad}=\,$\SI{4.75}{\meter}}%
                        \end{subfigure}%
                        %\hfill%
                        \begin{subfigure}{0.55\textwidth}%
                            \centering%
                            \includegraphics[width=0.8\textwidth]{%
                                content/figures/chapter4/MFR/new/phantom/2D/%
                                phantom_x575_sigma02.png}%
                            \caption{%
                                $R\ix{rad}=\,$\SI{5.75}{\meter}}%
                        \end{subfigure}%
                        }%
                    \caption{Phantom images consisting of a singular bright spot of \SI{0.2}{\meter} FWHM -  \textbf{(a)} inboard and \textbf{(b)} outboard anisotropies.}\label{fig:phantom_bright_spots_ani}%
                \end{figure}%
%
                A less complex, however more intricate to perform tomography with artificial radiation distribution is simultaneously introduced and reconstructively scanned by varying $k\ix{core}=\left\{0.4, 1\right\}\text{a.u.}$ in \cref{fig:phantom_scans_xx75_kani_core}. \autoref{fig:phantom_bright_spots_ani}:\textbf{(a)} shows a singular, large bright spot with omnidirectional Gaussian profile and \SI{0.2}{\meter} FWHM placed in $z=0$ and major radius $R\ix{rad}=$\SI{4.75}{\meter}, whereas in \textbf{(b)} the same is done for $R\ix{rad}=$\SI{5.75}{\meter}. Their maximum brightness is \SI{0.85}{\mega\watt\per\cubic\meter}.\\%
%
                \begin{figure}%
                    \centering%
                    \begin{subfigure}{0.45\textwidth}%
                        \centering%
                        \includegraphics[width=\textwidth]{%
                            content/figures/chapter4/MFR/scan/%
                            phantom_scan_pos_mesh_x475_kcore_scan_old.png}%
                        \caption{%
                            $R\ix{rad}=\,$\SI{4.75}{\meter}}%
                    \end{subfigure}%
                    \hfill%
                    \begin{subfigure}{0.45\textwidth}%
                        \centering%
                        \includegraphics[width=\textwidth]{%
                            content/figures/chapter4/MFR/scan/%
                            phantom_scan_pos_mesh_x575_kcore_scan_old.png}%
                        \caption{%
                            $R\ix{rad}=\,$\SI{5.75}{\meter}}%
                    \end{subfigure}%
                    \caption{Reconstruction scan for the corresponding phantom images in \cref{fig:phantom_bright_spots_ani}, a singular bright spot of \SI{0.2}{\meter} FWHM -  \textbf{(a)} inboard and \textbf{(b)} outboard anisotropies. In both, $k\ix{core}$ has been varied in the same range. The layout of the results is similar to \cref{fig:phantom_scans_fsring_kani_coreEdge}.}\label{fig:phantom_scans_xx75_kani_core}%
                \end{figure}%
%
                In the tomographic results on the left, for $k\ix{core}\le$\SI{0.75}{\arbitraryunit} multiple, erroneous maxima are found. The relative variance in the core is nearly constant with \SIrange{10}{13}{\percent}. On the other hand, the SOL error is significantly larger and varies strongly, with a steady decay towards $k\ix{core}=$\SI{0.95}{\arbitraryunit}. Obviously the shape of the absolute deviation is dominated by the outside error, hence the alike profile. In the phantom, the total integrated power from the two-dimensional distribution is entirely represented by the core and constant at \SI{0.39}{\mega\watt}. Results from the phantom also show congruence between the inside the separatrix and absolute calculated values throughout the $k\ix{core}$ spectrum. Finally, the fitness $\chi^{2}$ similarly decreases with larger $k\ix{core}$. Curiously, the Pearson coefficient $\rho\ix{c}$ indicates the opposite behaviour, i.e. increasing towards $k\ix{core}=$\SI{0.75}{\arbitraryunit} and then quickly to its maximum in \SI{0.95}{\arbitraryunit} before dropping conclusively.\\%
                The $k\ix{core}$ scan of the second tomogram in \cref{fig:phantom_bright_spots_ani}:\textbf{(b)} is shown in \cref{fig:phantom_scans_xx75_kani_core}:\textbf{(b)}. Here, the multiplicity and spread of the erroneous and individually reconstructed peaks is greatly increased compared to before. Until \SI{0.8}{\arbitraryunit}, six to seven maxima are found between $0.125r\ix{a}-0.65r\ix{a}$. Beyond, the number of points is decreased noticeably for larger $k\ix{core}$. Integrated error values from the SOL are also very large here. Since the core variance is comparatively constant around \SI{25}{\percent}, the total discrepancy throughout the $k\ix{core}$ variation is dominated by the difference in SOL emissivity. Fitness factor and Pearson coefficient both indicate an optimal reconstruction within the presented interval, where $\chi^{2}$ and $\rho\ix{c}$ are closest to unity, i.e. $\sim$\SI{1.5}{\arbitraryunit} and \SI{0.87}{\arbitraryunit} respectively.\\%
                The $k\ix{ani}$ variation results for the above, highly anisotropic phantom radiation distribution evidently pronounce a strong optimum within the limited range. For such clearly separated and singular emissivity features inside the separatrix, a factor of $k\ix{core}=$\SI{0.95}{\arbitraryunit} yields adequate tomographic reconstructions in regard to the quantitative analysis. Given the generality of the approach and underlying artificial images, one expects similar behaviour with larger scale asymmetric features in superimposed profiles or experimental data. However, the variance in quality of results in this small window somewhat mitigates the significance of said conclusions, since minor differences in either localisation, shape or $k\ix{ani}$ may lead to severe changes in the tomogram. The asymmetry in reconstruction parameter profiles around the optimum is of particular concern for the exploration of non-artificial radiation data. Furthermore, the same is true for the noticeable discrepancies between the in- and outboard anisotropy and hence the images presented in the left and right column of \cref{fig:phantom_scans_xx75_kani_core}, which can be trivially attributed to the geometry of multicamera bolometer system.\\%
%
                \newline%
                This concludes the efforts of benchmarking and evaluating the \textit{Minimum Fisher regularisation} algorithm with \textit{radially dependent anisotropy} weighting with a large set of deliberately designed and, albeit small scale, statistically varied set of phantom emissivity distributions as input. The propagation and their level of impact of errors in the underlying line-of-sight geometry has also been measured. Due to the nature of the multidimensional parameter dependency of the quality of MFR results, variations of such in an attempt to find optimal configurations or sets of coefficients like $k\ix{ani}$ have proven difficult. However, for the select set of artificial radiation profiles and hence conceptually it was shown that there does exist an ideal combination of MFR parameters for a given emissivity distribution. As per design, there it was also found that regularisation weights $k>1$ correspond to isotropic and coefficients $k<1$ to anisotropic brightness profiles with improved quality factors for fitness, $\chi^{2}$ and two-dimensional correlation, $\rho\ix{c}$.\\%
                Respective setups of symmetrical phantom images with reconstruction weights favouring smooth distributions have indicated geometric biases towards the individual camera apertures location, i.e. areas of increased lines-of-sight coverage along the lower part of the separatrix, inboard side and at the outboard tip of the triangular plane. Furthermore, from the perspective of the up-down symmetrically designed horizontal bolometer camera, this and the intrinsic tilt of the detector fan leads to a systematic and intrinsic asymmetry in the produced forward and backward calculated radiation powers. Individual case testing with deliberately symmetric-asymmetric superimposed artificial profiles presented results of strongly divergent qualities, further underlining the previous assessment. Simultaneously, variation in $k\ix{ani}$ highlighted significant discrepancies between the one- and two-dimensional pictures, hinting at independent qualitative developments in both.\\%
                Experimentally motivated phantom emissivity distributions, i.e. combined smooth, symmetric core and anisotropic SOL profiles mimicking hotspots in place of the magnetic islands prove to be a particular challenge for the MFR. With respect to the structure and localisation of the outside features, the regularisation weight optimisation is especially important. Addition of a hypothetical, inboard located horizontal camera has provided noticeable qualitative improvement for this type of phantom, suggesting and supporting the demand for an extension of the existing multicamera bolometer system. Finally, occlusion of the anisotropic profile on the inside by a homogenous, similarly bright ring yields promising results towards the inversion of experimental data compared to the initial case.\\%
                The previously introduced potential discrepancy between the assumed and actual \textit{in-situ} geometry of the bolometer camera array was tested and evaluated in regard to its impact as an unknown, systematic error to the forward and backward calculated line integrated signals. Alternation of the aforementioned variance produces characteristic divergence of corresponding orientation in the MFR tomograms and profiles. However, that effect is indistinguishable from the various other variabilities in the reconstruction and of no such noticeable magnitude. This then provides an additional uncertainty to be taken into account when evaluating actual measurement data.\\%
                Iteration in $k\ix{ani}$ and thorough quantitative analysis for principle emissivity shapes or combinations thereof was performed. Significant anti-/correlation between the individual variance, fitness and Pearson coefficients underlined the previously, only in isolated cases presented discrepancy to the subjectively perceived tomogram quality. The shown spectra also visually underline the proposed optima of MFR parameters through structural congruence or extremes in qualitative numbers.\\%
                An issue that has been highlighted very clearly in the previous sections is the difficulty of finding a single set of RDA regularisation coefficients that is applicable to all or even a larger set of similar radiation distributions, which is particularly true for asymmetric or complex anisotropic radiation distributions. The large impact of shape and position of emissivity localisations, relative to the composition of isotropic structures makes it even difficult to establish an optimal $k\ix{ani}$ profile. Also shown through the statistics for comparable phantoms, neither fitness parameters nor subjective reconstructive quality necessarily align and therefore complicate things further. However, from experience with the collected data a set of $k\ix{core},\,k\ix{edge}=$2.0\,a.u.,\,0.3\,a.u. is a solid choice to find smooth core distributions and localised edge radiation.%
%
        \subsection{Tomographic Reconstruction Statistics}\label{sec:phantomstat}%
%
            The final step is a \textit{pseudo-statistical} analysis of the accumulated phantom image data Minimum Fisher reconstructions with respect to properties already introduced in the benchmarking process. The goal here is to, on one hand find possible underlying correlations in these numbers for artificial radiation distributions and on the other evaluate the fit of the previous tomograms with their corresponding core plasma parameters, i.e. $P\ix{rad}$.%
%
            \begin{figure}[t]%
                \centering%
                \includegraphics[width=.9\textwidth]{%
                    content/figures/chapter4/MFR/new/%
                    phantom_statistics_comp.png}%
                \caption{%
                    Collected results of previously presented tomographic phantom radiation profile reconstructions in \cref{sec:phantoms}. A red, dashed line indicates congruence between the two, i.e. 1:1 relation. \textbf{(left):} From top to bottom: FWHM position, FWHM maximum, $P\ix{rad}$ from both cameras and two-dimensional, integrated emissivity. \textbf{(right):} From top to bottom: MSD over $\chi^{2}$, MSD and $\chi^{2}$ over Pearson coefficient $\rho\ix{C}$, $P\ix{rad}$ over integral of core emissivity and over total integrated radiation.}\label{fig:tomo_phantom_statistics}%
            \end{figure}%
%
            Combinations of the MFR phantom radiation image benchmark are collected and condensed into \cref{fig:tomo_phantom_statistics}. Included are tomograms that produced an absolute deviation from the phantom that is less than the integral of the radiation distribution. This set of proxy-reconstruction pairs is far larger than what has been discussed above, but for sake of clarity they can not all be presented individually. In the left column of plots, the abscissa values are populated by data from the artificial input distributions, while the ordinate is defined by their respective tomogram match. The right column shows combinations of quality parameters \textit{MSD}, Pearson coefficient $\rho\ix{c}$ and fitness factor $\chi^{2}$, as well as the two-dimensional integrated powers and the individual $P\ix{rad}$ extrapolations from the forward calculated detector signals. In all images if applicable, a 1:1 congruence line with \SI{45}{\degree} inclination is included to highlight agreement in the data.\\%
            At the top on the left, the FWHM locations in the phantoms are compared to their reconstructed counterparts. However, as it was shown before, equality in number of radial extremes for both is not trivial, so value pairs $\left(FWHM\ix{phan}, FWHM\ix{tom}\right)$ are produced here only where this is the case. Throughout the radial spectra, points are generally grouped close to or at the aforementioned line, with a few outliers mostly indicating smaller structures in the reconstruction. For the peak widths, the number of combinations is improved since every radial profile yields a result, as long as at least one extremum is produced. The maxima however do not show such behaviour, presenting the input FWHM widths to be consistently and considerably larger. In the next two plots, $P\ix{rad}$ from both forward calculated bolometer camera signals on the top and at the bottom the integrated two-dimensional radiation powers in the core and in total are compared. The prior shows great congruence with only one outlier, i.e. throughout the power spectrum $P\ix{rad,phan}=P\ix{rad,tom}$ within $\sigma\ix{rad}=$\,\SI{0.2}{\mega\watt}. No difference between HBCm and VBC is can be noted. Below the majority of the results are collected closely around the 1:1 line, however with $\sigma\ix{rad}=$\,\SI{0.7}{\mega\watt} and a greater amount of outliers. The integrated core and total emission, as well as tomogram and phantom values show no particular difference across the power spectrum.\\%
            In the first plot of the right column of \cref{fig:tomo_phantom_statistics}, the integrated average MSD is compared to the fitness factor between forward and backward calculated detector signals. A distinct line at $\chi^{2}=1$ between \SIrange{0.1}{0.8}{\percent} underlines the discrepancy among the quality parameters that was highlighted before during the benchmark. So does the group of results just below \SI{0.5}{\percent}, spreading across $\chi^{2}=$\,\SIrange{1.1}{1.8}{\arbitraryunit}, solidifying that the two-dimensional correlation not necessarily and positively correspond to the MFR targets. However, the few and sparsely spread outliers do indicate that the general significance of these quality parameters still holds, especially taking the lack of data for $\chi^{2}>1$ and >\SI{0.5}{\percent} into account. Conclusively, MSD and $\chi^{2}$ are compared to the third estimate for congruence, the Pearson coefficient in the next image. Separated by the 1:1 line, $\chi{2}$ almost exclusively yields data above and the MSD below. Here, many tighter and almost directed groupings of markers for each quantity indicate a similar set of phantoms, likely within a reconstructive iteration to find ideal settings therefor.\\%
            Finally, the last two plots in that column compare the individual cameras $P\ix{rad}$ extrapolation from the forward calculated detector signals to the two-dimensional integrated power in the tomograms core and in total. Additionally, separate linear fits for the HBC and VBC data are provided, including their parameters and colour-matched line plot. The corresponding fit function coefficients are calculated using a common \textit{least squares regression} algorithm, with $P\ix{rad}$ and $P\ix{2D}$ as the in-/dependent variables respectively. Again, distinct clouds of data points originate in identical or closely related phantom radiation distributions and their iterative tomographies.\\%
            Figure \ref{fig:tomo_phantom_statistics} visually highlights the deductions presented at the conclusion of the artificial radiation image benchmark of the introduced MFR algorithm. Results shown for FWHM location and value underline the radial matching capabilities, however at a lower fidelity in terms of the individual shape and given the discrepancy in multiplicity. The strong congruence in the comparison between bolometer camera radiative power loss extrapolations is evidence of, on one hand well preselected set of results and on the other equally effective tomography settings and success condition $\chi^{2}=$\,\SI{1}{\arbitraryunit} Similarly, the correspondence in absolute two-dimensional power values essentially adheres to the 1:1 line, however the degradation when compared to the above image can be attributed to the intrinsic ill-posedness and hence large number of unknowns. The same is true for the match between MSD and fitness factor, where for $\chi^{2}$ approaching or at unity, the number of results increases and the tomograms' deviation is low, though still retains a noticeable spread and points are found far beyond that algorithmic cut-off. Overall characteristic and therefore conclusions here are reflected below in the comparison of fitness and Pearson coefficient, i.e. the concentrated results at $\chi^{2}=$\,\SI{1}{\arbitraryunit} towards $\rho\ix{c}=1$ and point clouds around there. The accompanying MSD data is in line with the interpretation of said two-dimensional quality measures, near linearly decreasing with increasing phantom-tomogram cross-correlation and vanishing at unity. Given this profile, the group of outliers in the previous plot and here above a fitness of \SI{1}{\arbitraryunit} have to be certainly treated as such. They are connected to a particular set of artificial radiation distribution reconstructions of lesser adequacy - the corresponding MSD and $\rho\ix{c}$ however indicate a good fit of tomogram and phantom. At last, the individual $P\ix{rad}$ are not sufficiently represented by either the integrated core or total two-dimensional powers. The prior, at least for higher absolute radiation levels is potentially overestimated by the bolometer measurements, while their linear regression yields results supporting congruence at equal slopes for both cameras with minor absolute terms. Only the total radiation level presents somewhat linear coherence with $P\ix{rad}$ of any camera, though at an arguably worse regression result, i.e. larger deviation of the corresponding plots from the 1:1 line and poorer parameters. Hence, the bolometers' extrapolation underestimates the - phantom contained - total emission, given the set of distributions and fits. The distinctive gap in results here suggests that for general groups of radiation profiles, more accurate predictive linear models could be calculated, though at possibly greater discrepancy to the ideal match of $P\ix{rad}:P\ix{2D}$ or even $P\ix{tot}$.%
%
    \section{Tomography of Experimental Data}\label{sec:expdata}%
%
        At last, after finalizing the benchmarking of the MFR algorithm with an extensive set of artificial radiation distributions, regularisation weight coefficients, variations and input geometries, one will proceed to perform tomography with experimentally measured data. However, due to the limited scope of this work, the set of experiments investigated in this section will also be focused on discernable and reasonable features in their respective tomographic reconstructions of adequate fitness and $k\ix{ani}$. The latter are established by finding a $\chi^{2}$ closest to unity for a given optimum set of weighting parameters that is within a sufficiently, yet restrictively extended interval in line with the experiences and statistics of the previous segment. Introduction and approach to the below data is of exemplary character and will be done so with respect to the corresponding central plasma parameters. For sake of comprehensibility and manageability, presentation of the individual $P\ix{rad(2D)}$, $\chi^{2}$ and forward and backward calculated detector signal profiles will be omitted. Their relevance towards the evaluation of this nature is limited, though they are recorded regardless and employed later on in an overarching comparison for all and per discharge experimental and phantom reconstructions.%
%
        \subsection*{XP20180725.44}%
%
            \begin{figure}[t]%
                \centering%
                \captionsetup{width=.47\textwidth}%
                \begin{minipage}[c]{0.47\textwidth}%
                    \centering%
                    \includegraphics[width=\textwidth]{%
                        content/figures/chapter4/MFR/new/tomo/%
                        20180725.044/20180725_044_power_feedback.pdf}%
                    \caption{%
                        XP20180725.44:\\%
                        \textbf{(top left)}: ECRH, plasma energy and loss \textbf{(bottom left)}: $P\ix{rad}$ \textbf{(top right)}: Electron density \textbf{(center right)}: Electron temperature \textbf{(bottom right)}: Radiation fraction, target heat load and power balance.}\label{fig:20180725.44_PDF}%
                \end{minipage}%
                \hfill%
                \begin{minipage}[c]{0.47\textwidth}%
                    \centering%
                    \includegraphics[width=\textwidth]{%
                        content/figures/chapter4/MFR/new/tomo/%
                        20180725.044/20180725_044_dens_chord_frad_div.pdf}%
                \end{minipage}%
            \end{figure}%
%
            The central plasma parameters for experiment \textit{XP20180725.44} are shown in \cref{fig:20180725.44_PDF}, including ECRH, HBC and VBC radiation power $P\ix{rad}$, diamagnetic energy $W\ix{dia}$, its temporal derivative, a selection of \textit{Thomson scattering} LOS volumes and line integrated dispersion interferometry for electron density, as well as electron cyclotron emission temperature measurements. Thereof derived radiation fraction $f\ix{rad}$ and integrated divertor heat load are plotted separately. This was partially already introduced and discussed in \cref{fig:powerbal_0725.044}, only the electron quantities have been added in this case. Here, the magnetic configuration is that of a high mirror field (\textit{KJM}), hence the slightly altered shape of the triangular plane, island structure, reconstruction domain and subsequent pixel grid.\\%
%
            \begin{figure}[t]%
                \centering%
                \begin{subfigure}{0.475\textwidth}%
                    \centering%
                    \caption{\SI{3.05}{\second}}%
                    \vspace*{-0.1cm}%
                    \includegraphics[width=\textwidth]{%
                        content/figures/chapter4/MFR/new/tomo/20180725.044/2D/%
                        tomo_20180725.044aniM4_2.00_0.25_nT15_nW2_reduced_nigs1_sN8_30x20x150_1.25_3.05_mfr1D.png}%
                \end{subfigure}%
                \hfill%
                \begin{subfigure}{0.475\textwidth}%
                    \centering%
                    \caption{\SI{3.15}{\second}}%
                    \vspace*{-0.1cm}%
                    \includegraphics[width=\textwidth]{%
                        content/figures/chapter4/MFR/new/tomo/20180725.044/2D/%
                        tomo_20180725.044aniM4_2.00_0.25_nT15_nW2_reduced_nigs1_sN8_30x20x150_1.25_3.15_mfr1D.png}%
                \end{subfigure}%
                \newline%
                \vspace*{-0.1cm}%
                \begin{subfigure}{0.475\textwidth}%
                    \centering%
                    \caption{\SI{3.2}{\second}}%
                    \vspace*{-0.1cm}%
                    \includegraphics[width=\textwidth]{%
                        content/figures/chapter4/MFR/new/tomo/20180725.044/2D/%
                        tomo_20180725.044aniM4_2.00_0.25_nT15_nW2_reduced_nigs1_sN8_30x20x150_1.25_3.20_mfr1D.png}%
                \end{subfigure}%
                \hfill%
                \vspace*{-0.1cm}%
                \begin{subfigure}{0.475\textwidth}%
                    \centering%
                    \caption{\SI{3.3}{\second}}%
                    \vspace*{-0.1cm}%
                    \includegraphics[width=\textwidth]{%
                        content/figures/chapter4/MFR/new/tomo/20180725.044/2D/%
                        tomo_20180725.044aniM4_2.00_0.25_nT15_nW2_reduced_nigs1_sN8_30x20x150_1.25_3.30_mfr1D.png}%
                \end{subfigure}%
                \newline%
                \begin{subfigure}{0.475\textwidth}%
                    \centering%
                    \caption{\SI{3.4}{\second}}%
                    \vspace*{-0.1cm}%
                    \includegraphics[width=\textwidth]{%
                        content/figures/chapter4/MFR/new/tomo/20180725.044/2D/%
                        tomo_20180725.044aniM4_2.00_0.25_nT15_nW2_reduced_nigs1_sN8_30x20x150_1.25_3.40_mfr1D.png}%
                \end{subfigure}%
                \hfill%
                \begin{subfigure}{0.475\textwidth}%
                    \centering%
                    \caption{\SI{3.65}{\second}}%
                    \vspace*{-0.1cm}%
                    \includegraphics[width=\textwidth]{%
                        content/figures/chapter4/MFR/new/tomo/20180725.044/2D/%
                        tomo_20180725.044aniM4_2.00_0.25_nT15_nW2_reduced_nigs1_sN8_30x20x150_1.25_3.65_mfr1D.png}%
                \end{subfigure}%
                \caption{%
                    XP20180725.44:\\%
                    Tomograms for the above discharge on a $1.25\,r\ix{a}$ inflated grid with $n\ix{r}\times\,n_{\vartheta}=20\times150$ and \textit{KJM} high mirror magnetic configuration. RDA parameters applied here are $k\ix{core},\,k\ix{edge}=\left\{2, 0.25\right\}$ with $N\ix{T}=15$ and $N\ix{S}=2$. Central experiment parameters can be found in \cref{fig:20180725.44_PDF}.}\label{fig:tomo_20180725.44_times}%
            \end{figure}%
%
            In \cref{fig:tomo_20180725.44_times}, an excerpt of six points in time just before and across the pronounced global maximum in \SI{3.4}{\second} from the multicamera, bolometric measurement is reconstructed using the same set of $k\ix{ani}=\left\{2, 0.25\right\}$ and secondary coefficients $N\ix{T}=15$ and $N\ix{S}=2$ (see \cref{eq:kani}). This configuration strongly favours smooth and isotropic emissivity profiles inside the core up to two cells close to the separatrix and therefore more anisotropic structures beyond until the edge of the domain.\\%
            The first tomogram in \textbf{(a)} shows a relatively even and smooth, shallow profile along the LCFS of increased width. On the inboard side and towards the lower X-point, this is moved further inward and a distinct maximum can be found around $z=0$, while the correspond island is also slightly brighter than the rest. The core is mostly dark. In \textbf{(b)} this smooth and thin ring has shifted radially inwards, accompanied by a poloidal translation of the maximum closer to the HBCm aperture, adjacent to the lower outboard magnetic island. The inboard side of the profile is still noticeably brighter compared to the rest. Figure \textbf{(c)} shows a further shrunken distribution, however not homogeneously but rather by establishing a more pronounced edge towards the now entirely radiation-less SOL and separatrix area. Hence, the ring brightness grows slightly overall and the peak power can be found closest to the midplane, upper X-point. In tomogram \textbf{(d)} an again significantly, radially smaller profile of very smooth emissivity, besides the maximum towards the lower inboard magnetic island is presented. Coincidental with the peak in $P\ix{rad}$, the core $n\ix{e}$ measurements and $\diff W\ix{dia}/\diff t$, the second to last reconstruction in \textbf{(e)} shows the smallest brightness profile at only a radius of $0.2rix{a}$, also featuring the highest overall and maximum power. The highest radiation density is found towards the tip of the triangle-shaped plane. Still, this small ring appears hollowed like before and otherwise smooth. Radial width or decay of said feature is comparable to before, given the significantly different ratio between size and shape. Finally, at the equilibrated plateau shortly after that peak, tomogram \textbf{(f)} presents again a larger brightness profile, with a similar structure but only less than half the intensity of the previous reconstruction. Width, poloidal smoothness and hollowness are also nearly congruent to the prior. Like in the preceding three tomograms, no radiation is placed outside the edge of the distinctly confined distribution.\\%
            This first exemplary discharge in \cref{fig:20180725.44_PDF} and its corresponding MFR of experimental data in \cref{fig:tomo_20180725.44_times} yield plausible, albeit qualitative correlations. The behaviour of the two-dimensional radiation profiles corresponds well with the indicated plasma density and its temporal evolution, particularly the intrinsic discrepancy between the outer and inner LOS volumes. The same is true for the reduction and centralization of $T\ix{e}$ during that time, coinciding with an increase and narrowing in the tomogram radiation powers for a constant input microwave heating and negligible plasma stored energy. Furthermore, with respect to the total $P\ix{rad}$ of both cameras, the quantitative character of the brightness distribution across the selected points in time is also well-supported. Given the shape and structure of the reconstructed profiles, the applied anisotropic regularisation weight coefficients $k\ix{ani}=\left\{2, 0.25\right\}$, $N\ix{T}=15$ and $N\ix{S}=2$ produced adequate results that align with the established phantom image benchmarks.%
%
        \subsection*{XP20180809.13}%
%
            \begin{figure}[t]%
                \centering%
                \captionsetup{width=.47\textwidth}%
                \begin{minipage}[c]{0.47\textwidth}%
                    \centering%
                    \includegraphics[width=\textwidth]{%
                        content/figures/chapter4/MFR/new/tomo/%
                        20180809.013/20180809_013_power_feedback.pdf}%
                    \caption{%
                        XP20180809.13:\\%
                        \textbf{(top left)}: ECRH, plasma energy and loss \textbf{(bottom left)}: $P\ix{rad}$ \textbf{(top right)}: Electron density \textbf{(center right)}: Electron temperature \textbf{(bottom right)}: Radiation fraction and target heat load.}\label{fig:20180809.13_PDF}%
                \end{minipage}%
                \hfill%
                \begin{minipage}[c]{0.47\textwidth}%
                    \centering%
                    \includegraphics[width=\textwidth]{%
                        content/figures/chapter4/MFR/new/tomo/%
                        20180809.013/20180809_013_dens_chord_frad_div.pdf}%
                \end{minipage}%
            \end{figure}%
%
            The next exemplary discharge XP20180809.13 and its central plasma parameters are presented in \cref{fig:20180809.13_PDF}. It is particularly characterised by a constant heating power $P\ix{ECRH}$, generally very low radiation fraction $f\ix{rad}\le$\SI{0.25}{\arbitraryunit} and two distinct events that lead noticeable, large spikes in radiative power loss $P\ix{rad}$. These are caused by the injection of tertiary impurities through the ablation of metal particles by a \textit{Laser Blow-Off} (LBO) system - a concentrated, high power laser beam superheats, ablates and ejects from plasma-side coated glass tiles into the SOL. The injected amount is relatively small, though possible charge states increased, hence the negligible or small impact on $W\ix{dia}$ and the electron density and temperature. The input $P\ix{ECRH}$ is, after an initial step, set constant. Correspondingly, the plasma stored energy increases sharply in the beginning and also remains steady. As mentioned before, the two LBO events are highlighted here by large maxima in \SI{2}{\second} and \SI{3}{\second}. Both camera results are qualitatively in good agreement. From inside to out, the electron density decreases from interferometer measured to the most outward TS volumes, however remaining relatively constant throughout the discharge. The same is true for the corresponding temperatures, hence the discrepancy between ECE provided values and TS volumes is significantly increased. During the first only negligibly and after the second ablation, $T\ix{e,ECE}$ instantaneously drops, before quickly and steadily growing back to the previous level. Finally, as stated above, the $f\ix{rad}$ remains low throughout the experiment, only breaching that threshold in correspondence to the laser blow-offs.\\%
%
            \begin{figure}[t]%
                \centering%
                \begin{subfigure}{0.48\textwidth}%
                    \centering%
                    \caption{\SI{1.8}{\second}}%
                    \includegraphics[width=\textwidth]{%
                        content/figures/chapter4/MFR/new/tomo/20180809.013/2D/%
                        tomo_20180809.013aniM3_2.00_0.50_nT14_reduced_nigs1_sN8_30x20x150_1.35_1.80_mfr1D.png}%
                \end{subfigure}%
                \hfill%
                \begin{subfigure}{0.48\textwidth}%
                    \centering%
                    \caption{\SI{2.15}{\second}}%
                    \includegraphics[width=\textwidth]{%
                        content/figures/chapter4/MFR/new/tomo/20180809.013/2D/%
                        tomo_20180809.013aniM3_2.00_0.50_nT14_reduced_nigs1_sN8_30x20x150_1.35_2.15_mfr1D.png}%
                \end{subfigure}%
                \newline%
                \begin{subfigure}{0.48\textwidth}%
                    \centering%
                    \caption{\SI{2.2}{\second}}%
                    \includegraphics[width=\textwidth]{%
                        content/figures/chapter4/MFR/new/tomo/20180809.013/2D/%
                        tomo_20180809.013aniM3_2.00_0.50_nT14_reduced_nigs1_sN8_30x20x150_1.35_2.20_mfr1D.png}%
                \end{subfigure}%
                \hfill%
                \begin{subfigure}{0.48\textwidth}%
                    \centering%
                    \caption{\SI{2.8}{\second}}%
                    \includegraphics[width=\textwidth]{%
                        content/figures/chapter4/MFR/new/tomo/20180809.013/2D/%
                        tomo_20180809.013aniM3_2.00_0.50_nT14_reduced_nigs1_sN8_30x20x150_1.35_2.80_mfr1D.png}%
                \end{subfigure}%
                \caption{%
                    XP20180809.13:\\%
                    Tomographic reconstruction of experimental bolometer data on a $1.35\,r\ix{a}$ inversion grid of standard magnetic configuration, otherwise similar to \cref{fig:tomo_20180725.44_times}, between \SIrange{1.8}{2.8}{\second}. RDA parameters are constant throughout at $k\ix{core},\,k\ix{edge}=\left\{2, 0.5\right\}$ and $N\ix{T}=14$, $N\ix{S}=2$. Corresponding central experiment parameters can be found in \cref{fig:20180809.13_PDF}.}\label{fig:tomo_20180809.13_times}%
            \end{figure}%
%
            The selected points in time for the performed reconstructions in \cref{fig:tomo_20180809.13_times} coincide with just before $t=$\SI{1.8}{\second}, during \SIrange{2.15}{2.2}{\second} the first and again equilibrated after the second event in \SI{2.8}{\second}. They are all produced using the same set of $k\ix{ani}=\left\{2, 0.5\right\}$ and parameters $N\ix{T}=14$, $N\ix{S}=2$. Here, the underlying magnetic configuration for the experiment and constructed two-dimensional mesh is standard, while the domain size has been increased to $1.35r\ix{a}$. Plot \textbf{(a)} shows a maximum brightness outside the separatrix of particular, larger poloidal width, on the lower inboard side closest to the corresponding X-point. Along the LCFS, the smooth intensity profile shows a relatively constant brightness, though with a very minor up-down asymmetry. In \textbf{(b)}, after the injection of secondary impurities but before a bolometer measured response in plasma radiation, minute changes in the absolute level of emission and its radial position hint at a contraction of the latter ring. One can also now notice a similarly bright inboard island, expanding further on the lower peak from before. During the maximum in $P\ix{rad}$ reconstructed in \textbf{(c)}, the total emissivity is, on one hand significantly increased and on the other essentially constant in a fixed radius and FWHM on the inside of the separatrix. On the outboard side, the tomogram is distinctly limited, i.e. no radiation is reconstructed beyond the radial decay of the ring. Conclusively, in tomogram \textbf{(b)} at which the plasma radiation again has equilibrated but before the second LBO, a qualitatively congruent emissivity distribution is presented. While the absolute power level is increased slightly and the radial decay towards the outboard tip of the triangular plane is shortened, the remaining tomogram is otherwise the same as in \textbf{(a)}.\\%
            This set of MFR shows two major characteristics of interest with respect to the introduced data in \cref{fig:20180809.13_PDF}: first the contraction of the radiation distribution and its evolution under injection of external impurities and second the subsequent equilibration thereof. Introduction of additional high-Z material into the discharge leads to a reduction in $T\ix{e}$ under constant or minor increase of plasma stored energy and large power dissipation through radiation. The lack of precise temporal correlation between $P\ix{rad}$ and the other plasma parameters during that event is not yet entirely understood. However, correlation tests, for example superimposing the outermost channels of the bolometers with the ECRH, have shown no significant deviation and therefore ruled out lagging of the prior in this context. No feedback was performed in this case, i.e.no parasitic computational load on the system like in \cref{sec:drawbacks} is noted. The consequential centralisation of radiation is the result of the strong ionisation of said material and hence its transport across the separatrix into the core. At this time, the plasmas' emissivity is entirely dominated by the \textit{quasi} equilibrated impurity distribution, producing the smooth and constant profile aligned in the center and along the LCFS. The asymmetry or shift of the maximum brightness towards the inboard side is attributed to a combination of magnetic configuration, the location of the bolometer inside the W7-X vessel, the low $f\ix{rad}$ and therefore condensation of radiation in the LOS in and close to X-points. Lastly, this process proves to be reversible, as the ablated impurity is then again kinetically transported to the SOL and ejected from the plasma after relaxation of the central parameters. Again, this set of experimental MFR are well in agreement with the previous extensive verification benchmarks and accompanying diagnostic data in \cref{fig:20180809.13_PDF}.%
%
        \subsection*{XP20181010.32}%
%
            \begin{figure}[t]%
                \centering%
                \begin{subfigure}{0.45\textwidth}%
                    \centering%
                    \caption{\SI{0.65}{\second}\,,$f\ix{rad}=$\SI{0.33}{\arbitraryunit}}%
                    \includegraphics[width=\textwidth]{%
                        content/figures/chapter4/MFR/new/tomo/20181010.032/2D/%
                        tomo_20181010.032aniM4_2.00_0.35_nT13_nW2_reduced_nigs1_sN8_30x20x150_1.35_0.65_mfr1D.png}%
                    \end{subfigure}%
                \hfill%
                \begin{subfigure}{0.45\textwidth}%
                    \centering%
                    \caption{\SI{1.7}{\second}\,,$f\ix{rad}=$\SI{0.4}{\arbitraryunit}}%
                    \includegraphics[width=\textwidth]{%
                        content/figures/chapter4/MFR/new/tomo/20181010.032/2D/%
                        tomo_20181010.032aniM4_2.00_0.35_nT13_nW2_reduced_nigs1_sN8_30x20x150_1.35_1.70_mfr1D.png}%
                \end{subfigure}%
                \newline%
                \begin{subfigure}{0.45\textwidth}%
                    \centering%
                    \caption{\SI{1.8}{\second}\,,$f\ix{rad}=$\SI{0.45}{\arbitraryunit}}%
                    \includegraphics[width=\textwidth]{%
                        content/figures/chapter4/MFR/new/tomo/20181010.032/2D/%
                        tomo_20181010.032aniM4_2.00_0.35_nT13_nW2_reduced_nigs1_sN8_30x20x150_1.35_1.80_mfr1D.png}%
                    \end{subfigure}%
                \hfill%
                \begin{subfigure}{0.45\textwidth}%
                    \centering%
                    \caption{\SI{2.2}{\second}\,,$f\ix{rad}=$\SI{0.66}{\arbitraryunit}}%
                    \includegraphics[width=\textwidth]{%
                        content/figures/chapter4/MFR/new/tomo/20181010.032/2D/%
                        tomo_20181010.032aniM4_2.00_0.35_nT13_nW2_reduced_nigs1_sN8_30x20x150_1.35_2.20_mfr1D.png}%
                \end{subfigure}%
                \newline%
                \begin{subfigure}{0.45\textwidth}%
                    \centering%
                    \caption{\SI{2.95}{\second}\,,$f\ix{rad}=$\SI{0.95}{\arbitraryunit}}%
                    \includegraphics[width=\textwidth]{%
                        content/figures/chapter4/MFR/new/tomo/20181010.032/2D/%
                        tomo_20181010.032aniM4_2.00_0.35_nT13_nW2_reduced_nigs1_sN8_30x20x150_1.35_2.95_mfr1D.png}%
                    \end{subfigure}%
                \hfill%
                \begin{subfigure}{0.45\textwidth}%
                    \centering%
                    \caption{\SI{3.05}{\second}\,,$f\ix{rad}=$\SI{0.9}{\arbitraryunit}}%
                    \includegraphics[width=\textwidth]{%
                        content/figures/chapter4/MFR/new/tomo/20181010.032/2D/%
                        tomo_20181010.032aniM4_2.00_0.35_nT13_nW2_reduced_nigs1_sN8_30x20x150_1.35_3.05_mfr1D.png}%
                \end{subfigure}%
                \caption{%
                    XP20181010.32:\\%
                    Tomographic reconstruction of experimental data on a $1.35\,r\ix{a}$ inversion grid with $n\ix{r}\times\,n_{\vartheta}=30\times150$, based off of standard magnetic configuration. Throughout \SIrange{0.65}{3.05}{\second}, the RDA parameters are constant at $k\ix{core},\,k\ix{edge}=\left\{2, 0.35\right\}$. Corresponding central experiment parameters have been previously introduced in \cref{fig:20181010.32_PDF}.}\label{fig:tomo_20181010.32_times}%
            \end{figure}%
%
            The next experimental data tomography concerns the prime real-time bolometer feedback application in XP20181010.32. Background plasma parameters and scenario configurations have been illustrated in \cref{subsec:primew7xfeedback} and more specifically \cref{fig:20181010.32_PDF}. A selected set of points in time and tomograms have been produced in order to highlight the effects of the supplementary injection of gas through thermal gas valves using feedback. The results can be seen in \cref{fig:tomo_20181010.32_times}. Regarding the previous reconstructions, identical settings have been used to find the corresponding tomograms, though the experimental scenarios are vastly different.\\%
            In \textbf{(a)}:\SI{0.65}{\second} during a small plateau, i.e. before the injection via the thermal gas valves has begun and at low $f\ix{rad}$, the first tomogram yields a largely dim profile with a very concentrated, close to the lower inboard X-point and inside the lower outboard island located bright spot. Less intense, secondary features are shown next to the upper central X-point, as well as in and around the lower inboard island, connected by low emissivity around the separatrix and in the SOL. At \textbf{(b)}:\SI{1.7}{\second}, the overall quality of the profile is unchanged, though the maximum in the lower island at the tip of the triangular shaped plane is now much stronger and the relative intensity of the others is reduced. At that time, the QSB has continuously injected helium as an impurity into the SOL and the ECRH has reached its steady-state power level. In terms of plasma profiles, density, temperature and radiation are strongly increasing at this moment still.\\%
            With the earliest equilibration of $n\ix{e}$ and $T\ix{e}$, \textbf{(c)}:\SI{1.8}{\second} shows an emissivity distribution of similar magnitude, however now with the primary, global maximum in the upper central X-point. Shortly after in \textbf{(d)}:\SI{2.2}{\second}, a feedback gas injection is concluded and $P\ix{rad}$ has a first local maximum. Simultaneously, electron temperature and density show minor peaks across the radial spectrum. At this time, $f\ix{rad}$ reaches 75\% and the radiation power distribution condenses almost entirely in a singular, confined point close to the intersection of the two upper magnetic islands. At \textbf{(e)}:\SI{2.95}{\second}, the radiation fraction has decreased to a temporary minimum of $\sim80\%$ in reaction to another feedback gas injection pulse. Simultaneously, $T\ix{e}$ declines, while $n\ix{e}$ yields a small maximum. The total integrated radiation power loss has now essentially reached its \textit{quasi steady-state} level, given the oscillations due to the gas valve opening and closing. The overall maximum brightness in the corresponding tomogram is lowered, though the number of highlighted features of that intensity is increased. Besides the previous localisation in the upper central X-point, the concentrated emissivity of the lower outboard island is now the global maximum and moved inwards onto the LCFS.\\%
            Finally, \textbf{(f)}:\SI{3.05}{\second} corresponds to $f\ix{rad}\sim95\%$, coinciding with a local maximum in $T\ix{e}$ and minimum in $n\ix{e}$ while the feedback gas valves are closed for another \SI{0.1}{\second}. In the two-dimensional radiation distribution, the brightest feature is located again on the separatrix close to the lower outboard island, while previously noted localisations are reduced in intensity. With respect to the profile overall, its relative emissivity is more akin to that in \SI{2.95}{\second} and \SI{1.8}{\second}, i.e. less condensed in the aforementioned extremes.\\%
            This sequence of MFR profiles provides a fundamental insight into the behaviour and impact of high-$f\ix{rad}$ radiation feedback controlled thermal gas injections. It also graphically highlights the before outlined difficulty of defining and finding an ideal or even adequate LOS combination for the real-time bolometer evaluation metric - see \cref{sec:realtime_radiation} and \cref{sec:evalmetrics}. Both can be reduced to the very pronounced focussing of emission in points of intersecting closed field lines in the SOL and magnetic islands. Throughout the temporal evolution of the emissivity distribution, this characteristic is presented by a noticeable and significant correlation with the feedback controlled radiation fraction. Up to 80\% of $f\ix{rad}$, the majority brightness condenses and shifts from the lower outboard in counter-clockwise direction towards the upper central X-point. Beyond this level, though at a significantly increased absolute power, this trend is reversed and, with the radiation fraction growing towards unity, the majority of emissivities move to the lower outboard island. This process in itself is again reversible and between \SIrange{80}{100}{\percent}, this reallocation goes back and forth with the opening and closing of the thermal helium beam valves. That said, only during opening of said valves and injection of gaseous impurities, brightness profiles like in \textbf{(d)} are measured and the plasmas' response leads to conditions as in \textbf{(e)}-\textbf{(f)}. This is on one hand due to the location of the gas inlet and its connection along magnetic field lines to the bolometer measurement plane, and on the other the combination of $f\ix{rad}$ level and plasma profiles. The latter means that emissions are \textit{pushed out} towards the separatrix and SOL, as the temperature and density distributions peak and centralise, meaning a sharp drop-off at that radius - see STRAHL simulations based on data from this discharge in \cref{sec:strahlmodel}. Hence, the tomograms find the majority of emissivities only in these particular locations for $r\ge r\ix{a}$.\\%
            Regarding the configuration and selection of LOS for the real-time radiation estimate $P\ix{pred}$, this behaviour makes it particularly difficult to find an ideal or fitting set to adequately measure the power throughout the experiment. Hypothetically, a predictive collection only looking at the upper- and inboard-side SOL from both bolometer camera arrays inevitably overestimates, while in this case the opposite certainly underestimates the actual, total radiative power exhaust. A balanced set that watches all the above characteristics and also possibly integrates through intermittently lit areas is of great importance towards the success of such an application.\\%
            In conclusion, the presented results and their interpretation are essential for following real-time, bolometer controlled radiation feedback experiments. However, the limited set of effectively one data point in XP20181010.32 makes predictions difficult.%
%
        \subsection*{XP20181011.12}%
%
            \begin{figure}[t]%
                \centering%
                \captionsetup{width=0.49\textwidth}%
                \begin{minipage}[c]{0.49\textwidth}%
                    \centering%
                    \includegraphics[width=\textwidth]{%
                        content/figures/chapter4/MFR/new/tomo/%
                        20181011.012/20181011_012_power_feedback.pdf}%
                    \caption{%
                        XP20181011.12:\\%
                        \textbf{(top left)}: ECRH, plasma energy and loss \textbf{(bottom left)}: $P\ix{rad}$ \textbf{(top right)}: Electron density \textbf{(center right)}: Electron temperature \textbf{(bottom right)}: Radiation fraction and target heat load.}\label{fig:20181011.12_PDF}%
                \end{minipage}%
                \hfill%
                \begin{minipage}[c]{0.49\textwidth}%
                    \centering%
                    \includegraphics[width=\textwidth]{%
                        content/figures/chapter4/MFR/new/tomo/%
                        20181011.012/20181011_012_dens_chord_frad_div.pdf}%
                \end{minipage}%
            \end{figure}%
%
            The last example of experimental bolometer data \textit{Minimum Fisher tomographic reconstruction} is given with XP20181011.12, for which the core plasma parameters are shown in \cref{fig:20181011.12_PDF} and the respective two-dimensional emissivity distributions in \cref{fig:tomo_20181011.12_times}. This discharge is mostly characterised by two input power step-downs in the ECRH and intermittent impurity injections via the \textit{laser blow-off} system, leading to, on one hand very pronounced events and on the other larger gradients in $P\ix{rad}$. This experiment and hence the underlying MFR mesh are again based in a \textit{high mirror field} magnetic configuration (\textit{KJM}). Employed anisotropic regularisation weighting coefficients and parameters are kept constant at $k\ix{ani}=\left\{2, 0.35\right\}$ and $N\ix{T}=14$, $N\ix{S}=2$, similarly favouring smooth core and localised SOL emissivities.\\%
%
            \begin{figure}[t]%
                \centering%
                \begin{subfigure}{0.48\textwidth}%
                    \centering%
                    \caption{\SI{1.4}{\second}}%
                    \includegraphics[width=\textwidth]{%
                        content/figures/chapter4/MFR/new/tomo/20181011.012/2D/%
                        tomo_20181011.012aniM3_2.00_0.35_nT13_reduced_nigs1_sN8_30x20x150_1.35_1.50_mfr1D.png}%
                    \end{subfigure}%
                \hfill%
                \begin{subfigure}{0.48\textwidth}%
                    \centering%
                    \caption{\SI{2.7}{\second}}%
                    \includegraphics[width=\textwidth]{%
                        content/figures/chapter4/MFR/new/tomo/20181011.012/2D/%
                        tomo_20181011.012aniM3_2.00_0.35_nT13_reduced_nigs1_sN8_30x20x150_1.35_2.70_mfr1D.png}%
                \end{subfigure}%
                \newline%
                \begin{subfigure}{0.48\textwidth}%
                    \centering%
                    \caption{\SI{5.65}{\second}}%
                    \includegraphics[width=\textwidth]{%
                        content/figures/chapter4/MFR/new/tomo/20181011.012/2D/%
                        tomo_20181011.012aniM3_2.00_0.35_nT13_reduced_nigs1_sN8_30x20x150_1.35_5.65_mfr1D.png}%
                \end{subfigure}%
                \hfill%
                \begin{subfigure}{0.48\textwidth}%
                    \centering%
                    \caption{\SI{5.7}{\second}}%
                    \includegraphics[width=\textwidth]{%
                        content/figures/chapter4/MFR/new/tomo/20181011.012/2D/%
                        tomo_20181011.012aniM3_2.00_0.35_nT13_reduced_nigs1_sN8_30x20x150_1.35_5.70_mfr1D.png}%
                \end{subfigure}%
                \caption{
                    XP20181011.12:\\%
                    Tomographic reconstruction of experimental bolometer data on a $1.35\,r\ix{a}$ inversion grid with $n\ix{r}\times\,n_{\vartheta}=20\times150$ intersections, based off of the \textit{KJM} high mirror magnetic configuration. RDA parameters are constant throughout at $k\ix{core},\,k\ix{edge}=\left\{2, 0.35\right\}$ and $N\ix{T}=14$, $N\ix{S}=2$. Corresponding central experiment parameters can be found in \cref{fig:20181011.12_PDF}.}\label{fig:tomo_20181011.12_times}%
            \end{figure}%
%
            In \textbf{(a)}, the radiation is near evenly distributed inside the separatrix at the outboard side, magnetic islands on the bottom, top and the inboard side domain boundary. Only on the outside of the lower inboard X-point a strong and radially focused maximum can be found. Minor variations are presented along the lower inside the LCFS and in the core for $r<0.7r\ix{a}$. In \textbf{(b)}, the inboard localisation slightly shifts radially towards the X-point while its emissivity increases greatly. Simultaneously, the overall brightness in the core and closer to the tip of the triangular plane remains constant, therefore decreasing significantly in relative terms. The same behaviour as before at higher intensities is presented in \textbf{(c)}. However, the local maximum now expands from the inboard magnetic island to the separatrix and X-point. Finally, in \textbf{(d)}, the overall and maximum radiation density are increased again, as the localisation is pushed outward again towards the edge of the inboard island.\\%
            The distribution of emissivity and its evolution throughout the course of this discharge is dominated by the characteristic condensation at the lower inboard X-point. Indications thereof can be seen in \cref{fig:tomo_20180725.44_times} at $t=$\SI{3.05}{\second}, i.e. the tomogram with the highest $f\ix{rad}$ in this sequence, however the configuration of the experiment and condition of the machine did allow for and achieve this particular profile of radiation. Experiment XP20181011.12 took place after the second boronisation of the W7-X vessel - a cleaning process and condition of the first wall using diborane gas and long duration glow discharges, after which the performance in terms of radiative power loss, impurity content, plasma density and temperature at a given input power is greatly improved. Hence, the qualitative structure of this set of tomograms is more akin to high radiation fraction experiments like XP20181010.32 etc., with the majority of emissions in the SOL and separatrix. However, in the first reconstruction at the lowest $f\ix{rad}$, the brightness beyond the LCFS is comparatively smooth, contrary to the corresponding $k\ix{edge}=\,$\SI{0.35}{\arbitraryunit} which emphasizes anisotropic profiles. After a heating power step-down and external impurity injection, a similar contraction of emissivities like before applies to this distribution, condensing the now much larger total power to a singular point. Further $P\ix{ECRH}$ reduction, repeated ablation and therefore increase in radiation fraction conclusively leads to an inward shift of that feature. Finally, beyond that $f\ix{rad}$ threshold and total intensity due to another decrease in input power and LBO event, this begins to move outward again, though in and along the respective magnetic island.\\%
            Regarding the performance of the MFR itself, the perhaps at this point generalizing $k\ix{ani}$ weighting parameter configuration yields plausible results of adequate quality and is able to differentiate between minute changes between the selected profiles. The presented two-dimensional radiation distributions are also in line with the previously featured phantom image and experimental data reconstructions, i.e. transitions and proportions are in well in agreement with before. Particularly the capability to resolve single cell variations in location and brightness across a smaller range of intensity levels is proven here, especially given the increased uncertainty in actually measured data.\\%
%
            \newline%
            This concludes the, albeit limited application of the introduced and benchmarked MFR tomography algorithm on experimental bolometer data. Presented results have been achieved with confidence, particularly in light of the benchmarks performed in the previous section and underline the capabilities of said method. However, this is certainly only, as was stated in the beginning, a proof of concept. Implications should in no way be interpreted as generalized deductions, applicable to arbitrary input profiles, more so when taking into account the practically infinite experimental parameter space available.%
%
        \subsection{Tomographic Reconstruction Statistics}\label{sec:expstat}%
%
                \begin{figure}[t]%
                    \centering%
                    \captionsetup{width=.47\textwidth}%
                    \begin{minipage}[c]{0.47\textwidth}%
                        \centering%
                        \includegraphics[width=\textwidth]{%
                            content/figures/chapter4/MFR/new/%
                            exp_statistics_results_xp.png}%
                    \end{minipage}%
                    \hfill%
                    \begin{minipage}[c]{0.47\textwidth}%
                        \centering%
                        \caption{%
                            Collected results of previously presented experimental data tomographic reconstructions in \cref{sec:expdata}. Included are reconstructions that yielded $\chi^{2}<1000$. A dashed 1:1 line indicates congruence between the axis and abscissa. If applicable, a regression linear fit for both cameras derived individually is noted in the bottom. A transparent error bar (orange) is indicated around the projected congruence. \textbf{(top):} $P\ix{rad}$ of both bolometer cameras from experiment and tomogram. \textbf{(center):} $P\ix{rad}$ of both cameras over integral of core emissivity. \textbf{(bottom):} $P\ix{rad}$ of both cameras over total integral of emissivity.
                            }\label{fig:tomo_experiment_statistics}%
                    \end{minipage}%
                \end{figure}%
%
                Again, accumulated experimental bolometer data Minimum Fisher reconstructions with respect to properties already introduced in the benchmarking process are analysed. Comparing MFR tomography results across different experiments statistically has proven difficult and so far has not produced substantial insight, however not beyond what was previously condensed from the established $P\ix{rad}$ or chordal brightness data anyhow. This section will proceed and investigate the experimental inversions on a larger scale more closely, i.e. similarly to the evaluations above in \cref{sec:phantomstat} and for an applicable selection of parameters. The forward extrapolated and two-dimensional radiation powers in total and from the core are compared in \cref{fig:tomo_experiment_statistics}. Collected data is selected only for $\chi^{2}<1000$, i.e. sensible results of the MFR and presented like before, including a line for exact identity of the quantities and plots for any respective linear regression fits.\\%
                Keeping in mind that the presented results are collected from various experiments - data is included also for reconstructions that have not been explicitly discussed or shown in this thesis -, particularly of very characteristic emissivity distribution and transitions, the correlation between these and the images shown \cref{fig:tomo_phantom_statistics} is quite significant. Especially given the very generous threshold interval in $\chi^{2}$ for preselection, the congruence between the forward and backward calculated $P\ix{rad}$ is on par with the deliberately designed and adjusted phantom profile tomography results. The strong agreement is hence certainly a testament to the performance and algorithmic optimisation of the MFR. Looking back at the comparison between the extrapolated and two-dimensional integrated power from the core, the linear regression fits and their underlying data are noticeably improved with respect to the target proportionality line for the experimental reconstructions. Not only do both camera measurements yield the same prediction function within a certain rounding error, which is simultaneously very similar to the direct congruence line, the points are generally grouped closer around follow the latter more closely. However, this means that $P\ix{rad}$ underestimates the total, in the tomogram contained emissivity by an amount that is about equal to the radiation power in the SOL, since the last image shows comparably that $P\ix{tot}$ is consistently larger. The latter is also reflected in the accompanying fit results, which still deviate only slightly between the cameras here. In conclusion, the small set experimental data MFR has shown very promising results both qualitatively as seen previously, but also quantitatively, especially in respect to the phantom benchmark. Though the strong performance of the algorithm in combination with actual measurements is diminished or in question due to the reduced set of assessable parameters, i.e. additional $P\ix{2D}$ or $\rho\ix{c}$.%
%
                \subsubsection*{Power Balance}%
%
                    \begin{figure}[t]%
                        \centering%
                        \captionsetup{width=.47\textwidth}%
                        \begin{minipage}[c]{0.47\textwidth}%
                            \centering%
                            \caption{
                                XP20181010.32:\\%
                                Comparison between previously derived power balance for XP20181010.32 in \cref{fig:20181010.32_balance} and the balance calculated from the integrated two-dimensional radiation distribution of the tomographic reconstructions presented in \cref{sec:expdata}. \textbf{(top):} Central experiment parameters towards the power balance. \textbf{(bottom):} Comparison of power balances derived from a common $P\ix{rad}$ and the radiation distribution integrals.}\label{fig:tomo_experiment_20181010032_balance}%
                        \end{minipage}%
                        \hfill%
                        \begin{minipage}[c]{0.47\textwidth}%
                            \centering%
                            \includegraphics[width=\textwidth]{%
                                content/figures/chapter4/MFR/new/%
                                tomogram_power_balance_20181010.032.png}%
                        \end{minipage}%
                    \end{figure}%
%
                    The final application and evaluation of MFR data is done using a selection of the previously presented experimental results for different points in time during the discharges and using these to calculate their respective power balance. Firstly, central plasma parameters for XP20181010.32 as introduced and discussed in \cref{fig:20181010.32_PDF}, alongside said $P\ix{bal}$ and individual points for each MFR tomograms total and core $P\ix{2D}$ are given in \cref{fig:tomo_experiment_20181010032_balance}. The prior is included as direct reference as before. Below, a power balance is constructed using, on the one hand the continuous $P\ix{rad}$ from bolometer measurements as a profile and on the other the separate integrated two-dimensional core and total radiation powers from reconstructions of that experiment. For a detailed, in-depth treatment of the full $P\ix{bal}$ see \cref{fig:20181010.32_balance}, though one should note here that particularly in the very beginning of the discharge and throughout the continuing feedback injections and corresponding oscillations, it consistently fails to achieve equalisation, even under consideration of the intrinsically large standard deviation. Calculating $P\ix{bal}$ using the integrated emissivity only from the core of the reconstructed brightness distributions shows an overall much improved match, i.e. $P\ix{bal}\rightarrow0$. A reconstructed core brightness is in much better agreement with the prior premise like discussed in \cref{eq:power_balance}, as deviations from equalisation and variations due to the feedback are reduced compared to using $P\ix{rad}$ or $P\ix{2D}^{\text{tot}}$. The latter shows similar behaviour, though naturally at a greater distance from the target equilibrium, while still also featuring fewer oscillations by the injection of thermal gas impurities. In this particular case at larger radiation fractions, $P\ix{bal,tot}^{\text{tom}}$ increases its deviation from null and the core calculated value, though the latter remains close to $P\ix{bal}=0$. With respect to the previously shown phantom tomography benchmark and experimental reconstruction statistics, this is generally in line with the established picture there and the corresponding linear regression fit results. Both $P\ix{bal,tot}^{\text{tom}}$ and $P\ix{bal,core}^{\text{tom}}$ are seen as a parameter for calculating $P\ix{bal}$ and an improvement in qualitative and quantitative terms compared to the standalone extrapolation of detector measurements. Given the initial example, in special applications or for an optimised algorithm of allocating MFR samples - the computational load for a continuous solution thereby is hardly practical -, this approach may prove to be a better radiation power loss measure towards a global power balance. While a second set of $P\ix{bal}$ using reconstructed solutions will be featured, extending this to a statistical level for a more thorough and generalized evaluation is out of scope for this work.\\%
                    \newline%
%
                    \begin{figure}[t]%
                        \centering%
                        \captionsetup{width=.4\textwidth}%
                        \begin{minipage}[c]{.4\textwidth}%
                            \centering%
                            \caption{%
                                XP20180725.44:\\%
                                Comparison between power balances for XP20180725.44 as in \cref{fig:tomo_experiment_20181010032_balance}. \textbf{(top):} Central experiment parameters towards the discharge power balance. \textbf{(bottom):} Comparison of power balances derived from a common $P\ix{rad}$ and the radiation distribution integrals.}\label{fig:tomo_experiment_201810725044_balance}%
                        \end{minipage}%
                        \hfill%
                        \begin{minipage}[c]{0.53\textwidth}%
                            \centering%
                            \includegraphics[width=\textwidth]{%
                                content/figures/chapter4/MFR/new/%
                                tomogram_power_balance_20180725.044.png}%
                        \end{minipage}%
                    \end{figure}%
%
                    The second example is presented similarly to \cref{fig:tomo_experiment_201810725044_balance} for XP20180725.44. Central plasma parameters in the top image, as well as the plasma global power balance profile calculated using $P\ix{rad}$ from the extrapolated detector measurements at the bottom have previously been introduced and discussed in \cref{fig:powerbal_0725.044}, as well as \cref{fig:20180725.44_PDF} and the following MFR reconstructions. In conclusion, this has supported the quasi steady-state applicability of the equation in \cref{eq:power_balance} and difficulties during scenarios of fast or drastic plasma transitions. The current example features both rapid dis-/continuous changes in radiation regimes or events and steady-state equilibria. Results of reconstructed two-dimensional powers shown here are centred temporally around the mentioned plasma collapse and condensation of emissivity to the core. Immediately noticeable is the consistent and almost constant separation not only between the total and core integral calculated $P\ix{bal}$ but also among those, indicating a significant quantitative difference for various MFR configurations, i.e. $k\ix{ani}$ parameters. The power balance derived from $P\ix{2D}^{\text{core}}$ is close to or slightly below equality and its $P\ix{rad}$ counterpart and therefore the radiative loss too large. Throughout the series of reconstructions, the integrated two-dimensional powers lead to a continuously but slowly decreasing balance with a trend from \SIrange{0}{-1.5}{\mega\watt} to \SIrange{-0.5}{-1.8}{\mega\watt}. This variation is about double that of the $P\ix{bal}$ profile reduction after the event. Additional and eventually coincidental markers for a given point in time from core and absolute integrated radiation power originate in deliberate $k\ix{ani}$ variations in the reconstruction.\\%
                    Similar results as before in \cref{fig:tomo_experiment_20181010032_balance} are presented, as the contribution of the core integrated tomogram emissivity fits better to the target equalisation of $P\ix{bal}=0$ and the $P\ix{rad}$ calculated reference than its $P\ix{2D,tot}$ counterpart. Furthermore, both are less impacted by the previously discussed stronger and faster transitions noted in the standard derived plasma power balance. Despite the steady-state development, the values computed using tomography results deviate significantly from their target and more so than for the feedback controlled discharge. Though the discrepancy between the two set of points and for varying MFR parameter configurations is also of similar magnitude. The worse match of $P\ix{bal,core}^{\text{tom}}$ hence is likely due to the characteristic shape of the underlying reconstructed brightness profiles in \cref{fig:tomo_20180725.44_times} particularly after the collapse, since essentially no emissivity is placed here for $r>r\ix{a}$. Again, the ratio between the tomogram powers is well in line with the prior statistical analysis. Overall, this still remains a valid estimator of the radiative plasma power loss and input to its total equilibrium power balance.\\%
%
    \section{Conclusions}\label{sec:conclusionschap4}%
%
        This concludes the final physical chapter of this thesis and therefore the introduction, benchmarking and experimental evaluation of a tailored \textit{Minimum Fisher regularisation} tomography algorithm for the multicamera bolometer diagnostic at W7-X. Tests applying geometric perturbations and using different methods of sensitivity discretisation have shown significant robustness of the MFR against such variations and simultaneously established a set of configurations that provide adequate transmissivity at favourable computational costs. The impact of essentially unknown discrepancies of the actual \textit{in-situ} orientation compared to an \textit{as-designed} assembly of the bolometer have been estimated using forward calculation and found to be non-negligible. However, it was measured to be of similar magnitude as other variabilities to the MFR and hence, very unfavourably, indistinguishable in the final tomogram. Those have to be anticipated and deliberate inference benchmark calculations made for a given geometry - per experimental campaign - and parameter set. At the same time, supplementary artificial camera arrays of deliberately tailored geometry, specifically filling gaps in the previously examined setup, were introduced to the tomography and used to aid in the reconstruction of more complex phantom radiation distributions. STRAHL simulated radiation profiles have presented an intrinsic asymmetry in the symmetrical designed bolometer LOS configuration and corresponding transmissivity matrix $\mathbf{T}$, with a bias towards the upper separatrix area and SOL.\\%
        The benchmark has shown both the effectiveness and limits of said algorithm, underlining the existence of an ideal set of tomography parameters for a given input two-dimensional emissivity distribution. Anisotropy regularisation weight coefficients produced results of more localised ($k<1$) or smooth ($k>1$) radiation profiles as intended, hence corresponding also to improved quality measures $\rho\ix{c}$ etc. The latter have shown to not be exclusively congruent and suggest that the quantitative or qualitative optimal tomogram is not necessarily achieved for $\chi^{2}=1$. Experimentally motivated artificial phantom images were proven to be particularly difficult to reconstruct, therefore provided valuable experience towards the reconstruction of actual experimental data. Iteration in $k\ix{ani}$ across a much larger set of profiles than presented has further cemented the above evaluations. Statistical analysis across all artificial emissivity distributions similarly agreed with those assessments, while also producing linear regression results correlating the two-dimensional radiation power and forward calculated $P\ix{rad}$. Thereby, both cameras find that core emissions exclusively are slightly overrepresented, and the absolute integrated power significantly overestimated by their corresponding extrapolations. Conclusively, a set of $k\ix{ani}=\left\{2, 0.3\right\}$, $N\ix{T}=15$ can adequately support initial reconstructions of bolometer measurements from high density, high radiation discharges on a $n\ix{r}\times n\ix{$\vartheta$}=30\times150$ pixel grid, covering $1.3^{2}V\ix{P}$ of the torus volume at reasonable computational costs.\\%
        Practical application of the established knowledge about the RDA MFR algorithm, i.e. tomography of W7-X campaign bolometer measurements provided adequate and robust results, though does not yield insights beyond this initial impression due to its limited execution. However, evaluation of all experimental reconstructions combined presented nearly identical numbers compared to its artificial counterpart, evidently supporting the findings in this reduced set tomograms. Furthermore, at last, integrated powers were used to calculate global, \textit{quasi steady-state} power balances for selected samples in time in comparison to the continuous $P\ix{rad}$ profile. While no absolute improvement is measured for neither core nor total integral and the aforementioned discrepancy between $P\ix{rad}$ and $P\ix{2D}$ is reflected here significantly, both are characteristically more stable and a potential alternative for small interval inspections.\\%
        The Minimum Fisher regularisation tomography algorithm in combination with the radially dependent anisotropy weighting has hereby been formally and thoroughly benchmarked and tested for the particular application with the multicamera bolometer diagnostic at W7-X. Conclusions are robust and promise overall reliable results, though one has to keep in mind the difficulty of multidimensional parameter optimisation during such ill-posed reconstruction problems. Combination with previous synthetic data and geometric perturbations was successful in outlining the shortcomings of this setup and providing hypothetical upgrades thereto.%
%