%
\chapter{Conclusion and Outlook}\label{chap:conclusion}%
%
    The importance of radiative power exhaust, majorly dependent on the experimental configuration, i.e. input heating and density, as well as impurity composition and concentration is well-understood. Towards the operation of existing and the development of future fusion reactors, be it for research purposes or as a DEMO power plant, exploration of deliberately impurity seeded  and feedback controlled scenarios with focus on detachment physics for heat energy dissipation and machine safety is of great interest. Reliable and stable controlled radiative cooling in the confined region, in the SOL and divertor with radiation power fractions of $f\ix{rad}\ge95\%$ is therefore necessary.\\%
%
    \,\newline%
    A real-time bolometer radiation feedback system, working together with thermal impurity gas valves in the SOL for injection has been designed and implemented during the operational phase OP1.2b at W7-X. It was applied successfully, achieving stable hydrogen plasma with controlled helium edge cooling at radiation fractions $\ge85\%$ and peak values of up to $100\%$. Here, reduction of target heat loads of at least a factor of two was measured, while detachment of C$^{3+}$ ionisation radiation was visible already for $f\ix{rad}\sim50\%$. A limited set of at least three lines of sight was validated as a proxy for fast radiation power loss extrapolations. Simultaneously, comparisons with dimensionless scaled, raw signals of singular detectors have suggested the programming with individual LOS for steady-state scenarios. However, computational and algorithmic limitations infer non-negligible latencies to the bolometer feedback and were difficult to optimise for during commissioning. Despite this, the intrinsic connection of $P\ix{rad}$ and ultimately $f\ix{rad}$ to the detachment process makes this approach essential for future applications.\\%
%
    \,\newline%
    Exploration and inference of the corresponding experimental parameter space was not able to determine a particular scaling or underlying correlation for the corresponding low-Z impurity seeded feedback experiments. It was established however that moderately scaled gas puffs, i.e. medium length and intensity, generally were very capable of reliably performing edge cooling without terminal plasma perturbation. Rudimentary models for this injection process, proposing two and three chamber systems have been found both to be equally capable of representing exemplary radiation measurements of feedback activations. However, respective input parameters provided no further significant insight. Given the systems limitations, evaluation of the line of sight sensitivity towards relevant radiation scenarios showed detectors viewing the separatrix and SOL region to be most viable for \textit{a priori} real-time bolometer feedback configurations. No explicitly best set of three, five or seven detectors exists, though for feedback purposes, a robust selection of few vertical or/and horizontal bolometer camera channels can be provided that generally achieves $\ge85\%$ prediction accuracy when compared to the full data set. Agreement for both analysis regarding only radiation data and also central plasma parameter between the two cameras underline these results. The one-dimensional transport simulation code STRAHL was used in combination with experimental data from feedback applications and input parameters modelled for comparison purposes. Results have shown carbon impurities to be the dominant contributor to SOL emissivity, while reduction of diffusivity, as well as electron temperature and density around the separatrix provided radiation profiles similar to experimental findings. Both see an inward shift from outside to inside the separatrix for $f\ix{rad}\rightarrow1$. However, forward calculation of using STRAHL calculations produced no comparable level of asymmetry in the line of sight integrated chord brightness, suggesting larger discrepancies in the model ansatz.\\%
%
    \,\newline%
    Significant robustness and reliability for a tailored radially dependent anisotropy (RDA) weighting Minimum Fisher regularization (MFR) algorithm in combination with the W7-X multicamera bolometer system have been established through rigorous geometry perturbation testing and verification. A large scale benchmark thereof, involving numerous simple and complex phantom radiation profiles for reconstruction, has suggested the existence of an ideal set of anisotropy parameters for a given emissivity distribution, though the dimensionality of this ill-posed optimization problem makes it essentially impossible to actually find such a set $k\ix{ani}^{\ast}$. By that, an intrinsic bias towards the upper SOL and separatrix area in the transmissivity matrix $\mathbf{T}$ was found. Limited interval weighting factor variations have shown, as per design, $k\lessgtr1$ to correspond well to localised or smooth radiation structures in reconstruction, respectively. Evaluation of phantom image tomographies with additional, two-dimensional quality coefficients demonstrated potential misalignment between adequate fitness factors $\chi^{2}$ and truly optimal reconstructions. For the set of artificial brightness profiles, $P\ix{rad}$ consistently underestimates the actual, total power contained in the phantom. Lastly, application of this knowledge to experimental measurements presented results in line with the corresponding benchmark, and integration into power balance calculations provided a stable, albeit costly alternative to $P\ix{rad}$.\\%
%
    \,\newline%
    The central feature of this thesis, the core bolometer real time feedback system for radiative cooling and detachment purposes, is significantly hampered by its large intrinsic latency compared to other control candidates and corresponding computational limitations. Upgrades thereof will lead to a proportional improvement of the individual estimators and eventually also enable one to establish and implement a more complex and versatile $P\ix{pred}$ model, taking operational configurability into account on a \textit{per-experiment} basis. For that, a more delicate approach to an attempted scaling law would be beneficial and could add a predictive component to the provided radiation gauges. Similarly, a thorough analysis of the correlation between multichannel extrapolation $P\ix{pred}^{1}$ and single signal $P\ix{pred}^{2}$, with focus on the derivative character and omitted volumetric scaling could provide necessary insights to the bolometer feedback.\\%
    Results for the multi-chamber model have a promising outlook and an educated parametric optimisation process with respect to experimental data and other SOL models might be necessary to yield a better picture about corresponding population and transport regimes therein.\\%
    Finally, a more detailed approach with a larger set of relevant phantom images to supplementary artificial bolometer cameras at the current location can provide solid candidates for the practical extension thereof. Nevertheless, it certainly aids the confidence when performing Minimum Fisher tomographies of measured data when extending the set of benchmark reconstructions by more examples from the vast range of principle brightness profile shapes and their combinations that has not been touched upon in this work.%

