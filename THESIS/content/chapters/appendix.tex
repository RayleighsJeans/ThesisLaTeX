%
\chapter{Measurement algorithm}\label{apx:algorithm}%
%
    \begin{algorithm}[H]%
        \SetKwInOut{Input}{input}\SetKwInOut{Output}{output}%
        \IncMargin{1em}%

        \SetAlgoLined%
        \KwResult{calibration, measurement, real-time feedback}%
        \BlankLine%

        \Input{$\Delta t$, ranges, $S$, $c$, $N$, $M$, $V_{M}$ and $K_{M}$}%
        \Output{\emph{at runtime:} $P\ix{pred}^{\left(1\right)}$, $P\ix{pred}^{\left(2\right)}$; \emph{post:} calibration, $\kappa$, $\tau$, $R$, $\Delta U$, $P\ix{pred}^{\left(n\right)}$}%
        \BlankLine%

        \textbf{TRIGGER} $\rightarrow$ \emph{initialization} $\left[T1-\text{\SI{60}{\second}}\right]$\;%
        $K_{M}$, $V_{M}$ $\leftarrow$ input file (standard geometry)\;%\
        $S$, $c$ $\leftarrow$ file/visual input, selection and single channel\;%
        \BlankLine

        $U_{\text{AO},n}$ $\leftarrow$ memory adress, initialize \SI{0}{\volt}\;%
        \textbf{NI}\textsuperscript{\textregistered} \textbf{6321} $\leftarrow$ initialize two-channel analog output\;%
        \BlankLine%
    
        \tcp{output as long as main data cquisition}%
        \tcp{does not return successfully}%
        \While{measurement incomplete}{%
            \For{$n\in\left\{1,2\right\}$}{%
                $U_{\text{A0},n}$ $\leftarrow$ read $P\ix{pred}^{\left(n\right)}$\;
                device FIFO $\leftarrow$ $U_{\text{A0},n}$;%
                \hspace{10em}\raisebox{.1\baselineskip}[0pt][0pt]{$\left.\rule{0pt}{2.75\baselineskip}\right\}\ \mbox{parallel}$}\\%
                flush FIFO to voltage output\;
            }%
        }%

        \vdots%
        \caption{Bolometer measurement routine and feedback.}\label{alg:algorithm}%
    \end{algorithm}%
    \newpage%
%
    \addtocounter{algocf}{-1}%
    \begin{algorithm}[H]%
        \vdots%
        \textbf{mode register} $\leftarrow$ full range $\pm$\SI{80}{\milli\volt}\;%
        \textbf{filter register} $\leftarrow$ sample time \SI{0.4}{\milli\second}\;
        $V\ix{off}$ $\leftarrow$ \SI{10}{\second} offset measurement mean\;
        \BlankLine%

        \textbf{mode register} $\leftarrow$ two-stage calibration\;%
        \For{absorber$\,\,\in($measurement, reference$)$}{%
            \emph{\tcp{10000 samples over \SI{4}{\second}}}%
            \For{$i=0$, $i<10000$, $i\rightarrow 10000$}{%
                \If{$2000<i<6000$}{%
                    $R$, $\tau$ $\leftarrow$ max current and decay\;%
                    $\kappa$ $\leftarrow$ linear resistance change by ohmic heating\;%
                }%
            }%
        }%

        \textbf{mode register} $\leftarrow$ individual range (\SI{10}{\milli\volt}, \SI{20}{\milli\volt})\;%
        \textbf{filter register} $\leftarrow$ sample time $\Delta t$\;%
        \BlankLine%

        $^{\left(1\right)}$\text{FIFO} $\leftarrow$ initialize $\vert S\vert\times\left(M+2\right)$ array\;%
        $^{\left(2\right)}$\text{FIFO} $\leftarrow$ initialize $\left(M+1\right)$ array\;%
        $P\ix{pred}^{\left(n\right)}$, $\Delta U$ $\leftarrow$ initialize $N$ sample array\;%

        \vdots%
        \caption{Bolometer measurement routine and feedback.}%
    \end{algorithm}%
    \newpage%

    \addtocounter{algocf}{-1}%
    \begin{algorithm}[H]%
        \vdots%
        \tcp{final measurement}%
        \For{$i=0$, $i<N+1000$, $i\rightarrow N+1000$}{%
            \If{$i>1000$}{
                $\Delta U^{\left(i\right)}$ $\leftarrow$ $\Delta t$ integrated absorber voltage\;
                \tcp{store samples in FIFO}%
                \For{$j\in S$}{%
                    $^{\left(1\right)}$\text{FIFO}$_{j}^{\left(i\,\text{mod}\,M\right)}$ $\leftarrow$ $\Delta U_{j}^{\left(i\right)}$\;%
                }%
                $^{\left(2\right)}$\text{FIFO}$^{\left(i\,\text{mod}\,M\right)}$ $\leftarrow$ $\Delta U_{c}^{\left(i\right)}$\;%
                \BlankLine%

                \tcp{mean over previous $M$ samples}%
                \For{$j\in S$}{%
                    $^{\left(1\right)}$\text{FIFO}$_{j}^{\left(M+2\right)}$ $\leftarrow$ $^{\left(1\right)}$\text{FIFO}$_{j}^{\left(M+1\right)}$\;%
                    $^{\left(1\right)}$\text{FIFO}$_{j}^{\left(M+1\right)}$ $\leftarrow$ $1/M\,\sum_{k}^{M}\,^{\left(1\right)}\text{FIFO}_{j}^{\left(k\right)}$\;%
                    \BlankLine%

                    \tcp{derivative}%
                    $\diff\left(\Delta\widetilde{U}_{j}^{\left(i\right)}\right)/\diff t$ $\leftarrow$ $\left(^{\left(1\right)}\text{FIFO}_{j}^{\left(M+2\right)}+^{\left(1\right)}\text{FIFO}_{j}^{\left(M+1\right)}\right)/\Delta t$\;%
                }%
                $^{\left(2\right)}$\text{FIFO}$^{\left(M+1\right)}$ $\leftarrow$ $1/M\,\sum_{k}^{M}\,^{\left(2\right)}\text{FIFO}^{\left(k\right)}$\;%
                \BlankLine%

                \tcp{memory address for feedback}%
                $P\ix{pred}^{\left(1\right)}$ $\leftarrow$ \cref{eq:prediction} $\leftarrow$ $K_{M}$, $V_{M}$, $\diff\left(\Delta\widetilde{U}_{j}^{\left(i\right)}\right)/\diff t$, $^{\left(1\right)}\text{FIFO}^{\left(k\right)}$\;%
                $P\ix{pred}^{\left(2\right)}$ $\leftarrow$ \cref{eq:prediction2} $\leftarrow$ $^{\left(2\right)}\text{FIFO}^{\left(M+1\right)}$\;%
            }%
        }%
        \BlankLine%

        \tcp{done, upload and saving}%
        \textbf{HDD} $\leftarrow$ $P\ix{pred}^{\left(1\right)}$, $P\ix{pred}^{\left(2\right)}$, $\Delta U$, $\kappa$, $\tau$, $R$, calibration currents\;%
        \textbf{W7-X archive} $\leftarrow$ $P\ix{pred}^{\left(1\right)}$, $P\ix{pred}^{\left(2\right)}$, $\Delta U$, $\kappa$, $\tau$, $R$, calibration currents\;%
        \caption{Bolometer measurement routine and feedback.}%
    \end{algorithm}%

\chapter{Prediction of Feedback Impact on Plasma Parameters}\label{apx:peaktopeak}%
%
    In order to adjust and optimise the real time bolometer feedback with gas valves as actuators, the temporal and quantitative correlation between activation and reaction of plasma parameters has to be understood. Varying injection durations and amounts of gas from the feedback valve have equally varying impact on the evolution of plasma profiles. Furthermore, the latency between feedback action and plasma reaction can greatly change the effect the injected gas has on the discharge characteristics. This also may play a large role when reevaluating the applicability and performance of the real time bolometer feedback.\\%
    In an attempt to find possible correlations between the activation of the thermal gas valves and change in plasma parameters, data from the entire set of accomplished feedback experiments has been evaluated using an algorithm that \textit{finds peaks (local maxima)} and compares them between the actuator and given profiles. An example of such a \textit{peak-finding-and-matching} can be found in \cref{fig:peak_finding_examples}. Two sets of results for different algorithm parameters but the same profiles are shown here. The origin profile, i.e. the evolution of the actuator, from which found peaks are used to look forward in time for correlating reactions in other profiles, is represented by the activation of the AEH51 thermal gas valve. The \textit{target profile}, i.e. the line representing the reaction to the \textit{origin profile} (feedback activation), in which the algorithm looks for corresponding maxima to the previous peaks from the origin, is given by the $P\ix{rad}$ of the horizontal bolometer camera. Depending on a provided \textit{width, distance} and \textit{prominence}, the algorithm searches for local maxima in the origin, which are colour coded with transparent bars and matched in the same colours to peaks in the target profile. The corresponding maxima in the activation are interlaced with higher transparency in the same plot. Such a match is produced if a local maximum is found in the target profile within a certain time interval, i.e. $T\ix{peaks}<$\SI{450}{\milli\second}, after the location of the peak in the activation. For multiple occurrences, the first maximum forward in time is chosen. In the provided examples, peaks for a given minimum width and distance in-between of $\tau$=\SI{75}{\milli\second} and \SI{150}{\milli\second} are shown. The prominence of local maxima with significant width and distance to other peaks are evaluated based on the surrounding profile: for an increase of $>100\%$ of the variation within $\tau/2$ in the profile around the peak, a local maximum is found by the algorithm. For further details, see the documentation of the underlying core routine in \cite{ScipyFindPeaks}.\\%
%
    \begin{figure}[t]%
        \centering%
        \begin{subfigure}{0.49\textwidth}%
            \includegraphics[width=\textwidth]{%
                content/figures/chapter3/peaks/%
                peak_mathching_75ms_example.pdf}%
            \caption{}%
        \end{subfigure}%
        %\hfill%
        \begin{subfigure}{0.49\textwidth}%
            \includegraphics[width=\textwidth]{%
                content/figures/chapter3/peaks/%
                peak_mathching_150ms_example.pdf}%
            \caption{}%
        \end{subfigure}%
        \caption{%
            XP:20181010.32:\\%
            Example of how peaks in experimental data are detected. Shown here is $P\ix{rad,HBCm}$, as well as the feedback valve control in port AEH51. The peaks found in the lower plot are transparently overlaid onto the upper. Same colours indicate cause and effect relation between gas injection and plasma parameter ($P\ix{rad,HBCm}$) changes.  Algorithmic peak detection width set to \textbf{(a)} \SI{75}{\milli\second} and \textbf{(b)} \SI{150}{\milli\second}.}\label{fig:peak_finding_examples}%
    \end{figure}%
%
    The provided examples for different peak widths and distances $\tau$ in \cref{fig:peak_finding_examples} showcase the importance of careful adjustment of the selected algorithm parameters and their influence. For smaller $\tau$, potentially fewer local maxima are detected for the same prominence as for larger $\tau$, while increasing width and distance inevitably lead the algorithm to produced false positive matches. Reducing the interval in which corresponding peaks can be found to match also, on one hand act as a cut-off for false positives but on the other hide high latency responses in the target plasma profiles. Within the range of $\tau=$\SIrange{75}{150}{\milli\second} and, depending on the profile, $T\ix{peaks}<$\SI{450}{\milli\second} the algorithm works best with the provided feedback experiment data. However, as is already presented by the above examples, not all local maxima are found for the given parameter selection. A manual procedure to find peaks in the lines of data from feedback experiments that were related to the activation of the thermal gas valves is, given the large number of plasma parameters to search through and discharges performed, not within the scope of this work. That said, if not stated otherwise, the following results have been achieved with $\tau=$\SI{150}{\milli\second} and the previously outlined prominence - an increase $>100\%$ of the variance within $\tau/2$ around a local maximum. For electron density measurements of the dispersion interferometer, the interval was set to $T\ix{peaks}=$\SI{150}{\milli\second}, while for $P\ix{rad}$ from the bolometer camera system required a larger setting of $T\ix{peaks}=$\SI{450}{\milli\second}. Data from central electron densities measurements $n\ix{e}$ and heating power $P\ix{ECRH}$ for points in time of both origin and target profile peak have also been collected simultaneously.\\%
%
    \begin{figure}[t]%
        \centering%
        \includegraphics[width=\textwidth]{%
            content/figures/chapter3/peaks/%
            peaks_key_gas_val_H2_150ms.pdf}%
        \caption{Peak database of $P\ix{rad}$ for both bolometer cameras individually from hydrogen injection experiments. Changes in radiation power are shown for \textbf{(top left)} the change in control, \textbf{(top right)} the approximate injected gas amount, \textbf{(bottom left)} the time between the maximum peak values and \textbf{(bottom right)} the time delay between peaks over the respective feedback control change.}\label{fig:peak_database_H2}%
    \end{figure}%
%
    The results for the above described procedure from all thermal gas injection feedback experiments that featured hydrogen as the injected gas are presented in \cref{fig:peak_database_H2}. The compiled data points show the results for the horizontal and vertical bolometer cameras as the target profile individually. The top left shows the increase in $P\ix{rad}$ over the change in gas flow or valve activation $\Gamma$ - assuming constant gas pressure and temperature they are synonymous. The absolute gas flow is not known for all data points, hence a presentation in arbitrary units in favour of comparability is chosen. Below that the increase in radiative power loss as a function of gas injection duration $T_{\Gamma}$ can be seen. The top right image shows the change in $P\ix{rad}$ over the total injected amount of gas, i.e. the integrated valve activation peak $\int\Gamma$ over the interval $T_{\Gamma}$.\\%
    The presented results in \cref{fig:peak_database_H2}:(top left) show that for a large range of gas pulse strengths, $\Delta P\ix{rad}$ changes almost always with small increments, while few exceptions with no coherent pattern exist across the entire spectrum of valve activations. For smaller injection gas flows, i.e. between \SIrange{5}{10}{\arbitraryunit} a small group of points with larger corresponding $\Delta P\ix{rad}$ for both cameras can be seen, indicating that with a reduced flow rate the respective change in radiation increases. The aforementioned exceptions at higher radiation increments might be attributed to varying plasma parameters, which will be examined at a later point in this chapter. Similar findings are presented in the bottom left plot, where the majority of results are found at small $\Delta P\ix{rad}$ across the entire spectrum of gas injection durations $T_{\Gamma}$. However, a larger number of results for a radiation increase $>$\SI{0.25}{\mega\watt} is found for smaller durations, i.e. \SIrange{0.15}{0.3}{\second} by both cameras equally. Again, singular data points at high $\Delta P\ix{rad}$ across all $T_{\Gamma}$ show no distinct behaviour. This is also reflected in \cref{fig:peak_database_H2}:(top right), where for small to medium amounts of injected gas, i.e. up to \SI{20}{\arbitraryunit}, the majority of results are found below \SI{0.5}{\mega\watt} for both bolometer cameras. This can be linked to the previously discussed plots by either assuming small gas flows for longer injection durations or more intense, shorter pulses. Finally, the last image on the bottom right displays a significantly higher data point density for smaller gas flows and larger latencies between action and reaction in plasma radiation between \SIrange{0.2}{0.5}{\second}. For higher injection intensities, more results are found at $\Delta T\ix{peak}>$\,\SI{0.3}{\second}, further suggesting that higher gas flows do not necessarily, if at all lead to greater or quicker plasma radiation reactions.\\%
%
    \begin{figure}[t]%
        \centering%
        \includegraphics[width=\textwidth]{%
            content/figures/chapter3/peaks/%
            peaks_key_gas_val_He_150ms.pdf}%
        \caption{Similar plot as figure \cref{fig:peak_database_H2}, with the gas used for feedback being helium. Peak searching algorithm parameters are kept the same.}\label{fig:peak_database_He}%
    \end{figure}%
%
    The same layout and presentation of data has been used to collect and display results of the algorithm for \textit{helium impurity seeded} experiments, where the feedback controlled thermal gas valve was fed with molecular helium in \cref{fig:peak_database_He}. Far fewer results have been produced by the algorithm for the same set, or a selection of different sets of parameters at that, than for hydrogen feedback experiments. This is attributed to the greatly smaller number of performed experiments under such conditions. As for \cref{fig:peak_database_H2}, the top left image shows most of the points at $\Delta P\ix{rad}<$\,\SI{0.2}{\mega\watt}, while across the rest of the spectrum in gas flow no coherent pattern can be recognized. Similar findings are presented by the top right plot - the majority of the results show small injection gas amounts at $\Delta P\ix{rad}$<\,\SI{0.5}{\mega\watt}. Consequently, the point density in the bottom left image is highest up to $T_{\Gamma}<$\,\SI{0.5}{\second} and very small radiation power loss increments, while fewer results can also be found at up to $\Delta P\ix{rad}\sim$\,\SI{1}{\mega\watt}. Finally, the latency between injection of helium and changes in plasma radiation power, which is shown in the bottom right plot of \cref{fig:peak_database_He}, is generally found to be between \SIrange{0.1}{0.5}{\second} at small gas flows. For larger flows, no significant pattern is displayed here. In conclusion, very similar, results are presented by the algorithmically collected data points for helium injected feedback experiments as for their hydrogen counterparts. On one hand, this underlines the interpretation of the previously discussed \cref{fig:peak_database_H2}, since the experimental approach and goal were the same. On the other hand, this poses the question as to why a different impurity produces quantitatively comparable results to the injection of working gas. Hence, a more thorough analysis of the dataset is required, involving a larger set of evaluated plasma parameters than just $P\ix{rad}$ of the bolometer camera system.\\%
%
    \begin{figure}[t]%
        \centering%
        \includegraphics[width=\textwidth]{%
            content/figures/chapter3/peaks/%
            peaks_param_key_type_val_QSQ_150ms}%
        \caption{Selected experiment parameters for previously presented peaks in $P\ix{rad}$ from feedback experiments. \textbf{(top left)} change in radiation power over change in control with electron density colour map \textbf{(top right)} radiation fraction over heating power with $n\ix{e}$ map \textbf{(bottom left)} radiation power delta over temporal peak delay with ECRH colour code \textbf{(bottom right)} electron density by heating power with radiation fraction map.}\label{fig:peak_parameters_QSQ}%
    \end{figure}%
%
    The extended database of results previously shown and discussed in \cref{fig:peak_database_H2} and \ref{fig:peak_database_He}, including core plasma parameters $f\ix{rad}$, $P\ix{ECRH}$ and $n\ix{e}$ is shown in \cref{fig:peak_parameters_QSQ}. Results in all plots were reduced due to availability of the corresponding plasma parameters. The presentation of data points is very similar to the prior, where results are plotted for both bolometer cameras individually. Furthermore, the axis and abscissa on the two left images are kept the same. The two right figures feature, on the top, the radiation loss fraction $f\ix{rad}$ and, on the bottom, electron density $n\ix{e}$ over cyclotron heating power $P\ix{ECRH}$. In all four plots, a colour grading is imposed on the scattered points: the electron density is used on the top two images, while the bottom left shows the heating power and the bottom right the radiation fraction.\\%
    The superimposed core electron density on the top left plot in \cref{fig:peak_parameters_QSQ} indicates no noticeable behaviour across the entire spectrum of gas flows and radiation powers. On the bottom left plot, $P\ix{ECRH}$ is higher for larger changes in radiation power loss $\Delta P\ix{rad}$, while no correlation to the latency between feedback action and plasma reaction therein is found. The plot on the top right features a clear trend of higher electron densities for higher heating powers and radiation loss fractions, i.e. peaks have been found for $f\ix{rad}\sim100\%$ and $P\ix{ECRH}=\,$\SI{6}{\mega\watt} at $n\ix{e}\ge\,$\SI{10}{\per\cubic\meter}. One should note here that radiation fractions greater than unity are found by the searching algorithm, since momentary power exhaust from the discharge through radiation can be greater than the input heating power under the assumption that plasma stored energy is lost. Finally, the last image on the bottom right supports the observations of the previous plot, where larger heating powers and electron densities are also accentuated with higher $f\ix{rad}$.\\%
    The results presented in the two left images of \cref{fig:peak_parameters_QSQ} promote no further insight into possible correlations of feedback gas injection and changes in plasma radiation, besides the indication of a decreased latency between action and reaction. However, the findings in the top and bottom right plots describe an evident connection between the input heating power, electron core density and radiation loss fraction. Similar results have been previously examined and discussed for different divertor and magnetic field configurations, mainly under steady state conditions\cite{Klinger2016,Fuchert2018,Zhang2020}, implying that the feedback gas injection reactions are governed by the same plasma power scaling laws. Here, at higher heating powers $>\,$\SI{3}{\mega\watt}, peaks in the plasma radiation have been found for electron densities $n\ix{e}>\,$\SI{5e19}{\per\cubic\meter} and loss fractions $f\ix{rad}\in\left\{0.2,\,1.2\right\}$. Concurrently, for higher values of $P\ix{ECRH}$ and $n\ix{e}$, radiation power loss fractions significantly greater than unity, i.e. $f\ix{rad}>1$ are found. Examinations of the previous \cref{chap:realtimefeedback} and especially \cref{subsec:densityfeedback} have shown that higher electron densities combined with $P\ix{rad}$ close to the input heating power - $f\ix{rad}\ge80\%$ - are favourable to an improved radiative cooling of the scrape-off layer and target heat load reduction. The data presented in \cref{fig:peak_parameters_QSQ} therefore overall supports and is consistent with the findings of the latter chapter, however they so far do not exhibit any dependencies whatsoever that could be used to improve the performance of the real time radiation feedback.%
%
    \subsection{Comparison with Non-Feedback Gas Injection}\label{sec:compmainvalve}%
%
        \begin{figure}[t]%
            \centering%
            \includegraphics[width=\textwidth]{%
                content/figures/chapter3/peaks/%
                peaks_key_type_val_Main_150ms.pdf}%
            \caption{Peak database from main gas valve. Changes in radiation power $P\ix{rad}$ of both cameras for \textbf{(top left)} the change in gas flow, \textbf{(top right)} the approximate injected gas amount, \textbf{(bottom left)} the time between the maximum peak values and \textbf{(bottom right)} the time delay between peaks over the respective gas flow.}\label{fig:peak_database_main}%
        \end{figure}%
%
        In order to reference the results gathered from the peak finding algorithm for the thermal gas injection valve and feedback activation, the same has to be conducted for the main gas valves located around W7-X. The corresponding data points for both working gases, helium and hydrogen and a minimum peak width of $\tau=$\,\SI{150}{\milli\second}, like in \cref{fig:peak_database_H2} and \cref{fig:peak_database_He}, can be found in \cref{fig:peak_database_main}. The structure of plots and quantities displayed here are the same as before. One should note here that the scale for $\Gamma$ is again in arbitrary units, though the absolute value thereof is not necessarily comparable to that of the thermal gas valves of the helium beam diagnostic, since nozzle and gas injection characteristics, as well as pressures and temperatures of the gases are different.\\%
        In the top left image of \cref{fig:peak_database_main}, a similar behaviour as for the feedback gas valve is seen, where across the spectrum of gas flows $\Delta\Gamma$, the radiation power loss changes with small increments. For smaller fuelling intensities, i.e. <\SI{10}{\arbitraryunit} the response in $\Delta P\ix{rad}$ increases like before, quantitatively and qualitatively, while also no coherent pattern can be seen for larger $\Delta P\ix{rad}$ along the abscissa. The bottom left picture features a much narrower distribution of points than before. Up to $T_{\Gamma}\le\,$\SI{1}{\second}, a significant increase in radiation power loss is found towards $T_{\Gamma}\sim\,$\SI{0.15}{\second} with $\Delta P\ix{rad}>\,$\SI{1}{\mega\watt}. Furthermore, the majority of peaks is now located around an injection duration between \SIrange{0.3}{0.6}{\second} below \SI{0.5}{\mega\watt}. This may be attributed to, on one hand the method of injection - main gas valves are applied before the start of the discharge and again, if necessary, to feed the plasma during the experiment - and on the other the valve latency and their distance to the SOL. Beyond $T_{\Gamma}\sim\,$\SI{1.2}{\second} the point density is greatly reduced and almost no results can be found beyond this point. Scattered results at $\Delta P\ix{rad}>\,$\SI{0.5}{\mega\watt} show no coherent pattern. In combination of the latter two, the top right plot presents a continuation of the previous findings: around lower to medium amounts of injected gas, $\int\Gamma\le$\SI{20}{\arbitraryunit} the response in $P\ix{rad}$ is increased up to \SI{0.5}{\mega\watt}, while beyond that no significant differences can be seen. Finally, the last image on the bottom right also presents a higher data point density up to medium gas flows, $\Delta\Gamma\le\,10\,$a.u and higher latencies between injection and reaction $\Delta T\ix{peak}\ge\,$\SI{0.2}{\second}. However, no results are found for very small gas flows $\Delta\Gamma<$\SI{2}{\arbitraryunit} up to very high latencies of \SI{0.4}{\second}. At higher injection intensities, this density decreases greatly and for higher latencies $\Delta T\ix{peak}>\,$\SI{0.3}{\second} slightly more peaks are found still. This is in agreement with \cref{fig:peak_database_H2}: higher gas flows do not necessarily imply quicker or stronger reactions in radiation power loss, while slower, more moderate injections appear to yield better results for cooling the plasma.\\%
%
        \begin{figure}[t]%
            \centering%
            \includegraphics[width=\textwidth]{%
                content/figures/chapter3/peaks/%
                peaks_param_key_type_val_Main_150ms}%
            \caption{Experiment parameters for main gas valve. \textbf{(top left)} change in radiation power over change in feedback control with electron density colour map \textbf{(top right)} radiation fraction over heating power with $n\ix{e}$ map \textbf{(bottom left)} radiation power delta over temporal peak delay with ECRH colour code \textbf{(bottom right)} electron density by heating power with radiation fraction map.}\label{fig:peak_parameters_Main}%
        \end{figure}%
%
        In order to complete the comparison between the injection valves with respect to their impact on the radiation power loss, a counterpart to \cref{fig:peak_parameters_QSQ} involving additional plasma parameters has to be provided. The corresponding result can be found in \cref{fig:peak_parameters_Main}. The procedure to acquire the data points, as well as their presentation in the plots is the same as before.\\%
        In the top left of \cref{fig:peak_parameters_QSQ}, the superimposed core electron density slightly increases with larger $\Delta P\ix{rad}$ for smaller gas flows. At higher injection intensities, few data points scattered along the ordinate and at very low radiation impact also indicate higher $n\ix{e}$. Similar to \cref{fig:peak_parameters_QSQ}, on the bottom left larger $\Delta P\ix{rad}$ for small injection durations are noted with increasing input heating powers. The two right-hand plots show the same quantitative picture as before, where with larger $P\ix{ECRH}$ both electron density and radiation power loss fraction $f\ix{rad}$ increase. Furthermore, the overall structure of the data point distributions are very similar. The top right image however does not feature an accumulation of points towards very high $f\ix{rad}$ and heating powers with $n\ix{e}\ge\,$\SI{10}{\per\cubic\meter}.\\%
        From \cref{fig:peak_database_H2} through \cref{fig:peak_parameters_Main}, one can summarize that, in general, a gas injection with moderate intensity and duration is most suited for radiatively cooling the scrape-off layer and thereby potentially improving its overall performance with regard to increasing electron density and temperature. Besides the apparent differences in application, i.e. construction and methodology, no significant qualitative change in plasma radiation response between thermal gas feedback valve and main gas inlet is found. That said, this is concluded with respect to the implied direct causality (algorithm), where the action-reaction connection between the gas pulse and peak in radiation power loss is implied. As stated before, and supported by the very similar results in \cref{fig:peak_parameters_Main}, the algorithmically established peak database adheres to the same plasma (performance) scaling laws which have been established for steady state conditions. Finally, in reference to the scientific questions posed in the beginning of this \cref{chap:feedbackeval}, so far no correlation between the activation of the feedback, the radiation power loss and underlying plasma parameters could be established.%

%% RESIDUAL

% STRAHL
% \begin{align}%
%     \begin{split}%
%         \Phi\ix{0}=4\pi r\ix{bound}v\ix{0}n_{\text{i,}0}\left(r\ix{out}\right)\\%
%         S_{\text{i,}0\to 1}\overset{!}{=}S_{\text{i,}0}=n\ix{0}\left(r\right)n\ix{e}\left(r\right)Q_{\text{i,}0}\left(r\right)\\%
%         n\ix{0}\left(r\right)=\frac{n\ix{0}\left(r\ix{out}\right)r\ix{out}}{r}\exp{\left(-\displaystyle\int_{r}^{r\ix{out}}\frac{n\ix{e}Q_{\text{i,}0}}{v\ix{0}}\diff r\right)}%
%     \end{split}%
% \end{align}%
%
% \begin{align}%
%     \begin{split}%
%         D\ix{i,Z}\head{cl}=&\left(\frac{\partial\Psi\ix{pol}}{\partial r}\right)^{-2}\left\langle\frac{R^{2}B^{2}\ix{pol}}{B^[2]}\right\rangle\,\frac{m\ix{i}k\ix{B}T\ix{i}\nu\ix{i,D}}{e^{2}Z^{2}}\\%
%         v\ix{i,Z}\head{cl}=&D\ix{i,Z}\head{cl}Z\,\left(\frac{\diff\left(\ln\,n\ix{D}\right)}{\diff r}-\frac{1}{2}\frac{\diff\left(\ln\,T\ix{i}\right)}{\diff r}\right)\\%
%         \nu\ix{i,D}=&\frac{4e^{4}\sqrt{2\pi}}{m\ix{e}}\sqrt{\frac{m\ix{i}m\ix{D}}{m\ix{i}+m\ix{D}}}\frac{Z\ix{i}^{2}Z\ix{D}^{2}\ln\,\Lambda}{\left(k\ix{B}T\right)^{3/2}}\,n\ix{D}%
%     \end{split}%
% \end{align}%
%
% \begin{align}%
%     \begin{split}%
%         D\ix{i,Z}\head{PS}=&\left(\frac{\partial\Psi\ix{pol}}{\partial r}\right)^{-2}\left\langle RB\ix{tor}\right\rangle^{2}\left(\left\langle B^{-2}\right\rangle-\left\langle B^{2}\right\rangle^{-1}\right)K\ix{PS}\frac{m\ix{i}k\ix{B}T\ix{i}\nu\ix{i,D}}{e^{2}Z^{2}}\\%
%         v\ix{i,Z}\head{PS}=&D\ix{i,Z}\head{PS}\,Z\left(\frac{\diff\left(\ln n\ix{D}\right)}{\diff r}+F\ix{PS}\left(\varepsilon,\,v^{\ast}\ix{D}\right)\frac{\diff\left(\ln T\ix{i}\right)}{\diff r}\right)%
%     \end{split}%
% \end{align}%
%
% \begin{align}%
%     \begin{split}%
%         D\ix{i,Z}\head{bp}=&\left(\frac{\partial\Psi\ix{pol}}{\partial r}\right)^{-2}\frac{\left\langle RB\ix{tor}\right\rangle^{2}}{\left\langle B^{2}\right\rangle}\frac{k\ix{B}T\ix{i}\mu\ix{i,D}^{\ast}}{e^{2}Z^{2}n\ix{i}}\\%
%         v\ix{i,Z}\head{bp}=&D\ix{i,Z}\head{bp}\,Z\left(\frac{\diff \left(\ln n\ix{D}\right)}{\diff r}+F\ix{bp}\frac{\diff\left(\ln T\ix{i}\right)}{\diff r}\right)%
%     \end{split}%
% \end{align}%
%

% \subsubsection*{Fourier transform}%

%     Fourier transform.%

%     \begin{align}%
%         \begin{split}\label{eq:fourier_base}%
%             g\ix{i}=\mathcal{F}\left(P\ix{rad}\right)&\left(\omega\right)\,\qquad g\ix{j}=\mathcal{F}\left(P\ix{pred}^{\left(1\right)}\right)\left(\omega\right)\\%
%             g\ix{i,j}\left(\omega\right)&=g\ix{i}^{\ast}\left(\omega\right)g\ix{j}\left(\omega\right)%
%         \end{split}%
%     \end{align}%

%     \begin{align}%
%         \begin{split}\label{eq:fourier}%
%             \varphi\left(\omega\right)=\frac{\vert g\ix{i,j}^{2}\vert}{\vert g\ix{i}g\ix{j}\vert}\,,\qquad\vartheta=\int_{-\infty}^{\infty}\varphi\left(\omega\right)\diff \omega%
%         \end{split}%
%     \end{align}%

% \subsubsection*{Coherence}%

%     Coherence.%

%     \begin{align}%
%         \begin{split}\label{eq:coherence_base}%
%             g\ix{i,T}\left(\omega\right)=&\mathcal{F}\left(P\ix{rad}\left(t\right)f\ix{T}\left(t\right)\right)\left(\omega\right)\,,\qquad f\ix{T}\left(t\right)=\left\{%
%             \begin{array}{ll}%
%                 1&,\,t\in T\\%
%                 0&,\,\text{else}%
%             \end{array}\right.\\%
%             g\ix{j,T}\left(\omega\right)=&\mathcal{F}\left(P\ix{pred}^{\left(1\right)}\left(t\right)f\ix{T}\left(t\right)\right)\left(\omega\right)%
%         \end{split}%
%     \end{align}%

%     \begin{align}%
%         \begin{split}\label{eq:coherence_adv}%
%             S\ix{kk}\left(\omega\right)=&\displaystyle\lim_{T\to\infty}\,\frac{1}{T}\,\vert g\ix{k,T}\left(\omega\right)\vert^{2}\,, \qquad k\in\{i\,,j\}\\%
%             S\ix{ij}\left(\omega\right)=&\displaystyle\lim_{T\to\infty}\frac{1}{T}\left(g\ix{i,T}^{\ast}\ast\left(\omega\right)g\ix{j,T}\left(\omega\right)\right)\\%
%             =&\int_{-\infty}^{\infty}\left(\displaystyle\lim_{T\to\infty}\frac{1}{T}\int_{-\infty}^{\infty}P\ix{rad}\left(t-\tau\right)f\ix{T}\left(t-\tau\right)\times\right.\\%
%             &\hspace{1.25cm}\left.P\ix{pred}^{\left(1\right)}\left(t\right)f\ix{T}\left(t\right)\diff t\right)\euler^{-i2\pi\omega\tau}\diff\tau%
%         \end{split}%
%     \end{align}%

%     \begin{align}%
%         \begin{split}\label{eq:coherence}%
%             \varphi\left(\omega\right)=\frac{\vert S\ix{ij}\left(\omega\right)\vert^{2}}{S\ix{ii}\left(\omega\right)S\ix{jj}\left(\omega\right)}\,,\qquad\vartheta=\displaystyle\int_{-\infty}^{\infty}\frac{\vert S\ix{ij}\left(\omega\right)\vert^{2}}{S\ix{ii}\left(\omega\right)S\ix{jj}\left(\omega\right)}\diff \omega%
%         \end{split}%
%     \end{align}%
%