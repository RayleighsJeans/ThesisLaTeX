%
\chapter{Feedback impact and line of sight sensitivity analysis}\label{chap:feedbackeval}%
%
    The presented results and findings of the previous chapter have posed various challenges towards a more detailed understanding of the impact and effects of the real time bolometer feedback system with the thermal gas injection as an actuator. As repeatedly marked throughout the discussion in \cref{chap:realtimefeedback}, three major questions remain regarding the optimisation and evaluation of the feedback mechanism and the corresponding evolution of plasma parameters.%
%
    \begin{enumerate}%
        \item[1.]{%
            Does an optimal set of lines of sight $S$ for the real time bolometer feedback system exist?}%
        \item[2.]{%
            What is the dominant contributor towards the line of sight selection sensitivity of the real time bolometer feedback system?}%
    \end{enumerate}%
%
    The answers to those questions are, but not necessarily have to be, very much entangled with each other. On one hand, findings regarding the optimisation of the line of sight selection $S$ inevitably will support or contradict evaluations towards the local sensitivity of the bolometer LOS geometry and the enclosed real time feedback. On the other hand, any correlation between the thermal gas valve activation and evolution of plasma characteristics will have to play a large role in the selection of the ideal set of lines of sight for the bolometer feedback.\\%
    This chapter will attempt to, as thoroughly as possible within the scope of this work, answer those questions, using a combination of different physical models and variations thereof, available sets of data from feedback experiments and simulated plasma profiles from the equilibrium transport code STRAHL. Following this proposal, a simple impurity seeding model will be explored and applied to experimental data from the bolometer diagnostics. An extensive evaluation of the quality of LOS selections $S$, also with respect to core plasma characteristics is enclosed. Finally, the equilibrium impurity transport code STRAHL is employed for the purposes of this work.%
%
    \section{Impurity Seeding Modelling}\label{sec:seedmodel}%
%
        \begin{figure}[t]%
            \centering%
            \includegraphics[width=0.8\textwidth]{%
                content/figures/chapter3/chambers/%
                20180920_049_power_feedback.pdf}%
            \caption{XP20180920.49:\\%
                Experimental data of $P\ix{rad}$ for both bolometer cameras and PID components of gas seeding valve controller. This is an excerpt between \SI{4.2}{\second} and \SI{8.2}{\second} where two consecutive seeding stages where applied and the radiation power responded accordingly.}\label{fig:chamber_expdata}%
        \end{figure}%
%
        The results presented in \cref{chap:realtimefeedback} have shown a direct connection between the change in plasma radiation and gas injected through the valves of the feedback system. During experiments with interferometer-measured electron density feedback, a continuous gas injection resulted in the increase and quick saturation of the radiative power loss at constant heating power. Throughout the establishment and configuration of the feedback procedure, this pattern has been repeated multiple times, with occasionally multiple saturated \textit{plateaus} in the radiation power. An example for this behaviour in a continuously fueled discharge can be found in \cref{fig:chamber_expdata}, which shows an excerpt from XP20180920.49. The left plot features the individual $P\ix{rad}$ measurements from HBC and VBC. The right side shows the gas valve activation, which is directly linked to the gas flow. Over an interval of \SI{4}{\second}, two changes in injection intensity are applied. The first appears just before \SI{5}{\second} and linearly increases until \SI{6.2}{\second}. This suggests that an equilibration takes place at a given flux of gas particles from the valves into the scrape-off layer. A slow increase in $P\ix{rad}$ might also imply gas buffering upon injection, suggesting the pumping timescale towards the wall to be much larger than those of the transport. With respect to the overarching questions of this chapter, this observation facilitates another possibility to answer the question of connection between radiation reaction and feedback.\\%
        The theoretical explorations made with the enclosed model(s) are intended to help to better understand and prepare for impurity seeding feedback experiments, guided by the plasma radiation as measured by the bolometer diagnostics. Those are in particular different to feedback controlled fueling experiments, like it has been the case for data discussed in \cref{chap:realtimefeedback}, and have to be approached differently. Here, one assumes to treat the impurity content $c\ix{imp}$ as a distinct population within seperate compartments (see below) in contrast to $n\ix{e, fuel}$ the plasma density, however ultimately in the greater picture the local emissivity remains very much dependent on both quantities, i.e. $\varepsilon\ix{rad}\propto n\ix{e}^{2}c\ix{imp}L\ix{Z}$.%

%
        \subsection{Two Chamber Model}\label{subsec:twochamb}%
%
            \begin{figure}[t]%
                \centering%
                \includegraphics[width=0.5\textwidth]{%
                    content/figures/chapter3/chambers/%
                    twochamber_scheme_crop_colors.pdf}%
                \caption{Two chamber model schematic, with $N\ix{p}$ the plasma chamber population and $N\ix{w}$ the wall chamber population, including their respective source and loss rates $\tau\ix{x,y}$ and $\Gamma\ix{s}$.}\label{fig:twochamber_schematic}%
            \end{figure}%
%
            In a first attempt to better understand the saturation process, a simple model is established for comparison. The schematic representation of this model can be found in \cref{fig:twochamber_schematic}. Here, the total amount of gas is split into two separate populations, i.e.the feedback injected impurity. In this scenario, the gas flow $\Gamma\ix{s}$ contributes directly to the population inside the plasma chamber, including the SOL, $N\ix{P}$, neglecting any scattering processes or losses. This, in turn, is connected through particle exchange with the second population of the reactor walls (chamber) $N\ix{W}$. The different exchange flows are noted by their respective rate coefficients $\tau\ix{p,w}$, for transport from the plasma to the wall, $\tau\ix{w,p}$ vice versa, and $\tau\ix{P}$ exclusive losses from the plasma. The first order differential equations for both populations are as follows:%
%
            \begin{align}
                \begin{split}\label{eq:twochamber_base}%
                    \frac{\diff}{\diff t}N\ix{p}=\dot{N}\ix{p}=&\Gamma\ix{s}+N\ix{w}\tau\ix{w,p}-N\ix{p}\left(\tau\ix{p,w}f+\tau\ix{p}\right)\,\,,\\%
                    \frac{\diff}{\diff t}N\ix{w}=\dot{N}\ix{w}=&N\ix{p}\tau\ix{p,w}f-N\ix{w}\tau\ix{w,p}\,\,.%
                \end{split}%
            \end{align}%
%
            The first order differential equations of the individual quantities are composed of the respective sources and losses, where the term denoting the plasma-to-wall exchange $\tau\ix{p,w}$ is modified by an additional parameter $f$ to satisfy the equilibrium requirement at the threshold population. Therefore, the equation for the wall chamber population can be solved for approaching $N\ix{w,lim}$ the limit value at equilibrium, where its first order temporal derivative from \cref{eq:twochamber_base} becomes:%
%
            \begin{align}%
                \begin{split}\label{eq:twochamber_limes}%
                \displaystyle\lim_{N\ix{w}\to N\ix{w,lim}}&\dot{N}\ix{w}\overset{!}{=}0=N\ix{p}\tau\ix{wp}f-N\ix{w,lim}\tau\ix{w,p}\\%
                &\rightarrow f=\frac{N\ix{w,lim}\tau\ix{w,p}}{N\ix{p}\tau\ix{p,w}}\,\,.%
                \end{split}%
            \end{align}%
%
            The final equation for the plasma and wall chamber population then become:%
%
            \begin{figure}[t]%
                \parbox{0.02\linewidth}{%
                    \textbf{(a)}}%
                \quad%
                \parbox{0.42\linewidth}{%
                    \includegraphics[width=\linewidth]{%
                        content/figures/chapter3/chambers/%
                        twochamber_gammas_scan.pdf}}%
                \quad%
                \parbox{0.42\linewidth}{%
                    \includegraphics[width=\linewidth]{%
                        content/figures/chapter3/chambers/%
                        twochamber_Nwlim_scan.pdf}}%
                \quad%
                \parbox{0.02\linewidth}{%
                    \textbf{(b)}}%
                \,\\%
                \parbox{0.02\linewidth}{%
                    \textbf{(c)}}%
                \quad%
                \parbox{0.42\linewidth}{%
                    \includegraphics[width=\linewidth]{%
                        content/figures/chapter3/chambers/%
                        twochamber_tauwp_scan.pdf}}%
                \quad%
                \parbox{0.42\linewidth}{%
                    \includegraphics[width=\linewidth]{%
                        content/figures/chapter3/chambers/%
                        twochamber_N_w0_scan.pdf}}%
                \quad%
                \parbox{0.02\linewidth}{%
                    \textbf{(d)}}%
                \caption{Parameter variation for two chamber model in \cref{eq:twochamber}. Included are variations of seeding rate \textbf{(a)} $\Gamma\ix{s}$ and limit parameter \textbf{(b)} $N\ix{w,lim}$, as well as wall-to-plasma loss rate \textbf{(c)} $\tau\ix{p}$ and initial population value \textbf{(d)} $N\ix{w,0}$, while the other model parameters have been kept constant.%
                % $\{\tau\ix{w,p}, N\ix{w,lim}, \tau\ix{p}, \Gamma\ix{s}, N\ix{p,0}, N\ix{w,0}\}=\{1, 10, 1.0, 10, 10, 10\}\,\left[\text{\SI{}{\arbitraryunit}}\right]$
                }\label{fig:twochamber_scan}%
            \end{figure}%
%
            \begin{align}%
                \begin{split}\label{eq:twochamber}%
                    \dot{N}\ix{w}=&\left(N\ix{w,lim}-N\ix{w}\right)\tau\ix{w,p}\\%
                    \dot{N}\ix{p}=&\Gamma\ix{s}+N\ix{w}\tau\ix{w,p}-N\ix{w,lim}\tau\ix{w,p}N\ix{p}-N\ix{p}\tau\ix{p}\,\,.%
                \end{split}%
            \end{align}%
%
            Note that the above equations are no longer a function of the parameter $\tau\ix{p,w}$.\\%
            In order to understand the behaviour of the above two chamber model in \cref{eq:twochamber}, the included parameters are varied independently and their individual impact on the development on the plasma and wall chamber populations recorded. A straightforward explicit first order differential equation integration method (\textit{4th order Runge-Kutta method}\cite{WikiRungeKutta}) is used. The results can be found in \cref{fig:twochamber_scan}. For each variation of one variable, the evolution of the populations have been calculated for an interval of \SI{10}{\second} each, while all other parameters have been kept constant. If the respective variable is not changed, the initial values are - all in arbitrary units (\SI{}{\arbitraryunit}) or \SI{}{\arbitraryunit\per\second} - as follows: $\tau\ix{w,p}=$\SI{1}{\arbitraryunit\per\second}, $N\ix{w,lim}=$\SI{10}{\arbitraryunit}, $\tau\ix{p}=$\SI{1}{\arbitraryunit\per\second}, $\Gamma\ix{s}=$\SI{1}{\arbitraryunit\per\second}, $N\ix{p,0}=$\SI{10}{\arbitraryunit}, $N\ix{w,0}=$\SI{10}{\arbitraryunit} The first plot in \cref{fig:twochamber_scan}:(a) shows the variation of the injection flow $\Gamma\ix{S}$ between $\left\{0, 2\right\}$\SI{}{\arbitraryunit} Recorded results for both populations are displayed for each injection intensity, while the corresponding lines are drawn also with a colour scale between grey and red for $N\ix{p}$ and blue for $N\ix{w}$. With increasing $\Gamma\ix{S}$, the equilibrium plasma chamber population increases proportionally, while $N\ix{w}$ remains constant at its initial value. The tipping point from increasing to decreasing plasma chamber population is evidently reached at $\Gamma\ix{S}=\tau\ix{p}$ - for $\tau\ix{w,p}=\tau\ix{p,w}$) -, i.e. where the absolute loss from the plasma chamber population is balanced by the input gas flow. At very small $\Gamma\ix{s}$, $N\ix{p}$ even becomes negative at \SI{2}{\second} for this combination of parameters, providing an invalid solution for \cref{eq:twochamber}. The second plot in \cref{fig:twochamber_scan}:(b) shows the variation of the limit parameter $N\ix{w,lim}$ between $\left\{0, 20\right\}$\SI{}{\arbitraryunit} For smaller $N\ix{w,lim}$, $N\ix{p}$ increases within the first \SI{4}{\second} before levelling off at \SI{9}{\arbitraryunit}, while for larger limit values the plasma chamber population decreases and then approaches its equilibrium. The wall chamber population develops accordingly, where larger values of $N\ix{w,lim}$ lead to equally larger equilibria and vice versa. The variation of the wall-to-plasma loss rate $\tau\ix{w,p}$ between $\left\{0, 2\right\}$\,a.u is presented in the third plot in \cref{fig:twochamber_scan}:(c). Finally, the last plot in \cref{fig:twochamber_scan}:(d) shows the variation of the initial wall chamber population $N\ix{w,0}$ between $\left\{0, 20\right\}$\SI{}{\arbitraryunit} A similar, but inverted behaviour of $N\ix{p}$ as in plot (b) can be seen, since larger $N\ix{w,0}$ correspond to initially increasing plasma chamber populations and the other way around. Again, $N\ix{p}$ equilibrates at \SI{9}{\arbitraryunit} while the wall chamber population levels off at \SI{10}{\arbitraryunit}, the limit value $N\ix{w,lim}$ here, after increasing or decreasing from its respective smaller or larger starting point. With respect to the information presented in the plots of \cref{fig:twochamber_scan}, variations of the not previously presented and hence remaining parameters $N\ix{p,0}$, $\tau\ix{p,w}$ and $\tau\ix{p}$ result in no additional insight into the evolution of the model. Changes in the initial plasma chamber population mimic the behaviour of $N\ix{w}$ when $N\ix{w,0}$ is varied, while $\tau\ix{p}$ determines the boundaries of $N\ix{p}$ in a trivial way. If the plasma chamber loss vanishes, its respective population does not equilibrate and becomes infinite. For smaller $\tau\ix{p}$ the equilibrium of $N\ix{p}$ increases and vice versa.\\%
            After introducing a simple two-chamber model in \cref{eq:twochamber} and benchmarking it, one can now apply it to the experimental data presented in \cref{fig:chamber_expdata} in order to examine its suitability for reproducing the observed behaviour of the plasma radiation. The accompanying feedback activation data show two distinct injections steps, hence the model is applied similarly. A simple least-square fit procedure is employed to find the best parameter set representing the first and successive second stage. Only positive parameter solutions are allowed, since all particle exchange directions are accounted for by individual coefficients. The resulting populations $N\ix{p}$ and $N\ix{w}$ of the initial injection step up until \SI{2.2}{\second} of the feedback application are used as input for the evolution of the second stage for another \SI{2}{\second}. This fit is conducted assuming an injection rate of 200\% intensity of the priors. A presentation of this fit over the normalized experimental data, next to the temporal evolution of the individual populations can be found \cref{fig:twochamber_twostage}. The corresponding parameters are noted below.\\%
%
        \begin{figure}[t]%
            \centering%
            \includegraphics[width=0.8\textwidth]{%
                content/figures/chapter3/chambers/%
                chamber_simulation_two_stage.pdf}%
            \caption{Application of model \cref{eq:twochamber} on experimental data from XP20180920.49 in-between \SI{4.2}{\second} and \SI{8.2}{\second}. The respective parameters of this solution are $\Gamma\ix{s}=$\SI{0.648}{\arbitraryunit\per\second}, $\tau\ix{w,p}=$\SI{0.1}{\arbitraryunit\per\second}, $N\ix{w,lim}=$\SI{1.388}{\arbitraryunit}, $\tau\ix{p}=$\SI{0.436}{\arbitraryunit\per\second}, $N\ix{p,0}=$\SI{0.011}{\arbitraryunit} and $N\ix{w,0}=$\SI{0.075}{\arbitraryunit}}\label{fig:twochamber_twostage}%
        \end{figure}%
%
        This simple model is evidently very well capable of reproducing the behaviour of $P\ix{rad}$ in this example, though under the vastly simplifying assumption of direct proportionality between $P\ix{rad}\propto N\ix{p}$, which was ultimately motivated by $N\mathrel{\hat{=}}n\ix{imp}\propto c\ix{imp}\propto P\ix{rad}$. The wall chamber population has a very small positive slope, which is reflected by its comparatively large capacity ($N\ix{w,lim}$) of 1.388 and $\tau\ix{w,p}$:$\Gamma\ix{s}-\tau\ix{p}$. The ratio between the net growth of both chambers, $\dot{N}\ix{p}\sim\Gamma\ix{s}-\tau\ix{p}$ and $\dot{N}\ix{w}\sim\tau\ix{w,p}$ determines the development of the populations. Initial values $N\ix{p,0}$ and $N\ix{w,0}$ are negligible and can be accounted for by remaining gas in the discharge environment from previous experiments.\\%
        Although this trivial two-chamber model offers only a superficial understanding of the behaviour of the plasma radiation evolution in relation to the feedback gas injection, it is of significant interest explicitly for its simple approach and ability to reproduce the experimental data. It also shows that a direct proportionality between the feedback activation intensity - its length, strength and frequency - and its impact on plasma radiation is not obvious or trivially explored and that time constants might play a more important and complex role here. The above calculations are also capable of reproducing the behaviour in presence of a second feedback stage for the same set of parameters and with continuity at the transition. Therefore, the model and its respective implications, i.e. two separate compartments with finite capacity in exchange with each other and a single source and sink, provide a possible explanation for the correlation between feedback gas injection and plasma radiation from the initially posed questions.%
%
        \subsection{Three Chamber Model}\label{subsec:threechamb}%
%
            \begin{figure}[t]%
                \centering%
                \includegraphics[width=0.7\textwidth]{%
                    content/figures/chapter3/chambers/%
                    threechamber_scheme_crop_colors.pdf}%
                \caption{Three chamber model schematic, with the plasma chamber population $N\ix{p}$, $N\ix{w}$ the wall chamber population and an additional chamber for a scrape-off layer population $N\ix{s}$. The seeding location has changed to the SOL, including the loss channel from there. There is also a second pair of respective source and loss rate terms between the individual model chambers.}\label{fig:threechamber_schematic}%
            \end{figure}%
%
            The three-chamber model is a simple extension of the two-chamber model to the case of three separate chambers and is based on the latter. A scrape-off layer chamber (population) $N\ix{s}$ is added in-between the wall and plasma chamber as an additional buffer, which in this case features the single source and sink like the plasma chamber did before. This is assumed to be a more exact model for the feedback gas reservoirs in a discharge. A schematic can be found in \cref{fig:threechamber_schematic}. Plasma and SOL, as well as SOL and wall chamber are connected through two exchange terms and accompanying coefficients for both particle flow directions respectively. Source and sink are again represented by $\Gamma\ix{s}$ and $\tau\ix{s}$ (from $N\ix{s}$). Here, the \textit{finite capacity} boundary condition applies to the plasma and wall chamber, assuming absolute limits to the impurity content here respectively: the wall may be saturated with impurities, while there is a critical $c\ix{imp}$ in the plasma beyond which it is unstable or extinguished. This model will also be applied to the experimental data in \cref{fig:chamber_expdata} in same manner as before.\\%
            Given \cref{fig:threechamber_schematic}, the first order differential equations describing the populations in this model are as follows:%
%
            \begin{align}%
                \begin{split}\label{eq:threechamber_base}%
                    \dot{N}\ix{w}=&N\ix{s}\tau\ix{s,w}f\ix{w}-N\ix{w}\tau\ix{w,s}\,\,,\\%
                    \dot{N}\ix{s}=&\Gamma\ix{s}+N\ix{p}\tau\ix{p,s}f\ix{p}-N\ix{s}\left(\tau\ix{s,w}f\ix{w}+\tau\ix{s,p}+\tau\ix{s}\right)\,\,,\\%
                    \dot{N}\ix{p}=&N\ix{s}\tau\ix{s,p}-N\ix{p}\tau\ix{p,s}f\ix{p}\,\,.%
                \end{split}%
            \end{align}%
%
            Again, the respective source and loss terms of the scrape-off layer chamber are modified by two individual parameters $f\ix{w}$ and $f\ix{p}$ to satisfy the equilibrium condition of before, which is applied here to the wall and plasma chamber populations, $N\ix{w}$ and $N\ix{p}$. More specifically, when approaching equilibrium populations $N\ix{p,lim}$ and $N\ix{w,lim}$, the respective derivative should be zero.%
%
            \begin{align}%
                \begin{split}\label{eq:threechamber_limes}%
                    \displaystyle\lim_{N\ix{w}\to N\ix{w,lim}}&\dot{N}\ix{w}\overset{!}{=}0=N\ix{s}\tau\ix{s,w}f\ix{w}-N\ix{w,lim}\tau\ix{w,s}\\%
                    \displaystyle\lim_{N\ix{p}\to N\ix{p,lim}}&\dot{N}\ix{p}\overset{!}{=}0=N\ix{s}\tau\ix{s,p}-N\ix{p,lim}\tau\ix{p,s}f\ix{p}\\%
                    &\rightarrow f\ix{w}=\frac{N\ix{w,lim}\tau\ix{w,s}}{N\ix{s}\tau\ix{s,w}},\qquad f\ix{p}=\frac{N\ix{s}\tau\ix{s,p}}{N\ix{p,lim}\tau\ix{p,s}}%
                \end{split}%
            \end{align}%
%
            Finally, the equations for plasma, scrape-off layer and wall chamber populations in this simple model become:%
%
            \begin{align}%
                \begin{split}\label{eq:threechamber}%
                    \dot{N}\ix{w}=&\left(N\ix{w,lim}-N\ix{w}\right)\tau\ix{w,s}\,\,,\\%
                    \dot{N}\ix{s}=&\Gamma\ix{s}+\frac{N\ix{p}N\ix{s}\tau\ix{s,p}}{N\ix{p,lim}}-N\ix{w,lim}\tau\ix{w,s}-N\ix{s}\left(\tau\ix{s,p}+\tau\ix{s}\right)\,\,,\\%
                    \dot{N}\ix{p}=&N\ix{s}\tau\ix{s,p}-\frac{N\ix{p}N\ix{s}}{N\ix{p,lim}}\,\,.%
                \end{split}%
            \end{align}%
%
            Note that the final set of equations no longer contains terms of $\tau\ix{s,w}$ and $\tau\ix{p,s}$, reducing the set of parameters to six and increasing it by two from the previous model.\\%
            Similar to before, the model is benchmarked with respect to the influence of the different parameters on the evolution of the individual chamber populations. This is done using the \textit{4th order Runge-Kutta} integration method for first order ODEs. The parameters are varied in the same way as before, whereas the initial values are set to $\tau\ix{s,p}=$\SI{1}{\arbitraryunit\per\second}, $N\ix{p,lim}=$\SI{7}{\arbitraryunit}, $\Gamma\ix{s}=$\SI{20}{\arbitraryunit\per\second}, $N\ix{w,lim}=$\SI{5}{\arbitraryunit}, $\tau\ix{w,s}=$\SI{1}{\arbitraryunit\per\second}, $\tau\ix{s}=$\SI{1}{\arbitraryunit\per\second}, $N\ix{p,0}=0$, $N\ix{s,0}=$\SI{1}{\arbitraryunit} and $N\ix{w,0}=0$ if not stated otherwise. The results can be found in \cref{fig:threechamber_scan}. Each model is calculated for an interval of \SI{10}{\second}, while one of the parameters is changed and the rest is kept constant. Presentation and schematic of the corresponding figures are the same, with the only difference being the additional lines for $N\ix{s}$.\\%
            The first plot in the top left, \cref{fig:threechamber_scan}:(a) features the temporal evolution of the three populations at varying values of $N\ix{w,lim}$ between $\left\{0, 10\right\}$\SI{}{\arbitraryunit} In the top right, \cref{fig:threechamber_scan}:(b) shows the results for variations of $N\ix{p,lim}\in\left\{0, 20\right\}$\SI{}{\arbitraryunit} Higher values of $N\ix{p,lim}$ correspond to smaller SOL chamber equilibria, while the wall chamber capacity $N\ix{w,im}$ is not affected by changes in $N\ix{p,lim}$ and remains at \SI{5}{\arbitraryunit} For constant exchange coefficients, source and loss rates, the plasma chamber population takes longer to reach its maximum capacity the higher the value of $N\ix{p,lim}$. In \cref{fig:threechamber_scan}:(c), the evolution of the populations for SOL-to-plasma exchange rates $\tau\ix{s,p}$ between $\left\{0, 2\right\}$\,\SI{}{\arbitraryunit\per\second} is shown. The lower the value of $\tau\ix{s,p}$, the longer it takes the plasma chamber population to reach is capacity $N\ix{p,lim}$, though itself is not affected by it. The evolution of $N\ix{w}$ does not change. In \cref{fig:threechamber_scan}:(d), results for changes in the gas feeding rate $\Gamma\ix{s}$ over a large interval of $\left\{15, 60\right\}$\,\SI{}{\arbitraryunit\per\second} are presented. For the given set of parameters, the SOL chamber population increases strongly with larger values of $\Gamma\ix{s}$, while its equilibration latency stays the same throughout the spectrum. Again, $N\ix{w}$ is not perturbed by the change in gas injection rate. For smaller $\Gamma\ix{s}$ the process of reaching the steady state $N\ix{p,lim}$ takes longer, though the equilibrium value stays constant. \autoref{fig:threechamber_scan}:(e) contains variations of the wall-to-SOL chamber loss rate $\tau\ix{w,s}$ in the interval $\left\{0, 2\right\}$\,\SI{}{\arbitraryunit\per\second} Instinctively, one would expect larger wall-to-SOL loss rates to lead to higher equilibria in $N\ix{s}$ for a given set of $N\ix{w,lim}$, $\Gamma\ix{s}$ and $\tau\ix{s}$. However, the presented results show otherwise, i.e. increasing $\tau\ix{w,s}$ causes smaller plateaus of SOL chamber population, which is indeed in agreement with the model described by \cref{eq:threechamber}. Finally, \cref{fig:threechamber_scan}:(f) presents results for the variation of $\tau\ix{s}$ in between $\left\{0, 2\right\}$. With $\tau\ix{s}\rightarrow0$, the equilibrium SOL population increases until the system evidently becomes unbounded at $\tau\ix{s}=0$, since there is no more exhaust of gas particles.\\%
%
            \begin{figure}[t]%
                \parbox{0.02\linewidth}{%
                    \textbf{(a)}}%
                \quad%
                \parbox{0.42\linewidth}{%
                    \includegraphics[width=\linewidth]{%
                        content/figures/chapter3/chambers/%
                        threechamber_Nwlim_scan.pdf}}%
                \quad%
                \parbox{0.42\linewidth}{%
                    \includegraphics[width=\linewidth]{%
                        content/figures/chapter3/chambers/%
                        threechamber_Nplim_scan.pdf}}%
                \quad%
                \parbox{0.02\linewidth}{%
                    \textbf{(b)}}%
                \,\\%
                \parbox{0.02\linewidth}{%
                    \textbf{(c)}}%
                \quad%
                \parbox{0.42\linewidth}{%
                    \includegraphics[width=\linewidth]{%
                        content/figures/chapter3/chambers/%
                        threechamber_tausp_scan.pdf}}%
                \quad%
                \parbox{0.42\linewidth}{%
                    \includegraphics[width=\linewidth]{%
                        content/figures/chapter3/chambers/%
                        threechamber_gammas_scan.pdf}}%
                \quad%
                \parbox{0.02\linewidth}{%
                    \textbf{(d)}}%
                \,\\%
                \parbox{0.02\linewidth}{%
                    \textbf{(e)}}%
                \quad%
                \parbox{0.42\linewidth}{%
                    \includegraphics[width=\linewidth]{%
                        content/figures/chapter3/chambers/%
                        threechamber_tauws_scan.pdf}}%
                \quad%
                \parbox{0.42\linewidth}{%
                    \includegraphics[width=\linewidth]{%
                        content/figures/chapter3/chambers/%
                        threechamber_taus_scan.pdf}}%
                \quad%
                \parbox{0.02\linewidth}{%
                    \textbf{(f)}}%
                \caption{Parameter variation for three chamber model in \cref{eq:twochamber}. Included are variations of limit parameters \textbf{(a)} $N\ix{w,lim}$ and \textbf{(b)} $N\ix{p,lim}$,  plasma source and feedback injection rates \textbf{(c)} $\tau\ix{s,p}$ and $\Gamma\ix{s}$, as well as \textbf{(e)} $\tau\ix{w,s}$ source and \textbf{(f)} $\tau\ix{s}$ loss rates of the plasma scrape-off layer, while other model parameters have been kept constant.%
                % $\{\tau\ix{s,p}, N\ix{p,lim}, \tau\ix{p,s}, \Gamma\ix{s}, N\ix{w,lim}, \tau\ix{w,s}, \tau\ix{s}, N\ix{p,0}, N\ix{s,0}, N\ix{w,0}\}=\{1, 10, 0.7, 20, 5, 1, 1, 0, 1, 0\}\,\left[\text{\SI{}{\arbitraryunit}}\right]$
                }\label{fig:threechamber_scan}%
            \end{figure}%
%
            Of particular interest here is the influence of $\tau\ix{s}$, $\tau\ix{s,p}$ and their ratio, as both impact the equilibrated SOL population $N\ix{s}$ and hence the overall ratio between the chambers. Lower SOL loss  and plasma exchange rates correlate, as expected, shifting the contents so that the middle chamber is enriched without depletion of the others. As was discussed earlier, boundary conditions are only considered for the wall and plasma chamber, making the SOL an open container in this scenario for the introduced species. Assuming this to be an extrinsic impurity, strongly radiating in the core and edge, this can also be used to define a setting for a brighter SOL that may support radiative cooling in front of the target. Also, scaling in the wall inventory, i.e. the equilibrated $N\ix{w,lim}$, has great influence on the SOL population, essentially siphoning off particles for constant flow and loss rates.\\%
            The benchmark of the model and its accompanying parameters is thereby completed. The addition of the scrape-off layer chamber, which also incorporates the source and sink of the system, was found to act as a buffer for the injected feedback gas and an exchange reservoir of the remaining two plasma and wall chambers. Evidently, $N\ix{w}$ is only affected by changes in its respective chamber capacity and $\tau\ix{w,s}$, not however by perturbations in the plasma or SOL chamber populations. The behaviour of $N\ix{p}$ is as expected, where larger transfer coefficients in this direction and/or higher capacities in the neighbouring SOL chamber lead to accelerated population of the plasma chamber.\\%
%
            \begin{figure}[t]%
                \centering%
                \includegraphics[width=0.8\textwidth]{%
                    content/figures/chapter3/chambers/%
                    chamber_simulation_two_stage_threechamber.pdf}%
                \caption{Application of the model from \cref{eq:threechamber} on the experimental data of XP20180920.49 in-between \SI{4.2}{\second} and \SI{8.2}{\second}. The final set of parameters from a least-square fit is $\tau\ix{s,p}=$\,\SI{0.1}{\arbitraryunit\per\second}, $N\ix{p,lim}=$\,\SI{17.9}{\arbitraryunit}, $\tau\ix{p,s}=$\,\SI{14.931}{\arbitraryunit\per\second}, $\Gamma\ix{s}=$\,\SI{18.08}{\arbitraryunit\per\second},  $N\ix{w,lim}=$\,\SI{2.579}{\arbitraryunit}, $\tau\ix{w,s}=$\,\SI{2.561}{\arbitraryunit\per\second}, $\tau\ix{s}=$\,\SI{2.772}{\arbitraryunit\per\second}, $N\ix{p,0}=0$, $N\ix{s,0}=0$, $N\ix{w,0}=1$. The weighting factor is $f=10.957$.}\label{fig:threechamber_twostage}%
                %
            \end{figure}%
%
            With its introduction in \cref{eq:threechamber} and the corresponding benchmark in \cref{fig:threechamber_scan}, the three chamber model can now also be applied to reproduce the results of \cref{fig:chamber_expdata} with the same approach as in \cref{fig:twochamber_twostage}. Again, a least-square fit is used to find the best parameter solution for $N\ix{p}+fN\ix{s}\propto P\ix{rad}$ from the distinct two stage injection. The difference to the prior fit is that the scrape-off layer is now incorporated into the modelling of the plasma radiation. A weighting factor $f$ between the two contributions is also included in the fit. Only a set of positive parameters is allowed. The model is applied to the first \SI{2.2}{\second} of the injection, after which the resulting populations are used as input for further calculations beyond that point in time. The fit is performed to find a set of parameters that represents the first and second feedback stage best while satisfying the above requirement.\\%
            \autoref{fig:threechamber_twostage} shows the results from the fit. The left plot presents the original experimental data with the superimposed sum of $N\ix{p}=fN\ix{s}$. A line of the fit for the first stage only is also included, verifying the results for the complete data set. On the right the evolution of the individual three populations can be found, scaled up by separate factors for better visibility. From the corresponding display scaling factors and the value of $f\sim11$, one can immediately determine the dominating contribution of this model solution to the representation of $P\ix{rad}$. The final input of $N\ix{p}$ to the plasma radiation is more than $10^{2}$ times smaller than that of the SOL chamber population. The individual populations of $N\ix{w}$ and $N\ix{p}$ are significantly smaller than $N\ix{s}$. The content of the wall chamber is a factor of $10^{2}$ smaller than those of the others and can be neglected. Furthermore, its respective capacity and exchange rate - read \textit{loss rate} - $\tau\ix{w,s}$ to $N\ix{s}$ are small and of almost the same magnitude, which means particles transported into that chamber immediately are lost to the neighbouring SOL. The plasma chamber population is about $10$ times larger and features an increased positive slope compared to the latter. This is the result of the large ratio between $\tau\ix{p,s}$:$\tau\ix{s,p}\approx150$, which also greatly impairs this population to grow. By about a factor of ten, the scrape-off layer chamber population is the largest in this model. For the given injection intensity and model loss rate, the SOL is beginning to saturate towards the end of this first step, when the term $\propto N\ix{s}\left(\tau\ix{s}+\tau\ix{s,p}\right)$ becomes larger than the net gain of $N\ix{s}$ in \cref{eq:threechamber}.\\%
            The presented results in \cref{fig:threechamber_twostage} are consistent with the measurements in \cref{fig:chamber_expdata} and results in \cref{fig:twochamber_twostage}. They underline the previous findings and show that the extended model is also capable of reproducing the experimental data through $P\ix{rad}\propto N\ix{p}+fN\ix{s}$. Furthermore, it offers a deeper understanding of the behaviour of injected gas during feedback experiments. By extending the prior set of equations and adding a scrape-off layer compartment, into which the gas is injected, this model is able to  mimic the effects of radiative cooling through seeding of impurities, like it was pursued by the underlying experimental setup. The results from the fitting procedure of $P\ix{rad}\propto N\ix{p}+fN\ix{s}$ show a strong correlation between the SOL impurity content and the radiation power loss. On one hand, the contribution of $N\ix{s}$ to $P\ix{rad}$ is, comparatively, ten times larger than that of the plasma chamber population. And on the other hand, the absolute content of the SOL is largest among the three chambers in this model by at least a factor of ten. This results in a very dominating contribution of that species to the radiation power loss prediction. In \cref{subsec:densityfeedback} and especially the C-III emission measurements of \cref{fig:20180920.32_cIII}, it was seen that, with higher gas injection intensity, the radiation in the edge of the plasma and SOL increased, suggesting an accumulation of externally seeded impurities in that region. Although these findings fit the conclusions from the respective experimental results rather well, there still remain discrepancies either between both the corresponding models and measurement data. In the extended three chamber model, no equilibration process is seen in the evolution of the final parameter set of \cref{fig:threechamber_twostage}, contrary to the results in \cref{fig:twochamber_twostage} or \cref{fig:chamber_expdata}. This behaviour is expected of the wall chamber population, where the physical plasma boundary, i.e. a steel wall, is able to retain a certain amount of impurity gases through kinetic or chemical deposition\cite{Winter1992}. Similarly, there exists a maximum, sustainable impurity concentration of the core plasma. Beyond this level of saturation, a radiative collapse of the discharge through removal of plasma stored energy by radiation is very likely\cite{Zhang2021_2}.\\%
%
            \newline%
            This concludes the analysis of the saturation behaviour in radiation power loss during gas injection plasma feedback experiments. Simple two and three chamber models have been established and explored, through which the impact of extrinsically seeded impurities on the plasma radiation can be better understood. Applying said models to measurement data from the bolometer diagnostic has yielded plausible results which coincide with experimental data. With regard to the initially posed questions at the beginning of this \cref{chap:feedbackeval}, this partially answers \textit{item 1.}, since a strong correlation between the impurity contents of the scrape-off layer and plasma edge from extrinsic seeding and plasma radiation was found.%
%
    \section{Line of Sight Sensitivity Evaluation}\label{sec:evalmetrics}%
%
        A key feature and issue at the same time of the real-time bolometer feedback system is the extrapolation and application of a limited line of sight selection $S$ to the control variable $P\ix{rad,S}\overset{!}{=}P\ix{pred}$ - in this section, only the formerly introduced $P^{\text{(1)}}\ix{pred}$ from \cref{eq:prediction} is of interest, hence the simplification. To approach this challenge and adequately evaluate the capabilities of the prediction, a metric for measuring the quality of the individual LOS selection has to be established and thoroughly exercised across a wide selection of experimental data sets that ideally represent as many different plasma scenarios as possible. This in itself poses another problem, in which a single such metric might not be sufficient to address this task. For example, most noteably in \cref{sec:feedbackachieve}, the latency and frequency between the feedback injection and reaction of the plasma radiation were found to play a large role in the success of the experiment. A simple, normalized deviation between the prediction and full set $P\ix{rad}$ across the experiment time interval is not capable of representing the difference in spectral densities between the two quantities. Hence, a variation of different metrics will be adequately introduced and applied individually on the same set of experimental data to thoroughly evaluate the impact of LOS selection for feedback control quality.%
%
        \subsection{Evaluation Metric}\label{subsec:eval_metric}%
%
            In order to establish a unified terminology for all metric variations for the sake of simplifying equations, references and plots, first a basic set of equations and maps have to been introduced. The underlying prediction for the plasma radiation power loss, as measured by the bolometer diagnostic, for a line of sight selection $S$ shall be the same as previously introduced in \cref{eq:prediction}:%
%
            \begin{align}%
                P\ix{pred}\overset{!}{=}P\ix{S}=\frac{V\ix{P,tor}}{V\ix{S}}\sum_{M}^{S}\frac{P_{M}V_{M}}{K_{M}}\,\,,\qquad S\subset\left(S\ix{HBC},S\ix{VBC}\right)\,.\nonumber%
            \end{align}%
%
            Two functions have to be defined next. First, one providing a metric for the quality of the prediction relative to the \textit{full LOS set} $P\ix{rad}$. The domain of this function is the time interval of the experiment. Second, a function that transforms the result of the latter from the temporal or frequency domain to a single number. This value then represents the quality of the prediction for the LOS set $S$ with respect to the corresponding metric. The following are therefore introduced:%
%
            \begin{empheq}[box=\fbox]{align}%
                \begin{split}\label{eq:metric}%
                    \varphi\coloneqq &f\left(t,S,P\ix{rad}\right)\colon\mathbb{N}^{m}\times\mathbb{R}\to\mathbb{R}\\%
                    \vartheta\coloneqq &h\left(S,\,P\ix{rad},\varphi\right)\colon\mathbb{N}^{m}\times\mathbb{R}\to\mathbb{R}%
                \end{split}%
            \end{empheq}%
%
            For simplification, from here on $\varphi=\varphi\left(t\right)$ will be called metric and $\vartheta=\vartheta\left(S\right)$ map to the prediction quality. The evaluation process is repeated for all following metrics in the same way. In the beginning, before beginning the calculations for the first $\varphi\left(t\right)$, a large collection of LOS sets is established. These contain combinations of three, five and seven channels individually, where each is constructed using all possible permutations of the respective three, five or seven subset ranges of channels from the horizontal \textit{or} vertical bolometer camera. For example, for LOS sets of three channels from the HBC, permutations of from ranges $\left\{1, \dots, 10\right\}$, $\left\{12, \dots, 21\right\}$ and $\left\{22, \dots, 30\right\}$ resulted in a total collection of \textit{nine hundred} different sets $S$. The individual subset ranges are also later easily deductible in the plots presenting the results, i.e. \cref{fig:results_weighted_deviation}. The selection of underlying experiments for conducting the feedback evaluation is the entire set of feedback controlled discharges from the previous campaign. For every provided \textit{full} $P\ix{rad}$ of every individual experiment, all $P\ix{pred}=P\ix{pred}\left(S\right)$ of the corresponding permutations $S$ are calculated using \cref{eq:prediction}. Applying those results to \cref{eq:metric} yields the prediction qualities $\vartheta\left(S\right)$ for the corresponding permutations of LOS and incorporated metric $\varphi$. Finally, evaluation of this data set for each channel $n$ and combination size $m$ produces an average \textit{line of sight sensitivity in plasma feedback scenarios} $\overline{\vartheta}^{\left(\text{m}\right)}\ix{n}$.%
%
            \begin{align}%
                \begin{split}\label{eq:sensitivity}%
                    \overline{\vartheta}\ix{n}^{\left(\text{m}\right)}=\frac{1}{N^{\left(\text{n,m}\right)}}\sum_{S}\vartheta\left(S^{\left(\text{m}\right)}\right)\\%
                    N^{\left(\text{n,m}\right)}=\sum_{S}\delta\ix{n}\left(S^{\left(\text{m}\right)}\right)\\%
                    \delta\ix{n}\left(S\right)=\left\{%
                    \begin{array}{ll}%
                        1,&n\in S\\%
                        0,&\text{else}%
                    \end{array}\right.%
                \end{split}%
            \end{align}%
%
            Here, $N^{\left(\text{n,m}\right)}$ is the amount of permutations of size $m$ that feature channel number $n$. The left-hand expression of the first equation in \cref{eq:sensitivity} can be seen as the contribution of channel $n$ to the expected quality of the prediction with selection size $m$. In other words, $\vartheta\ix{n}^{\left(\text{m}\right)}$ describes the sensitivity of channel $n$ to local plasma radiation emissions and therefore significance to the representation of the total power loss $P\ix{rad}$.\\%
            Finally, after calculating the quality using \cref{eq:metric} for the given metric, the full set of experimental data with all permutations of size 3, 5, and 7 channels individually, the evaluation is concluded. This strategy is applied to every following metric $\varphi\left(t\right)$ and map $\vartheta\left(S\right)$ and their results are examined and compared with each other.%
%
            \subsubsection*{Weighted Deviation}%
%
                The first metric introduced in \cref{eq:weighted_deviation} will be referred to as a \textit{weighted mean deviation}. This map is defined by the ratio between prediction $P\ix{pred}$ and the full data set $P\ix{rad}$. The closer to the expected value and therefore more accurate the prediction, the larger the value $\varphi\left(t\right)$. However, the metric only yields values in the temporal interval of the experiment and for predictions with deviations to $P\ix{rad}$ smaller than the respective reference. This punishes LOS sets $S$ that greatly overestimate $P\ix{rad}$ by not producing a contribution in such cases. Hence, assuming $P\ix{rad}>P\ix{pred}>0$, the metric is positive-definite, and or more specifically $\varphi\in\left(0, 1\right)$. The quality $\vartheta$ of LOS selection $S$ and corresponding prediction is calculated by integrating the metric across the experiment interval and normalizing by $T\ix{stop}-T\ix{start}$.%
%
                \begin{align}
                    \begin{split}\label{eq:weighted_deviation}%
                        \varphi\left(t\right)=&\left\{\begin{array}{ll}%
                            1-\frac{\left\|P\ix{pred}-P\ix{rad}\right\|\left(t\right)}{P\ix{rad}\left(t\right)}&,\,\left(\left\|P\ix{pred}-P\ix{rad}\right\|<P\ix{pred}\right)\land\\%
                            &\qquad\left(T\ix{start}<t<T\ix{stop}\right)\\%
                            0&,\,\text{else}%
                        \end{array}\right.\\%
                        \vartheta=&\frac{1}{T\ix{stop}-T\ix{start}}\int_{T\ix{start}}^{T\ix{stop}}\varphi \left(t\right)\diff t%
                    \end{split}%
                \end{align}%
%
                The corresponding results for \cref{eq:weighted_deviation} are presented in \cref{fig:weighted_deviation}. On the left-hand side, an example of how such a metric for a LOS selection of \textit{three channels} and experiment XP20181010.32 looks like can be seen. Also indicated here is a dotted black line, which represents the \textit{norm} of $\vartheta\left(t\right)$, i.e. the metric of a selection $S$ that yields the largest possible quality $\vartheta\left(S\right)$ or for $P\ix{pred}=P\ix{rad}$. The calculated value of $\vartheta\left(S\right)$ with respect to $P\ix{rad,HBC}$ is noted with 0.846. The left ordinate corresponds to the values of radiation power loss, while the right refers to the quality of the LOS set in \SI{}{\arbitraryunit}. The right-hand side plot shows the overall results $\vartheta$ for \cref{eq:weighted_deviation} for the same experiment as on the left, number XP20181010.32, and all LOS selections with three channels, $m=3$.\\%
%
                \begin{figure}[t]%
                    \centering%
                    \begin{subfigure}{0.47\textwidth}%
                        \includegraphics[width=\textwidth]{%
                            content/figures/chapter3/training/best_chans/%
                            best_chans_wghtd_dev_C[_8_7_23].pdf}%
                        \caption{}%
                    \end{subfigure}%
                    \hfill%
                    \begin{subfigure}{0.47\textwidth}%
                        \includegraphics[width=\textwidth]{%
                            content/figures/chapter3/training/%
                            weighted_deviation_3_HBCm_combinations.pdf}%
                        \caption{}%
                    \end{subfigure}%
                    \caption{Example of how the quality of the prediction for the \textit{weighted deviation} metric is calculated for the previously discussed XP20181010.32. \textbf{(a)} Comparison of traces $\varphi\left(t\right)$, calculated using \cref{eq:weighted_deviation}, $P\ix{rad}$ and $P\ix{pred}^{\left(1\right)}$ for a subset of three channels of the HBC. \textbf{(b)} Overview of 900 different combinations of three channel subsets for $\vartheta$.}\label{fig:weighted_deviation}%
                \end{figure}%
%
                The averaged quality $\vartheta$ of three, five and seven channel LOS selections for the \textit{weighted deviation metric} is calculated using \cref{eq:sensitivity} and shown in \cref{fig:results_weighted_deviation} for both bolometer cameras $P\ix{rad}$ individually. The left and right plot show the results for the HBC and VBC respectively, with different colours and markers, including errors bars derived from the standard deviation in $\vartheta\left(S\right)$. On the lower ordinate, the channel number is noted and on the upper the corresponding minimum, effective plasma radius along the LOS. The \textit{sensitivity} of the horizontal camera is generally, except for number 20 for combinations of three channels, well above 0.7. In cases of $m=3$, the average quality of the channels is around 0.8, with LOS around $\pm0.5r\ix{a}$ having higher sensitivities of up to 0.9. A small local minimum of 0.75 can be found around $\sim -r\ix{a}$, as well as a local maximum at the opposite side at $r\ix{a}$ with $\sim 0.85$. Lines of sight viewing the plasma core close to the magnetic axis $r\sim0$ all have a similar sensitivity of 0.83. Channel number 20, which is viewing the center of the triangular plane and area close to the inboard, lower magnetic \textit{X-point}, features a much lower sensitivity than any other of 0.58. The error bars on the individual channels do not exceed $\pm0.05$, where most have noticeably smaller ones. Especially LOS closer to the magnetic axis have smaller deviations than those around the local extremes. For $m=5$, the average quality is significantly increased and fluctuations between the individual channels in their contribution to the prediction are reduced. The expected quality of $P\ix{pred}$ is between 0.84\,-\,0.9. A local minimum and maximum can be also found around $\pm0.5r\ix{a}$, respectively. The error bars are reduced, similarly to before. And again, the same is repeated for the average sensitivity of channel selections of the HBC with size $m=7$. Here, the overall prediction quality is increased even more to between 0.91\,-\,0.92. The spectrum is hence narrowed further, with negligible error bars and almost no variations across the channels. Around $\pm0.75r\ix{a}$, where the local extremes in the previous profiles have been observed, minor deviations from the average sensitivity are found, indicating that the highest prediction quality is achieved by those lines of sight.\\%
%
                \begin{figure}[t]%
                    \centering%
                    \begin{subfigure}{0.47\textwidth}%
                        \includegraphics[width=\textwidth]{%
                            content/figures/chapter3/training/%
                            weighted_deviation_sensitivity_combs_HBCm.pdf}%
                        \caption{}%
                    \end{subfigure}%
                    \hfill%
                    \begin{subfigure}{0.47\textwidth}%
                        \includegraphics[width=\textwidth]{%
                            content/figures/chapter3/training/%
                            weighted_deviation_sensitivity_combs_VBC.pdf}%
                        \caption{}%
                    \end{subfigure}%
                    \caption{Average prediction quality $\vartheta$ over a large number of experiments for combinations of three, five and seven channels using the \textit{weighted deviation} metric. Prediction selections were limited to one of the individual camera arrays \textbf{(a)} HBCm and \textbf{(b)} VBC.}\label{fig:results_weighted_deviation}%
                \end{figure}%
%
                On the right-hand side, the sensitivity of the vertical bolometer camera is presented in the same way. Again, for all combination sizes, the average quality is generally above 0.75. Lines of sight with $r\sim-0.5r\ix{a}$, as well as $-0.1r\ix{a}$ for $m=3$ have higher sensitivities of up to 0.9. However, for higher $m$ there is a local minimum instead of a maximum at $-0.13r\ix{a}$. The results for combinations with $m=3$ show the highest uncertainties, while the error bars decrease with larger prediction selection size like before. More specifically, channels number 56, 65 and 77 feature especially large uncertainties of up to 0.1 or $\sim20\%$. The profiles of different $m$ again provide visual aid for the gaps in their respective counterparts. Though, distinct differences can be noted in comparison to the results of the horizontal bolometer camera. There is no correlation between selection size $m$ and average LOS sensitivity and a gap remains in the spectrum around $-0.36r\ix{a}$.\\%
                In summary, the \textit{weighted deviation} metric shows that the expected prediction quality and therefore sensitivity of the individual channels for LOS selections of at least three channels is above 70\% for both subsets of HBC and VBC lines of sight. In case of the horizontal camera, this increases proportionally to the amount of channels $m$ in the selection, simultaneously reducing the variation in $\vartheta$ between combination predictions. Local minima in the sensitivity profile of the horizontal camera are found at radii that view the plasma core and inboard side X-points, while maxima are produced by lines of sight integrating through the center and opposite the camera located magnetic island. The vertical camera achieves result of similar quality, though the sensitivity does not scale with selection size. Maxima can be found again for LOS that mainly view the plasma core and inboard side magnetic island edges. Here, the peak height in sensitivity correlates with the integration length through the triangular plane, as is expected for this type of analysis. Due to the camera orientation, the single local minimum is found at a different location. However, this still corresponds to a LOS viewing an inboard side X-point. The discrepancy between the profiles around $-0.13r\ix{a}$ can be attributed to the poor variation in the $m=3$ sensitivity.%
%
            \subsubsection*{Correlation}%
%
                The next metric and sensitivity map incorporate the \textit{correlation} integral between the prediction $P\ix{pred}$ and the full data set $P\ix{rad}$. The approach is analogous to the prior weighted deviation method in \cref{eq:weighted_deviation}, where the metric yields a function in temporal space $\varphi\left(t\right)$ and the map transforms and normalizes this to a single prediction quality value for the subset $S$ $\vartheta\left(S\right)$. The \textit{(cross) correlation} integral describes the likeness of signals along the same domain, calculating a profile in which higher deviations from the self-correlation $\varphi^{\ast}$ - the cross correlation of a signal with itself - correspond to lesser similarity between them. Assuming $P\ix{rad},\,P\ix{pred}>0$, the metric and quality map both yield positive-definite results, which are presented in arbitrary units.%
%
                \begin{align}%
                    \begin{split}\label{eq:correlation}%
                        \varphi\left(t\right)=&\int_{-\infty}^{\infty}P\ix{pred}\left(\tau\right)P\ix{rad}\left(t+\tau\right)\diff \tau\\%
                        \vartheta=&\frac{1}{T\ix{stop}-T\ix{start}}\int_{T\ix{start}}^{T\ix{stop}}\left|\varphi^{\ast}\left(t\right)-\varphi\left(t\right)\right|\diff t\\%
                        \varphi^{\ast}\left(t\right)=&\int_{-\infty}^{\infty}P\ix{rad}\left(\tau\right)P\ix{rad}\left(t+\tau\right)\diff \tau%
                    \end{split}%
                \end{align}%
%
                \begin{figure}[t]%
                    \centering%
                    \begin{subfigure}{0.47\textwidth}%
                        \includegraphics[width=\textwidth]{%
                            content/figures/chapter3/training/best_chans/%
                            best_chans_self_cross_corr_C[_0_7_30_19_12_5_24].pdf}%
                        \caption{}%
                    \end{subfigure}%
                    \hfill%
                    \begin{subfigure}{0.47\textwidth}%
                        \includegraphics[width=\textwidth]{%
                            content/figures/chapter3/training/%
                            self_correlation_3_HBCm_combinations.pdf}%
                        \caption{}%
                    \end{subfigure}%
                    \caption{Example of how the quality of the prediction for the \textit{correlation} metric is calculated for the previously discussed XP20181010.32. \textbf{(a)} Comparison of traces $\varphi\left(t\right)$, calculated using \cref{eq:correlation}, $P\ix{rad}$ and $P\ix{pred}^{\left(1\right)}$ for a subset of three channels of the HBC. \textbf{(b)} Overview of 900 different combinations of three channel subsets for $\vartheta$.}\label{fig:correlation}%
                \end{figure}%
%
                An example for this metric and selection quality $\vartheta$ can be found in \cref{fig:correlation}. The plot on the left-hand side is presented in a way similar to before, including a line representing the \textit{metric norm} or self-correlation. The results shown for $\varphi\left(t\right)$ are calculated using a LOS selection with $m=7$ from the horizontal bolometer camera and achieve a quality of $\vartheta\left(S\right)=$\SI{1.1}{\arbitraryunit} Experiment number XP20181010.32 is again chosen as the comparing data set and example here. On the right-hand side, a plot of the spectrum of selection qualities, calculated using \cref{eq:correlation} for the same $\sim$\,900 combinations and experiment XP20181010.32 as before with $m=3$ can be found. Values range from \SIrange{1}{28}{\arbitraryunit} and vary between single instances of $S$ with one channel exchanged up to \SI{12}{\arbitraryunit}\\%
%
                In \cref{fig:results_correlation}, the averaged line of sight sensitivity for selection sizes of three, five and seven channels from both bolometer cameras for the \textit{cross-correlation metric} are displayed similarly to before in \cref{fig:results_weighted_deviation}. Shown profiles depict an entirely different behaviour than those calculated using the weighted deviation metric where channel sensitivities have little variation across the camera array, except for two very distinct local maxima around $-0.9r\ix{a}$ and $0.85r\ix{a}$. The different selection sizes are congruent here. Error bars are, compared to the prior metric, very large with up to 40\% and only barely decrease in size with increasing LOS selection size $m$. The data points of the individual profiles are all within their adjacent standard deviations, which increase around the location of local maxima. In contrast to before, the orientation of the quality ordinate is different here due to the definition in \cref{eq:correlation}, where smaller values equal higher sensitivity of the channels to the prediction results. This means that the previously noted maxima correspond to less sensitive lines of sight for the plasma radiation power loss. Larger selection size profiles show slightly smaller average values, indicating a generally improved prediction quality for those sets of $S$. For the vertical bolometer camera, a similar image can be seen here, with comparatively large error bars and nearly congruent profiles for the different LOS selection sizes $m$. The variation across the individual profiles is slightly larger than that of the HBC. The results for the different $m$ are also in agreement with each other with consideration of the aforementioned error bars. However, in contrast to before, the larger size LOS selections yield higher average values and therefore lower local sensitivity for $r>0$. On the other end of the spatial spectrum the profiles are congruent. With the results of the horizontal camera in mind, the global maximum here and therefore the lowest local sensitivity coincides with the upper inboard  X-point location that was featured before.\\%
%
                \begin{figure}[t]%
                    \centering%
                    \begin{subfigure}{0.47\textwidth}%
                        \includegraphics[width=\textwidth]{%
                            content/figures/chapter3/training/%
                            self_correlation_sensitivity_combs_HBCm.pdf}%
                        \caption{}%
                    \end{subfigure}%
                    \hfill%
                    \begin{subfigure}{0.47\textwidth}%
                        \includegraphics[width=\textwidth]{%
                            content/figures/chapter3/training/%
                            self_correlation_sensitivity_combs_VBC.pdf}%
                        \caption{}%
                    \end{subfigure}%
                    \caption{Average prediction quality $\vartheta$ over a large number of experiments for combinations of three, five and seven channels using the \textit{correlation} metric. Prediction selections were limited to one of the individual camera arrays \textbf{(a)} HBCm and \textbf{(b)} VBC.}\label{fig:results_correlation}%
                \end{figure}%
%
                The correlation metric provides an entirely different approach to measuring the efficacy and quality of the plasma radiation power loss prediction for varying sizes $m$ of the LOS selection. This method focuses on both the likeness in absolute value and congruence in temporal behaviour between $P\ix{pred}$ and $P\ix{rad}$. Most channels, with exceptions for the local extremes, of the horizontal and vertical bolometer camera feature a similar sensitivity. In comparison to the weighted deviation, the lowest prediction qualities for the HBC are found for channels that previously presented local maxima, i.e. the highest radiation sensitivity. Lines of sight viewing along the inside of the separatrix and SOL across the triangular plane of W7-X and the inboard side located X-points show the lowest prediction quality for this metric. With respect to the prior analysis, this can also be said for the results of the VBC on the right-hand side - the less sensitive channels, measured by the correlation metric, of the vertical array only watch part of the core and the inboard X-points. For the VBC, detectors viewing the edges of the magnetic island adjacent to those X-points present profile minima and therefore higher local sensitivities. Contrasting results for increasing prediction selection sizes are found for the different cameras. The HBC shows the highest sensitivity for $m=7$, while the VBC depicts the opposite behaviour for fewer channels or no correlation between size $m$ and average quality. In both cases, increased standard deviations are found for all selection sizes, thus the prediction metric variation is very large. To some extent, the results from the weighted deviation metric analysis are supported by those findings, however the local extremes are orientated the other way and underline the opposite argument. This metric finds lines of sight from those ranges to contribute worse to the overall prediction quality than others, indicating that they provide unfavourable temporal correlation between prediction and total radiation power loss. With the previous, contrasting results in mind, this suggests that the correlation emphasizes discrepancies created due to the localized pertubation of the radiation distribution from (feedback) gas puffs.% 
%
            \subsubsection*{Mean Deviation}%
%
                The third metric is introduced in \cref{eq:mean_deviation} and uses a modified variance to calculate a \textit{mean deviation}. The corresponding prediction quality is defined by the square root of the integral of that mean deviation, normalized by the length of the measurement interval. In contrast to the weighted deviation from above, a smaller profile is equal to a higher likeness or better congruence between prediction and full data set. The opposite is true for the quality map due to its inverse correlation with the metric. This prediction evaluation method does not take the temporal relation between the two signals into account explicitly, similar to the prior weighted deviation metric, however it does so without cut-offs in the spectrum of $P\ix{pred}$. Both expressions in \cref{eq:mean_deviation} yield positive-definite results.%
%
                \begin{align}%
                    \begin{split}\label{eq:mean_deviation}%
                        \varphi\left(t\right)=&\frac{1}{T\ix{stop}-T\ix{start}}\left(P\ix{rad}\left(t\right)-P\ix{pred}\left(t\right)\right)^{2}\\%
                        \vartheta=&\sqrt{\left(T\ix{stop}-T\ix{start}\right)\left(\int_{T\ix{start}}^{T\ix{stop}}\varphi\left(t\right)\diff t\right)^{-1}}%
                    \end{split}%
                \end{align}%
%
                An example of the mean deviation metric is shown in \cref{fig:results_mean_deviation}. The left-hand side plot shows the results for a single LOS selection and horizontal bolometer camera measurement from discharge XP20181010.32 for comparison in the same way as before. The average prediction quality $\vartheta$ for this set of $S$ is calculated to \SI{5.12e-6}{\arbitraryunit} This is an exceptionally small value, as is reflected by the profile of $\varphi$ in the same figure, which corresponds to a strong agreement between prediction and $P\ix{rad}$. Hence, the \textit{norm} here is a baseline at zero. On the right-hand side all, unsorted prediction qualities in reference to XP20181010.32 for line of sight selections with $m=3$ are shown. Results range from \SI{0.51e-5}{\arbitraryunit} down to near zero, while individual selections $S$ with one channel exchanged can differ by up to \SI{0.4}{\arbitraryunit}.\\%
%
                \begin{figure}[t]%
                    \centering%
                    \begin{subfigure}{0.47\textwidth}%
                        \includegraphics[width=\textwidth]{%
                            content/figures/chapter3/training/best_combs/%
                            best_comb_mean_dev_C[10_21_26].pdf}%
                        \caption{}%
                    \end{subfigure}%
                    \hfill%
                    \begin{subfigure}{0.47\textwidth}%
                        \includegraphics[width=\textwidth]{%
                            content/figures/chapter3/training/%
                            mean_deviation_3_HBCm_combinations.pdf}%
                        \caption{}%
                    \end{subfigure}%
                    \caption{Example of how the quality of the prediction for the \textit{mean deviation} metric is calculated for the previously discussed XP20181010.32. \textbf{(a)} Comparison of traces $\varphi\left(t\right)$, calculated using \cref{eq:mean_deviation}, $P\ix{rad}$ and $P\ix{pred}^{\left(1\right)}$ for a subset of three channels of the HBC. \textbf{(b)} Overview of 900 different combinations of three channel subsets for $\vartheta$.}\label{fig:mean_deviation}%
                \end{figure}%
%
                The averaged prediction quality $\vartheta$ of three, five and seven channel line of sight selections for the \textit{mean deviation} metric, calculated by \cref{eq:mean_deviation} is shown in \cref{fig:results_mean_deviation}. The left and right side plot show the results for the horizontal and vertical bolometer camera respectively the same way as before. On the left the highest values, i.e. generally with $\sim$\SI{1.2e-4}{\arbitraryunit} are produced by $m=7$, while the respective variation across the spectrum of channels is lowest. Standard deviations are comparatively large at \SI{3e-5}{\arbitraryunit} or $\sim$40\%. While for larger selections sizes no or only minor extrema exist, $m=3$ has stronger local maxima at $-0.28r\ix{a}$ and $0.6r\ix{a}$. The right-hand side shows the results for the vertical bolometer camera in the same manner. Similar to the HBC, the highest values are produced by selection size $m=7$, while overall the profiles are grouped closer here. The average sensitivity here is higher than before, with error bars of up to \SI{5e-5}{\arbitraryunit} or $\sim$25\%. For $m=3$, the profile is lowest in this plot with values between \SIrange{1}{1.7e-4}{\arbitraryunit} and even more prominent extremes in similar locations to the other lines, i.e. $-0.13r\ix{a}$. An additional local minimum can be found around $0.7r\ix{a}$.\\%
                The \textit{mean deviation} metric offers another approach to analysing the difference in quality of individual channel selections. In the above results, the local extremes are in the same locations as for the weighted deviation, though being far less pronounced in both cameras. However, some features are missing when comparing to the latter, i.e. the small minimum and maximum outside the separatrix on both ends of the spectrum are absent here. More prominently, the profiles standard deviation is overall unfavourable with at least 20\% and an order of magnitude larger than that of the weighted deviation and similarly sized compared to the correlation metric. No outliers in local radiation sensitivity are found with this method. Like in the case of \textit{weighted deviation}, this metric promotes channels with higher values, i.e. horizontal camera LOS from the edge of the array, close to and outside of $\pm r\ix{a}$, to be best suited for predicting $P\ix{rad}$. The global minimum of the vertical camera profiles is located in the same position as well in the beginning. Scaling of the quality $\vartheta$ with selection size $m$ however yields the same result, increasing the prediction quality consistently while losing previously presented characteristic features. Qualitatively, the VBC again is in agreement - as was the case for the weighted deviation -, and local extrema also consist of LOS that view one lower inboard and one upper inboard island. Overall, with a selection size $m=7$, a good prediction quality can be achieved for any channel combination from both cameras viewing close to and outside the separatrix as calculated by the mean deviation metric. With respect to the previous prediction evaluation methods, this finds channels viewing the core to consistently and positively impact the performance. The respective mathematical formulas in \cref{eq:mean_deviation} do not yield significantly different results comparing to \cref{fig:results_weighted_deviation}, i.e. temporal and absolute deviations are not weighted differently like the correlation does. However, that said, the exceptionally large error bars have to be taken into account during later assessments of the real-time radiation feedback. Taking the prior two metrics evalutations into account, an exclusive rather than inclusive deduction can be made: the outermost LOS viewing the edge of the plasma towards the domain boundary promote the least sensitivity and are generally unfavourably for $P\ix{rad}$ prediction.%
%
                \begin{figure}[t]%
                    \centering%
                    \begin{subfigure}{0.47\textwidth}%
                        \includegraphics[width=\textwidth]{%
                            content/figures/chapter3/training/%
                            mean_deviation_sensitivity_combs_HBCm.pdf}%
                        \caption{}%
                    \end{subfigure}%
                    \hfill%
                    \begin{subfigure}{0.47\textwidth}%
                        \includegraphics[width=\textwidth]{%
                            content/figures/chapter3/training/%
                            mean_deviation_sensitivity_combs_VBC.pdf}%
                        \caption{}%
                    \end{subfigure}%
                    \caption{Average prediction quality $\vartheta$ over a large number of experiments for combinations of three, five and seven channels using the \textit{mean deviation} metric. Prediction selections were limited to one of the individual camera arrays \textbf{(a)} HBCm and \textbf{(b)} VBC.}\label{fig:results_mean_deviation}%
                \end{figure}%
%
                \newline%
                One should note that in \cref{sec:fouriercorrelate} an additional metric is introcduced and discussed that features a cross correlation between discrete fourier transforms of $P\ix{rad}$ and the prediction. However, due to its complexity, inconclusive data set and evaluation it will not be part of the latter discussion here.\\%
                This concludes the application and examination of the individual evaluation metrics for the same multifaceted, large set of experiments with feedback application. For all metrics, the plasma radiation prediction has been characterized for different selection sizes $m$ and the local sensitivity of such towards the full set $P\ix{rad}$ explored. Different functional patterns towards a singular quality measure for this kind of prediction have been exercised and their impact measured. The results are used to understand the contribution of individual lines of sight, given the manifold of metric characteristics magnetic configuration of W7-X, to the predictability of qualitative and quantitive features in $P\ix{rad}$. With respect to the initially posed questions in the beginning of this \cref{chap:feedbackeval}, findings from the above investigations point to lines of sight of the HBC at or close to the separatrix and scrape-off layer to be most promising for predicting the radiative power loss from a smaller data set. More specifically, when also looking at channels of the VBC, information from the upper inboard magnetic island and X-point yield the largest contribution towards favourable $P\ix{red}$ for feedback experiments. To conclude the presented investigation differently: for smaller sets of $S$, LOS viewing the core and outermost regions of the plasma boundary towards the wall are unfavourable for feedback purposes. For larger selections, i.e. $m=7$, both of the prior biases become obsolete, as was expected given the premise of the calculations.\\%
                An additional study has been carried out involving the underlying plasma parameters and the above sensitivities to querry whether there are differences in discharge scenarios and what their potential impact on this study is. Their results can be found in \cref{sec:senseresults}.%
%
%
    \section{STRAHL Modelling}\label{sec:strahlmodel}%
%
        The one-dimensional transport simulation code STRAHL is designed to analyse the radial distribution and radiation profile of impurities within the plasma bulk. It solves the radial continuity equation for each ionization stage of the impurity in a one-dimensional geometry and uses an ansatz of anomalous diffusivities and radial drift velocities. STRAHL can incorporate full neoclassical transport treatment as needed. The simulation focuses on dynamics of impurity transport and its radiative effects, using parameters derived from experimental data for the background plasma and is therefore used to understand and predict their behaviour. Results outside the confinement area are inherently wrong due to the assumptions made about the derivations used in STRAHL.\\%
        The equilibrium impurity transport and radiation simulation is employed to investigate the previously discussed results of the local LOS sensitivity towards the bolometer feedback. In particular, the calculated one-dimensional impurity emission profiles will be used to find possible correlations between the results in \cref{sec:evalmetrics}, kinetic profiles and transport coefficients, which are varied as input parameters into STRAHL. The goal is to link data from the previously examined results of experiment XP20181010.32, which is the prime candidate for radiation feedback control, to particular variations in the chord brightness profiles and impurity radiation around the separatrix that have been observed in the previous chapter. For given profiles of electron temperature and density, transport coefficients, magnetic field geometry and strength, impurity sources and atomic spectral and energy data, STRAHL solves \cref{eq:strahl_radtrans} (see below) and finds the corresponding distribution and emissivity profile. Only approximative terms for parallel transport to the plasma limiting surface are considered in the SOL, while the kinetic profiles are estimated to decay exponentially outside and the magnetic geometry and have to be extrapolated from the LCFS. Results here can not be used for further discussion, particularly not in the context of W7-X. A more detailed inspection with additional background information can be found in \cite{Dux2006,Goncharov2007,Behringer1987}.%
%
        \subsection{Introduction}\label{subsec:strahl}%
%
            STRAHL solves the radial continuity for each ionisation stage $i$, i.e. of particle density $n\ix{i,Z}$ of a given impurity $Z$ by the above paradigms. Let $S\ix{i,Z}$ be sources and sinks of said ion population. The simulation focuses on transport perpendicular to the flux surfaces (FS), assuming a constant $n\ix{i,Z}$ along them, however poloidal variations in distance between them also leads to a net impurity density and temperature gradient and hence flux. For establishing a differential expression to solve numerically, one finds the flux surface average of $n\ix{i,Z}$ and applies that to the continuity equation, producing a cylinder coordinate form in which an ansatz of diffusive and convective terms for the flux is applied. This gives, taking the flux surface averaged radial diffusion coefficient $D^{\ast}$ and drift velocity $v^{\ast}$:%
%
            \begin{align}%
                \begin{split}\label{eq:strahl_radtrans}%
                    &\frac{\partial n_{\text{i,}Z}}{\partial t}=\frac{1}{r}\frac{\partial}{\partial r}\,r\left(D^{\ast}\frac{\partial n_{\text{i,}Z}}{\partial r}-v^{\ast}n_{\text{i,}Z}\right)+S_{\text{i,}Z}\,\,,\\%
                    D^{\ast}=&\left\langle D\left(\theta\right)\vert \nabla r\vert^{2}\right\rangle\ix{FS}\coloneqq \frac{1}{4\pi^{2}R\ix{M}r}\displaystyle\oint\ix{FS}D\left(\theta\right)\vert\nabla r\vert\diff S\,\,\\%
                    v^{\ast}=&\left\langle v\left(\theta\right)\vert\nabla r\vert\right\rangle\ix{FS}=\frac{1}{4\pi^{2}R\ix{M}r}\displaystyle\oint\ix{FS}v\diff S\,\,.%
                \end{split}%
            \end{align}%
%
            Outside the separatrix, the one dimensional model is not fit and the parallel transport to the divertor or limiter is considered by a loss time $\tau\ix{||}$, which is considered by an additional term $-n\ix{i,Z}/\tau\ix{||}$ in the SOL on the right-hand side of the modified continuity equation. The sources and sinks $S\ix{i,Z}$ are described by coupling neighbouring ion stages with their respective rate coefficients $Q\ix{i,Z}$ for ionisation ($ion$), radiative and di-electronic ($rec$) as well as charge exchange ($cex$) recombination.
%
            \begin{align}%
                \begin{split}%
                    S_{\text{i,}Z}=-&n_{\text{i,}Z}\left(n\ix{e}Q\head{ion}_{\text{i,}Z}+n\ix{e}Q_{\text{i,}Z}\head{rec}+n\ix{H}Q_{\text{i,}Z}\head{cex}\right)+n\ix{e}Q_{\text{i,}Z-1}\head{ion}n_{\text{i,}Z-1}+\\%
                    &n_{\text{i,}Z+1}\left(n\ix{e}Q_{\text{i,}Z+1}\head{rec}+n\ix{H}Q_{\text{i,}Z+1}\head{cex}\right)%
                \end{split}%
            \end{align}%
%
            Neoclassical transport is accounted for by two individual tools in STRAHL. \textit{NeoArt} is another simulation code by Arthur Peeters, solving the set of linear coupled equations for the parallel velocities in arbitrary toroidally symmetric geometry for all collision regimes\cite{Peeters2000}. The second approach finds approximative analytical expressions for the diffusion and drift according to Hirshman and Sigmar et al.\cite{Hirshman1981}\\%
            The result in \cref{eq:strahl_radtrans} for a number of $Z$ transport equations can be expressed and solved algebraically and therefore numerically, using $\vec{n}$ the density of said impurity as well as $\mathbf{D}$, $\mathbf{v}$, $\mathbf{S}$ and $\mathbf{R}$ (diagonal) matrices of transport and rate coefficients. The latter two denote products of the previously mentioned parameters and electron number densities $n\ix{e}$. A source term for neutral ionisation is given by $\vec{d}=\vec{e}\ix{1}$.%
%
            \begin{align}%
                \begin{split}%
                    \frac{\partial\vec{n}}{\partial t}=&\mathbf{D}\frac{\partial^{2}\vec{n}}{\partial t^{2}}+\left(\left(\frac{1}{r}+\frac{\diff}{\diff r}\right)\mathbf{D}-\mathbf{v}\right)\frac{\partial\vec{n}}{\partial r}-\\%
                    &\left(\frac{1}{r}+\frac{\diff}{\diff r}\right)\mathbf{v}\vec{n}-\mathbf{S}\vec{n}-\mathbf{R}\vec{n}+\vec{d}%
                \end{split}%
            \end{align}%
%
            The stable and computationally effective \textit{Crank-Nicholson}\footnote[1]{an implicit, finite difference method used for numerically solving the heat equation and similar partial differential equations; giving second-order convergence in time}\cite{WikiCrankNicholson} method finds an equation for temporal iteration to $\vec{n}^{\left(l+1\right)}$:%
%
            \begin{align}%
                \begin{split}%
                    &\vec{n}^{\left(l+1/2\right)}=\frac{\vec{n}^{\left(l\right)}+\vec{n}^{\left(l+1\right)}}{2}\,\,\\%
                    \frac{\vec{n}^{\left(l+1\right)}-\vec{n}^{\left(l\right)}}{\Delta t}=&\left[\mathbf{D}\frac{\partial^{2}}{\partial r^{2}}+\left(\left(\frac{1}{r}+\frac{\diff}{\diff r}\right)\mathbf{D}-\mathbf{v}\right)\frac{\partial}{\partial r}-\right.\\%
                    &\quad\left.\left(\frac{1}{r}+\frac{\diff}{\diff r}\right)\vec{v}-\mathbf{S}-\mathbf{R}\right]\vec{n}^{\left(l+1\right)}+\vec{d}\,\,.%
                \end{split}%
            \end{align}%
%
            Neutrals (impurities) have so far only been accounted for as sources to the first ionisation stage. In STRAHL, they are modelled as uniformly moving radially inwards, decaying in density with decreasing radius as the plasma is assumed to be sufficiently dense and hot to fully ionize the influx of particles.\\%
%
            \begin{figure}[t]%
                \centering%
                \begin{minipage}[c]{0.48\textwidth}%
                    \centering%
                    \subcaptionbox{}{%
                        \includegraphics[width=\textwidth]{%
                            content/figures/chapter3/STRAHL/nete/%
                            compare_ne_Te_91_92_full.png}
                    }%
                \end{minipage}%
                \hfill%
                \begin{minipage}[c]{0.48\textwidth}%
                    \centering%
                    \subcaptionbox{}{%
                    \includegraphics[width=\textwidth]{%
                        content/figures/chapter3/STRAHL/diag_lines/%
                        compare_strahl_rad_91_92_full_CO.pdf}%
                    }%
                \end{minipage}%
                \captionof{figure}{Comparison between \textit{Thomson scattering} data and corresponding STRAHL simulation results for $f\ix{rad}=33\%$ and $66\%$ from experiment XP20181010.32. \textbf{(a)}: Thomson scattering electron density and temperature data with individual second order spline interpolated lines (\textit{interp.}), which were used as input for STRAHL simulations. \textbf{(b)}: One-dimensional STRAHL simulation results of total carbon impurity radiation intensity.}\label{fig:nete_total_rad_91_92}%
            \end{figure}%
%
            \,\newline%
            Analysis of the previously discussed experimental achievements have been focused around the plasma edge and SOL, however, due to the intrinsic inaccuracy in this particular area, modelling using STRAHL can not be used directly to mimic those profiles but to achieve a better understanding of the radiation distribution regarding individual impurities and changes in transport. For this purpose, a small set of variations of transport coefficients for constant input kinetic profiles and simulation parameters will be discussed.\\%
            Data for $T\ix{e}$ and $n\ix{e}$ are taken from discharge XP20181010.32 at a radiation fraction of $f\ix{rad}=$100\%. If not stated otherwise, the STRAHL configuration is the same for all presented results. An inexhaustible neutral gas background was considered for collisions with the modelled impurity ion species. For convenience, local quasi-neutrality $n\ix{e}\left(r\right)=n\ix{i}\left(r\right)$ and equilibrium $T\ix{e}\left(r\right)=T\ix{i}\left(r\right)$ is assumed. Corresponding data for atomic energies levels and spectral information were collected from the central \textit{Atomic Spectra Database}\cite{AtomicSpectra} and/or using \textit{ADAS}\cite{ADAS}. As was extensively explored at the stellarator W7-X during its past experimental campaigns and its predecessors\cite{Hirsch2008}, in general, the most relevant impurities towards radiative power loss and plasma performance were oxygen and carbon\cite{Wang2020,Sereda2020,Klinger2016}. Helium, which during previous operational phases and before the first boronisation also played a large role in plasma-wall interactions, will be omitted in the calculations, since the above data are collected after treatment of the reactor wall by plasma-chemical deposition of boron. If not stated otherwise, a set of default parameters, modelled to represent a plasma environment as encountered during the feedback experiments, is deliberately chosen and used throughout the different calculations. Magnetic field data and grid point information, i.e. location of separatrix etc. are derived using the standard configuration. Neutral impurities are injected, modelling the exhaust and pumping of the wall, with an energy of \SI{1}{\electronvolt} at a rate of \SI{9e20}{\per\second} from $r\ix{LCFS}+$\SI{7.5}{\centi\meter}, where $r\ix{LCFS}$ is the radial location of the separatrix in STRAHL, while recombinations are configured to occur at the surface or close to it. The impurity density, as well as the $T\ix{e}$ and $n\ix{e}$ profile decay length is set to \SI{5}{\centi\meter} outside the separatrix, which mimics profile shapes across magnetic islands. A limiting surface, i.e. the divertor is placed at $r\ix{LCFS}+$\SI{6.5}{\centi\meter}, whereas the calculation boundary is at $+$\SI{8}{\centi\meter}. Any provided grid-based data, like the experimental Thomson scattering measurements for electron temperatures and densities, are smoothed upon ingest by STRAHL through \textit{cubic spline interpolation} to ensure stability during numerical iteration.%
%
            \begin{figure}[t]%
                \centering%
                \captionsetup{width=.47\textwidth}%
                \begin{minipage}[c]{0.48\textwidth}%
                    \centering%
                    \subcaptionbox{}{%
                        \includegraphics[width=\textwidth]{%
                            content/figures/chapter3/STRAHL/diag_lines/%
                            compare_strahl_rad_combine_91_92_full.png}%
                    }%
                \end{minipage}%
                \hfill%
                \begin{minipage}[c]{0.48\textwidth}%
                    \centering%
                    \caption{Comparison between STRAHL simulation results for $f\ix{rad}=33\%$ and $66\%$ from experiment XP20181010.32 of carbon impurities: the top shows the fractional abundances of the individual ionisation stages of carbon, the bottom depicts the corresponding radiation intensities and total power density.}\label{fig:nete_abund_lines_91_92}%
                \end{minipage}%
            \end{figure}%
%
            In \cref{fig:nete_total_rad_91_92}, a set of kinetic radial profiles $n\ix{e}$ and $T\ix{e}$ for two selected radiation fraction levels $f\ix{rad}=33\%$ and $66\%$ is shown next to the corresponding STRAHL simulation results of radial, total radiation intensity profiles for carbon and oxygen. The Thomson scattering measurements and interpolations on the left are extracted from the results of XP20181010.32 at \SI{0.621}{\second} and \SI{2.22}{\second}, respectively. The position of the separatrix $r\ix{a}$ is indicated by a dotted, grey line around \SI{0.52}{\meter}. \autoref{fig:fluid_coeffs} shows the corresponding radial fluid drift and diffusion coefficient profiles, applied for all calculations equally if not stated otherwise. The anomalous diffusion profile is general and experimentally motivated, as indicated by Wegner et al. and the \textit{Laser Blow-Off system}\cite{Klinger2019}. The convective drift was conveniently kept constant at zero, as it was not well understood and very much subject to current investigation at the time of calculation. Anomalous diffusion is found to be more than two orders of magnitude larger than the neoclassical transport, strongly suggesting dominant turbulent transport\cite{Klinger2019}. Therefore, the chosen profile indicates a larger diffusivity of carbon impurities in the core towards the separatrix, while their transport outside the LCFS is still greater than at the plasma center. In the bottom left and top right of the respective kinetic profile plots in \cref{fig:nete_total_rad_91_92}, the separatrix temperatures and densities for the individual radiation fractions are noted for comparison. On the right, the top and bottom plots show the total impurity radiation power density profile for oxygen and carbon respectively for the corresponding $f\ix{rad}$ and input data on the left. They represent the sum of all considered ionization stages integral emissivity per radial bin in STRAHL. For both plots, the shape of the profiles outside the LCFS, where no data can be collected from the Thomson Scattering diagnostic, is defined by the previously introduced generalization about kinetic decay lengths in the SOL of \SI{5}{\centi\meter}. On the right, the corresponding impurity emissivities for both $f\ix{rad}$ stages immediately show a strong mismatch in the absolute values and shapes of the radial plots between the two elements. In particular, the radiation power loss through oxygen impurities $P\ix{tot,O}$ is $10^{1}-10{2}$ times smaller than the corresponding exhaust through carbon. For both impurities and radiation fraction levels, the general shape of the radial radiation plots are very similar. The emission from the plasma core is consistently higher with \SI{0.65}{\kilo\watt\per\cubic\meter} than from the SOL for oxygen, while carbon impurity particles irradiate the strongest at \SI{45}{\kilo\watt\per\cubic\meter} beyond the separatrix. At both radiation fraction levels, $P\ix{tot,O}$ has its maximum close to $0.8r\ix{a}$ The absolute levels so far scale linearly with $f\ix{rad}$, i.e. a factor of two is directly represented in $P\ix{tot}$ by both impurities. Due to the intrinsic ansatz for particle transport, the emissivity for oxygen has distinct features at the separatrix that are significantly different from the surrounding profile. However, such characteristics can not be found in carbon, where in fact the global minimum in emissivity is located at the LCFS.\\%
%
            \begin{figure}[t]%
                \centering%
                \captionsetup{width=.47\textwidth}%
                \begin{minipage}[b]{0.48\textwidth}%
                    \centering%
                    \subcaptionbox{}{%
                        \includegraphics[width=\textwidth]{%
                            content/figures/chapter3/STRAHL/transport/%
                            compare_anomal_transp_91_92_full.png}
                    }%
                \end{minipage}%
                \hfill%
                \begin{minipage}[b]{0.48\textwidth}%
                    \centering%
                    \caption{Standard radial fluid drift and diffusion profiles, which were used as input for the above - and if not stated otherwise below - STRAHL simulations.}\label{fig:fluid_coeffs}%
                \end{minipage}%
            \end{figure}%
%
            Based on the observations about the total amount of radiation, the following evaluations will omit oxygen as a factor to the analysis of radial emissivity profiles from impurities under feedback conditions. Hence, \cref{fig:nete_abund_lines_91_92} only focuses on the individual ionization stages of carbon and their corresponding fractional abundances, i.e. $f\ix{A,Z}=n\ix{i,Z}/n\ix{tot,C}$. Here, the relative density (top) and emissivity (bottom) of each carbon ion species across the radius for the previous radiation fractions of 33\% and 66\% is shown. The presented results correspond to the same kinetic input profiles as before in \cref{fig:nete_total_rad_91_92}. Carbon impurity species are exclusively represented by the fully ionized C$^{6+}$ up until $\sim0.6r\ix{a}$, where its fractional abundances decreases and those of the next lower stages, C$^{5+}$ and C$^{4+}$ begin to increase. Outside the separatrix, the lesser ionized particle counts start to increase. Carbon atoms can only be found at the very end of the domain with a sharp cut-off towards the edge due to recombination reactions at the plasma limiting surface, acting as a source of neutral impurities. The change in $f\ix{rad}$ has a varying impact on the individual ion abundances. Increasing the radiation fraction leads to a slight reduction of fully ionized C$^{6+}$, whereas the next two lower stages show a small uplift and shift in their maximum location inwards and outwards for C$^{5+}$ and C$^{4+}$ respectively. At the bottom in \cref{fig:nete_abund_lines_91_92}, the relative radiation power loss fraction of each of the above ionization stages is shonw. With respect to the individual line radiation coefficients and kinetic temperature and density profiles, they directly correspond to the previous fractional abundances.\\%
%
            \newline%
            Due to the poor quality of input data, particularly the kinetic profiles from the unreliable \textit{Thomson scattering} of that time that did not produce measurements outside the LCFS, and the aforementioned intrinsic inaccuracy of STRAHL to adequately describe transport processes across the seperatrix and in the SOL, further investigations involving more experimental backgrounds will be omitted here. However, a small selection can be found still in \cref{apx:strahlexp} for reference.%
%
        \subsection{Parameter Variation}\label{subsec:parametervar}%
%
            The following model variations are limited to the transport coefficient profiles $D$ and the absolute level of the separatrix electron temperature and density, while all other parameters are kept constant. The impact of each individual variation will only be examined for the kinetic data collected at $f\ix{rad}=100\%$ in XP20181010.032, like it was done before for $f\ix{rad}=33\%$ and 66\%. Additional, less relevant parameter or configuration variations can be found \cref{apx:strahlvar}.%
%
            \subsubsection*{Diffusion Coefficient Variations}%
%
                \begin{figure}[t]%
                    \centering%
                    \begin{minipage}[b]{0.48\textwidth}%
                        \centering%
                        \subcaptionbox{}{%
                            \includegraphics[width=\textwidth]{%
                                content/figures/chapter3/STRAHL/transport/%
                                compare_anomal_transp_62_66_full.png}
                        }%
                    \end{minipage}%
                    \hfill%
                    \begin{minipage}[b]{0.48\textwidth}%
                        \centering%
                        \subcaptionbox{}{%
                            \includegraphics[width=\textwidth]{%
                                content/figures/chapter3/STRAHL/fract_abund/%
                                compare_fract_abund_62_66_full.png}%
                        }%
                    \end{minipage}%
                    \captionof{figure}{Comparison between two sets of radial transport coefficient profiles and corresponding STRAHL simulation results - the fractional abundance is shown here -, both for $f\ix{rad}=100\%$ from experiment XP20181010.32. \textbf{(a)}: normalized drift velocity profile and diffusion coefficient. \textbf{(b)}: fractional abundances of carbon from STRAHL simulation results, corresponding to the different profiles on the left.}\label{fig:transp_abund_62_66}%
                \end{figure}%
%
                Changing the transport profiles will be conducted in two steps: first, a displacement of the significant peak, after which another set of different absolute height is applied in the simulation.\\%
                In \cref{fig:transp_abund_62_66}, the first two diffusion coefficient $D$ profile variations are shown alongside their corresponding impact on the fractional abundances of the previously described carbon ion species. The profiles are produced by a radial shift either \SI{0.05}{\meter} towards ($D\ix{2}$) or \SI{0.15}{\meter} away ($D\ix{1}$) from the separatrix, while the respective maximum and peak shape remains the same. The convective drift velocity is unchanged. When comparing with \cref{fig:nete_abund_lines_93_94_edge}, both sets of diffusion coefficients yield far larger plateaus, spanning nearly the entire core. Also, fully ionized carbon C$^{6+}$ remains the dominant species throughout the domain, while for $D\ix{1}$ its abundance $f\ix{6+}$ drops and the next lower stage C${4+}$ increases at the LCSF. All other ion stages show negligible changes outside the separatrix. \autoref{fig:rad_ratios_total_62_66} shows the corresponding relative emissivities and their respective absolute and total radiation powers of the individual carbon ion stages. From here on, the focus is again on the radial spectrum close to and outside the LCFS. No significant changes for C$^{6+}$ can be noted. The next lower ionization level features a higher emissivity overall for $D\ix{1}$, whereas C$^{4+}$ shifted radially outward by \SI{0.15}{\meter} and increased its maximum in $0.9r\ix{a}$. The overall increase is mimicked in the SOL as well. Comparing with the initial results in \cref{fig:total_rad_93_94}, $P\ix{3+}$ here also has a significant relative emissivity outside for $f\ix{rad}=100\%$ and both diffusion profiles. Its maximum is also higher than before. The remaining impurity emissions of C$^{2}$ through atomic carbon are, for $D\ix{1}$ and $D\ix{2}$, centred close to or around the LCFS. Results for C$^{2+}$ and $D\ix{2}$ are very similar to the initially presented profile, though smaller in value, however for $D\ix{1}$ the SOL emissivity is strongly reduced with increasing distance from $r\ix{a}$. The lowest charge state C$^{1+}$ previously had almost no emissions inside and the majority outside the separatrix. Here, for both sets of transport coefficients, $P\ix{1+}$ is slightly larger and nearly symmetrical around $r\ix{a}$. The same is true for atomic carbon.\\%
%
                \begin{figure}[t]%
                    \centering%
                    \begin{minipage}[b]{0.48\textwidth}%
                        \centering%
                        \subcaptionbox{}{%
                            \includegraphics[width=\textwidth]{%
                                content/figures/chapter3/STRAHL/diag_lines/%
                                compare_strahl_rad_ratios62_66_edge.png}
                        }%
                    \end{minipage}%
                    \hfill%
                    \begin{minipage}[b]{0.48\textwidth}%
                        \centering%
                        \subcaptionbox{}{%
                            \includegraphics[width=\textwidth]{%
                                content/figures/chapter3/STRAHL/diag_lines/%
                                compare_strahl_rad_62_66_edge.png}%
                        }%
                    \end{minipage}%
                    \captionof{figure}{STRAHL simulation results of relative and total impurity radiation from carbon for the previously presented sets of radial drift and diffusion profiles in \cref{fig:transp_abund_62_66} for $f\ix{rad}=100\%$ from experiment XP20181010.32. \textbf{(a)}: relative radiation intensity for each individual ionisation stage of carbon. \textbf{(b)}: Absolute radiation power and integrated loss for the same ion species.}\label{fig:rad_ratios_total_62_66}%
                \end{figure}%
%
                On the right in \cref{fig:rad_ratios_total_62_66}, the individual absolute carbon emissivities largely reflect the results on the left, where the relative changes in C$^{6+}$, C$^{5+}$ and C$^{4+}$ align and both only yield small contributions to the plasma core for every $D\ix{i}$ profile. The SOL part of C$^{3+}$ radiation is a steady continuation of the pronounced inside profile peak for $D\ix{1}$. For $D\ix{2}$, the much lower and broader maximum is similar to the initial result in \cref{fig:rad_ratios_total_62_66}, where beyond the LCFS the profile is defined by the decay length. This is also the case for both sets of profiles for C$^{2+}$ and C$^{1+}$, though with decreasing intensity. Hence, $D\ix{1}$ yields a much higher and sharper radiation power loss maximum inside the separatrix than $D\ix{2}$.\\%
%
                \begin{figure}[t]%
                    \centering%
                    \begin{minipage}[b]{0.48\textwidth}%
                        \centering%
                        \subcaptionbox{}{%
                            \includegraphics[width=\textwidth]{%
                                content/figures/chapter3/STRAHL/transport/%
                                compare_anomal_transp_103_104_full.png}%
                        }%
                    \end{minipage}%
                    \hfill%
                    \begin{minipage}[b]{0.48\textwidth}%
                        \centering%
                        \subcaptionbox{}{%
                            \includegraphics[width=\textwidth]{%
                                content/figures/chapter3/STRAHL/fract_abund/%
                                compare_fract_abund_103_104_full.png}%
                        }%
                    \end{minipage}%
                    \captionof{figure}{Comparison between two sets of radial transport coefficient profiles and corresponding STRAHL simulation results - the fractional abundance is shown here -, both for $f\ix{rad}=100\%$ from experiment XP20181010.32. \textbf{(a)}: normalized drift velocity profile and diffusion coefficient. \textbf{(b)}: fractional abundances of carbon from STRAHL simulation results, corresponding to the different profiles on the left.}\label{fig:transp_abund_103_104}%
                \end{figure}%
%
                The next set of images contain the second variation of diffusion profiles $D$, where the absolute value instead of radial location is changed. Here, in \cref{fig:transp_abund_103_104}, $D\ix{3}$ represents the initially used profile and $D\ix{1}$ a much lower but similarly shaped version of the latter. The individual fractional abundances on the right for both sets of transport coefficients from the left indicate minor differences. The relative amount of fully ionized carbon increases for $D\ix{3}$ in the core, while its value at and beyond the LCFS is reduced. The $f\ix{A}$ profiles of the next two lower charge states therefore have shifted towards the SOL and increased where $f\ix{6+}$ is decreased. The remaining ion stages behave similarly, though only featuring significant abundances outside the LCFS with decreasing relative contributions the lower the ionization. Furthermore, for a increased diffusivity of $D\ix{3}$, C$^{3+}$ and C$^{1+}$ show direct proportionality in the SOL. Again, carbon atom numbers increase closer to the divertor, with negligible influence of the different transport coefficient profiles.\\%
%
                \begin{figure}[t]%
                    \centering%
                    \begin{minipage}[b]{0.48\textwidth}%
                        \centering%
                        \subcaptionbox{}{%
                            \includegraphics[width=\textwidth]{%
                                content/figures/chapter3/STRAHL/diag_lines/%
                                compare_strahl_rad_ratios103_104_edge.png}
                        }%
                    \end{minipage}%
                    \hfill%
                    \begin{minipage}[b]{0.48\textwidth}%
                        \centering%
                        \subcaptionbox{}{%
                            \includegraphics[width=\textwidth]{%
                                content/figures/chapter3/STRAHL/diag_lines/%
                                compare_strahl_rad_103_104_edge.png}%
                        }%
                    \end{minipage}%
                    \captionof{figure}{STRAHL simulation results of relative and total impurity radiation from carbon for the previously presented sets of radial drift and diffusion profiles in \cref{fig:transp_abund_62_66} for $f\ix{rad}=100\%$ from experiment XP20181010.32. \textbf{(a)}: relative radiation intensity for each individual ionisation stage of carbon. \textbf{(b)}: Absolute radiation power and integrated loss for the same ion species.}\label{fig:rad_ratios_total_103_104}%
                \end{figure}%
%
                The corresponding relative emissivities, as well as absolute impurity ion emissions and total radiation power loss for the above diffusion coefficient profile variation can be found in \cref{fig:rad_ratios_total_103_104}. On the left, the previously examined impact of changing from the initial transport coefficients $D\ix{3}$ to $D\ix{1}$ is reflected here in a similar way. Radiation from C$^{6+}$ is slightly reduced in front of the separatrix, while the next lower ion stage emissions are increased by 5\% in the same spatial range. Again, $P\ix{4+}$ is shifted towards the LCFS similarly to its fractional abundance profile. However, no emissivity can be found in the SOL despite the plateau in $f\ix{4+}$. All other carbon ions behave accordingly to their respective observed differences in fractional abundance, while their individual emissions decrease with the distance to the domain boundary. Most prominently, C$^{3+}$ features a sharper and higher maximum closer to separatrix with larger relative emissivity in the SOL.\\%
                \cref{fig:rad_ratios_total_103_104}:\textbf{(b)} presents the corresponding total radiation power loss for each individual ionization level at the top and the carbon impurity emissions as a whole on the bottom. Profiles and their changes between the different input coefficients are generally in strong agreement with the noted variations in their fractional abundances. For $D\ix{1}$, C$^{5+}$ and C$^{4+}$ have significantly increased emissions only from the core. Both of the next two lower ion stage peaks slightly shift towards the LCFS. Outside the separatrix, C$^{3+}$ decays like before with a constant $\lambda$, while C$^{2+}$ has a greatly more intense second local maximum at $1.05r\ix{a}$. Contrary to the relation between the two $f\ix{1+}$ profiles, $P\ix{1+}$ is larger for $D\ix{1}$. In summary, the variations through increase in C$^{5+}$ and C$^{4+}$ emissions in the core, as well as C$^{4+}$ and C$^{3+}$ close to and beyond the LCFS with their respective shift towards $r\ix{a}$ dominate the changes in $P\ix{tot}$ on the bottom. Around the separatrix, a more than twofold increase can be noted.\\%
                For both parameter variations in the radial diffusivity profile $D$ from \cref{fig:transp_abund_62_66} and \ref{fig:transp_abund_103_104}, a significant increase in $f\ix{a}$ close to and beyond the separatrix for C$^{4+}$ and lower ionization stages can be found. Consequently, this also entails a reduction of C$^{5+}$ and C$^{6+}$ population here. This, in combination with the respective atomic radiation coefficients and the greatly reduced plasma background electron and ion temperature and density, also leads to similar changes in $P\ix{diag}$ and a growth in maximum total radiation power loss inside the LCFS of up to or 120\%. This indicates that a reduction of particle transport in locations of low $T\ix{e}$ and therefore ionization regimes greatly dictates the resulting carbon impurity emissions in shape and absolute value.%
%
            \subsubsection*{Seperatrix Electron Profile Variation}%
%
                \begin{figure}[t]%
                    \centering%
                    \begin{minipage}[b]{0.48\textwidth}%
                        \centering%
                        \subcaptionbox{}{%
                            \includegraphics[width=\textwidth]{%
                                content/figures/chapter3/STRAHL/nete/%
                                compare_ne_Te_82_90_edge.png}%
                        }%
                    \end{minipage}%
                    \hfill%
                    \begin{minipage}[b]{0.48\textwidth}%
                        \centering%
                        \subcaptionbox{}{%
                            \includegraphics[width=\textwidth]{%
                                content/figures/chapter3/STRAHL/fract_abund/%
                                compare_fract_abund_82_90_edge.png}%
                        }%
                    \end{minipage}%
                    \captionof{figure}{Comparison between two sets of input electron profiles and corresponding STRAHL simulation results around the separatrix, both for $f\ix{rad}=100\%$ from experiment XP20181010.32. The Thomson scattering measurement of electron density and temperature which is radially closest to the separatrix was rescaled. \textbf{(a)}: Electron density and temperature profiles from Thomson scattering measurements. \textbf{(b)}: fractional abundances of carbon from STRAHL simulation results, corresponding to the different profiles on the left.}\label{fig:nete_abund_82_90}%
                \end{figure}%
%
                Lastly, a re-scaling of the input electron separatrix temperature and density at the LCFS is carried out. The very last input grid point for both quantities is reduced to, in this particular case, only 10\% of the original value and the intrinsic STRAHL ingest extra-/interpolation is used to yield continuous profiles to calculate off of. In order to achieve reasonable gradients in either kinetic input profile on both sides, the respective numbers at the LCFS are not scaled down directly. The resulting plots can be found, next to their corresponding fractional abundances, in \cref{fig:nete_abund_82_90}.\\%
                At a re-scaled separatrix electron density of 10\%, the spline interpolated profile in the top left features a roughly halved absolute value at the location of the LCFS and a shallower decay in the SOL. Due to the lower $n\ix{e,a}$, and a consequently steeper slope close to the separatrix, the spline fits a slightly higher core electron density towards $0.8r\ix{a}$. Only very minor differences in $T\ix{e}$ profiles between the two scalings can be found, with negligible changes in the core and a slightly smaller local characteristic close to the LCFS. At $r\ix{a}$, the electron temperature is reduced by $\sim$1/3, beyond which the SOL characteristics are virtually indistinguishable. On the right, the fractional abundance of fully ionized C$^{6+}$ increases along the radius for a reduced separatrix electron temperature and density with respect to the initial profile.\\%
%
                \begin{figure}[t]%
                    \centering%
                    \begin{minipage}[b]{0.48\textwidth}%
                        \centering%
                        \subcaptionbox{}{%
                            \includegraphics[width=\textwidth]{%
                                content/figures/chapter3/STRAHL/diag_lines/%
                                compare_strahl_rad_ratios82_90_edge.png}
                        }%
                    \end{minipage}%
                    \hfill%
                    \begin{minipage}[b]{0.48\textwidth}%
                        \centering%
                        \subcaptionbox{}{%
                            \includegraphics[width=\textwidth]{%
                                content/figures/chapter3/STRAHL/diag_lines/%
                                compare_strahl_rad_82_90_edge.png}%
                        }%
                    \end{minipage}%
                    \captionof{figure}{STRAHL simulation results of relative and total impurity radiation from carbon for the previously presented sets of electron density and temperature profiles in \cref{fig:nete_abund_82_90} for $f\ix{rad}=100\%$ from experiment XP20181010.32. \textbf{(a)}: relative radiation intensity for each individual ionisation stage of carbon. \textbf{(b)}: Absolute radiation power and integrated loss for the same ion species.}\label{fig:rad_ratios_total_82_90}%
                \end{figure}%
%
                The conclusive relative emissivities and absolute radiation powers of the individual carbon ionization stages, as well as the total loss through impurity emissions can be found in \cref{fig:rad_ratios_total_82_90}. For a scaled down electron temperature and density input profile, the relative emissivities on the left of all impurity ions are shifted inwards. The relative height of $P_{i}$ inside the separatrix for C$^{2+}$, C$^{4+}$ and C$^{5+}$ are slightly reduced, while those of atomic carbon, C$^{2+}$ and C$^{6+}$ remain the same. However, $P\ix{1+}$ increases. Outside the LCFS, all ions besides C$^{1+}$ are significantly decreased. On the right, the previously observed shift and reduction in intensity inside the separatrix is now true for all ionization stages. A relative shift of \SIrange{2}{5}{\centi\meter} and varying reductions in absolute value can be found, where C$^{3+}$, C$^{2+}$ and C$^{1+}$ feature the largest changes. Outside the LCFS, the emissivity of all contributing species is reduced by an increasing amount the higher the respective value at $r\ix{a}$. This observation is directly reflected in the bottom total radiation power from carbon impurities for a lower separatrix electron temperatures and density.\\%
                Rescaling and thereby decreasing the input electron temperature and density close to the LCFS leads to a significant reduction of total impurity emissions inside and outside the separatrix, as well as a minor inward shift of the individual and hence cumulated carbon radiation power. The observed changes between the scaling in the fractional abundances can be attributed to the corresponding reaction coefficients of the individual ionization stages, i.e. fully ionized carbon is transported further out due to the significantly lower electron density and therefore collision and recombination rate. Furthermore, at the given kinetic inputs and for constant transport coefficients, dictated by the atomic spectral rates, both the total and individual maximum carbon impurity emissivities are displaced towards the plasma center. Most prominently, C$^{3+}$ and C$^{4+}$ experience the largest reduction and shift before and outside the separatrix, while also contributing the strongest to the aforementioned maximum. Coincidentally, the core emissions only exhibit negligible changes. The input and impact of this approach are found to be more akin to the differences between $f\ix{rad}=90\%$ and 100\% than the previous parameter variations.\\%
%
                \newline%
                With respect to the initially posed objective - finding relevant contributors to the local emissivity and LOS sensitivity of the bolometer -, changes in transport coefficients and kinetic profiles around the separatrix have a strong influence on shape and value of the emission profile on the inside and close to the LCFS. Both a reduction in absolute diffusivity and in electron density and temperature close to $r\ix{a}$ impact the impurity radiation in such a way that underlines and supports the findings in \cref{subsec:primew7xfeedback} and \ref{subsec:densityfeedback}, in which the profiles peak shifts inwards, indicating a noticeable level of detachment. These also reflect the experimentally measured changes between the previously exercised radiation fraction stages, where $n\ix{e}$ and $T\ix{e}$ are gradually decreasing with increasing $f\ix{rad}$ and an accompanying reduction of transport in that area.%
%
        \subsection{Impact on Chord Brightness Profile}\label{subsec:strahlchord}%
%
            A final and necessary step of evaluating the one-dimensional impurity transport simulation results from STRAHL is calculating the corresponding chord brightness profiles using the bolometer diagnostics local sensitivities provided through \cref{eq:diff_etendues}, \ref{eq:emissivity} and \cref{fig:los_grid_emiss}. In order to find the two-dimensional radiation distribution $\mathrel{\hat{g}}$ for any given STRAHL profile one takes advantage of the poloidal ($\vartheta$) invariance of the results and therefore extends the emissivity along the angular spectrum by  $2\pi$, thus yielding $\mathrel{\hat{g}}\left(r,\vartheta\right)$. Given the camera geometry matrix $\mathbf{T}\left(r,\vartheta\right)$, the radiation power loss as measured by the bolometer $\vec{b}$ becomes%
%
            \begin{align}%
                \begin{split}\label{eq:forward_lint}%
                    \mathrel{\hat{g}}\ix{sim}\coloneqq\mathrel{\hat{g}}\ix{sim}\left(r,\vartheta\right)\,&\equiv\,\mathrel{\hat{g}}\ix{sim}\left(r\right)\,\,\Rightarrow\,\,\vec{x}\ix{sim}\,\,,\\%
                    \text{chord brightness:}\qquad\qquad\qquad\mathbf{T}\vec{x}\ix{sim}&=\vec{b}\ix{sim}\quad.%
                \end{split}%\
            \end{align}%
%
            \begin{figure}[t]%
                \centering%
                \begin{minipage}[b]{0.48\textwidth}%
                    \centering%
                    \subcaptionbox{}{%
                        \includegraphics[width=\textwidth]{%
                            content/figures/chapter3/STRAHL/%
                            chordal_profile_HBCm_20181010_032.png}%
                    }%
                \end{minipage}%
                \hfill%
                \begin{minipage}[b]{0.48\textwidth}%
                    \centering%
                    \subcaptionbox{}{%
                        \includegraphics[width=\textwidth]{%
                            content/figures/chapter3/STRAHL/forward_lint/%
                            forward_chord_HBCm_00091_00092_00093_00094_min.png}%
                    }%
                \end{minipage}%
                \captionof{figure}{Forward integrated chord brightness profiles for the HBC at different $f\ix{rad}$ \textbf{(a)} from experiment XP20181010.32 and \textbf{(b)} for STRAHL simulation results of total carbon impurity radiation distributions from previously presented electron density and temperature profiles in \cref{fig:nete_abund_lines_91_92} and \cref{fig:nete_abund_lines_93_94_edge} of XP20181010.32.}\label{fig:chord_forward_exp_vs_STRAHL}%
            \end{figure}%
%
            A comparison of the results produced by \cref{eq:forward_lint}, using the horizontal bolometer camera geometry, for the previously employed radiation fraction stages and their corresponding experimentally measured and modelled after profiles can be found in \cref{fig:chord_forward_exp_vs_STRAHL}. This particular camera is chosen because it provides a full and even coverage of the triangular plane. The results for the HBC on the left are found for the same electron temperature and density profiles as the STRAHL calculations on the right are performed with, i.e. of experiment XP20181010.32 and at different radiation fractions. The underlying simulation results correspond to the initially calculated profiles without input parameter variations. In both images, reflections of the opposite side profiles - mirrored at $r=0$ - are superimposed semi-transparently with a dash-dotted line in order to find any asymmetries more easily. The mirrored lines make the discrepancies between left and right half of the profile obvious, with large differences in respective radial position and absolute height in each radiation fraction level. On the left, this peaks location, as produced by the HBC geometry, corresponds to the inboard portion of the separatrix and its lower side and midplane magnetic islands. The right side of the profile is linked to the outboard side and tip of the triangular plane with its two SOL islands.\\%
            The image on the right shows the STRAHL results, created using the same camera geometry as on the left and \cref{eq:forward_lint}. Chord brightness profiles of the same order of magnitude can be found, however with a much more clear separation between the individual $f\ix{rad}$ stages and higher level of symmetry. For a radiation fraction of 33\%, a similar emissivity in the core and up to $\pm\,r\ix{a}$ as on the left is presented. The profiles mirror reflection is congruent with the original on both sides. At $f\ix{rad}=66\%$, the location of the local maxima and the profiles symmetric shape remain the same, though with a generally increased intensity. Going to 90\% leads to a shift of the two maxima towards $\pm\,r\ix{a}$, however still outside in the SOL. Very small, however negligible for now variations between the left and right reflection of the profile can be found around the separatrix. For a radiation fraction of 100\%, both local maxima are now located inside the LCFS and decrease sharply beyond $\pm\,r\ix{a}$. The discrepancy in peak is also reflected by the misalignment of the mirrored profile on both sides. Since the underlying impurity emission distribution calculated by STRAHL and extended in poloidal space is intrinsically symmetric, this can only be explained by minor differences in the LOS geometry of the upper and lower half of the HBC camera.\\%
%
            \begin{figure}[t]%
                \centering%
                \begin{minipage}[b]{0.48\textwidth}%
                    \centering%
                    \subcaptionbox{}{%
                        \includegraphics[width=\textwidth]{%
                            content/figures/chapter3/STRAHL/forward_lint/%
                            forward_chord_HBCm_00082_00090_min.png}%
                    }%
                \end{minipage}%
                \hfill%
                \begin{minipage}[b]{0.48\textwidth}%
                    \centering%
                    \subcaptionbox{}{%
                        \includegraphics[width=\textwidth]{%
                            content/figures/chapter3/STRAHL/forward_lint/%
                            forward_chord_HBCm_00102_00103_00104_min.png}%
                    }%
                \end{minipage}%
                \captionof{figure}{Forward integrated chord brightness profiles for the HBC from STRAHL simulation results of total carbon radiation distributions for \textbf{(a)} different electron density and temperature profiles \textbf{(b)} varying transport coefficient profiles from \cref{fig:transp_abund_62_66} and \cref{fig:transp_abund_103_104}, based on data from experiment XP20181010.32 at $f\ix{rad}\approx100\%$.}\label{fig:forward_neTe_vs_transport}%
            \end{figure}%
%
            Next, in order to further evaluate the impact of the STRAHL input parameter variations, a comparison between the impact of the separatrix $n\ix{e}$ and $T\ix{e}$ reduction and transport coefficient profile changes is presented in \cref{fig:forward_neTe_vs_transport}. The results are achieved in the same way as before, using \cref{eq:forward_lint} and the horizontal bolometer camera geometry.\\%
            On the left, the plot for $T,\,n\ix{e,a}\approx100\%$ is the same as the previously shown chord brightness at $f\ix{rad}=100\%$, while the lower profile of a reduced separatrix temperature and density corresponds to the results in \cref{fig:rad_ratios_total_82_90}. The latters reduced total radiation power is underlined by a lower plasma core emissivity and a roughly halved peak intensity on both sides. However, their maxima location is now further inward around $0.8r\ix{a}$. On the right, the chord brightness profile of diffusion coefficient profile $D\ix{1}$ is akin to the initial results in \cref{fig:chord_forward_exp_vs_STRAHL} at $f\ix{rad}=100\%$, while $D\ix{2}$ and $D\ix{3}$ are calculated using the emissivities in \cref{fig:rad_ratios_total_62_66} and \ref{fig:rad_ratios_total_103_104}. For $D\ix{2}$, the overall emissivity is reduced in the core and at the edges by 50-130\%. The location of the local maxima is unchanged and remains inside the LCFS. With a transport coefficient profile of $D\ix{3}$, the chord brightness as measured by the HBC is greatly reduced to a peaked profile in the plasma center with linear decay to the SOL. No addition localized features can be found here.\\%
            Finalizing the forward modelling of the STRAHL results and their parameter variation using the bolometer camera geometry is \cref{fig:core_v_sol_comparisons}. Both images show the same approach, where the individual carbon impurity ion emissivities are used with \cref{eq:forward_lint} to find their respective chord brightnesses and split them each into a SOL and plasma core portion, i.e. how much of the radiation in $P\ix{chord}$ is located outside or inside the separatrix. Because the conservation of volumes along magnetic field lines is no longer valid outside the separatrix, a simple geometric approach for calculating the impurity emissions can not be used. An experimentally motivated method that also can be used for measured data will be introduced. Let us assume that the LOS cone of channel $j$ has a volume of $V\ix{j}^{\text{core}}$ in the core and $V\ix{j}^{\text{SOL}}$ in the SOL - for example, the outermost HBC channels have no plasma core volume. The respective total powers for camera $C$, i.e. the horizontal bolometer camera, become%
%
            \begin{align}%
                P\ix{core}=\frac{V\ix{P,tor}}{\sum\limits_{j}^{C}V\ix{j}^{\text{core}}}P\ix{rad,C}%
                \,\,,\quad%
                P\ix{SOL}=\frac{V\ix{P,tor}}{\sum\limits_{j}^{C}V\ix{j}^{\text{SOL}}}P\ix{rad,C}\,\,.%
                \label{eq:prad_corevsol}%
            \end{align}%
%
            Both of the above, $P\ix{core}$ and $P\ix{SOL}$ are toroidal extrapolations from the bolometer plane, which implicitly assumes a constant distribution of radiation in the machine. However, this was already true for the initially posed \cref{eq:prad_total} for $P\ix{rad}$, which has generally proven to be valid in most scenarios\cite{Klinger2019,Zhang2018}, particularly with a stronger irradiating plasma core.\\%
            The left of \cref{fig:core_v_sol_comparisons} shows the radiation distribution as calculated using \cref{eq:prad_total} and \cref{eq:prad_corevsol} for the initial STRAHL simulation results at $f\ix{rad}=33\%$, 66\%, 90\% and 100\%. A grey background is added to underline the distinction between core and SOL emission majority, symbolizing the assumed validity in modelling results given the inaccuracy on open field lines (see later). The figure is split into the core (top), SOL (center) and relative (bottom) radiation parts, where the latter only features the total power of the prior two. On the top and in the center, the individual carbon ion emissivities are split into their relative core and SOL portion. One can immediately note that C$^{6+}$ through C$^{4+}$ emissions are entirely, besides a small deviation at $f\ix{rad}=66\%$, represented in the plasma center. The same is true for C$^{3+}$ through C$^{1+}$ and outside the separatrix until a radiation fraction of 66\%. Beyond, $P\ix{SOL}$ of atomic carbon and C$^{1+}$ remains above 95\%, while that of C$^{3+}$ and C$^{2+}$ drop at $f\ix{rad}=100\%$. Consequently, $P\ix{core}$ presents nearly no change for the prior and an increase in the latter two to 30\% and 55\% respectively. At the bottom, the relative integrated impurity radiation from the core and SOL are shown for the same $f\ix{rad}$ as above. Using the same approach as \cref{eq:prad_corevsol} and two prior plots with $P\ix{tot}$ instead of the individual ion emissivities, this plot shows a distinct turnover between plasma core SOL emission from 90\% to 100\% radiation fraction. Initially, $P\ix{core}$ to $P\ix{SOL}$ are distributed nearly evenly until $f\ix{rad}=66\%$. Beyond that, the bolometer measured radiation power loss from the core increases, while the SOL part drops accordingly. At 100\% radiation fraction, this ratio is now inverted and $P\ix{core}=0.7$, $P\ix{SOL}=0.3$ is achieved.\\%
            The plot on the right of \cref{fig:core_v_sol_comparisons} shows the same type of results, i.e. the core and SOL portion of the individual carbon impurity ion emissions and their cumulative total radiation power loss over the different transport coefficient profiles that have been discussed in the previous sections. For all of $D\ix{3}$ through $D\ix{1}$, where $D\ix{3}$ is equal to the diffusion parameter used on the left, C$^{6+}$, C$^{5+}$ and C$^{4+}$ exclusively feature radiation from the core and atomic carbon from the SOL. The next two lower ionization stages C$^{3+}$ and C$^{2+}$ have a $P\ix{core}$ between 45\%-60\% and 20\%-45\%, while their respective plasma emissions are highest for $D\ix{2}$ lowest for $D\ix{1}$. Lastly, C$^{1+}$ shows almost none of its emissions in the core similarly as the next higher charge states. The relative integrated $P\ix{tot}$ at the bottom presents a near linearly declining core radiation and hence increasing relative SOL emissivity. At $D\ix{3}$, the result from the left set of plots is mirrored, from which e.g. $P\ix{SOL}$ grows for $D\ix{2}$ and $D\ix{1}$, while the core radiation decreases conclusively. In this case, the dominance of $P\ix{core}$ over $P\ix{SOL}$ is not reversed as it was before.\\%
%
            \begin{figure}[t]%
                \centering%
                \begin{minipage}[b]{0.48\textwidth}%
                    \centering%
                    \subcaptionbox{}{%
                        \includegraphics[width=\textwidth]{%
                            content/figures/chapter3/STRAHL/core_v_sol/%
                            compare_strahl_corevsol91_92_93_94__full.png}
                    }%
                \end{minipage}%
                \hfill%
                \begin{minipage}[b]{0.48\textwidth}%
                    \centering%
                    \subcaptionbox{}{%
                        \includegraphics[width=\textwidth]{%
                            content/figures/chapter3/STRAHL/core_v_sol/%
                            compare_strahl_corevsol102_103_104__full.png}%
                    }%
                \end{minipage}%
                \captionof{figure}{Individual and total radiation power loss from carbon impurities, separated into core and SOL parts, for \textbf{(a)} different levels of $f\ix{rad}$, which are modelled from electron temperature and density profiles of experiment XP20181010.32, and \textbf{(b)} $f\ix{rad}\sim100\%$ with similar underlying profiles but varying transport coefficients - they are the same as in \cref{fig:transp_abund_62_66} and \cref{fig:transp_abund_103_104}.}\label{fig:core_v_sol_comparisons}%
            \end{figure}%
%
            \newline%
            The comparison between experimental and STRAHL chord brightness profiles in \cref{fig:chord_forward_exp_vs_STRAHL} shows acceptable agreement in absolute intensity at low to high $f\ix{rad}$. However, the stark contrast between the highest radiation fraction emissivities, especially at the center of the plasma indicates inaccuracies in the evaluation of STRAHL profiles using the bolometer geometry for this purpose. A homogenous, poloidal distribution of radiation power, which was assumed for the forward STRAHL calculations, does not apply to the experimental case. The relation between the highest three emission levels in the bolometers' chord brightness profiles, which are all within their respective error margins across the radius, is not supported by the simulation. Furthermore, those experimentally measured lines feature a strong asymmetry, which also can be found for the simulated data, although to a much smaller degree. The high-side inboard located local maximum inside the separatrix though supports the STRAHL results at $f\ix{rad}=100\%$ and their shifted peaks inward of $r\ix{a}$.\\%
            Incorporating the previous separatrix temperature and density and transport coefficient parameter variations with this approach achieves \cref{fig:forward_neTe_vs_transport}. The individual diffusivities show quantitative, though insignificant qualitative differences besides $D\ix{3}$, which yields negligible emissivities overall. A decrease in impurity transport inside the core and close to the separatrix alone does not produce the necessary detachment of plasma radiation from the divertor to inside the LCFS. Reducing the electron profiles $n\ix{e}$ and $T\ix{e}$ at the separatrix and consequently inside the SOL on the other hand produces a smaller, though noticeably further inward shifted chord brightness at $f\ix{rad}=100\%$. Most prominently, a strongly reduced $T\ix{e,a}$ corresponds to a reduced ionization of carbon and a shift in fractional abundances towards lower charge states at that location, which in turn yield a higher emissivity at that temperature.\\%
            Lastly, the results in \cref{fig:core_v_sol_comparisons} further reinforce the latter results and again underline the qualitative shift in relative radiation distribution between the core and SOL. In addition, this alleviates the intrinsic inaccuracy in STRAHL for emissions outside closed flux surfaces, since the majority of $P\ix{rad}$ now comes from inside at $f\ix{rad}=100\%$ and more so for lowered $n\ix{e}$ and $T\ix{e}$ at the separatrix. In turn, this therefore strongly supports the initially posed hypotheses, that at very high radiation fractions in feedback scenarios, plasma radiation detachment was observed by the bolometer and that carbon impurities played a major role. All the above has also reiterated the importance of LOS channels close to and around the separatrix. They especially measure the relevant maxima and shifts, which again provide an important tool to accurately qualitatively and quantitatively assess emissivities. Conclusively, this also expands the so far acquired understanding of feedback radiation scenarios and the local LOS sensitivities towards such plasma.%
%
    \section{Conclusions}\label{sec:conclusionschap3}%
%
        This chapter was introduced with two questions regarding the activation of the gas feedback, the real time bolometer system, its LOS geometry and the underlying plasma parameters. This included a model evaluation for the injection of gas into the SOL and its transport to the plasma and walls. Finally, the one-dimensional impurity transport and radiation code STRAHL was employed to verify and further explore the behaviour of oxygen and carbon emissivities for various parameter combinations, using experimentally measured data and motivated coefficients as input.\\%
        Calculation and comparisons of the LOS selection prediction quality, using a variety of measurements metrics, selection sizes and the previously issued large number of feedback experiments as input data have been performed in this chapter. The many, functionally very different metrics have provided results that point to a higher local sensitivity towards feedback scenarios for channels that view around or at the separatrix, especially in the horizontal bolometer camera. Such lines-of-sight also might focus at the inboard and/or lower side magnetic islands and X-points. Both bolometer camera datasets are generally in good agreement, supporting the individual results and conclusions.\\%
        The one-dimensional impurity radiation and transport simulation code \textit{STRAHL} was employed to model plasma profiles and transport coefficients that yield a significant contribution to local emissivities inside and around the separatrix that correlate to the previous findings of LOS prediction qualities. Electron density and temperature data from the prime bolometer feedback experiment XP20181010.32 and experimentally motivated transport coefficients, as well as results from \textit{ADAS} were used as input for calculation. Initially, oxygen and carbon were considered, where it was found that the latters total radiation emission is insignificant already at very low $f\ix{rad}$. The base model simulation featured a large shift of impurity emissions - from outside the SOL - towards the plasma center when going to very high radiation fractions, as was measured by the bolometer for the very same profiles and $f\ix{rad}$. Input parameter variations in STRAHL have revealed that, in this scenario, both a reduction of diffusivity and electron temperature and density around the separatrix provided such results that underlined the experimental findings of the bolometer diagnostic and the findings of the LOS sensitivity evaluation, where a shift of the maximum emissivity from the SOL to inside $r\ix{a}$ and beyond is found. The experimentally measured and STRAHL simulation profiles therefore together suggest some form of plasma (radiation) detachment for $f\ix{rad}\Rightarrow\,1$.\\%
        Conclusively, a comparison between horizontal bolometer camera measured and calculated chord brightness profiles was performed. The large asymmetry in the experimental data is not reproduced by the forward model results, where the poloidal extrapolation of the radial STRAHL profile yields only negligible variations due to the minor differences in the underlying HBC LOS geometry. This indicates that there is both a significant inadequacy in the forward modelling approach and a stronger poloidal asymmetry in the actual plasmas' emissivity, i.e. a higher power loss closer to the VBC and lower inboard X-points where the magnetic field lines from the feedback gas valves connect to. The simulated chord brightness provides the desired and experimentally recorded inward shift of the maximum emissivity - though the model does so from outside the separatrix to inside, while the bolometer finds all peaks there - and the corresponding high-side inboard located local maximum inside the separatrix for very high $f\ix{rad}$. Incorporating the parameter variations in STRAHL further underlined the previous assessments. Evaluations of the core and SOL radiation distribution from the forward model profiles gave additional credibility to this application of a one-dimensional impurity transport code. The application of STRAHL here thereby supports the initial hypotheses of feedback driven plasma radiation detachment towards very high radiation fractions, while also providing a prominent contributor for such scenarios in carbon impurities. Previously recorded LOS sensitivities are underlined by the latter chord brightness comparisons due to the correspondingly positioned local maxima and shifts in their respective profiles.\\%
        Circling back to, at the beginning of this chapter posed questions, one can answer confidently as follows:%
%
    \begin{enumerate}%
        \item[1.]{%
            %Does an optimal set of lines of sight $S$ for the real time bolometer feedback system exist?}%
            No explicitly best set of 3, 5 or 7 LOS for finding $P\ix{rad}$ exists. However, for feedback purposes, an established and robust selection of few vertical or/and horizontal bolometer camera channels can be provided that generally can achieve at least 85\% prediction accuracy when compared to the full data set. Such LOS are among those looking at inboard and low-side located X-points and magnetic islands or along the separatrix. To formulate this in exclusivity: detectors viewing only the outermost plasma edge or, in smaller sets, core are not suited for feedback purposes.}%
        \item[2.]{%
            %What is the dominant contributor towards the line of sight selection sensitivity of the real time bolometer feedback system?}%
            Carbon impurity is found to be a significant contributor towards the observed crucial changes in emission distribution for plasma radiation detachment scenarios that are aimed for by the application of the feedback system. Similarly important is the reduction of both separatrix and SOL electron temperature and density and the diffusion profile in this location.}%
    \end{enumerate}%