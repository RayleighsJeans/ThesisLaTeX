%
\checkoddpage\ifoddpage\clearpage\else\cleardoublepage\fi%
%
\section*{Thesis Overview}%
%
    The importance of impurities, both intrinsic and extrinsic towards not only the overall performance of a fusion plasma but explicitly its control and the machine safety is well-established. Exploration of radiative power exhaust scenarios corresponding to detachment physics and its research of deliberate and feedback controlled impurity seeding is of great interest. This thesis aims to establish a diagnostic framework for and conclusively implement a real-time global radiation power loss feedback system with the bolometer diagnostic for achieving stable detachment through low-Z impurity thermal gas seeding. Based on experimental achievements thereof, a multidimensional, statistical parameter analysis is performed, focusing on the sensitivity of the application with respect to emissivity distributions and plasma scenarios. At last, a custom tailored tomographic inversion algorithm is benchmarked using artificial radiation profiles and its robustness under geometric perturbations evaluated.%
%
    \paragraph*{Bolometry of Fusion Plasmas}%
%
       This chapter introduces and focuses on the bolometer diagnostic system implemented at the Wendelstein 7-X stellarator, covering the requirements, construction, and operational principles used to measure spatial and temporal evolution of plasma radiation. It further details the specific challenges faced in the fusion environment and describes the used metal resistor type, emphasizing its advantages and limitations within this context. The bolometer system at Wendelstein 7-X is highlighted for its reliability and capability to withstand extreme conditions and provide critical data for power balance and transport studies.%
%
    \paragraph*{Plasma Radiation Feedback Control}%
%
        This chapter focuses on the configuration and operational details of a real-time radiation feedback control system developed for Wendelstein 7-X. It outlines the system's design, performance, and experimental achievements, showcasing how the system contributes to controlling plasma radiation levels. The chapter illustrates the feedback system's impact on plasma parameters under the influence of bolometer data and its comparison with other feedback control strategies, emphasizing the significance of maintaining optimal plasma conditions and enhancing the overall efficiency of fusion experiments.%
%
    \paragraph*{Feedback Impact and Line of Sight Sensitivity Analysis}%
%
        Chapter \ref{chap:feedbackeval} presents a comprehensive analysis of the feedback controls' impact on plasma parameters, including impurity seeding modelling and line of sight sensitivity evaluation. It discusses the theoretical predictions and experimental validation of feedback effects on plasma behaviour, offering insights into the optimization of impurity seeding strategies. The chapter also delves into the sensitivity analysis of bolometer camera lines of sight and their implications for accurate measurement and control of plasma radiation, highlighting the use of STRAHL modelling to understand and mitigate feedback-related challenges.%
%
    \paragraph*{Two-dimensional Radiation Inversion}%
%
        The focus of \cref{chap:inversions} is on the techniques and challenges associated with two-dimensional radiation profile inversion in fusion plasmas with regard to the W7-X bolometer. It explores the application of minimum Fisher regularization and discusses the sensitivity of camera geometry to line of sight perturbations. The chapter introduces phantom radiation profiles to assess the inversion process's accuracy and discusses the tomography of experimental data. It emphasizes the importance of accurate radiation profile inversion for understanding plasma behaviour.%
%