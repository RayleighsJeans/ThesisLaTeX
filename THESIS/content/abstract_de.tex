
\checkoddpage\ifoddpage\clearpage\else\cleardoublepage\fi%

    \chapter*{Zusammenfassung}%

        Strahlungsverluste im Rahmen der Leistungsabfuhr sind für den Betrieb bestehender und die Entwicklung künftiger Fusionsreaktoren von großer Bedeutung. Die gezielte Verunreinigung des Plasmas mit Fokus auf den Ablöseprozess, bei dem sich das ionisierte Gas von der physischen Wand des Reaktorgefäßes entfernt, für die Abfuhr von Wärmeenergie und zum Schutz der Maschine ist von großem wissenschaftlichen Interesse. Eine zuverlässige und stabile, kontrollierte Strahlungskühlung im magnetisch eingeschlossenen Bereich, im Plasmarand und nahe dem Divertor mit einem Verhältnis zwischen Strahlungs- und Heizleistung ($f\ix{rad}$) von $\ge95\%$ ist daher notwendig.\\%
        Während OP1.2b wurde am Stellarator Wendelstein 7-X (W7-X) ein Rückkopplungssystem des Bolometers mit Echtzeit-Funktion für die Gasinjektion von Verunreinigungen mit niedriger Kernladungszahl entworfen und implementiert. Damit wurde ein stabiles Wasserstoffplasma mit kontrollierter Strahlungskühlung bei $f\ix{rad}\ge85\%$ erreicht. Eine Verringerung der Wärmebelastung des Divertors um mindestens den Faktor zwei wurde gleichzeitig gemessen, während die Ablösung der Strahlung aus C$^{3+}$-Ionen bereits für $f\ix{rad}\sim50\%$ sichtbar war. Ein Satz von mindestens drei Sichtlinien wurde als Proxy für schnelle Extrapolationen der Strahlungsverluste validiert.\\%
        Gasestösse von moderater quantitativer Stärke und Länge waren im Allgemeinen zuverlässig in der Lage Strahlungskühlung ohne terminale Plasmastörungen zu bewirken. Rudimentäre Modelle für diesen Injektionsprozess konnten beispielhafte Strahlungsmessungen von Rückkopplungsexperimenten darstellen. Detektoren, welche die Separatrix- und Plasmarand-Region betrachten, erwiesen sich als am brauchbarsten für diesen Ansatz. Eine Auswahl von drei bis sieben vertikalen und/oder horizontalen Bolometerkanälen erreichen eine Vorhersagegenauigkeit von $\ge85\%$ im Vergleich zum vollständigen System. Der eindimensionale Transport- und Strahlungssimulationscode STRAHL wurde in Kombination und zum Vergleich mit experimentellen Daten aus Rückkopplungsszenarien eingesetzt. Verunreinigungen durch Kohlenstoff stellen hier den größten Beitrag an Emissionen am Plasmarand dar, während die Verringerung der Diffusivität, sowie der Elektronentemperatur und -dichte um die Separatrix zu Experiment-ähnlichen Strahlungsprofilen führte. Eine Verschiebung von außerhalb nach innerhalb der Separatrix war für $f\ix{rad}\rightarrow1$ zu beobachten.\\%
        Die Zuverlässigkeit eines optimierten Minimum Fisher Tomographie Algorithmus in Kombination mit dem W7-X-Bolometersystem wurde durch rigorose Störungstests der Geometrie nachgewiesen. Ein Benchmark dieser Methode hat gezeigt, dass es einen idealen Satz von radialen Gewichtungskoeffizienten $k\ix{ani}$, welche für die Einstellung der Regularisierung genutzt werden, für eine gegebene Emissivitätsverteilung gibt. In der Transmissionsmatrix $\mathbf{T}$ wurde eine intrinsische Asymmetrie in Richtung des oberen SOL- (\textit{"scrape-off layer"}) und Separatrixbereichs festgestellt. Variationen des Gewichtungsfaktors haben gezeigt, dass $k\lessgtr1$ in der Rekonstruktion gut mit lokalisierten bzw. glatten Strahlungsstrukturen korreliert. Die Auswertung von Phantombildern mit zweidimensionalen Qualitätskoeffizienten zeigte eine signifikante Diskrepanz zwischen diesen und $\chi^{2}$. Bei den künstlichen Strahlungsbildern unterschätzten die errechneten $P\ix{rad}$ konsequent die tatsächliche, darin enthaltene Gesamtstrahlungsleistung um $\sim30\%$.%
%