\documentclass[final]{beamer}
% beamer 3.10: do NOT use option hyperref={pdfpagelabels=false}!
% don't show navigation symbols
% 2016 A. Kleiber
% 2018 A. Kleiber

\beamertemplatenavigationsymbolsempty
% \usepackage{2019_05_beamerouterthemeposterasdex}
\usepackage{beamerouterthemeposterw7x}
\usepackage[english]{babel}
\usepackage[latin1]{inputenc}
\usepackage{tcolorbox}
\usepackage{amsmath, amsthm, amssymb, latexsym, stackrel}
\usepackage{array}
\usepackage{units}
\usepackage{siunitx}
\usepackage{mathtools}

% Arial-Schrift
%\usepackage{ipp}
%\usepackage[scaled]{uarial}
%\usepackage{times}\usefonttheme{professionalfonts}  % times is obsolete
%\usefonttheme[onlymath]{serif}

\boldmath
\usepackage[orientation=portrait,size=a0,scale=1.4,debug]{%
    beamerposteripp}  % e.g. for DIN-A0 poster

%\usepackage[orientation=portrait,size=a1,scale=1.4,grid,debug]{%
%   beamerposter}  % e.g. for DIN-A1 poster, optional grid and debug output
%\usepackage[size=custom,width=200,height=120,scale=2,debug]{%
%   beamerposter}  % e.g. for custom size poster
%\usepackage[orientation=portrait,size=a0,scale=1.0,%
%            printer=rwth-glossy-uv.df]{beamerposter}
% e.g. for DIN-A0 poster with rwth-glossy-uv printer check

\setbeamertemplate{bibliography item}{\insertbiblabel}
% Damit im Literaturverzeichnis Zahlen stehen.
\setbeamertemplate{itemize item}[triangle]

% title of the presentation
\title{%
    Local Radiated Power Sensitivity and Intrinsic Impurity Correlation %
    Analysis at the Stellarator Wendelstein 7-X}%
\author[Author]%
    {\underline{P.~Hacker\inst{1,2}*},%
     ~D.~Zhang\inst{1},~F.~Reimold\inst{1},~%
     ~R.~Burhenn\inst{1} %
     and T.~Klinger\inst{1},~%
     for the Wendelstein 7-X Team-Collaboration\inst{1}}
% email address of the corresponding author

% Definieren mail corresponding author
\newcommand{\emailcorrespondingauthor}{philipp.hacker@ipp.mpg.de}
% Definieren Name Conference
\newcommand{\nameconference}%
    {Deutsche~Physikalische~Gesellschaft~-~%
    Spring~Meeting~Munich~2019,~%
    Matter and Cosmos Section (SMuK)}

\newcommand{\diff}{\text{d}}
\newcommand{\tenpo}[1]{\cdot 10^{#1}}
\newcommand{\ix}[1]{_\text{#1}}
\newcommand{\imag}{\mathbf{i}}
\newcommand{\fett}[1]{\textbf{#1}}
\newcommand{\tilt}[1]{\textit{#1}}
\newcommand\inlineeqno{\stepcounter{equation}\ \quad\quad(\theequation)}

\institute[]{%\hspace{5cm}
    \inst{1}%
        Max~Planck~Institute~for~Plasma~Physics,~Wendelsteinstr.~1,~%
        D-17491~Greifswald,~Germany,\\%
    \inst{2}%
        Ernst-Moritz-Arndt~University~Greifswald,~%
        Rubenowstr.~1,~D-17489~Greifswald,~Germany}
% set date of the talk
\date{09.03.2020}

\begin{document}\begin{frame}\frametitle{}%
    \begin{minipage}[t]{0.39\textwidth}%
        \vfill%
        %\vspace*{.0cm}%
        \begin{kasten}{\large%
            Bolometer Diagnostic}% at Wendelstein 7-X}%
            \vspace*{.25cm}%
            \large{{\color{ipp}\fett{%
                Goals}}}%
            \vspace*{.25cm}%
            \small{\begin{itemize}%
                \item{%
                    investigate total radiation powerloss through %
                    impurities and its distribution}%
                \item{%
                   global \& local power balance, as well as impurity %
                   and transport studies through tomographic %
                   inversion}%
                \color{red}{\item{%
                   real time plasma feedback control based off of the %
                   radiation power loss and its distribution to %
                   achieve improved detachment, adjust thermal loads of %
                   in-vessel components \& explore high radiation scenarios}}%
            \end{itemize}}%
            \vspace*{.25cm}%
            \large{{\color{ipp}\fett{%
                Motivation}}}%
            \vspace*{.25cm}%
            \small{\begin{itemize}%
                \item{%
                    averaged thermal load of in-vessel components %
                    expected to be up to %
                    \mbox{\SI{100}{\kilo\watt/\square\meter}} mainly by %
                    radiation and non-absorbed heating power}%
                \item{%
                    calculating temporal and spatial evolution of %
                    the radiation loss previously only after the plasma %
                    has been terminated}
                \item{%
                    investigate radiation scaling, i.e. %
                    importance of intrinsic \& extrinsic impurities %
                    and their location}%
            \end{itemize}}
            \vspace*{.25cm}%
            \large{{\color{ipp}\fett{%
                Design}}}%
            \vspace*{.25cm}\small{%
            \begin{itemize}%
                \item{%
                    multi-device system: horizontal bolometer camera %
                    (HBC, 32 channels) and vertical bolometer camera %
                    (VBC, 20 channels for each of two subdetectors)%
                    \newline$\Rightarrow$ more detectors with different %
                    filters/coatings available, e.g. for investigation %
                    of soft x-ray radiation}%
                % \item{%
                %     fan-shaped lines of sight provide full plasma %
                %     coverage at $\SI{5}{\centi\meter}$ spatial resolution %
                %     at magnetic axis}
                \item{%
                    steady state operation at discharges with up to %
                    $\SI{30}{\minute}$ of $\SI{10}{\mega\watt}$ heating %
                    power ensured by cooling system with %
                    graphite elements and water cooling structures}
                \item{%
                    detectors are carbon coated Au-foil %
                    on $\SI{5}{\micro\meter}$ Si$_{3}$N$_{4}$ %
                    substrate, backed by a $\SI{30}{\micro\meter}$ %
                    platin meander with a $\SI{0.25}{\milli\second}$ %
                    response time; temporal resolution of %
                    $\SI{0.8}{\milli\second}$ to $\SI{6.4}{\milli\second}$}%
            \end{itemize}}%
            \vspace*{.25cm}%
            \begin{columns}%
                \column{0.48\textwidth}%
                    \centering{\small{\color{ipp}%
                        \textbf{\underline{%
                            HBC}}}}%
                \column{0.48\textwidth}%
                    \centering{\small{\color{ipp}%
                        \textbf{\underline{%
                            VBCr/VBCl}}}}%
            \end{columns}%
            \vspace*{.25cm}%
            \begin{center}%
                \includegraphics[width=0.7\textwidth]%
                    {figures/content/linesofsight_in_vessel.pdf}%
                \vspace*{.5cm}%
                \newline\tiny{%
                    LOS for HBC (32 ch.) and VBC (two 20-ch. subdetector %
                    arrays) with individual apertures, retracted into %
                    the vacuum vessel behind wall elements; located in %
                    the triangle-shaped plane at W7-X.\cite{Zhang2010}}%
                \newline%
            \end{center}
        \end{kasten}%
        \vspace*{.25cm}%
        \begin{kasten}{Equations}\vspace*{.25cm}\small{%
            The radiation power observed by the bolometers equals to:
            $$P\ix{rad,bolo}\propto\sum\ix{Z} n\ix{e}\cdot %
                n\ix{Z}\cdot L\ix{Z}\left(T\ix{e}, T\ix{i}, T\ix{Z},%
                \,\,\dots\right)$$%
            where $L\ix{Z}$ is the line radiation function by %
            species $Z$. For each channel the observed %
            power $P\ix{ch}$ can be calculated by %
            using\cite{Gianone2002}:%
            $$P\ix{ch}=F\ix{ch}\cdot\left(\tau\ix{ch}%
                \frac{\diff(\Delta U)}{\diff t}+%
                f_{\tau, \text{ch}}\cdot(\Delta U)\right)$$%
            with $\Delta U\propto\Delta T\propto\Delta P$ the change in %
            measurement voltage, absorber temperature and incident radiation %
            power. Properties denoting $(\cdot)\ix{ch}$ refer to the %
            individual channel/foil characteristics, e.g. cooling %
            time ($\tau$) and $f_{\tau, \text{ch}}$, $F\ix{ch}$ %
            numbers calculated from cable attributes, detector %
            resistance and heat capacity.}%
            \vspace*{1.cm}\\%
            \normalsize{\color{ipp}\fett{\underline{%
                Global Power Estimate:}}}%
            \vspace*{.5cm}\\%
            \small{%
                For each camera (VBC, HBC) individually, the total %
                radiation loss can be calculated like:%
                $$P\ix{rad, cam}=\frac{V\ix{P,tor}}{V\ix{cam}}\cdot%
                    \sum\ix{ch}^{\text{cam}}\frac{V\ix{ch}}{K\ix{ch}}%
                    \cdot\frac{P\ix{ch}}{53\%}$$%
                \quad\quad with:%
                $$V\ix{cam}=\sum\ix{ch}V\ix{ch}\,\,.$$%
                The volume and geometry of the detectors lines %
                of sight and corresponding aperture %
                are noted as $V\ix{ch}$, $K\ix{ch}$ hence %
                $V\ix{cam}$ is the total volume investigated %
                by a camera. Using EMC3-Eirene simulation the %
                estimated plasma volume from which radiation is %
                emitted is approximated to be $V\ix{P, tor}$.}%
            \vspace*{1.cm}\\%
            \normalsize{\color{ipp}\fett{\underline{%
                Real Time Prediction:}}}%
            \vspace*{.5cm}\\%
            \small{%
                Due to technical limitations, feedback was only %
                possible to be calculated based off of a selection S %
                of lines of sights instead of a full array. %
                $\widetilde{P\ix{ch}}$ notes a %
                10-sample average to suppress noise without %
                sacrificing temporal responsiveness:%
                $$P\ix{pred}=P\ix{rad,S}=\frac{V\ix{P}}{V\ix{S}}\cdot%
                    \sum\ix{ch}^{\text{S}}\frac{V\ix{ch}}{K\ix{ch}}%
                    \cdot\frac{\widetilde{P\ix{ch}}}\inlineeqno\label{eq:predict}\,\,.$$%
                }%
        \end{kasten}%

        \begin{kasten}{References}%
            \fontsize{12}{12}{%
            \begin{thebibliography}{}%
                \bibitem{Zhang2010} %
                    "Design Criteria of the Bolometer diagnostic %
                    for steady-state operation of the W7-X stellarator"; %
                    Zhang, D. et al.; %Burhenn, R., Koenig, %
                    %R., Giannone, L., Grodzki, P.A., Klein, B., %
                    %Grosser, L., Baldzuhn, J., Ewert, K., %
                    %Erckmann, V., Hirsch, M., Laqua, H.P., %
                    %Oosterbeek, J.W.; %
                    Review of Scientific Instruments, %
                    Jan 1st, 2010; DOI:10.1063/1.3483194
        %         \bibitem{Zhang2016} %
        %             "The bolometer diagnostic at stellarator %
        %             Wendelstein 7-X and its first results in the %
        %             initial campaign"; D. Zhang, et al. %
        %             %R. Burhenn, A. Alonso, B. Buttenschön, %
        %             %Y. Feng, L.Giannone, M.Hirsch, U.Höfel, R.%
        %             %Lauber, M.Marquardt, K.Rahbarnia, J.Svensson, 
        %             %G.A.Wurden, R.Brakel, O.Grulke, J.Knauer, R.
        %             %König, H.Laqua, S.Marsen T.Stange, T.%
        %             %Schröder, H.Thomsen, G.M. Weir, A.Werner and %
        %             and the W7-X Team; Stellarator-New 2017
        %         \bibitem{Mast1991} %
        %             "A low noise highly integrated bolometer array %
        %             for absolute measurement of VUV and soft x %
        %             radiation"; K. F. Mast et. al; %
        %             % J. C. Vallet, C. Andelfinger, %
        %             % P. Betzler, H. Kraus, and %
        %             %G. Schramm;%
        %             Review of Scientific Instruments 62, 744 (1991);
        %             DOI: 10.1063/1.11.42078%
        %         \bibitem{VMEC} %
        %             "Steepest descent moment method for three %
        %             dimensional magnetohydrodynamic equilibria"; %
        %             Hirshman, S.P. et al.; %Whitson, J.C.; %
        %             Physics of Fluids 26, 3553, (1983); %
        %             DOI: 10.1063/1.864116%
        %         \bibitem{Wesson} %
        %             "Tokamaks"; Wesson, J.; Clarendon Press, Oxford; 1987%
        %         \bibitem{Feng} %
        %             "Numerical investigation of plasma edge %
        %             transport and limiter heat fluxes in %
        %             Wendelstein 7-X startup plasmas with %
        %             EMC3-EIRENE"; %
        %             Effenberg, F., Feng, Y. et al. %
        %             Nucl. Fusion 57 (2017) 036021 (15pp); %
        %             DOI: 10.1088/1741-4326/aa4f83%
                \bibitem{Gianone2002} %
                    "Derivation of bolometer equations relevant to %
                    operation in fusion experiments"; %
                    Gianone, L. et al.; Review of Scientific Instruments; %
                    20th of November, 2002; DOI: 10.1063/1.1498906%
        %         \bibitem{Zhang2018} %
        %             "Results of the bolometer diagnostic at %
        %             OP 1.a of W7-X"; internal review of the %
        %             physics plan during the %
        %             second operational phase at the stellarator %
        %             W7-X; 28.02.2018%
        %         \bibitem{Yamada2005} %
        %             "Characterization of energy confinement in %
        %             net-current free plasmas using the %
        %             extended International Stellarator Database"; %
        %             H. Yamada et al.; %
        %             INSTITUTE OF PHYSICS PUBLISHING and %
        %             INTERNATIONAL ATOMIC ENERGY AGENCY; %
        %             Nucl. Fusion 45 (2005) 1684¿1693%
            \end{thebibliography}}%
        \end{kasten}

    \end{minipage}%
    \hfill%
    \begin{minipage}[t]{0.6\textwidth}%
        \vfill%
        %\vspace*{.0cm}%
        \begin{kasten}{Real Time Feedback}%
            %\vspace*{0.5cm}%
            \begin{columns}%
                \column{0.49\textwidth}\small{%
                    \begin{itemize}%
                        \item{%
                            during last experiment campaign \tilt{OP1.2b}: %
                            real time plasma feedback control using %
                            in-situ calibrations %
                            and measurements of radiation loss distribution}%
                        \item{%
                            actuator is fast, thermal He-beam valve - %
                            gas flow adjusted for target %
                            where $P\ix{rad}\sim f(n\ix{e}, T\ix{e}, \dots)$.}%
                        \item{%
                            during experimental campaign used %
                            selection $\widetilde{S}_{5}$ based on an %
                            \tilt{educated guess} with 5 channels covering %
                            the plasma core, edge and %
                            \tilt{scrape-off layer} SOL}%
                    \end{itemize}}%
                    \vspace*{0.25cm}%

                    \begin{center}
                        \includegraphics[width=0.8\textwidth]%
                            {figures/content/%
                             overview20181010_032_prad_fb_qsq_div.png}%
                        \vspace*{.25cm}%
                        \newline\tiny{%
                            \fett{(top):} %
                            Comparison of input heating power $P\ix{ECR}$ and %
                            radiation power loss, as measured by the two %
                            camera arrays. Added are also the two feedback %
                            lines provided for the He beam valves. One %
                            features a predictive calculation like %
                            eq. (1), whereas the other uses one %
                            raw single channel signal multiplied by a %
                            manually adjusted factor. %
                            \fett{(middle):} %
                            Valve actuation of the thermal He beam feedback %
                            actuator. This signal indicates whether and how %
                            far the valve has been opende to fuel hydrogen %
                            H$_{2}$ into the plasma. %
                            \fett{(bottom):} %
                            Individual divertor module heat loads in W7-X (colored lines) and total integrated power.}%
                        \newline%
                    \end{center}%

                    \begin{kasten3}{Post-Feedback Sensitivity Evaluation}%
                        \vspace*{.25cm}
                        \small{\centering{\color{red}{%
                            What combination of or individual channels $S$ %
                            would have yielded the best %
                            feedback performance?}}%

                        \vspace*{0.25cm}\flushleft%
                        Using the prediction by a set $S$ from %
                        eq. (1) one can define a %
                        \tilt{weighted (normalised) deviation}-like %
                        cost function as:%
                        $$d\ix{S}(t)=\|P\ix{rad, cam}(t)-%
                            P\ix{pred, S}(t)\|$$%
                        $$\varepsilon\ix{S}(t)=%
                            \left\{\begin{array}{ll}%
                            1-\frac{d\ix{S}(t)}{P\ix{rad, cam}(t)}&,%
                                \,\,d\ix{S}<P\ix{rad, cam}\\%
                            0&,\text{ else}
                            \end{array}\right\}\inlineeqno\label{eq:eps}$$%
                        $$\vartheta\ix{S}=\int_{T}%
                            \varepsilon\ix{S}(t)\diff t%
                            \inlineeqno\label{eq:theta}$$%

                        \begin{center}
                            \includegraphics[width=.8\textwidth]%
                                {figures/content/%
                                 wghtd_dev_C[_5_16_27].png}%
                            \vspace*{.25cm}%
                            \newline\tiny{
                                Comparison of $P\ix{rad}$ from XPID: %
                                20181010.032 as calculated %
                                from the LOS of the full HBC array and a %
                                trace example for the subset %
                                $S=\{5, 16, 27\}$ $P\ix{pred, S}$. %
                                The purple line %
                                is calculated using eq. (2), while %
                                the number $\vartheta\ix{HBC, S}$ is produced %
                                by eq. (3).}%
                            \newline%
                        \end{center}%

                        \flushleft%
                        For example for $N_{3}\sim10^{4}$ subsets $S\ix{3}$ of %
                        $n=3$ lines of sight $\vartheta\ix{HBC, S}$ has been %
                        calculated. Let $N^{\text{ch}}\ix{n}$ be the number of %
                        $S^{\text{ch}}\ix{n}$ where %
                        detector $ch$ is incoorporated, the %
                        \tilt{average sensitivity of %
                        channel ch} becomes:%
                        $$\Omega^{\text{n}}\ix{ch}\,\,=\,\,%
                            \frac{1}{N^{\text{n}}\ix{ch}}\,\,%
                            \sum_{s(ch)}^{N}%
                            \,\,\vartheta\ix{HBC, s(ch)}%
                            \inlineeqno\label{eq:avsense}$$

                        \begin{center}%
                            \includegraphics[width=.8\textwidth]%
                                {figures/content/%
                                 spectrum_analysis_weighted_%
                                 deviation_C3HBCm.png}%
                            \vspace*{.25cm}%
                            \newline\tiny{
                                Results from eq. (4) for %
                                $S_{3}$ and XPID: 20181010.032. %
                                The abscissa is taken as the minimum %
                                effective plasma radius along the LOS of a %
                                detector. The combinatory space for the %
                                subsets has been restricted to reduce %
                                computational excess, which yields three %
                                distinguishable ranges on the left, %
                                center and right.%
                            }\newline%
                        \end{center}%
                    }\end{kasten3}%

                \column{0.49\textwidth}%
                    \begin{kasten4}{%
                        STRAHL Simulations of Carbon Radiation}%
                        \begin{center}%
                            \includegraphics[width=.7\textwidth]%
                                {figures/content/%
                                 radiational_fraction_%
                                 20181010_032_edge_top.png}%
                            \vspace*{.25cm}%
                            \newline\tiny{%
                                ECR heating power and $P\ix{rad,HBC}$ for %
                                radiation feedback controlled W7-X experiment %
                                XPID: 20181010.032. Indicated also the %
                                corresponding radiation power loss fraction %
                                $f\ix{rad}=P\ix{ECR}/P\ix{rad}$. For %
                                different radiation loss regimes %
                                $f\ix{rad}$ has been marked.}%
                        \end{center}%
                        \vspace*{.25cm}%
                        \begin{center}%
                            \includegraphics[width=.7\textwidth]%
                                {figures/content/%
                                 chordal_profile_HBCm_20181010_032.png}%
                            \vspace*{.25cm}%
                            \newline\tiny{%
                                Chordal brightness profiles for the HBC %
                                camera array, noted over the minimum %
                                $r\ix{eff}$ along the lines of sight, %
                                for points in time taken from the %
                                radiation fraction on the top.}%
                        \end{center}%

                        \small{\begin{itemize}%
                            \item{%
                                line-int. chordal profile shows %
                                majority of radiation coming from region close %
                                to separatrix or SOL}%
                            \item{%
                                increasing radiation fraction shows inward %
                                shift of brightness away from last closed %
                                fluxsurface}%
                        \end{itemize}}%
                        \vspace*{.25cm}%
                        \centering{\color{red}{%
                            What causes this behaviour given %
                            the 1D radiation distribution %
                            and plasma profiles?}}%
                        \vspace*{.25cm}%

                        \flushleft{\normalsize{\color{ipp}\fett{\underline{%
                            STRAHL:}}}}\\%
                        \vspace*{.25cm}%
                        \small{\begin{itemize}%
                            \item{%
                                assuming 1D distribution, majority of %
                                radiation coming from inside LCFS}%
                            \item{%
                                impurity transport \& radiation in %
                                coronal equilibrium modelled using %
                                \tilt{STRAHL} code and \tilt{ADAS} %
                                atomic database}%
                            \item{%
                                calculating radial transport $\Gamma\ix{i,Z}$ %
                                and emission of impurity $i$ and ion-stage %
                                $Z$ solving continuity equation %
                                using ansatz of anomalous diffusivities %
                                $D^{*}$ and radial drift velocities $v^{*}$:}%
                        \end{itemize}}
                        \begin{align}%
                            \frac{\partial n\ix{i,Z}}{\partial t}=%
                                &-\nabla\,\Gamma\ix{i,Z}+Q\ix{i,Z}\nonumber\\%
                                =&\frac{1}{r}\frac{\partial}{\partial r}r\left(%
                                D^{*}\frac{\partial n\ix{i,Z}}{\partial r}-%
                                v^{*}n\ix{i,Z}\right)+Q\ix{i,Z}\nonumber%
                        \end{align}
                        \vspace*{.25cm}%

                        \begin{center}%
                            \includegraphics[width=.7\textwidth]%
                                {figures/content/%
                                 compare_ne_Te_93_94_edge.png}%
                            \vspace*{.25cm}%
                            \newline\tiny{%
                                \tilt{Thomson Scattering} profiles for cases %
                                of high $f\ix{rad}$, as depicted in the %
                                top-page graph on XPID: 20181010.032, as %
                                well as spline-interpolated smooth traces %
                                for STRAHL input with exponentially decaying %
                                density \& temperature beyond $r\ix{LCFS}$.}%
                        \end{center}%
                        \vspace*{.25cm}%
                        \begin{center}%
                            \includegraphics[width=.7\textwidth]%
                                {figures/content/%
                                 compare_strahl_rad_93_94_edge_bottom.png}%
                            \vspace*{.25cm}%
                            \newline\tiny{%
                                Line radiation for all ion stages %
                                C$^{\text{X}+}$ of carbon in the coronal %
                                equilibrium according to the two radiation %
                                regimes shown above. When integrated for the %
                                plasma volume of W7-X the total radiation %
                                matches experimental levels.}%
                        \end{center}%
                    \end{kasten4}%

                    \begin{kasten2}{Conclusions}\small{%
                        \begin{itemize}%
                            \item{%
                                benchmarks using eq.\,(1) on different %
                                scenarios, cost metrics and camera/channels %
                                subsets (up to $n=9$) show similar results}
                            \item{%
                                Bolometer most sensitive to changes %
                                in radiation distribution along separatrix %
                                and SOL}%
                            \item{%
                                sensitivity analysis in STRAHL input %
                                parameters yields small changes in %
                                $P\ix{diag}$}%
                            \item{%
                                STRAHL shows strong radial dependence of %
                                intrinsic impurity radiation regarding %
                                temperature profile input}%
                            \item{%
                                carbon radiation possible indicator for %
                                regimes of detachment as main power sink}%
                            % \item{%
                            %     need 2D inversion of radiation distribution %
                            %     to validate results and extend investigations}%
                        \end{itemize}%
                    }\end{kasten2}%

            \end{columns}%
        \end{kasten}%
        \vfill%
    \end{minipage}%
    %\hfill%

\end{frame}\end{document}

        % QR AND SHIT
        % \vspace*{.5cm}%
        % \begin{columns}%
        %     \column{0.7\textwidth}%
        %         \centering{%
        %             \normalsize{\color{ipp}{%
        %               You want to look at this poster later or get %
        %                 in contact with the author?}}
        %         }
        %     \column{0.2\textwidth}%
        %         \centering{%
        %             \includegraphics[width=0.7\textwidth]%
        %                 {figures/header/Posters.png}%
        %         }
        % \end{columns}