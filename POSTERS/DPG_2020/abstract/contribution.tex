\documentclass{scdpg}
%
\usepackage[utf8]{inputenc} % Required for inputting international characters
\usepackage[T1]{fontenc} % Output font encoding for international characters
%
\usepackage{units}
\usepackage{siunitx}
%
\begin{document}
%
\scBookLanguage{de}
\begin{scAbstract}
%
\scLanguage{en}
\scTitle{%
    Local radiated power sensitivity and intrinsic impurity correlation analysis at the stellarator Wendelstein 7-X%
    }%
%
\scAuthor{*}{Philipp}{Hacker}{1}
%
\scAuthor{}{Felix}{Reimold}{1}
\scAuthor{}{Daihong}{Zhang}{1}
\scAuthor{}{Rainer}{Burhenn}{1}
\scAuthor{}{Thomas}{Klinger}{1}
%
\scAffiliation{1}{%
    Max-Planck Institute for Plasma Physics, Greifswald, Germany}%
% \scAffiliation{2}{%
%     Ernst-Moritz-Arndt Universität Greifswald, D-17491 Greifswald, Germany}
%
\scBeginText
%
The two-camera resistive bolometer system at the stellarator Wendelstein 7-X with its blackened gold absorbers has provided a real time evaluation of the total radiated power for plasma feedback during the last experiment campaign. Based on the assumption of poloidal symmetry the radiated power loss of the plasma can been estimated independently for both cameras and each channel from line-integrated measurements. Using a limited set out of the total available fan of sight lines covering most radial emission shells the radiation level was calculated for plasma feedback control with fast auxiliary gas fueling as an actuator. Investigations regarding the best set of sight lines predicting the radiated power loss have been done for all camera and channel combinations as well as different mathematical weighting methods. Normalisation with individual cross correlations functions of single line integrated signals yields a set of channels with particular relevance for the total radiated power. Incorporating results from the 1-D impurity transport code STRAHL and spectroscopic diagnostics we attempt to link the contribution of different intrinsic impurities to the loss distribution.
% At the stellarator Wendelstein 7-X a two-camera bolometer system consisting of detectors with blackened gold foil absorbers has been used in the previous experiment campaign to implement and optimize a real time evaluation of the radiated power. The calculated line integrated radiation intensity was used for feedback control of the plasma discharge with auxiliary gas fueling as an actuator. The bolometer views the plasma at a triangular cross-section of W7-X horizontally and vertically across a poloidal position. Its fan-shaped lines of sight provide full coverage of the studied plasma at this cross-section with a spatial resolution of \SI{5}{\centi\meter} on the magnetic axis. Based on the line-integrated measurements the radiated power loss of the plasma has been estimated independently for both cameras. Different methods of estimation have been used to access the radiated power in near real time. A single channel signal and weighting factor was used for edge radiating plasma. As a second estimator, a selection of sightlines were used together with their geometrical properties to extrapolate the power loss by radiation, as is done for the offline analysis of the radiated power. Feedback results will be shown, including benchmarks of the global power balance using the calculated radiated power.
%
% The bolometer diagnostic at the stellarator Wendelstein 7-X (W7-X), using metal resistive detectors, aims to investigate the features of the plasma radiation mainly from the impurities and to provide the total radiated power loss for global power balance study.  A two-camera system consisting of detector arrays with blackened gold-foil absorbers has been installed at W7-X. They view the plasma at a triangular cross section horizontally and vertically, respectively. The fan-shaped lines of sight provide full coverage of the studied plasma with a spatial resolution of \SI{5}{\centi\meter}. Based on their line-integrated measurements the total radiated power loss of the divertor plasma has been estimated independently. The initial results for helium and hydrogen plasma at different magnetic configurations and heating powers will be presented.
%
%Results of the bolometer diagnostic studying plasmas generated in the stellarator Wendelstein 7-X with a first test divertor and all graphite tile baffles will be presented. The bolometer diagnostic is a metal film resistive detector system used to study the total radiated power without limitation of frequency bands. Two gold-foil detectors with up to 64 equidistant channels and collimating apertures are located on the inner side and bottom of the vacuum vessel at the same poloidal position, yielding a horizontal and vertical cut through the same plasma volume. Their fan-shaped line of sight provides full coverage of the stellarator cross-section and ensures maximum power absorption at a spatial resolution of $\sim$\SI{5}{\centi\meter}. The devices promise a long-term, stable response and high absorption coefficients in UV (sensibility >85\%) and SXR (>95\%) radiation. Resistive calibration and offset correction through a shutter are performed in situ and on each corresponding discharge. The camera systems are water cooled and connected to graphite tiles and thermally high conductive structures to avoid thermal drifts of the electric components. Sufficient microwave shielding against stray radiation from the ECR heating has been achieved by placing virtually permeable wire-meshes in front of the detector.
%
\scEndText
\scConference{Hannover 2020}
\scPart{P}
\scContributionType{Vortrag;Talk}
\scTopic{Helmholtz Graduate School for Plasma Physics}
\scEmail{philipp.hacker@ipp.mpg.de}
\scCountry{Germany}
%
\end{scAbstract}
%
\end{document}
