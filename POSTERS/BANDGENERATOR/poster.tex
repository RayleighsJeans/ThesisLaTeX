\documentclass[final]{beamer}
% beamer 3.10: do NOT use option hyperref={pdfpagelabels=false}
% don't show navigation symbols
% 2016 A. Kleiber
%
\beamertemplatenavigationsymbolsempty
\usepackage{beamerouterthemew7xposter}
\usepackage[ngerman]{babel}
\usepackage[LGR,T1]{fontenc}
\usepackage[latin1]{inputenc}
\usepackage{tcolorbox}
%
\usepackage{amsmath, amsthm, amssymb, latexsym, stackrel}
\usepackage{mathtools}
% \usefonttheme[onlymath]{serif}
\boldmath
\usepackage[orientation=portrait,size=a4,scale=1.4]{beamerposterippw7x}
\setbeamertemplate{bibliography item}{\insertbiblabel}
%
% \usepackage[orientation=portrait,size=a1,scale=1.4,grid,debug]{beamerposter}
% e.g. for DIN-A1 poster, with optional grid and debug output
% \usepackage[size=custom,width=200,height=120,scale=2,debug]{beamerposter}
% e.g. for custom size poster
%
% my packages
\usepackage{units}
\usepackage{siunitx}
\usepackage{lipsum}
\usepackage{xcolor}
%
% title of the presentation, short will be shown in the footer
\title{Wenn der Funke �berspringt: der Bandgenerator}
%
% authors of the presentation
% lecturer (and maybe place and date) will be shown in the footer
\author{}
\institute{}
%
% set date of the talk
\date{14.09.2018}
%
\newcommand{\diff}{\text{d}}
\newcommand{\tenpo}[1]{\cdot 10^{#1}}
\newcommand{\ix}[1]{_\text{#1}}
\newcommand{\imag}{\mathbf{i}}
\newcommand{\fett}[1]{\textbf{#1}}
\newcommand{\tilt}[1]{\textit{#1}}
\newcommand{\textgreek}[1]{\begingroup\fontencoding{LGR}\selectfont#1\endgroup}
\newcommand\inlineeqno{\stepcounter{equation}\ \quad\quad(\theequation)}
%
\begin{document}
    \begin{frame}{}
%
        Jeder kennt es: Man geht �ber einen Kunstoffboden oder Polyesterteppich, ber�hrt anschlie�end eine T�rklinke und ... *zapp* bekommt einen kleinen elektrischen Schlag. Aber was ist eigentlich passiert?
        \vspace*{0.5cm}
%
        Die meisten K�rper sind elektrisch neutral, was bedeutet, dass sie die gleiche Anzahl von negativen (Elektronen) und positiven (Protonen) Ladungen enthalten. Um einen K�rper elektrisch aufzuladen, muss die Gleichgewicht also gest�rt werden, entweder durch Hinzuf�gen oder Enternen von Ladungen.\\
        \vspace*{0.5cm}
%
        Beim Laufen �ber den Teppich nehmen zBsp. Schuhsohlen aus Gummi �ber die Zeit Elektronen auf, womit sich der K�rper insgesamt negativ aufl�dt. Beim Ber�hren der T�rklinke aus Metall springen die gesammelten Elektronen �ber und erzeugen einen Funken.\\
        \vspace*{0.5cm}
%
        Aber nicht nur zuhause auf dem Fu�boden findet diese Ladungstrennung statt: in vornehmen, antiken Haushalten nutzte man bereits gro�e Bernsteine als Kleiderb�rste. Das Gleiten �ber den Stoff hat den Stein negativ aufgeladen, womit dieser den positiv geladenen Staub und Fuselteilchen anzog. Das altgriechische Wort f�r Bernstein ist n�mlich
        �lektron (\textgreek{hlektron}). Dieser Name wurde sp�ter zum Begriff f�r das einfach negativ geladene Elementarteilchen des Elektrons und Namensgeber der Elektrizit�t.\\
        \vspace*{0.5cm}
%
        Mit einem Bandgenerator kann man eletrischen Ladungen trennen und hohe Gleichspannungen mit �ber \SI{150}{\kilo\volt} bei geringen Stromst�rken von etwa \SI{6}{\micro\ampere} erzeugen.\\
        Ein solcher Generator besteht aus einer gro�en Metallkugel, innerhalb welcher ein Gummiband einen Metallkamm streift. Dieser ist mit der Kugel verbunden. Durch die Reibung zwischen Metall und Gummi kommt es zur beschriebenen Ladungstrennung, bei der Elektronen von der Kugel abwandern, und diese damit positiv ``aufladen''.\\
        �ber einen zweiten Metallkamm und Leiter k�nnen die abgef�hrten Elektronen auf eine zweite Kugel gebracht werden.\\
        \vspace*{0.5cm}
%
        Was passiert, wenn sich diese beiden Metallkugeln n�her kommnen?
%
    \end{frame}
\end{document}
