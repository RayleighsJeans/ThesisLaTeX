\documentclass{scdpg}
%
\usepackage[utf8]{inputenc} % Required for inputting international characters
\usepackage[T1]{fontenc} % Output font encoding for international characters
%
\usepackage{units}
\usepackage{siunitx}
%
\begin{document}
%
\scBookLanguage{de}
\begin{scAbstract}
%
\scLanguage{en}
\scTitle{The bolometer diagnostic at the stellarator Wendelstein 7-X}
%
\scAuthor{*}{Philipp}{Hacker}{1,2}
\scAuthor{}{Daihong}{Zhang}{1}
\scAuthor{}{Rainer}{Burhenn}{1}
\scAuthor{}{Birger}{Buttenschön}{1}
\scAuthor{}{Thomas}{Klinger}{1}
\scAuthor{}{W7-x}{Team}{1}
%
\scAffiliation{1}{Max-Planck Institut für Plasmaphysik, EURATOM Association, D-17491 Greifswald, Germany}
\scAffiliation{2}{Ernst-Moritz-Arndt Universität Greifswald, D-17491 Greifswald, Germany}
%
\scBeginText
%
The bolometer diagnostic at the stellarator Wendelstein 7-X (W7-X), using metal resistive detectors, aims to investigate the features of the plasma radiation mainly from the impurities and to provide the total radiated power loss for global power balance study.  A two-camera system consisting of detector arrays with blackened gold-foil absorbers has been installed at W7-X. They view the plasma at a triangular cross section horizontally and vertically, respectively. The fan-shaped lines of sight provide full coverage of the studied plasma with a spatial resolution of \SI{5}{\centi\meter}. Based on their line-integrated measurements the total radiated power loss of the divertor plasma has been estimated independently. The initial results for helium and hydrogen plasma at different magnetic configurations and heating powers will be presented.
%
%Results of the bolometer diagnostic studying plasmas generated in the stellarator Wendelstein 7-X with a first test divertor and all graphite tile baffles will be presented. The bolometer diagnostic is a metal film resistive detector system used to study the total radiated power without limitation of frequency bands. Two gold-foil detectors with up to 64 equidistant channels and collimating apertures are located on the inner side and bottom of the vacuum vessel at the same poloidal position, yielding a horizontal and vertical cut through the same plasma volume. Their fan-shaped line of sight provides full coverage of the stellarator cross-section and ensures maximum power absorption at a spatial resolution of $\sim$\SI{5}{\centi\meter}. The devices promise a long-term, stable response and high absorption coefficients in UV (sensibility >85\%) and SXR (>95\%) radiation. Resistive calibration and offset correction through a shutter are performed in situ and on each corresponding discharge. The camera systems are water cooled and connected to graphite tiles and thermally high conductive structures to avoid thermal drifts of the electric components. Sufficient microwave shielding against stray radiation from the ECR heating has been achieved by placing virtually permeable wire-meshes in front of the detector.
%
\scEndText
\scConference{Erlangen 2018}
\scPart{P}
\scContributionType{Poster}
\scTopic{Helmholtz Graduate School for Plasma Physics}
\scEmail{philipp.hacker@ipp.mpg.de}
\scCountry{Germany}
%
\end{scAbstract}
%
\end{document}
