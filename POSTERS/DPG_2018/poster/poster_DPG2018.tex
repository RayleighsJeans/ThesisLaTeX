\documentclass[final]{beamer}
% beamer 3.10: do NOT use option hyperref={pdfpagelabels=false}
% don't show navigation symbols
% 2016 A. Kleiber

\beamertemplatenavigationsymbolsempty
\usepackage{beamerouterthemew7xposter}
\usepackage[english]{babel}
\usepackage[latin1]{inputenc}
\usepackage{tcolorbox}

\usepackage{amsmath, amsthm, amssymb, latexsym, stackrel}
\usepackage{mathtools}
%\usefonttheme[onlymath]{serif}
\boldmath
\usepackage[orientation=portrait,size=a0,scale=1.4]{beamerposterippw7x}
\setbeamertemplate{bibliography item}{\insertbiblabel}

%\usepackage[orientation=portrait,size=a1,scale=1.4,grid,debug]{beamerposter}
% e.g. for DIN-A1 poster, with optional grid and debug output
%\usepackage[size=custom,width=200,height=120,scale=2,debug]{beamerposter}
% e.g. for custom size poster

% my packages
\usepackage{units}
\usepackage{siunitx}
\usepackage{lipsum}
\usepackage{xcolor}

% title of the presentation, short will be shown in the footer
\title{The bolometer diagnostic at the stellarator Wendelstein 7-X}

% title zu lang, auf zwei Zeilen umgebrochen werden: 
%\title{Effects
%of collisions on the saturation dynamics\linebreak \hspace*{6ex}
%of TAEs in tokamaks and stellarators}

% authors of the presentation
% lecturer (and maybe place and date) will be shown in the footer
\author{\underline{P.~Hacker\inst{1,2}}, \and D.~Zhang\inst{1}, \and %
        R.~Burhenn\inst{1}, \and B.~Buttensch�n\inst{1}, \and %
        T.~Klinger\inst{1}, \,\, and the W7-X~Team\inst{1}}
\institute{\inst{1}Max Planck Institute for Plasma Physics, %
           Wendelsteinstr. 1, D-17491 Greifswald, Germany %
           \inst{2}Ernst-Moritz-Arndt University Greifswald, Domstr. 11, %
                   D-17489 Greifswald, Germany}

% set date of the talk
\date{06.03.2018}

\newcommand{\diff}{\text{d}}
\newcommand{\tenpo}[1]{\cdot 10^{#1}}
\newcommand{\ix}[1]{_\text{#1}}
\newcommand{\imag}{\mathbf{i}}
\newcommand{\fett}[1]{\textbf{#1}}
\newcommand{\tilt}[1]{\textit{#1}}
\newcommand\inlineeqno{\stepcounter{equation}\ \quad\quad(\theequation)}

\begin{document}
  %\begin{frame}{}
  \begin{frame}
    \frametitle{}
%
    \begin{minipage}[t]{0.32\textwidth}
      \begin{kasten}{\large Bolometer}
        \vspace*{0.5cm}
        \small{{\color{ipp}\fett{%
          Goals}}}
        \vspace*{0.5cm}%
        \tiny{%
          \begin{itemize}
            \item{%
              %bolometric diagnostic system
              to investigate total radiation powerloss through impurities %
              %loss through radiation
              and its distribution%
              %in triangle-shaped %
              %plane at 108� toroidaly in W7-X%
              }%
            \item{global/local power balance, as well as (later) impurity and %
              transport studies through tomographic inversion}%
          \end{itemize}%
        }
        \vspace*{0.5cm}%
        \small{{\color{ipp}\fett{%
          Performance}}}%
        \vspace*{0.5cm}%
        \tiny{%
          \begin{itemize}
            \item{multi-device system: horizontal bolometer camera %
              (HBC, 32 channels) and vertical bolometer camera %
              (VBC, 20 channels for each of two subdetectors)%
              %used for main evaluation%
              \newline$\Rightarrow$ more detectos with different %
              filters/coatings available, e.g. for investigation of %
              soft x-ray radiation}%
            \item{VBC/HBC detector arrays with carbon coated, %
              $\SI{5}{\micro\meter}$ thick gold-foil absorbers for %
              maximum absorbtion at sensitivity of $\SI{200}{\nano\watt}$ %
              and minimum reflectivity %
              (visible light to SXR between %
              \SIrange{600}{0.2}{\nano\meter})}%
            \item{fan-shaped lines of sight provide full plasma coverage at %
              $\SI{5}{\centi\meter}$ spatial resolution}%
            \item{Au-foil on $\SI{7.5}{\micro\meter}$ SiN %
              substrate, backed by a gold meander with %
              $\SI{0.25}{\milli\second}$ response time %
              due to thermal diffusion}%
            \item{temporal resolution in range of %
              \SIrange{0.08}{1.6}{\milli\second}, depending on experiment %
              and data economy}%
          \end{itemize}%
        }%
        \vspace*{0.5cm}%
        \small{{\color{ipp}\fett{%
          Design Criteria}}}%
        \vspace*{0.5cm}%
        \tiny{
          \begin{itemize}%
            \item{steady state operation at discharges with up to %
              $\SI{30}{\minute}$ of $\SI{10}{\mega\watt}$ heating power %
              ensured by cooling system with %
              graphite elements and water cooling structures}
            \item{impact of electron cyclotron resonance heating (ECRH) %
              stray radiation (several %
              $\SI{10}{\kilo\watt\per\square\meter}$ %
              at $\SI{140}{\giga\hertz}$) reduced by a conductive %
              wire-mesh of thickness $\SI{90}{\micro\meter}$ and %
              $\SI{0.24}{\milli\meter}$ spacing in front of the detector %
              array, as well as a ceramic TiO/Al$_{2}$O$_{3}$ coating inside %
              the camera housing%
              \newline$\Rightarrow$ 3\% microwave transmission, %
              optical transmission factor 53\%}
          \end{itemize}
        }
        \vspace*{0.5cm}
        \begin{columns}%
          \column{0.4\textwidth}%
            \centering{%
              \small{%
                \color{ipp}%
                \textbf{%
                  \underline{%
                    HBC%
                  }%
                }%
              }%
            }%
          \column{0.4\textwidth}%
            \centering{%
              \small{%
                \color{ipp}%
                \textbf{%
                  \underline{%
                    VBCr/VBCl%
                  }%
                }%
              }%
            }%
        \end{columns}%
        \vspace*{0.5cm}%
        \centering{%
        \includegraphics[width=0.9\textwidth]%
            {figures/content/linesofsight_in_vessel_reff.pdf}%
        }%
        \newline
        %caption
        \tiny{%
          Lines of sight for HBC (32 channels) and VBC (two 20-channel %
          subdetector arrays) with individual apertures, %
          retracted into the vacuum vessel behind protective wall elements. %
          Located in the triangle-shaped plane at W7-X.%
          \cite{Zhang2010}
        }%
        \vspace*{1cm}
        \newline%
        \centering{%
        \includegraphics[width=0.8\textwidth]%
            {figures/content/torus_full_banana.pdf}%
        }%
        \newline
        %caption
        \tiny{%
          W7-X plasma vessel (torus, center) with equilibrium fluxsurfaces %
          (red), calculated by VMEC\cite{VMEC} at triangle- (top-left, 108�) %
          and "\tilt{bean}"-shaped (bottom-right, 0�) planes.%
        }%
      \end{kasten}%
    \end{minipage}%
    \hfill%
    \begin{minipage}[b]{0.32\textwidth}%
      \begin{kasten}{\large Components}%
        \vspace*{0.5cm}
        \centering{%
          \small{%
            \color{ipp}%
              \textbf{%
                \underline{%
                  Camera Head Construction%
                }%
              }%
            }%
          }%
        \vspace*{0.5cm}%
        \linebreak%
        \includegraphics[width=0.8\textwidth]%
            {figures/content/component_3d.pdf}%
        \par%
        %caption
        \tiny{%
          Camera head (VBC) construction. Subdectector arrays on water %
          cooled holdings with optic baffles (left). %
          Capped graphite tile with stainless steel aperture for thermal %
          protection.\cite{Zhang2010}%
        }%
        \vspace*{1cm}
        \linebreak%
        \centering{%
          \small{%
            \color{ipp}%
              \textbf{%
                \underline{%
                  Detector
                }%
              }%
            }%
          }%
        \vspace*{1cm}%
        \linebreak%
        \includegraphics[width=0.75\textwidth]%
            {figures/content/goldmeander_new.pdf}%
        \par%
        %caption
        \tiny{%
          Scheme of a single detector channel with housing/holder, %
          absorption foil, substrate and meander.%
          \cite{Gianone2002}%\cite{Mast1991}%
        }%
      \end{kasten}%
      %\vspace*{0cm}
      \begin{kasten}{\large References}%
        \fontsize{14}{14}{%
        \begin{thebibliography}{}%
          \bibitem{Zhang2010} "Design Criteria of the Bolometer diagnostic %
                              for steady-state operation of the W7-X %
                              stellarator"; %
                              Zhang, D. et al.; %
                              %Burhenn, R., Koenig, %
                              %R., Giannone, L., Grodzki, P.A., Klein, B., %
                              %Grosser, L., Baldzuhn, J., Ewert, K., %
                              %Erckmann, V., Hirsch, M., Laqua, H.P., %
                              %Oosterbeek, J.W.; %
                              Review of Scientific Instruments, %
                              Jan 1st, 2010; DOI:10.1063/1.3483194
          \bibitem{Zhang2016} "The bolometer diagnostic at stellarator %
                              Wendelstein 7-X and its first results in the %
                              initial campaign"; %
                              D. Zhang, et al. %
                              %R. Burhenn, A. Alonso, B. Buttensch�n, %
                              %Y. Feng, L.Giannone, M.Hirsch, U.H�fel, R.%
                              %Lauber, M.Marquardt, K.Rahbarnia, J.Svensson, 
                              %G.A.Wurden, R.Brakel, O.Grulke, J.Knauer, R.
                              %K�nig, H.Laqua, S.Marsen T.Stange, T.%
                              %Schr�der, H.Thomsen, G.M. Weir, A.Werner and %
                              and the W7-X Team; Stellarator-New 2017
          \bibitem{Mast1991} "A low noise highly integrated bolometer array %
                              for absolute measurement of VUV and soft x %
                              radiation"; %
                              K. F. Mast et. al; %
                              %J. C. Vallet, %
                              %C. Andelfinger, P. Betzler, H. Kraus, and %
                              %G. Schramm;%
                              Review of Scientific Instruments 62, 744 (1991);
                              DOI: 10.1063/1.11.42078%
               \bibitem{VMEC} "Steepest descent moment method for three %
                              dimensional magnetohydrodynamic equilibria"; %
                              Hirshman, S.P. et al.; %
                              %Whitson, J.C.; %
                              Physics of Fluids 26, 3553, (1983); %
                              DOI: 10.1063/1.864116%

             \bibitem{Wesson} "Tokamaks"; %
                              Wesson, J.; %
                              Clarendon Press, Oxford; %
                              1987%
               \bibitem{Feng} "Numerical investigation of plasma edge %
                              transport and limiter heat fluxes in %
                              Wendelstein 7-X startup plasmas with %
                              EMC3-EIRENE"; %
                              Effenberg, F., Feng, Y. et al. %
                              Nucl. Fusion 57 (2017) 036021 (15pp); %
                              DOI: 10.1088/1741-4326/aa4f83%
        \bibitem{Gianone2002} "Derivation of bolometer equations relevant to %
                              operation in fusion experiments"; %
                              Gianone, L. et al.; %
                              Review of Scientific Instruments; %
                              20th of November, 2002; %
                              DOI: 10.1063/1.1498906%
          \bibitem{Zhang2018} "Results of the bolometer diagnostic at %
                              OP 1.a of W7-X"; %
                              internal review of the physics plan during the %
                              second operational phase at the stellarator %
                              W7-X; 28.02.20i18%
         \bibitem{Yamada2005} "Characterization of energy confinement in %
                              net-current free plasmas using the %
                              extended International Stellarator Database"; %
                              H. Yamada et al.; %
                              INSTITUTE OF PHYSICS PUBLISHING and %
                              INTERNATIONAL ATOMIC ENERGY AGENCY; %
                              Nucl. Fusion 45 (2005) 1684�1693%
        \end{thebibliography}%
        }
      \end{kasten}%
    \end{minipage}%
    \hfill%
    \begin{minipage}[t]{0.32\textwidth}%
      \begin{kasten}{\large Calibration}%
        \vspace*{1cm}
        \tiny{%
          \begin{itemize}%
            \item{\textit{in-situ calibration} when ultra-high vacuum %
              pneumatic shutter is closed before every experiment}
            \item{reference correlation with un-exposed, %
              indentical part of the detectors \textit{Wheatstone bridge} %
              with two calibration gold-foils of same thickness and size %
              %($\SI{5x10}{\milli\meter}$)
              }%
            \item{for each channel, calibration is performed by a %
              $\SI{5}{\watt}$ pre-heating with closed shutter, regarding %
              cooling time, capacity and resistance\cite{Gianone2002}\\%
              $\Rightarrow$ temperature change in Au film leads to change in %
              resistance of underlying gold meanders}
          \end{itemize}%
        }%
        \vspace*{1.5cm}%
        \begin{columns}
          \column{0.45\textwidth}
            \centering{%
              \small{%
                \color{ipp}%
                \textbf{%
                  \underline{
                    Detector Array%
                  }%
                }
              }%
            }%
          \column{0.45\textwidth}
            \centering{%
              \small{%
                \color{ipp}%
                \textbf{%
                  \underline{
                    Wheatstone Bridge%
                  }%
                }
              }%
            }%
          \end{columns}%
          \vspace*{2cm}%
          \centering
          \includegraphics[width=0.95\textwidth]%
            {figures/content/wheatstone_mitschema.pdf}%
            \tiny{
              (Left) Detector array with sealed reference array, meander and %
              holder. (Right) Wheatstone bridge with radiated part (yellow) %
              and reference (green), left un-exposed throughout operation. %
              Reference and measurement resistances of the foils are the %
              same.\cite{Gianone2002}%\cite{Mast1991}%
            }%
        \vspace*{2cm}
        \centering
        \includegraphics[width=0.6\textwidth]%
          {figures/content/coating_results.pdf}%
          \newline
          \tiny{
            Test results of a bolometer prototype in a strong microwave %
            background provided by MISTRAL\cite{Zhang2016} with an withouth %
            microwave suppresion measures (see left).
          }%
        \vspace*{2cm}%
      \end{kasten}%
    \end{minipage}%
    \vspace*{0.5cm}
%
    \begin{kasten}{\large Results}%
      \begin{columns}%
        \hspace*{1cm}%
        \column{0.28\textwidth}%
          \begin{kasten}{Equations}
            \tiny{%
              The radiation power observed by the bolometers equals to:
              $$P_{rad,bolo}\propto\sum_{Z} n_{e}\cdot n_{Z}\cdot L_{Z}$$%
              \quad where $L_{Z}$ is the line radiation function by %
              impurities ($Z$): 
              $$L_{Z}=f\left(T_{e}, T_{i}, T_{Z}, %
                                 wall\,\,material/%
                                 conditions,\,\,\dots\right)$$%
              For each channel the observed power $P_{Ch}$ can be calculated %
              by using:%
              $$P_{ch}=\frac{2}{V_{eff}}\cdot%
                       \left(R_{ch}+2R_{C}\right)\cdot\kappa_{ch}%
                       \sqrt{g_{C}}\cdot\left(\tau_{ch}%
                       \frac{\diff(\Delta U)}{\diff t}+%
                       f_{\tau}\cdot(\Delta U)\right)%
                \inlineeqno\text{\cite{Gianone2002}}$$%
              \quad\quad with $\Delta U\propto\Delta T$ the change in %
              measurement voltage and absorber temperature%
              %$$V_{eff}=\frac{U_{5V}R_{ch}}{R_{ch}+2R_{C}}%
              %,\quad%
              %g_{C}=1+\left(\omega\cdot\left(R_{ch}+R_{C}\right)^{2}%
              %      \right)%
              %,\quad%
              %\omega=2\pi\cdot f_{Bridge}C_{cab}$$%
              %$$f_\tau=1-U_{eff}^{2}\cdot\beta%
              %\,,\quad\quad
              %\beta=\frac{1-\left(\omega R_{ch}\right)^{2}+%
              %        \left(\omega R_{C}\right)^{2}}%
              %        {1+\left(\omega\cdot\left(R_{ch}+R_{C}\right)%
              %        \right)}$$%
              \vspace*{0.5cm}%
              \begin{itemize}
                \item{properties denoting $(\cdot)_{ch}$ are the %
                  individual channel/foil characteristics, e.g. cooling %
                  time ($\tau$), heat capacity ($\kappa$) and resistance %
                  ($R$)}
                \item{$R_{C}$ and $C_{cab}$ are the connection cable %
                  resistance and %
                  capacity with \SI{41}{\ohm} and \SI{9}{\nano\farad} %
                  respectively}%
                \item{$f_{Bridge}$ is a dimensionless scaling factor of %
                  the Wheatstone bridge}%
              \end{itemize}%
            }%
            \vspace*{2.5cm}
            \tiny{\color{ipp}\fett{\underline{%
              Global Power Estimate:}}}%
            \tiny{ %
              for each camera (VBC, HBC) individually
              $$P_{rad}=\frac{V_{P,tor}}{V_{cam}}\cdot%
                  \sum_{ch}\frac{V_{ch}}{K_{ch}}\cdot\frac{P_{ch}}{53\%}%
                \inlineeqno$$%
              \quad\quad with:
              $$V_{cam}=\sum_{ch}V_{ch}$$%
              %,\,\,
              %V_{P,tor}=52381772\,\text{cm}^{3}$$
              \begin{itemize}
                \item{$V_{ch}$ the volume of the polygon created by the %
                  lines of sight of each detector and the corresponding %
                  aperture $\Rightarrow V_{cam}$ is the total volume %
                  investigated by the camera}%
                \item{$P_{ch}$ is calculated via eq. (1)}
                \item{scaling with 53\% due to the reduced input intensity of %
                  the stray radiation wire mesh}
                \item{$V_{P,tor}$ the estimated plasma volume from which radiation is %
                  emitted, approximated using one field configuration in an %
                  EMC3-Eirene simulation}
              \end{itemize}
            }
          \end{kasten}%
          % \includegraphics[width=0.9\textwidth]%
          %     {figures/content/raw_volt_1207051.pdf}%
          % \vspace*{0.5cm}%
          % \newline%
          % \tiny{%
          %   Raw signal of innermost/uppermost and plasma %
          %   core viewing channels for VBC/HBC (21/20 channels). %
          %   The vertical array used consists of a individual channels %
          %   taken from the two subdetector arrays of VBC. %
          %   One sample corresponds to $\Delta t=\SI{1.6}{\milli\second}$. %
          %   Standard deviation is \SIrange{0.3}{2}{\micro\volt}, %
          %   e.g. $\ll$ 1\%.%
          % }%
          \vspace*{2cm}%
          \begin{columns}%
            \column{0.48\textwidth}%
              \centering
              \includegraphics[width=\textwidth]%
                    {figures/content/oxygen_carbon_rad.pdf}%
              \vspace*{0.5cm}%
              \newline%
              \tiny{%
                Impurity radiation intensities for most relevant %
                elements in the device. Carbon and, but most importantly %
                oxygen impurity radiation is assumed to contribute the %
                most to line radiation.\cite{Wesson}%\cite{Post}
              }%
            \column{0.48\textwidth}
              \vspace*{0.5cm}
              \centering
              \includegraphics[width=\textwidth]%
                    {figures/content/emc3_sim_oxygen.pdf}%
              \vspace*{2cm}%
              \newline%
              \tiny{%
                EMC3-Eirene simulation results, which show the relative %
                radiation profile/emissivity for 80\% oxygen impurities %
                beyond the last closed fluxsurface (LCFS).\cite{Feng}%
              }%
          \end{columns}
        \column{0.35\textwidth}%
          \vspace*{1cm}
          \centering%
          \includegraphics[width=0.9\textwidth]%
              {figures/content/steady_state_2017111452.pdf}%
          \vspace*{0.5cm}%
          \newline%
          \tiny{%
            XP.20171114.052: (LEFT) Calculated powers from eq. (1) for the inner-/uppermost %
            channels from the VBC/HBC. One sample equals %
            \SI{1.6}{\milli\second}. A \tilt{Savitzky-Golay} filter with a %
            31 sample window is used on both raw signal/derivative to %
            minimize bitwise noise from the A2D converter. %
            (MIDDLE) Experiment parameters for the same discharge %
            (\SI{2.7}{\second}, terminated). %
            (RIGHT) channel power vs. effetive plasma radius, %
            e.g. $\rho_{eff}=\sqrt{\Psi_{N}}$ the distance from the magnetic %
            field center, for both camera.
          }%
          \vspace*{0.5cm}%
          \begin{kasten}{\small Conclusion}
          \vspace*{0.5cm}%
            \begin{itemize}
              \item{high-density, high-radiation discharges can show%
                $$f_{rad}=\frac{P_{rad}}{P_{ECRH}}\approx 1$$}%
              \item{inboard and lower plasma regions irradiate more intense %
                in EJM magnetic field configuration}
              \item{steady state: mainly radiation from outer plasma %
                regions/LCFS}
              \item{radiative collapse: strong centrification of plasma source %
                (see $r_{eff}$, core very bright)}
              \item{after $W_{dia}$ the plasma-stored energy begins to %
                decrease, discharge has already begun to collapse and %
                the input power is irradiated, hence $f_{rad}\ge 1$}
              \item{collapse likely triggered around $t\approx$\SI{2.3}{\second}, %
                by first pellet injection - edge localized radiation? -, %
                causing the plasma to shrink}
              \item{in radiative collapse case: thermal instabilities %
                (see $T_{e}$) triggered around plasma edge\\%
                $\Rightarrow$ local events, cooling effect by increasing %
                $P_{rad}$}
              \item{however, operational phase 1.2a (OP1.2a) at W7-X %
                showed possible high-performance discharges with great %
                power-exhaust capabilities around LCFS/scrape-off-layer\\%
                $\Rightarrow$ detachment with plasma shrinkage, less %
                impurity, better wall conditioning ...}
              %\item{operational limit given by the radiated power and %
              %  other plasma parameters?}
            \end{itemize}
          \vspace*{0.5cm}%
          \end{kasten}
          % \vspace*{0.5cm}
          % \centering%
          % \includegraphics[width=\textwidth]%
          %     {figures/content/reff_vbcandhbc2017120751.png}%
          % %\vspace*{0.5cm}%
          % \newline%
          % \tiny{%
          %   (TOP) Surface plot of measured radiated power over the effective %
          %   plasma radius, e.g. $\rho_{eff}=\sqrt{\Psi_{N}}$ the distance %
          %   from the magnetic field center. (BOTTOM) Extrapolated total %
          %   radiated power. (Discharge XP.20171207.051)
          % }%
        \column{0.35\textwidth}%
          \centering%
          \includegraphics[width=\textwidth]%
              {figures/content/radiative_collapse_2017112136_3.pdf}%
          \vspace*{0.5cm}%
          \newline%
          \tiny{%
            XP.20171121.036: (LEFT) Channel number vs. power, e.g. the %
            center corresponds to the center of the plasma core. %
            (MIDDLE) The most important discharge properties. Similar setup %
            as for the left discharge, but with forward feeding by frozen %
            H$_{2}$ pellet injection. The discharge collapses where the %
            stored, diamagnetic energy in the plasma drops drastically %
            around \SI{2.9}{\second}. P$_{rad}$ shows that this %
            corresponds to a radiative collapse, where the plasma shrinks %
            towards the magnetic axis (as seen in the left figures), and all %
            the input and stored energy is irradiated. %
            (RIGHT) The channel power over the effective plasma radius.%
          }%
          \vspace*{0.5cm}
          \begin{kasten}{\small OP1.2a}
            \begin{columns}%
              \hspace*{3cm}
              \column{0.35\textwidth}
                \vspace*{1cm}
                \newline%
                %\centering%
                \tiny{%
                  Collected database of different discharge %
                  setups/characteristics where radiation fraction %
                  reached unity. %
                  Results based on Yamada et al.\cite{Yamada2005} show %
                  good agreement for the energy confinement time scaling %
                  for: \cite{Zhang2018}}\\%
                  \centering%
                  \small{\color{ipp}%
                    $\tau_{E, ISS04}\propto n_{e}^{0.54}$}
              \column{0.65\textwidth}
                \centering%
                \includegraphics[width=0.7\textwidth]%
                    {figures/content/power_diagnostic_own_2.pdf}%
            \end{columns}
          \end{kasten}
      \end{columns}%
    \end{kasten}%
%
    %\begin{kasten}{\large Outlook}
    %  \begin{columns}
    %    \column{0.48\textwidth}
    %      \begin{itemize}
    %        \item{test}
    %      \end{itemize}
    %    \column{0.48\textwidth}
    %  \end{columns}
    %\end{kasten}
%
    %\hfill%
    %\begin{kasten}{Outlook}%
    %\end{kasten}%
%
    \vfill%
    \centerline{\tiny 82. Jahrestagung der DPG und DPG-Fr�hjahrstagung %
                       der Sektion AMOP, Erlangen, 04.03. - 09.03.2018}%
    \centerline{\tiny Authors correspondence:\,\,%
                P. Hacker, philipp.hacker@ipp.mpg.de; %
                D. Zhang, daihong.zhang@app.mpg.de}%
  \end{frame}%
\end{document}
