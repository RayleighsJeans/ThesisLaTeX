\documentclass[final]{beamer} 
% beamer 3.10: do NOT use option hyperref={pdfpagelabels=false}
% don't show navigation symbols
% 2018 A. Kleiber

\beamertemplatenavigationsymbolsempty
\usepackage{2018_10_beamerouterthemeposterw7x}
\usepackage[english]{babel}
\usepackage[latin1]{inputenc}
\usepackage{tcolorbox}
\usepackage{amsmath, amsthm, amssymb, latexsym, stackrel}
\usepackage{units}
\usepackage{siunitx}
\usepackage{mathtools}

% Arial-Schrift
% \usepackage{ipp}
% \usepackage[scaled]{uarial}
% \usepackage{times}\usefonttheme{professionalfonts}  % times is obsolete
% \usefonttheme[onlymath]{serif}

\boldmath
\usepackage[orientation=portrait,size=a0,scale=1.4,debug]{%
    2018_10_beamerposteripp}  % e.g. for DIN-A0 poster

% \usepackage[orientation=portrait,size=a1,scale=1.4,grid,debug]{beamerposter}
% e.g. for DIN-A1 poster, with optional grid and debug output

% \usepackage[size=custom,width=200,height=120,scale=2,debug]{beamerposter}
% e.g. for custom size poster

% \usepackage[orientation=portrait,size=a0,scale=1.0,%
%   printer=rwth-glossy-uv.df]{beamerposter}
% e.g. for DIN-A0 poster with rwth-glossy-uv printer check

% Damit im Literaturverzeichnis Zahlen stehen.
\setbeamertemplate{bibliography item}{\insertbiblabel}
\setbeamertemplate{itemize item}[triangle]

% title of the presentation
\title{%
    Consistently calculating the radiated power in near real time at the %
    stellarator Wendelstein 7-X}%
% Sollte die Ueberschrift zu lang sein, kann sie wie
% folgt auf zwei Zeilen umgebrochen werden: 
% \title{Effects of collisions on the saturation dynamics\linebreak%
%   \hspace*{6ex}%
%   of TAEs in tokamaks and stellarators}

% authors of the presentation
\author[Author]%
    {\underline{P.~Hacker\inst{1,2}*},~D.~Zhang\inst{1},~F.~Reimold\inst{1},~%
     M.~Krychowiak\inst{1},~R.~Burhenn\inst{1} and T.~Klinger\inst{1},~%
     for the Wendelstein 7-X Team-Collaboration\inst{1}}
% email address of the corresponding author

% Definieren mail corresponding author
\newcommand{\emailcorrespondingauthor}{philipp.hacker@ipp.mpg.de}

% Definieren Name Conference
\newcommand{\nameconference}%
    {Deutsche~Physikalische~Gesellschaft~-~%
    Spring~Meeting~Munich~2019,~%
    Matter and Cosmos Section (SMuK)}

% institutes of the authors
% \institute[IPP Greifswald]%
%   {\inst{1}Max Planck Institute for Plasma Physics, %
%    Wendelsteinstr. 1, D-17491 Greifswald, Germany}

\institute[]{%\hspace{5cm}
    \inst{1}%
        Max~Planck~Institute~for~Plasma~Physics,~Wendelsteinstr.~1,~%
        D-17491~Greifswald,~Germany,\\%
    \inst{2}%
        Ernst-Moritz-Arndt~University~Greifswald,~%
        Rubenowstr.~1,~D-17489~Greifswald,~Germany}

% name of the conference
% \nameconference{Name der Konferenz Da.Tu.M000}
% set date of the talk
\date{\today}

%%%%%%%%%%%%%%%%%%%%%%%%%%%%%%%%%%%%%%%%%%%%%%%%%%%%%%%%%%%%%%%%%%%%%%%%%%%%%%
\begin{document}
    \begin{frame}
        \frametitle{}
        \begin{minipage}[t]{0.30\textwidth}
            \vfill%
            \begin{kasten3}{\large%
                Motivation}\vspace*{0.5cm}\small{%
                \begin{itemize}%
                    \item{%
                        averaged thermal load on in-vessel components %
                        expected to be up to %
                        \mbox{\SI{100}{\kilo\watt/\square\meter}} mainly by %
                        radiation and non-absorbed heating power}%
                    \item{%
                        calculating the temporal and spacial evolution of %
                        the radiation loss previously only after the plasma %
                        has been terminated}
                    \item{%
                        dynamically adjust heat loads on components and %
                        investigate radiation regimes of high temperature %
                        plasmas with possibility of improved detachment %
                        experiments.}
                    \item{%
                        possibly find gas puff-radiation scaling law %
                        for purposes of extensive plasma control, i.e. %
                        density, power and radiation}
                \end{itemize}}\vspace*{0.5cm}%
            \end{kasten3}%
            \vspace*{0.7cm}%\hfill%
            \begin{kasten}{\large%
                Recap}%
                \vspace*{0.5cm}%
                \large{{\color{ipp}\fett{%
                  Design}}}%
                \vspace*{0.5cm}\small{%
                \begin{itemize}%
                    \item{%
                        multi-device system: horizontal bolometer camera %
                        (HBC, 32 channels) and vertical bolometer camera %
                        (VBC, 20 channels for each of two subdetectors)%
                        \newline$\Rightarrow$ more detectors with different %
                        filters/coatings available, e.g. for investigation %
                        of soft x-ray radiation}%
                    \item{%
                        fan-shaped lines of sight provide full plasma %
                        coverage at $\SI{5}{\centi\meter}$ spatial resolution}
                    \item{%
                        steady state operation at discharges with up to %
                        $\SI{30}{\minute}$ of $\SI{10}{\mega\watt}$ heating %
                        power ensured by cooling system with %
                        graphite elements and water cooling structures}
                \end{itemize}}%
                \vspace*{0.5cm}%
                \begin{columns}%
                    \column{0.4\textwidth}%
                        \centering{\small{\color{ipp}%
                            \textbf{\underline{%
                                HBC}}}}%
                    \column{0.4\textwidth}%
                        \centering{\small{\color{ipp}%
                            \textbf{\underline{%
                                VBCr/VBCl}}}}%
                \end{columns}%
                \vspace*{0.5cm}%
                \begin{center}%
                    \includegraphics[width=0.9\textwidth]%
                        {figures/content/linesofsight_in_vessel.pdf}%
                    \newline\tiny{%
                        LOS for HBC (32 ch.) and VBC (two 20-ch. subdetector %
                        arrays) with individual apertures, retracted into %
                        the vacuum vessel behind wall elements; located in %
                        the triangle-shaped plane at W7-X.\cite{Zhang2010}}%
                    \newline%
                \end{center}
                \large{{\color{ipp}\fett{%
                  Performance}}}%
                \vspace*{0.5cm}\small{%
                \begin{itemize}%
                    \item{%
                        VBC/HBC detector arrays with carbon coated, %
                        $\SI{5}{\micro\meter}$ thick gold-foil absorbers for %
                        maximum absorbtion at sensitivity of %
                        $\SI{200}{\nano\watt}$ and minimum reflectivity %
                        (visible light to SXR between %
                        \SIrange{600}{0.2}{\nano\meter})}%
                    \item{%
                        Au-foil on $\SI{5}{\micro\meter}$ Si$_{3}$N$_{4}$ %
                        substrate, backed by a \SI{30}{\micro\meter} platin %
                        meander with a $\SI{0.25}{\milli\second}$ response %
                        time}%
                    \item{%
                        temporal resolution in range of %
                        \SIrange{0.08}{6.4}{\milli\second}, depending on %
                        experiment and data economy}%
                    \item{%
                        impact of electron cyclotron resonance heating %
                        (ECRH) stray radiation (several %
                        $\SI{10}{\kilo\watt\per\square\meter}$ %
                        at $\SI{140}{\giga\hertz}$) reduced down to 3\% %
                        microwave transmission, opt. transmission factor 53\%}
                \end{itemize}}%
                \vspace*{1.0cm}%
                \begin{columns}%
                    \column{0.47\textwidth}%
                        \centering{%
                            \includegraphics[width=1.\textwidth]%
                                {figures/content/torus_full_banana.pdf}%
                        }\vspace*{0.5cm}\newline%
                        \tiny{%
                            W7-X plasma vessel (torus, center) with %
                            equilibrium fluxsurfaces (red), calculated %
                            by VMEC\cite{VMEC} at triangle- (top-left, %
                            108\°) and "\tilt{bean}"-shaped (bottom-right, %
                            0\°) planes.%
                        }%
                    \column{0.47\textwidth}
                        \vspace*{0.6cm}
                        \centering{%
                            \includegraphics[width=1.\textwidth]%
                                {figures/content/detector_me.pdf}
                        }\vspace*{0.5cm}\newline%
                        \tiny{%
                            Scheme of a single detector channel with %
                            housing/holder, absorption foil, substrate and %
                            meander.\cite{Gianone2002} Layer dimensions are %
                            noted.%
                        }%
                \end{columns}%
            \end{kasten}%
            \vspace*{0.7cm}%\hfill%
            \begin{kasten}{\large%
                References}
                \fontsize{14}{14}{%
                \begin{thebibliography}{}%
                    \bibitem{Zhang2010} %
                        "Design Criteria of the Bolometer diagnostic %
                        for steady-state operation of the W7-X stellarator"; %
                        Zhang, D. et al.; %Burhenn, R., Koenig, %
                        %R., Giannone, L., Grodzki, P.A., Klein, B., %
                        %Grosser, L., Baldzuhn, J., Ewert, K., %
                        %Erckmann, V., Hirsch, M., Laqua, H.P., %
                        %Oosterbeek, J.W.; %
                        Review of Scientific Instruments, %
                        Jan 1st, 2010; DOI:10.1063/1.3483194
                    \bibitem{Zhang2016} %
                        "The bolometer diagnostic at stellarator %
                        Wendelstein 7-X and its first results in the %
                        initial campaign"; D. Zhang, et al. %
                        %R. Burhenn, A. Alonso, B. Buttenschön, %
                        %Y. Feng, L.Giannone, M.Hirsch, U.Höfel, R.%
                        %Lauber, M.Marquardt, K.Rahbarnia, J.Svensson, 
                        %G.A.Wurden, R.Brakel, O.Grulke, J.Knauer, R.
                        %König, H.Laqua, S.Marsen T.Stange, T.%
                        %Schröder, H.Thomsen, G.M. Weir, A.Werner and %
                        and the W7-X Team; Stellarator-New 2017
                    \bibitem{Mast1991} %
                        "A low noise highly integrated bolometer array %
                        for absolute measurement of VUV and soft x %
                        radiation"; K. F. Mast et. al; %
                        %J. C. Vallet, C. Andelfinger, P. Betzler, H. Kraus, and %
                        %G. Schramm;%
                        Review of Scientific Instruments 62, 744 (1991);
                        DOI: 10.1063/1.11.42078%
                    \bibitem{VMEC} %
                        "Steepest descent moment method for three %
                        dimensional magnetohydrodynamic equilibria"; %
                        Hirshman, S.P. et al.; %Whitson, J.C.; %
                        Physics of Fluids 26, 3553, (1983); %
                        DOI: 10.1063/1.864116%
                    \bibitem{Wesson} %
                        "Tokamaks"; Wesson, J.; Clarendon Press, Oxford; 1987%
                    \bibitem{Feng} %
                        "Numerical investigation of plasma edge %
                        transport and limiter heat fluxes in %
                        Wendelstein 7-X startup plasmas with %
                        EMC3-EIRENE"; %
                        Effenberg, F., Feng, Y. et al. %
                        Nucl. Fusion 57 (2017) 036021 (15pp); %
                        DOI: 10.1088/1741-4326/aa4f83%
                    \bibitem{Gianone2002} %
                        "Derivation of bolometer equations relevant to %
                        operation in fusion experiments"; %
                        Gianone, L. et al.; Review of Scientific Instruments; %
                        20th of November, 2002; DOI: 10.1063/1.1498906%
                    \bibitem{Zhang2018} %
                        "Results of the bolometer diagnostic at %
                        OP 1.a of W7-X"; internal review of the %
                        physics plan during the %
                        second operational phase at the stellarator %
                        W7-X; 28.02.2018%
                    \bibitem{Yamada2005} %
                        "Characterization of energy confinement in %
                        net-current free plasmas using the %
                        extended International Stellarator Database"; %
                        H. Yamada et al.; %
                        INSTITUTE OF PHYSICS PUBLISHING and %
                        INTERNATIONAL ATOMIC ENERGY AGENCY; %
                        Nucl. Fusion 45 (2005) 1684¿1693%
                    \bibitem{FIFOSource} %
                        ``Introduction to the Queue Data Structure – Array %
                        Implementation''; %
                        URL: https://www.thecodingdelight.com/%
                        queue-data-structure-array-implementation/
                \end{thebibliography}}%
            \end{kasten}
            % QR AND SHIT
            \vspace*{0.5cm}%
            \begin{columns}%
                \centering{%
                \column{0.7\textwidth}%
                    \centering{%
                    \normal{\color{ipp}{%
                        You want to look at this poster later or get %
                        in contact with the author?}}}
                \column{0.2\textwidth}%
                    \centering{%
                        \includegraphics[width=0.7\textwidth]%
                            {figures/header/Posters.png}}%
            }\end{columns}
        \end{minipage}%
        \hfill%
        \begin{minipage}[t]{0.69\textwidth}%
            \vfill%
            \begin{kasten}{\large%
                Setup}%
                \vspace*{1.cm}
                \centering{%
                    \includegraphics[width=1.\textwidth]%
                        {figures/content/mm.pdf}}%
            \end{kasten}%
            \vspace*{1.0cm}%\hfill%
            \begin{kasten3}{\large%
                Outlook}\small{%
                \begin{columns}
                    \column{0.5\textwidth}
                        \begin{itemize}
                            \item{%
                                for the upcoming campaign OP2, %
                                starting early/mid 2020, the system has %
                                to be technically improved to account for %
                                longer discharges \& more accurate %
                                predictions, by e.g. automatic channel and %
                                scaling selection based off of configuration %
                                or programing}%
                            \item{%
                                extensive analysis of impact of helium %
                                beam gas fueling on radiation loss in %
                                comparison to the main valves, since it is %
                                located closer to the divertor/SOL}%
                        \end{itemize}}%
                    \column{0.48\textwidth}
                        \large{\color{red!75!black}{%
                        On the horizon:}}\\%
                        \small{%
                        Implemented until next campaign will be a setup for %
                        bayesian \& non-bayesian systems to calculate %
                        poloidal tomography profiles. This will also %
                        aid the developement of the introduced %
                        feedback alogrithm, since one might be able to %
                        weight the different lines of sight individually %
                        given the configuration \& plasma shape.}
                \end{columns}
            \end{kasten3}
        \end{minipage}%
        \vfill%
    \end{frame}
\end{document}

% GRAVEYARD
% \tiny{%
%     Combination of feedback signals, total $P_{rad,cam}$ %
%     from post-processing for significance analysis and %
%     the actual sampled results at the thermal He beam %
%     diagnostic. (XP-ID: 20181018.39)
% 
%     Full offset-corrected and filtered samples for the inboard VBC %
%     channels. Notice the slight slopes at the beginning of the signal.
%
%     Selection of most important plasma parameters, i.e. %
%     ECRH input power, total radiation loss (P$_{rad}$), core and SOL plasma
%     temperatures (T$_{e}$), as %
%     well as line integrated density (n$_{e}$) and stored %
%     plasma energy (W$_{dia}$).
%
%     Contour plot of the fully calibrated and filtered %
%     channel power, unscaled with plasma/torus volume %
%     and lines of sight. Note the non-functioning channel %
%     #7 (6+1) in the HBC array. The channel number is relative %
%     to the assembly of the indiv. arrays.
%
%     Same inboard VBC channels as before, but now in %
%     units of power by the bolometer equation and filtered %
%     using the \textit{Savitzki-Golay} 2nd order.
%
%     The feedback of course is done fully in parallel with %
%     the rest of the acquistion, which is why it is shown to %
%     the right here. Careful tweaking resulted in the best %
%     possbile temporal resolution of the system while %
%     keeping SNR and delay to a minimum.
%
% }
%
% \begin{itemize}
%     \item{%
%         testing feedback system internally prior to implementation/usage
%     }
%     \item{%
%         defined solid-state laser pulses with cycle %
%         frequency \SI{1}{\hertz}~-~\SI{1}{\kilo\hertz} %
%         and \SI{7}{\milli\watt} power onto detector
%     }%
%     \item{%
%         feedback loop and calculation algorithm limit %
%         the system to a temporal response of %
%         $\ge$\SI{14}{\milli\second}
%     }%
%     \item{%
%         mainly due to $F\cdot\Delta t$ where %
%         $\Delta t$ is the sample rate and $F$ the number %
%         of \textit{FIFO} array elements to average
%     }%
%     \item{%
%         shallow slope of rising edge in feedback is %
%         result of detector response and neglected %
%         dynamic (see bolometer %
%         equation~\cite{Gianone2002})}
% \end{itemize}
%
% \begin{columns}%
%     \column{0.5\textwidth}%
%     \begin{kasten4}{\large%
%         Prelimanary results}
%         \small{%
%         \begin{itemize}%
%             \item{%
%                 feedback signals have been successfully commissioned %
%                 and provided nearly for all experiments throughout the %
%                 previous campaign OP1.2b}%
%             \item{%
%                 focused tweaking of raw scaling factor for %
%                 a single SOL channel in VBC array w/o %
%                 averaging~\&~scaling suprisingly sufficient}%
%         \end{itemize}}%
%         \vspace*{0.7cm}%
%         \centering{%
%             \includegraphics[width=1.0\textwidth]%
%                 {figures/content/%
%                  18101032_successfull_feedback.pdf}}%
%         \vspace*{0.7cm}\newline\tiny{%
%             XPID: 20181010.32; the thermal helium beam %
%             diagnostic fueled He according to the %
%             parametrisation (lime, bottom right) to dial in %
%             the radiated power loss - implicitly given %
%             through the feedback $P_{rad}$ - %
%             \mbox{at $\ge\,0.9P_{ECRH}$.} The oscillations %
%             are a result of the feedback control via the %
%             bolometry signal. The goal was to achieve %
%             stabilized detachment from the divertor %
%             elements. Note how the radiation front along the %
%             SOL (observed by \~ch. 3-6~\&~21-26) oscillates %
%             as well.}
%         \begin{itemize}\small{%
%             \item{%
%                 trying to find possible scaling between %
%                 input power, density, gas fueling and %
%                 radiated power loss, e.g. for plasma control}%
%             \item{%
%                 making simple 3 parameter %
%                 $\lbrace a,b,c\rbrace$ intereference %
%                 assumption like:\vspace*{0.5cm}%
%                 \begin{align}
%                     P_{rad}[\text{MW}]&\propto a%
%                         \lbrace n_{e}%
%                         [10^{19}\text{m}^{-3}]\rbrace^{b}%
%                         \lbrace P_{ECRH}[\text{MW}]%
%                         \rbrace^{c}\nonumber\\
%                     \text{or}&\nonumber\\
%                     &\propto a%
%                         \lbrace f_{H2}%
%                         [\text{mbar\,s/l}]\rbrace^{b}%
%                         \lbrace P_{ECRH}[\text{MW}]%
%                         \rbrace^{c}\nonumber
%                 \end{align}}
%         }\end{itemize}%
%         \vspace*{0.7cm}%
%         \centering{%
%             \includegraphics[width=1.0\textwidth]%
%                 {figures/content/scaling_test.pdf}}%
%         \vspace*{0.7cm}\newline\tiny{%
%             Attempt of finding a possible scaling between %
%             input power, density/main gas fueling in %
%             H$_{2}$ and radiation loss in arbitrary magnetic %
%             configurations. Seperated however are the stages %
%             after the fueling, i.e. right at the %
%             peak of $P_{rad}$ and when equilibrated %
%             (steady).}
%     \end{kasten4}%
% \end{columns}%