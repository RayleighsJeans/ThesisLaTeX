\documentclass{beamer}

% use predefined IPP style
% options: St,noSt : with/without stellarator theory logo
% options: Eurofusion, noEurofusion : with/without Eurofusion logo
\useoutertheme[Eurofusion, w7x]{ippw7x}

% don't show navigation symbols
\beamertemplatenavigationsymbolsempty
% show navigation symbols
%\setbeamertemplate{navigation symbols}[only frame symbol]{}

% transition effects
% \transblindshorizontal
%    Vertical blinds pulled away
% \transblindsvertical
%    Move to center from all sides
% \transboxin
%    Move to all sides from center
% \transboxout
%    Slowly dissolve what was shown before
% \transdissolve
%    Glitter sweeps in specified direction
% \transglitter
%    Sweeps two vertical lines in
% \transslipverticalin
%    Sweeps two vertical lines out
% \transslipverticalout
%    Sweeps two horizontal lines in
% \transhorizontalin
%    Sweeps two horizontal lines out
% \transhorizontalout
%    Sweeps single line in specified direction
% \transwipe
%    Show slide specified number of seconds
% \transcover
% \transfade
% \transpush
% \transuncover
% \transduration{2}
% \addtobeamertemplate{background canvas}{\transfade}{}

% change left/right margins
% \setbeamersize{text margin left=20pt,text margin right=20pt}

% extra packages
\usepackage[english]{babel}
\usepackage{amsmath,amsfonts,amssymb,stackrel}
\usepackage{changepage}
\usepackage{tikz}
\usepackage{array}
\usepackage{units}
\usepackage{siunitx}

\newcommand{\backgroundlogo}{%
    \tikz[overlay,remember picture]{%
    \node[at=(current page.west)] (source) {};%
    \node[opacity = 0.02] {%
    \includegraphics[height=1.\paperheight]%
        {figures/header/minerva}%
    }%
  }
}

\newcommand{\diff}{\text{d}}
\newcommand{\tenpo}[1]{\cdot 10^{#1}}
\newcommand{\ix}[1]{_\text{#1}}
\newcommand{\imag}{\mathbf{i}}
\newcommand{\fett}[1]{\textbf{#1}}
\newcommand{\tilt}[1]{\textit{#1}}
\newcommand\inlineeqno{\stepcounter{equation}\ \quad\quad(\theequation)}

%%%%%%%%%%%%%%%%%%%%%%%%%%%%%%%%%%%%%%%%%%%%%%%%%%%%%%%%%%%%%%%%%%%%%%%%%%%%%%

\begin{document}
% \selectlanguage{german}

% title of the presentation
% short title will be shown in the footer
\title[Report]{Report}

% authors of the presentation
% lecturer (and maybe place and date) will be shown in the footer
\author[P.Hacker]{P. Hacker\inst{1, 2}}

% institutes of the authors
\institute[MPI for Plasmaphysics Greifswald]{%
    \inst{1}%
        Max-Planck-Institute for Plasmaphysics, %
        Wendelsteinstr. 1, Greifswald, Germany \and%
    \inst{2}%
        University of Greifswald, Rubenowstr. 6, Greifswald, Germany}

% set date of the talk
\date{\today}

    % first frame
    \begin{frame}
        % show title of talk and authors
        \titlepage

        % show Logos Helmholtz, Max-Planck and EUROfusion
        \begin{minipage}[]{0.35\textwidth}
            \includegraphics[height=6ex]%
                {figures/header/2017_H_Logo_CMYK_untereinander_EN}
        \end{minipage}
            \hfill
        \begin{minipage}[]{0.2\textwidth}
            \begin{center}
                \includegraphics[height=4ex]{figures/header/minerva}
            \end{center}
        \end{minipage}
        \hfill
        \begin{minipage}[]{0.35\textwidth}
            \begin{flushright}
                \includegraphics[height=5ex]%
                    {figures/header/EUROfusion-LOGO-PANTONE_REFL_BLUE}
            \end{flushright}
        \end{minipage}

        % show acknoledgement from EUROfusion
        \acknowledgement
    \end{frame}

    \begin{frame}{Protocoll}
        \begin{block}{Protocoll 2019/12/16}
            \only<1>{%
                \begin{itemize}
                    \item[+]{\color{orange}{%
                        last time agreed upon thoroughly, step-by-step redo the STRAHL scan; i.e. change transport, profiles and values at the LCFS on at a time to distinguish their effects
                    }}%
                    \item[+]{\color{red}{%
                        do easy, quick and dirty Minimum Fisher Regularization inversions of the same radiation distribution using Daihongs tool
                    }}%
                    \item[+]{\color{orange}{%
                        find missing parts in total radiation distribution and final/last ionistation stage in STRAHL for accurate calculations
                    }}
                \end{itemize}
            }%
        \end{block}
    \end{frame}

    \begin{frame}{STRAHL simulations}
        \begin{block}{Simulations}
            \only<1>{%
                \begin{itemize}
                    \item[+]{%
                        mainly two sets of Thomson scattering profile shapes: orignal and spline interpolated, using different orders (3, 5) of polynoms and number of points
                    }
                    \item[+]{%
                        values at ($\rho_{pol}=1.0$) the LCFS change between 80\% and 33\% of previous value
                    }
                    \item[+]{%
                        different decay lengths of \SIrange{1}{5}{\centi\meter}; transport $D=$\SIrange{2}{4}{\meter\per\square\second}
                    }
                    \item[+]{
                        $D/v$ shape either flat or electron root-like
                    }
                \end{itemize}
            }

            \only<2-5>
                \only<2>{%
                    \centering%
                    \includegraphics[width=.6\textwidth]%
                        {figures/content/compare_ne_Te_00012_00013_full}
                    \\full profiles of $n_{e}$ and $T_{e}$ at $f_{rad}=0.9$(12) and $1.0$(13)
                }
                \only<3>{%
                    \centering%
                    \includegraphics[width=.6\textwidth]%
                        {figures/content/compare_ne_Te_00012_00013_edge}
                    \\full profiles of $n_{e}$ and $T_{e}$ at $f_{rad}=0.9$(12) and $1.0$(13)
                }
                \only<4>{%
                    \centering%
                    \includegraphics[width=.6\textwidth]%
                        {figures/content/compare_strahl_rad_00012_00013_full}
                    \\radiation of each diagnostic line/ionisation at $f_{rad}=0.9$(12) and $1.0$(13)
                }
                \only<5>{%
                    \centering%
                    \includegraphics[width=.6\textwidth]%
                        {figures/content/compare_strahl_rad_00012_00013_edge}
                    \\radiation of each diagnostic line/ionisation at $f_{rad}=0.9$(12) and $1.0$(13)
                }
            }

            \only<6-10>
                \only<6>{%
                    \centering%
                    \includegraphics[width=.5\textwidth]%
                        {figures/content/compare_ne_Te_00012_00016_full}
                    \\full profiles of $n_{e}$ and $T_{e}$ at $f_{rad}=0.9$ with 5cm(12) and 1cm(16) decay length
                }
                \only<7>{%
                    \centering%
                    \includegraphics[width=.5\textwidth]%
                        {figures/content/compare_ne_Te_00012_00016_edge}
                    \\edge profiles of $n_{e}$ and $T_{e}$ at $f_{rad}=0.9$ with 5cm(12) and 1cm(16) decay length
                }
                \only<8>{%
                    \centering%
                    \includegraphics[width=.5\textwidth]%
                        {figures/content/compare_strahl_rad_00012_00016_edge}
                    \\radiation of each diagnostic line/ionisation at $f_{rad}=0.9$ with 5cm(12) and 1cm(16) decay length
                }
                \only<9>{%
                    \centering%
                    \includegraphics[width=.5\textwidth]%
                        {figures/content/compare_strahl_rad_00013_00017_edge}
                    \\radiation of each diagnostic line/ionisation at $f_{rad}=1.0$ with 5cm(13) and 1cm(17) decay length
                }
                \only<10>{%
                    \centering%
                    \includegraphics[width=.5\textwidth]%
                        {figures/content/compare_strahl_rad_00016_00017_edge}
                    \\radiation of each diagnostic line/ionisation at $f_{rad}=0.9$(16) and 1.0(17) with 1cm decay length
                }
            }

            \only<11-12>
                \only<11>{%
                    \centering%
                    \includegraphics[width=.5\textwidth]%
                        {figures/content/compare_strahl_rad_00016_00020_edge}
                    \\radiation of each diagnostic line/ionisation at $f_{rad}=0.9$ with D=2m/s$^{2}$(16) and 4m/s$^{2}$(20)
                }
                \only<12>{%
                    \centering%
                    \includegraphics[width=.5\textwidth]%
                        {figures/content/compare_strahl_rad_00017_00021_edge}
                    \\radiation of each diagnostic line/ionisation at $f_{rad}=1.0$ with D=2m/s$^{2}$(16) and 4m/s$^{2}$(20)
                }
            }

            \only<13-16>
                \only<13>{%
                    \centering%
                    \includegraphics[width=.5\textwidth]%
                        {figures/content/compare_anomal_transp_00021_00025_full}
                    \\profile shapes for D and D/v
                }
                \only<14>{%
                    \centering%
                    \includegraphics[width=.5\textwidth]%
                        {figures/content/compare_strahl_rad_00020_00024_edge}
                    \\radiation of each diagnostic line/ionisation at $f_{rad}=0.9$ with(24) and without(20) D/v profile
                }
                \only<15>{%
                    \centering%
                    \includegraphics[width=.5\textwidth]%
                        {figures/content/compare_strahl_rad_00021_00025_edge}
                    \\radiation of each diagnostic line/ionisation at $f_{rad}=1.0$ with(25) and without(21) D/v profile
                }
                \only<16>{%
                    \centering%
                    \includegraphics[width=.5\textwidth]%
                        {figures/content/compare_strahl_rad_00024_00025_edge}
                    \\radiation of each diagnostic line/ionisation at $f_{rad}=0.9$(24) and 1.0(25) with D/v profile
                }
            }

            \only<17-21>{%
                \frametitle{Spline interp., 80\%, k=3, N=25}
                \only<17>{%
                    \centering%
                    \includegraphics[width=.5\textwidth]%
                        {figures/content/compare_ne_Te_00012_00028_full}
                    \\full profiles of $n_{e}$ and $T_{e}$ at $f_{rad}=0.9$
                }
                \only<18>{%
                    \centering%
                    \includegraphics[width=.5\textwidth]%
                        {figures/content/compare_ne_Te_00013_00029_full}
                    \\full profiles of $n_{e}$ and $T_{e}$ at $f_{rad}=1.0$
                }
                \only<19>{%
                    \centering%
                    \includegraphics[width=.5\textwidth]%
                        {figures/content/compare_ne_Te_00013_00029_edge}
                    \\edge profiles of $n_{e}$ and $T_{e}$ at $f_{rad}=1.0$
                }
                \only<20>{%
                    \centering%
                    \includegraphics[width=.5\textwidth]%
                        {figures/content/compare_strahl_rad_00012_00028_edge}
                    \\radiation of each diagnostic line/ionisation at $f_{rad}=0.9$ from orig.(12) and spline interp.(28) profile
                }
                \only<21>{%
                    \centering%
                    \includegraphics[width=.5\textwidth]%
                        {figures/content/compare_strahl_rad_00028_00029_edge}
                    \\radiation of each diagnostic line/ionisation at $f_{rad}=0.9$(28) and 1.0(29) from spline interp. profile
                }
            }

            \only<22-26>{%
                \frametitle{Spline interp., 33\%, k=35 N=25}
                \only<22>{%
                    \centering%
                    \includegraphics[width=.5\textwidth]%
                        {figures/content/compare_ne_Te_00029_00033_full}
                    \\full profiles of $n_{e}$ and $T_{e}$ at $f_{rad}=0.9$ with k=3(29) and 5(33)
                }
                \only<23>{%
                    \centering%
                    \includegraphics[width=.5\textwidth]%
                        {figures/content/compare_strahl_rad_00029_00033_edge}
                    \\radiation of each diagnostic line/ionisation at $f_{rad}=0.9$ with k=3(29) and 5(33)
                }
                \only<24>{%
                    \centering%
                    \includegraphics[width=.5\textwidth]%
                        {figures/content/compare_ne_Te_00029_00045_edge}
                    \\full profiles of $n_{e}$ and $T_{e}$ at $f_{rad}=1.0$ with values of 80\%(29) and 33\%(45) at LCFS
                }
                \only<25>{%
                    \centering%
                    \includegraphics[width=.5\textwidth]%
                        {figures/content/compare_strahl_rad_00029_00045_edge}
                    \\radiation of each diagnostic line/ionisation at $f_{rad}=1.0$ with values of 80\%(29) and 33\%(45) at LCFS
                }
                \only<26>{%
                    \centering%
                    \includegraphics[width=.5\textwidth]%
                        {figures/content/compare_strahl_rad_00044_00045_edge}
                    \\radiation of each diagnostic line/ionisation at $f_{rad}=0.9$(44) and 1.0(45)
                }
            }

        \end{block}
    \end{frame}

    \begin{frame}{Conclusions}

        \only<1-5>{%
            \begin{block}{STRAHL results}
                \begin{itemize}
                    \only<1>{%
                        \item[+]{%
                            simple TS profiles with 'standard' settings yield small to no seperations between 90\% and 100\% radiation fracation in radial location shift of line maximum
                        }
                        \item[+]{%
                            lower ionisation stages seemingly don't move at all, higher charges shift roughly \SIrange{10}{15}{\centi\meter} at a quarter of the intensity
                        }
                    }
                    \only<2>{%
                        \item[+]{%
                            smaller decay length modelled for SOL without adjunct plasma island region, where profiles would be expected to appear flat or even peaked
                        }
                        \item[+]{%
                            slightly moved peak positions of lower stages, only around a few centimeters (insignificant, not within Bolometer resolution); intensity at LCFS increases noteably
                        }
                    }
                    \only<3>{%
                        \item[+]{%
                            additional increase of D from 2 to \SI{4}{\meter\per\square\second} yields neither a change in maximum position nor change in intensity
                        }
                        \item[+]{%
                            likewise picking a D/v profile according to effects of electron root heating adds no new information
                        }
                    }
                    \only<4>{%
                        \item[+]{%
                            inclusion of spline smoothed profile instead of the original, 'rough' TS datasets holds two effects:
                        }
                        \item[1)]{%
                            STRAHL mapping of profile intput points gives large hill right before LCFS, where the manually 80\% input follows
                        }
                        \item[2)]{%
                            unrelated towards its effect, higher ionisation levels move even further from 90\% to 100\% $f_{rad}$
                        }
                        \item[+]{%
                            changing the LCFS values level shifts the lower ions especially even disregarding the large bump created by STRAHL
                        }
                    }
                    \only<5>{%
                        \item[+]{%
                            possibly biggest distinguishable shift from high to highest fraction in spline interpolated profiles with 33\% at LCFS
                        }
                        \item[+]{%
                            the higher the profile gradiant towards LCFS, the greater the shift inside; governing factor is edge profile shape
                        }
                    }
                \end{itemize}
            \end{block}
        }

        \only<6>{%
            \begin{block}{Inversion \& Exp. Comparison}
                \begin{itemize}
                    \item[+]{%
                        could not access inversion tool, AFS access not granted properly and deprecated version on E5 server ..., waiting for Daihongs eMail response to copy the most recent edition to drive
                    }
                    \item[+]{%
                        added all radiation sources (incl. brems., cont. imp. etc.) to calculated data from STRAHL routine
                    }
                    \item[+]{%
                        started calculations based off of STRAHL simulations to compare experimental results; projection tool for slanted fluxsurfaces to map to from VMEC? (forgot about it ...)
                    }
                \end{itemize}
            \end{block}
        }

    \end{frame}

    % BACKUP
    % \begin{frame}
    % \end{frame}

\end{document}
