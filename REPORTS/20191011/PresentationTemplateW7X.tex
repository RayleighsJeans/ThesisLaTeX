\documentclass{beamer}

% use predefined IPP style
% options: St,noSt : with/without stellarator theory logo
% options: Eurofusion, noEurofusion : with/without Eurofusion logo
\useoutertheme[Eurofusion, w7x]{ippw7x}

% don't show navigation symbols
\beamertemplatenavigationsymbolsempty
% show navigation symbols
%\setbeamertemplate{navigation symbols}[only frame symbol]{}

% transition effects
% \transblindshorizontal
%    Vertical blinds pulled away
% \transblindsvertical
%    Move to center from all sides
% \transboxin
%    Move to all sides from center
% \transboxout
%    Slowly dissolve what was shown before
% \transdissolve
%    Glitter sweeps in specified direction
% \transglitter
%    Sweeps two vertical lines in
% \transslipverticalin
%    Sweeps two vertical lines out
% \transslipverticalout
%    Sweeps two horizontal lines in
% \transhorizontalin
%    Sweeps two horizontal lines out
% \transhorizontalout
%    Sweeps single line in specified direction
% \transwipe
%    Show slide specified number of seconds
% \transcover
% \transfade
% \transpush
% \transuncover
% \transduration{2}
% \addtobeamertemplate{background canvas}{\transfade}{}

% change left/right margins
% \setbeamersize{text margin left=20pt,text margin right=20pt}

% extra packages
\usepackage[english]{babel}
\usepackage{amsmath,amsfonts,amssymb,stackrel}
\usepackage{changepage}
\usepackage{tikz}
\usepackage{array}
\usepackage{units}
\usepackage{siunitx}

\newcommand{\backgroundlogo}{%
    \tikz[overlay,remember picture]{%
    \node[at=(current page.west)] (source) {};%
    \node[opacity = 0.02] {%
    \includegraphics[height=1.\paperheight]%
        {figures/header/minerva}%
    }%
  }
}

\newcommand{\diff}{\text{d}}
\newcommand{\tenpo}[1]{\cdot 10^{#1}}
\newcommand{\ix}[1]{_\text{#1}}
\newcommand{\imag}{\mathbf{i}}
\newcommand{\fett}[1]{\textbf{#1}}
\newcommand{\tilt}[1]{\textit{#1}}
\newcommand\inlineeqno{\stepcounter{equation}\ \quad\quad(\theequation)}

%%%%%%%%%%%%%%%%%%%%%%%%%%%%%%%%%%%%%%%%%%%%%%%%%%%%%%%%%%%%%%%%%%%%%%%%%%%%%%

\begin{document}
% \selectlanguage{german}

% title of the presentation
% short title will be shown in the footer
\title[Meet-Up Report]{Meet Report 10/11/2019}

% authors of the presentation
% lecturer (and maybe place and date) will be shown in the footer
\author[P.Hacker]{P. Hacker\inst{1, 2}}

% institutes of the authors
\institute[MPI for Plasmaphysics Greifswald]{%
    \inst{1}%
        Max-Planck-Institute for Plasmaphysics, %
        Wendelsteinstr. 1, Greifswald, Germany \and%
    \inst{2}%
        University of Greifswald, Rubenowstr. 6, Greifswald, Germany}

% set date of the talk
\date{2019/10/11}


    % first frame
    \begin{frame}
        % show title of talk and authors
        \titlepage

        % show Logos Helmholtz, Max-Planck and EUROfusion
        \begin{minipage}[]{0.35\textwidth}
            \includegraphics[height=6ex]%
                {figures/header/2017_H_Logo_CMYK_untereinander_EN}
        \end{minipage}
            \hfill
        \begin{minipage}[]{0.2\textwidth}
            \begin{center}
                \includegraphics[height=4ex]{figures/header/minerva}
            \end{center}
        \end{minipage}
        \hfill
        \begin{minipage}[]{0.35\textwidth}
            \begin{flushright}
                \includegraphics[height=5ex]%
                    {figures/header/EUROfusion-LOGO-PANTONE_REFL_BLUE}
            \end{flushright}
        \end{minipage}

        % show acknoledgement from EUROfusion
        \acknowledgement
    \end{frame}

    \begin{frame}{Protocoll of meeting}
        \begin{block}{Protocoll}
            \only<1>{%
                Last protocoll, 2019/07/11:
                \begin{itemize}
                    \item[1]{%
                        calculate sensitivity for channels -- localistaion
                    }%
                    \item[2]{%
                        rudimentally same channel selection (with %
                        regards to the number) seem to be mor pontent of %
                        information about plasma radiation%
                    }
                    \item[3]{\color{orange}{%
                        why is that the case? differences in radiation locals%
                    }}%
                    \item[4]{\color{red}{%
                        applicable conclusions for feedback system
                    }}
                \end{itemize}
            }%
            \only<2->{%
                \begin{itemize}
                    \only<2>{%
                        \section{Protocoll}
                        \item[+]{\color{orange}{%
                            get n$_{e}$ and T$_{e}$ profiles regarding the %
                            analised XP IDs accordingly from QTB or divertor %
                            spectroscopy/MPM%
                        }}
                        \item[+]{%
                            just look at O2/O in HEXOS lines to figure stuff %
                            out, also C maybe%
                        }
                        \item[+]{\color{orange}{%
                            P$_{rad}$ not always maximised at LCFS or %
                            island necessarily, rather %
                            $f\left(n_{e},\,T_{e}\right)$ (moving in/out)
                        }}
                    }
                \end{itemize}
            }
        \end{block}
    \end{frame}

    \begin{frame}
        \begin{block}{TS/QTB Profiles}
            \only<1>{%
                \centering
                \includegraphics[height=0.7\textheight]%
                    {figures/content/thomson/%
                     TS_profile_at_0_920516096s_combined.pdf}
            }
            \only<2>{%
                \centering
                \includegraphics[height=0.7\textheight]%
                    {figures/content/thomson/%
                     TS_profile_at_0_6871616s_combined.pdf}
            }
            \only<3>{%
                \centering
                \includegraphics[height=0.7\textheight]%
                    {figures/content/thomson/%
                     avg_Profile_thomson.pdf}
            }
        \end{block}
    \end{frame}

    \begin{frame}
        \begin{block}{Radiational Fraction}
            \only<1>{%
                \centering
                \includegraphics[height=0.7\textheight]%
                    {figures/content/%
                     radiational_fraction_20181010_032.pdf}
            }
            \only<2>{%
                \centering
                \includegraphics[height=0.7\textheight]%
                    {figures/content/%
                     radiational_fraction_20181010_032_edge.pdf}
            }
        \end{block}
    \end{frame}

    \begin{frame}
        \begin{block}{Chordial Profile and HEXOS Lines}
            \only<1>{%
                \centering
                \includegraphics[height=0.7\textheight]%
                    {figures/content/%
                     chordal_profile_HBCm.pdf}
            }
            \only<2>{%
                \centering
                \includegraphics[height=0.7\textheight]%
                    {figures/content/hexos/%
                     hexos_and_channels_20181010_032_%
                     [-0_37451663,1_62548337]['C_II','C_IV','C_V','C_VI'].pdf}
            }
            \only<3>{%
                \centering
                \includegraphics[height=0.7\textheight]%
                    {figures/content/hexos/%
                     hexos_and_channels_20181010_032_%
                     [1_2323633699999998,3_23236337]%
                     ['C_II','C_IV','C_V','C_VI'].pdf}
            }
            \only<4>{%
                \centering
                \includegraphics[height=0.7\textheight]%
                    {figures/content/hexos/%
                     hexos_and_channels_20181010_032_%
                     [2_43684337,4_43684337]['C_II','C_IV','C_V','C_VI'].pdf}
            }
        \end{block}
    \end{frame}

    \begin{frame}{Conclusions and further Proceedings}
        \begin{block}{Protocoll}
            \begin{itemize}
                \item[+]{%
                    QTB profiles most likely of insufficient accuracy and %
                    range for match with r$_{eff}$ and ionisation energies %
                    or photon levels%
                }%
                \item[+]{%
                    L$_{Z}$ needed and transport to link spacial position of %
                    radiation sensitivity (see channel number and analysis) %
                    to a possible ion/impurity species%
                }
                \item[+]{%
                    using STRAHL code with already in the past set up %
                    geometric factors to predict radiation per channel and %
                    hence local emissivity of ion species%
                }
                \item[+]{%
                    already have profiles to feed into STRAHL, else see %
                    MPM or maybe (more work) He-beam%
                }
            \end{itemize}
        \end{block}
    \end{frame}

    % BACKUP
    \begin{frame}
    \end{frame}

\end{document}
