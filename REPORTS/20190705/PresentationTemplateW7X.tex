\documentclass{beamer}

% use predefined IPP style
% options: St,noSt : with/without stellarator theory logo
% options: Eurofusion, noEurofusion : with/without Eurofusion logo
\useoutertheme[Eurofusion, w7x]{ippw7x}

% don't show navigation symbols
\beamertemplatenavigationsymbolsempty
% show navigation symbols
%\setbeamertemplate{navigation symbols}[only frame symbol]{}

% transition effects
% \transblindshorizontal
%    Vertical blinds pulled away
% \transblindsvertical
%    Move to center from all sides
% \transboxin
%    Move to all sides from center
% \transboxout
%    Slowly dissolve what was shown before
% \transdissolve
%    Glitter sweeps in specified direction
% \transglitter
%    Sweeps two vertical lines in
% \transslipverticalin
%    Sweeps two vertical lines out
% \transslipverticalout
%    Sweeps two horizontal lines in
% \transhorizontalin
%    Sweeps two horizontal lines out
% \transhorizontalout
%    Sweeps single line in specified direction
% \transwipe
%    Show slide specified number of seconds
% \transcover
% \transfade
% \transpush
% \transuncover
% \transduration{2}
% \addtobeamertemplate{background canvas}{\transfade}{}

% change left/right margins
% \setbeamersize{text margin left=20pt,text margin right=20pt}

% extra packages
\usepackage[english]{babel}
\usepackage{amsmath,amsfonts,amssymb,stackrel}
\usepackage{changepage}
\usepackage{tikz}
\usepackage{array}

\newcommand{\backgroundlogo}{%
    \tikz[overlay,remember picture]{%
    \node[at=(current page.west)] (source) {};%
    \node[opacity = 0.02] {%
    \includegraphics[height=1.\paperheight]%
        {figures/header/minerva}%
    }%
  }
}

\newcommand{\diff}{\text{d}}
\newcommand{\tenpo}[1]{\cdot 10^{#1}}
\newcommand{\ix}[1]{_\text{#1}}
\newcommand{\imag}{\mathbf{i}}
\newcommand{\fett}[1]{\textbf{#1}}
\newcommand{\tilt}[1]{\textit{#1}}
\newcommand\inlineeqno{\stepcounter{equation}\ \quad\quad(\theequation)}

%%%%%%%%%%%%%%%%%%%%%%%%%%%%%%%%%%%%%%%%%%%%%%%%%%%%%%%%%%%%%%%%%%%%%%%%%%%%%%

\begin{document}
% \selectlanguage{german}

% title of the presentation
% short title will be shown in the footer
\title[Meet-Up Report]{Meeting Report 05.07.19}

% authors of the presentation
% lecturer (and maybe place and date) will be shown in the footer
\author[P.Hacker]{P. Hacker\inst{1, 2}}

% institutes of the authors
\institute[MPI for Plasmaphysics Greifswald]{%
    \inst{1}%
        Max-Planck-Institute for Plasmaphysics, %
        Wendelsteinstr. 1, Greifswald, Germany \and%
    \inst{2}%
        University of Greifswald, Rubenowstr. 6, Greifswald, Germany}

% set date of the talk
\date{2019/07/05}


    % first frame
    \begin{frame}
        % show title of talk and authors
        \titlepage

        % show Logos Helmholtz, Max-Planck and EUROfusion
        \begin{minipage}[]{0.35\textwidth}
            \includegraphics[height=6ex]%
                {figures/header/2017_H_Logo_CMYK_untereinander_EN}
        \end{minipage}
            \hfill
        \begin{minipage}[]{0.2\textwidth}
            \begin{center}
                \includegraphics[height=4ex]{figures/header/minerva}
            \end{center}
        \end{minipage}
        \hfill
        \begin{minipage}[]{0.35\textwidth}
            \begin{flushright}
                \includegraphics[height=5ex]%
                    {figures/header/EUROfusion-LOGO-PANTONE_REFL_BLUE}
            \end{flushright}
        \end{minipage}

        % show acknoledgement from EUROfusion
        \acknowledgement
    \end{frame}

    % new frame
    \begin{frame}{Contents}
        % display all sections
        \tableofcontents%
        \backgroundlogo%
    \end{frame}%

    \begin{frame}{Protocoll of meeting}
        \section{Protocoll}
        \begin{block}{Protocoll}
            To summarize:
            \begin{itemize}
                \item[1]{%
                    calculate sensitivity for channels -- localistaion%
                }
                \item[2]{%
                    check whether this is generally applicable or a %
                    function of different system variables%
                }
                \item[3]{%
                    if necessary, focus on detachment experiments where %
                    feedback is applied and hence the channel selection %
                    does matter%
                }
                \item[4]{%
                    why is that the case? differences in radiation locals%
                }
                \item[5]{%
                    applicable conclusions for feedback system}
            \end{itemize}
        \end{block}
    \end{frame}

    \begin{frame}{Weighted deviation}
        \only<1>{%
            \section{Examples}
            \begin{block}{Weighted deviation}%
                \centering%
                \vspace*{.75cm}%
                $d_{diff}(t)=\|P_{rad}(t) - P_{prediction}(t)\|$\\%
                \vspace*{.75cm}%
                $\varepsilon(t)=%
                    \left\{\begin{array}{ll}%
                    1-\frac{d_{diff}(t)}{P_{rad}(t)}&,%
                        \,\,d_{diff}<P_{rad}\\%
                    0&,\text{ else}
                    \end{array}\right\}$\\%
                \vspace*{.75cm}%
                $\vartheta=\overline{\varepsilon(t)}$%
                \vspace*{.75cm}%
            \end{block}
        }%
        \only<2-4>{%
            \begin{block}{Example and spectrum: feedback 20171207.24 @ EJM}%
                \centering%
                \begin{minipage}{0.47\textwidth}%
                    \only<2-3>{%
                    \includegraphics[width=1.0\textwidth]%
                        {figures/content/20171207.024/wd/3_HBC/%
                         wghtd_dev_C[5_16_27].pdf}%
                    }
                    \only<4>{%
                    \includegraphics[width=1.0\textwidth]%
                        {figures/content/20171207.024/wd/3_VBC/%
                         wghtd_dev_C[64_76_56].pdf}%
                    }
                \end{minipage}%
                \begin{minipage}{0.47\textwidth}%
                    \only<2>{%
                        \includegraphics[width=1.0\textwidth]%
                            {figures/content/20171207.024/wd/3_HBC/%
                             wghtd_dev_spectrum_C3.pdf}%
                    }%
                    \only<3>{%
                        \includegraphics[width=1.0\textwidth]%
                            {figures/content/20171207.024/wd/3_HBC/%
                             spectrum_analysis_weighted_deviation.pdf}%
                    }
                    \only<4>{%
                        \includegraphics[width=1.0\textwidth]%
                            {figures/content/20171207.024/wd/3_VBC/%
                             spectrum_analysis_weighted_deviation.pdf}%
                    }
                \end{minipage}%
            \end{block}%
        }%
    \end{frame}%

    \begin{frame}{Cross correlation}
        \only<1>{%
            \section{Cross correlation}
            \begin{block}{Cross correlation}%
                \centering%
                \vspace*{.75cm}%
                $C_{corr}=%
                    \int (P_{rad}*P_{prediction})%
                    (\tau)\diff\tau$\\%
                    \vspace*{.75cm}\hspace*{1.6cm}%
                    $=\iint P_{rad}(t)P_{prediction}(t+\tau)\diff t\diff\tau$%
                \vspace*{.75cm}%
            \end{block}
        }
        \only<2-4>{%
            \begin{block}{Example and spectrum: feedback 20171207.24 @ EJM}%
                 \centering%
                \begin{minipage}{0.47\textwidth}%
                    \only<2-3>{%
                    \includegraphics[width=1.0\textwidth]%
                        {figures/content/20171207.024/cc/3_HBC/%
                         self_cross_corr_C[5_16_27].pdf}%
                    }
                    \only<4>{%
                    \includegraphics[width=1.0\textwidth]%
                        {figures/content/20171207.024/cc/3_VBC/%
                         self_cross_corr_C[64_76_56].pdf}%
                    }
                \end{minipage}%
                \begin{minipage}{0.47\textwidth}%
                    \only<2>{%
                        \includegraphics[width=1.0\textwidth]%
                            {figures/content/20171207.024/cc/3_HBC/%
                             self_cross_corr_spectrum_C3.pdf}%
                    }%
                    \only<3>{%
                        \includegraphics[width=1.0\textwidth]%
                            {figures/content/20171207.024/cc/3_HBC/%
                             spectrum_analysis_self_correlation.pdf}%
                    }
                    \only<4>{%
                        \includegraphics[width=1.0\textwidth]%
                            {figures/content/20171207.024/cc/3_VBC/%
                             spectrum_analysis_self_correlation.pdf}%
                    }
                \end{minipage}%
            \end{block}%
        }%
    \end{frame}

    \begin{frame}{More channels}
        \only<1>{%
            \section{More channels}
            \begin{block}{Possible parameter space}%
                \centering%
                \vspace*{.75cm}%
                    C = ALL, HBC, VBC and HBC \& VBC\newline%
                    A = 1 (normalization), 3, 4, 5, 6, 7, 8, 9 %
                \vspace*{.75cm}%
            \end{block}
        }
        \only<2-4>{%
            \begin{block}{Example and spectrum: feedback 20171207.24 @ EJM}%
                 \centering%
                \begin{minipage}{0.47\textwidth}%
                    \only<2-3>{%
                    \includegraphics[width=1.0\textwidth]%
                        {figures/content/20171207.024/wd/5_HBC/%
                         wghtd_dev_C.pdf}%
                    }
                    \only<4>{%
                    \includegraphics[width=1.0\textwidth]%
                        {figures/content/20171207.024/wd/5_VBC/%
                         wghtd_dev_C.pdf}%
                    }
                \end{minipage}%
                \begin{minipage}{0.47\textwidth}%
                    \only<2>{%
                        \includegraphics[width=1.0\textwidth]%
                            {figures/content/20171207.024/wd/5_HBC/%
                             wghtd_dev_spectrum_C5.pdf}%
                    }%
                    \only<3>{%
                        \includegraphics[width=1.0\textwidth]%
                            {figures/content/20171207.024/wd/5_HBC/%
                             spectrum_analysis_weighted_deviation.pdf}%
                    }
                    \only<4>{%
                        \includegraphics[width=1.0\textwidth]%
                            {figures/content/20171207.024/wd/5_VBC/%
                             spectrum_analysis_weighted_deviation.pdf}%
                    }
                \end{minipage}%
            \end{block}%
        }%
    \end{frame}

    \begin{frame}{Different configuration}
        \only<1-3>{%
            \begin{block}{Example and spectrum: feedback 20181009.046 @ KJM}%
                 \centering%
                \begin{minipage}{0.47\textwidth}%
                    \only<1-2>{%
                    \includegraphics[width=1.0\textwidth]%
                        {figures/content/20181009.046/wd/3_HBC/%
                         wghtd_dev_C[5_16_27].pdf}%
                    }
                    \only<3>{%
                    \includegraphics[width=1.0\textwidth]%
                        {figures/content/20181009.046/wd/3_VBC/%
                         wghtd_dev_C[64_76_56].pdf}%
                    }
                \end{minipage}%
                \begin{minipage}{0.47\textwidth}%
                    \only<1>{%
                        \section{Different configuration}
                        \includegraphics[width=1.0\textwidth]%
                            {figures/content/20181009.046/wd/3_HBC/%
                             wghtd_dev_spectrum_C3.pdf}%
                    }%
                    \only<2>{%
                        \includegraphics[width=1.0\textwidth]%
                            {figures/content/20181009.046/wd/3_HBC/%
                             spectrum_analysis_weighted_deviation.pdf}%
                    }
                    \only<3>{%
                        \includegraphics[width=1.0\textwidth]%
                            {figures/content/20181009.046/wd/3_VBC/%
                             spectrum_analysis_weighted_deviation.pdf}%
                    }
                \end{minipage}%
            \end{block}%
        }%
    \end{frame}

    \begin{frame}{Fit and comparison configurations}
        \begin{block}{HBC 3 channel combination sensitivity}%
            \only<1-2>{%
                \begin{minipage}{0.47\textwidth}%
                    \only<1-2>{%
                    \includegraphics[height=.8\textheight]%
                        {figures/content/20181009.046/%
                         best_cell_and_fit_withLOS.pdf}%
                    }
                \end{minipage}%
                \begin{minipage}{0.47\textwidth}%
                    \only<1>{%
                        \section{Fit and comparison}%
                        \includegraphics[height=.8\textheight]%
                            {figures/content/20181009.046/%
                             best_channels_and_fit_withLOS.pdf}%
                    }%
                    \only<2>{%
                        \includegraphics[height=.8\textheight]%
                            {figures/content/20181009.046/%
                             worst_cell_and_fit_withLOS.pdf}%
                    }
                \end{minipage}%
            }%
            \only<3>{%
                \centering%
                \includegraphics[height=.8\textheight]%
                    {figures/content/av_sense_crossXperiment.pdf}%
            }%
            \only<4>{%
                \centering%
                \includegraphics[height=.8\textheight]%
                    {figures/content/best_chans_crossXperiment.pdf}%
            }%
        \end{block}%
    \end{frame}

    \begin{frame}{Protocoll of meeting}
        \begin{block}{Protocoll}
            \only<1>{%
                Last protocoll, 2019/05/11:
                \begin{itemize}
                    \item[1]{\color{green}{%
                        calculate sensitivity for channels -- localistaion
                    }}%
                    \item[2]{\color{orange}{%
                        check whether this is generally applicable or a %
                        function of different system variables%
                    }}
                    \item[3]{\color{orange}{%
                        why is that the case? differences in radiation locals%
                    }}%
                    \item[4]{\color{red}{%
                        applicable conclusions for feedback system
                    }}
                \end{itemize}
            }%
            \only<2->{%
                \begin{itemize}
                    \only<2>{%
                        \section{Protocoll}
                        \item[+]{
                            check for victoria winters feedback session %
                            with different density stages, power leves and %
                            CH4
                        }%
                        \item[+]{%
                            just look at O2/O in HEXOS lines to figure stuff %
                            out, also C maybe%
                        }
                        \item[+]{%
                            get n$_{e}$ and T$_{e}$ profiles regarding the %
                            analised XP IDs accordingly from QTB or divertor %
                            spectroscopy/MPM%
                        }
                    }
                    \only<3>{%
                        \item[+]{
                            IN PARTICULAR:\\%
                            what makes those 'best channels' so important %
                            and distinguishes them from others
                        }%
                        \item[+]{%
                            where and what is in magnetic/plasma surface %
                            connection (toroidally)
                        }
                        \item[+]{%
                            P$_{rad}$ not always maximised at LCFS or %
                            island necessarily, rather %
                            $f\left(n_{e},\,T_{e}\right)$ (moving in/out)
                        }
                    }
                    \only<4>{%
                        \item[+]{
                            at roughly 30\% f$_{rad}$ we have 50\,eV along the LCFS which results in hight oxygen radiation %
                            fractions, while the islands slowly start %
                            radiating
                        }%
                        \item[+]{\color{red}{%
                            What is the most sensitive and important factor %
                            in the spatial radiation profile for the %
                            feedback?}
                        }
                    }
                \end{itemize}
            }
        \end{block}
    \end{frame}

\end{document}
