\documentclass[
  fontsize=11pt,
  paper=a4,
  parskip=half,
  enlargefirstpage=on,    % More space on first page
  fromalign=right,        % PLacement of name in letter head
  fromphone=on,           % Turn on phone number of sender
  fromrule=aftername,     % Rule after sender name in letter head
  addrfield=off,           % Adress field for envelope with window
  backaddress=off,         % Sender address in this window
  subject=beforeopening,  % Placement of subject
  locfield=narrow,        % Additional field for sender
  foldmarks=on,           % Print foldmarks
]{scrlttr2}

\usepackage[T1]{fontenc}
\usepackage[utf8]{inputenc}
\usepackage[german]{babel}
\usepackage{graphicx}

\setkomafont{fromname}{\sffamily \LARGE}
\setkomafont{fromaddress}{\sffamily}%% statt \small
\setkomafont{pagenumber}{\sffamily}
\setkomafont{subject}{\bfseries}
\setkomafont{backaddress}{\mdseries}

\LoadLetterOption{DIN}
\setkomavar{fromname}{Philipp Scholl}
\setkomavar{fromaddress}{%
    5 Merchants Place, Stephen Street\\%
    Dunlavin\\%
    Co. Wicklow\\%
    W91 C3Y6\\%
    Ireland}
\setkomavar{fromphone}{+353 87 1922755}
\setkomavar{fromemail}{rayleighsjeans@gmail.com}
\setkomavar{backaddressseparator}{\enspace\textperiodcentered\enspace}
\setkomavar{signature}{(Philipp Scholl)}
\setkomavar{place}{Dunlavin}
\setkomavar{date}{\today}
\setkomavar{enclseparator}{: }

\begin{document}
\begin{letter}{}
  \setkomavar{subject}{Einspruchsbegründung zum Bescheid 2023 über Einkommensteuer und Solidaritätszuschlag vom 25.03.2024}
  \opening{Sehr geehrte Damen und Herren,}

  Hiermit übermittle ich Ihnen die entsprechende Begründung des Einspruches zum Bescheid für 2023 über Einkommensteuer und Solidaritätszuschlag vom 25.03.2024 der Eheleute Stella (IdNr. Ehefrau 69 248 720 538) und Philipp (IdNr. Ehemann 78 630 294 856) Scholl zur Steuernummer 185/368/14414.\\%

  Diesen Einspruch begründen wir wie folgt:\\%

  Mit dem oben genannten Bescheid wurden die im Rahmen des Antwortschreibens zum Fragebogen zum häuslichen Arbeitszimmer erläuterten und begründeten Werbungskosten für das häusliche Arbeitszimmer nicht berücksichtigt. Nach §~9 Abs. 5 S. 1 EStG gelten die Vorschriften über die Abzugsfähigkeit von Aufwendungen für häusliche Arbeitszimmer iSd §~4 Abs. 5 Nr. 6b EStG sinngemäß. Damit greift zwangsweise die gesetzliche Gleichstellung der Gewinn- und Überschusseinkünfte (wie auch in der sinngemäßen Anwendung der Vorschriften über die Abschreibung) auch im Hinblick des Arbeitsplatzes des Steuerpflichtigen. Darüber hinaus sind Aufwendungen für häusliche Arbeitszimmer sowie die Kosten der Ausstattung grundsätzlich nach §~4 Abs. 5 Nr. 6b S. 1 EStG nicht abzugsfähig. Die Ausnahme besteht allerdings dann, wenn das Arbeitszimmer den Mittelpunkt der gesamten betrieblichen und beruflichen Betätigung bildet. Dieses Merkmal differenziert sich in qualitativer wie auch (sekundär) quantitativer Hinsicht.\\[0.5cm]%
  Das Ihnen vorliegende Antwortschreiben zum Fragebogen zum häuslichen Arbeitszimmer stellt einen – vom Rest der Wohnung – völlig abgegrenzten und ausschließlich zu dem Zweck der Arbeit ausgestatteten Raum dar, welcher auch getrennt von der zivilen Wohnung erreichbar ist. Die berufliche Betätigung bei meinem Arbeitgeber - KMW – erstreckt sich u.a. über die Entwicklung von Fahrzeugdynamiken und -physik sowie deren KI (Künstliche Intelligenz) für das digitale Training und verschiedene Simulationskomponenten für verschiedenes schweres Kriegsmaterial. Die Arbeit wird nur durch eine geeignete IT-Ausstattung (Server, Netzwerkinfrastruktur) ermöglicht, wie Sie im Arbeitszimmer auch vorliegt und genutzt wird. Ein Verbot für die Tätigkeit im „Home-Office“ lag seitens meines Arbeitgebers nicht vor.\\[0.5cm]%
  Des Weiteren wurde das häusliche Arbeitszimmer – auch aufgrund der Ausstattung - dazu genutzt, meine Dissertation fertig zu stellen. Die Dissertation, anders als es ggf. aus dem Antwortschreiben hervorgegangen ist, begründet keine weitere berufliche oder selbstständige Tätigkeit (im Sinne bspw. eines Hilfswissenschaftlers), sondern ist als Weiterbildung im Rahmen der beruflichen Tätigkeit bei KMW zu verstehen. So war meine Qualifikation als angehender Doktor in der Physik ein nicht unerhebliches Kriterium für meine Einstellung – wie auch aus der bereits genannten Erläuterung zum Aufgabengebiet meiner Tätigkeit zu erkennen ist. Aufgrund der Sicherheitsmaßnahmen bei KMW konnten allerdings keine Softwareprogramme, die notwendig waren zur Forschung und Erstellung meiner Dissertation, auf gegebener Hardware installiert noch genutzt werden. Aufgrund dessen ist die Promotion – in Absprache mit dem Arbeitgeber – im dafür eingerichteten häuslichen Arbeitszimmer erstellt worden. Die Bestrebungen in der Erreichung eines weiteren akademischen Grades waren dem Arbeitgeber nicht nur bekannt, sondern wurden auch entsprechend gefördert.\\[0.5cm]%
  Somit ist das Arbeitszimmer nach quantitativer wie auch qualitativer Hinsicht als Mittelpunkt der zudem einzigen beruflichen Betätigung zu sehen.\\%
  Aufgrund der og. Argumente sehe ich daher keine Begründung darin eine Abzugsfähigkeit des häuslichen Arbeitszimmers erstmalig nicht stattzugeben.\\%

  Mit freundlichen Grüßen\\%
  \includegraphics[width=3.5cm]{%
    ../figures/images/signature_scholl.png}%
  \hspace*{1cm}%
  \includegraphics[width=2.5cm]{%
    ../figures/images/signature_stella.png}\\[-0.7cm]%
  (Philipp Scholl)\hspace*{2cm}(Stella Scholl)%

\end{letter}
\end{document}